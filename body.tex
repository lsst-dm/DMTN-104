\section{Introduction}


\subsection{Obective}

The objective is to provide a clear picture on the DM management products, including description, characterization and dependencies.


\subsection{Definitions}

Following definitions are relevant for this document.


\subsubsection{Product} \label{sec:product}

A product is a component of the DM product tree, that contribute to satisfy the DM requirements.

Each product is characterize by:

\begin{itemize}
  \item \textbf{Name}: it is the main identification of the product. In some cases a \textit{short name} can be provided for documentation purposes.
  \item \textbf{Unique Key}: it is a unique string mainly for programmatic usage. The product manager/owner can provide one, or it will be assigned by the Architecture team.
  \item \textbf{Owner}: who is responsible for the quality and acceptance of a particular product (\citeds{LDM-294} \S6.6).
  \item \textbf{Manager}: {T/CAM} who has managerial and financial responsibility for the engineering teams within DM (\citeds{LDM-294} \S6.14.1).
  \item \textbf{WBS}: the WBS identification.
  \item \textbf{Team}: the team in which the product is developed.
  \item \textbf{Description}: it provides a short overview of the product purpose.
  \item \textbf{Git Package}: the Git package that implementes the product.
  \begin{itemize}
    \item \textbf{software products}: it shall be the top levelpackage as per \citeds{DMTN-106} definition. 
    \item \textbf{services}: it should be the Git package where the Dockerfile defining the service is defined. The same repository could be used to keep under configuratino control other configuration files related to kubernettes definition (except secrets).
  \end{itemize}
  \item \textbf{Dependencies}: the list of products required to implement this product.
  \item \textbf{Upstream Products}: the list is products that use this product.
  \item \textbf{Links}: internet links that provide inportant information in order to clearly define the product.
  \item \textbf{Documentation}: list of Rubin documents that characterize the product.
  \item \textbf{Requirements}: list of requirements that are related in the model to the product.
\end{itemize}


\subsubsection{Software Product} \label{sec:swproduct}

Please refer to \citeds{DMTN-106} section \$2 for the Software Product definition.


\subsubsection{Product Tree} \label{sec:ptree}

The product tree is the graphical representation of the DM products.


\subsubsection{Dependencies} \label{sec:dependencies}

The dependencies listed in this document are functional dependencies.
The aim is to provide, for each product, the list of other products needed for its implementation or execution. 


\subsection{Applicable Documents}

Following documents are applicable:

\citeds{LDM-294} Data Management Organization and Management\\
% \citeds{LDM-148} Data Management System Design\\


\subsection{Document Overview}

Information in sections \ref{sec:top} and \ref{sec:sups} is extracted from MagicDraw and provides:

\begin{itemize}
\item the DM top level product tree: it includes all products required to satisfy the DMS requirements (\citeds{LSE-61}).
\item the DM support product tree: it includes all products required for construction and maintenance.
\end{itemize}

The information regarding the Git packages implementing the above products (in sections \ref{sec:top} and \ref{sec:sups}) is provided in section \ref{sec:low}. 
This information is extracted from GitHub.

The section \ref{sec:nondm} lists all non DM products that are required in order to fulfill the DMS requirements.
This information is extracted from MagicDraw.
No GitHub package details are provided for these products. 

The last section \ref{sec:jiracomponents} will document all components in the DM Jira project.
 

\subsection{Consistency and Completeness}\label{sec:cons-comp}

The information collected in the product tree shall be consistent and complete.

In order to be consistent, following rules need to be fulfilled:

\begin{itemize}
\item Facilities: shall host Enclaves
\item Enclaves: shall host services
\item Services: shall be implemented using DM Software Products or COTS
\item DM Software Products: shall have only one GitHub package
\item Low level dependencies to be made explicit in section 4
\item Dependencies to 3rd party libraries should (not yet, but will be) derived from GitHub
\item COTS and 3rd party libraries: shall have a link to a documentation page available in the global internet
\end{itemize}

In order to be complete, following rules need to be fulfilled:

\begin{itemize}
\item All products shall have an owner
\item All packages (subsection, paragraphs) shall have a manager
In case of multiple managers in the underlying products, no manager is specified and Arch Team will review it.
\item Each manager and each owner needs ensure that:
\begin{itemize}
\item She/he is the right person of taking care of that package/product
\item The information provided for each package/product is sufficient to characterize it.
\end{itemize}
\item Architecture Team shall ensure the consistency of the document following the rules listed above.
\end{itemize}


\newpage
\section{Top Level Product Tree}\label{sec:top}

The products listed in this section are maintained in MagicDraw by the DM System Engineering group. 

These products are meant to be operational, i.e. will process the survey source data and produce L1 or L2 data products, or will be distributed to permit the external collaborators and the science community to provide their contribution to the project.

These products are organized and maintained in MagicDraw by the DM System Engineering group.

\newpage
\subsection{Data Management Operational Products}\label{sec:dmtop}

% auto generated from MagicDraw (revision) 300 - DO NOT EDIT!
% using template at <template>.
% Collecting data for component: ""
% using docsteady version 
%
% This file is meant to be included in LaTeX document in order to provide:
%   -  MagicDraw Top Level Product Tree (section 2)

\begin{longtable}{p{3.7cm}p{3.7cm}p{3.7cm}p{3.7cm}}\hline
\textbf{Manager} & \textbf{Owner} & \textbf{WBS} & \textbf{Team} \\ \hline
\parbox{3.5cm}{
Wil O'Mullane
\vspace{2mm}%
} &
\begin{tabular}{@{}l@{}}
\parbox{3.5cm}{
Leanne Guy
\vspace{2mm}%
} \\
\end{tabular}
 &
\begin{tabular}{@{}l@{}}
1.02C \\
\end{tabular} &
\begin{tabular}{@{}l@{}}
 \\
\end{tabular} \\ \hline
\multicolumn{4}{c}{
{\footnotesize ( Data Management
 - DM ) }
}\\ \hline
\citeds{LSE-61} &
\multicolumn{3}{l}{ Data Management System (DMS) Requirements }
\\ \cdashline{1-4}
\citeds{LDM-639} &
\multicolumn{3}{l}{ LSST Data Management Acceptance Test Specification }
\\ \cdashline{1-4}
\citeds{LDM-148} &
\multicolumn{3}{l}{ Data Management System Design }
\\ \cdashline{1-4}
\citeds{LDM-692} &
\multicolumn{3}{l}{ DM Verification Control Document }
\\ \cdashline{1-4}
\end{longtable}

  The products listed in the following subsections are required in order
to satisfy the DM requirements at all levels (LSE-61, interfaces and low
level flow down requirements).\\[2\baselineskip]These products are
organized in the following groups:

\begin{itemize}
\tightlist
\item
  service products
\item
  software products (developed by DM)
\item
  resource products (hardware, COTS and external provided softwares)
\item
  infrastructure products (physical or logical)
\item
  data products (reference data for development and operations)
\end{itemize}

~

The product tree graph for DM is available
\href{https://DMTN-104.lsst.io/Main_mixedLand.pdf}{ here }.

\newpage
\subsection{Service Products}\label{dmsrv}
\begin{longtable}{p{3.7cm}p{3.7cm}p{3.7cm}p{3.7cm}}\hline
\textbf{Manager} & \textbf{Owner} & \textbf{WBS} & \textbf{Team} \\ \hline
\parbox{3.5cm}{
Wil O'Mullane
\vspace{2mm}%
} &
\begin{tabular}{@{}l@{}}
\parbox{3.5cm}{
Leanne Guy
\vspace{2mm}%
} \\
\end{tabular}
 &
\begin{tabular}{@{}l@{}}
 \\
\end{tabular} &
\begin{tabular}{@{}l@{}}
 \\
\end{tabular} \\ \hline
\multicolumn{4}{c}{
{\footnotesize ( Services
 - DMSRV ) }
}\\ \hline
\end{longtable}

  This product includes all services required for:

\begin{itemize}
\tightlist
\item
  data distribution (DBB)
\item
  data processing (prompt and offline)
\item
  data access (LSP)
\item
  supporting activities (IT, monitoring, DBs, etc)
\end{itemize}

All services should be characterize by the implementing products and by
a configuration file (docker or similar) maintained in GitHub
repository.

The configuration file should follow the same development and releases
processes defined for the DM software products.

~

All services shall also specify which software products, DM, third party
or COTS, are implementing it. This information is shown in the
\textbf{Product Dependencies} table, \emph{Uses} column.


\includegraphics[max width=\linewidth]{subtrees/Main_DMSRV.pdf}

\newpage
\subsubsection{Prompt Services}\label{prsrv}
\begin{longtable}{p{3.7cm}p{3.7cm}p{3.7cm}p{3.7cm}}\hline
\textbf{Manager} & \textbf{Owner} & \textbf{WBS} & \textbf{Team} \\ \hline
\parbox{3.5cm}{
Michelle Butler
\vspace{2mm}%
} &
\begin{tabular}{@{}l@{}}
\parbox{3.5cm}{
Multiple
\vspace{2mm}%
} \\
\end{tabular}
 &
\begin{tabular}{@{}l@{}}
 \\
\end{tabular} &
\begin{tabular}{@{}l@{}}
 \\
\end{tabular} \\ \hline
\multicolumn{4}{c}{
{\footnotesize ( Prompt Services
 - PRSRV ) }
}\\ \hline
\citeds{LDM-533} &
\multicolumn{3}{l}{ LSST Level 1 System Test Specification }
\\ \cdashline{1-4}
\end{longtable}

  DM Services that need to run in a prompt manner.


   \newpage
\begin{longtable}{p{3.7cm}p{3.7cm}p{3.7cm}p{3.7cm}}\toprule
\multicolumn{2}{l}{\large \textbf{ \hypertarget{popsrv}{Planned Observation Publication} } }
& \multicolumn{2}{l}{(product in: Prompt Services
)}
\\ \hline
\textbf{\footnotesize Manager} & \textbf{\footnotesize Owner} &
\textbf{\footnotesize WBS} & \textbf{\footnotesize Team} \\ \hline
\parbox{3.5cm}{
Michelle Butler
\vspace{2mm}%
} &
\begin{tabular}{@{}l@{}}
\parbox{3.5cm}{
Tim Jenness
\vspace{2mm}%
} \\
\parbox{3.5cm}{
Kian-Tat Lim
\vspace{2mm}%
} \\
\end{tabular} &
\begin{tabular}{@{}l@{}}
1.02C.07.06.02 \\
\end{tabular} & \begin{tabular}{@{}l@{}}
LDF \\
\end{tabular} \\ \hline
\multicolumn{4}{c}{
{\footnotesize ( Planned Obs Pub
 - POPSRV ) }
}\\ \hline
\end{longtable}

This service receives telemetry from the OCS describing the next visit
location and the telescope scheduler's predictions of its future
observations. It publishes these as an unauthenticated,
globally-accessible web service comprising both a web page for human
inspection and a web API for usage by automated tools.


\begin{longtable}{p{3.7cm}p{3.7cm}p{3.7cm}p{3.7cm}}\hline
\textbf{\footnotesize Uses:}  & & \textbf{\footnotesize Used in:} & \\ \hline
\multicolumn{2}{c}{
\begin{tabular}{c}
\hyperlink{obspub}{Planned Observation Publication SW} \\ \hline
\end{tabular}
} &
\multicolumn{2}{c}{
\begin{tabular}{c}
\hyperlink{encprb}{Prompt Base Enclave} \\ \hline
\end{tabular}
} \\ \bottomrule
\multicolumn{4}{c}{\textbf{Related Requirements} } \\ \hline
{\footnotesize DMS-REQ-0353 } &
\multicolumn{3}{p{11.1cm}}{\footnotesize Publishing predicted visit schedule } \\ \cdashline{1-4}
\end{longtable}

     \newpage
\begin{longtable}{p{3.7cm}p{3.7cm}p{3.7cm}p{3.7cm}}\toprule
\multicolumn{2}{l}{\large \textbf{ \hypertarget{prpingsrv}{Prompt Processing Ingest} } }
& \multicolumn{2}{l}{(product in: Prompt Services
)}
\\ \hline
\textbf{\footnotesize Manager} & \textbf{\footnotesize Owner} &
\textbf{\footnotesize WBS} & \textbf{\footnotesize Team} \\ \hline
\parbox{3.5cm}{
Michelle Butler
\vspace{2mm}%
} &
\begin{tabular}{@{}l@{}}
\parbox{3.5cm}{
Steve Pietrowicz
\vspace{2mm}%
} \\
\end{tabular} &
\begin{tabular}{@{}l@{}}
1.02C.07.06.02 \\
\end{tabular} & \begin{tabular}{@{}l@{}}
LDF \\
\end{tabular} \\ \hline
\multicolumn{4}{c}{
{\footnotesize ( Prompt Proc Ing
 - PRPINGSRV ) }
}\\ \hline
\end{longtable}

This service is implemented in two instances that capture
crosstalk-corrected images from the LSSTCam and ComCam Camera Data
Systems along with selected metadata from the OCS and/or EFD and
transfer them to the Prompt Processing service in the Prompt NCSA
Enclave. There is no Prompt Processing Ingest instance for the auxiliary
telescope spectrograph.


\begin{longtable}{p{3.7cm}p{3.7cm}p{3.7cm}p{3.7cm}}\hline
\textbf{\footnotesize Uses:}  & & \textbf{\footnotesize Used in:} & \\ \hline
\multicolumn{2}{c}{
\begin{tabular}{c}
\hyperlink{iip}{Image Ingest and Processing} \\ \hline
\end{tabular}
} &
\multicolumn{2}{c}{
\begin{tabular}{c}
\hyperlink{encprb}{Prompt Base Enclave} \\ \hline
\hyperlink{encprn}{Prompt NCSA Enclave} \\ \hline
\end{tabular}
} \\ \bottomrule
\multicolumn{4}{c}{\textbf{Related Requirements} } \\ \hline
{\footnotesize CA-DM-CON-ICD-0007 } &
\multicolumn{3}{p{11.1cm}}{\footnotesize Provide Data Management Conditions data } \\ \cdashline{1-4}
{\footnotesize CA-DM-CON-ICD-0008 } &
\multicolumn{3}{p{11.1cm}}{\footnotesize Data Management Conditions data latency } \\ \cdashline{1-4}
{\footnotesize CA-DM-CON-ICD-0014 } &
\multicolumn{3}{p{11.1cm}}{\footnotesize Provide science sensor data } \\ \cdashline{1-4}
{\footnotesize CA-DM-CON-ICD-0015 } &
\multicolumn{3}{p{11.1cm}}{\footnotesize Provide wavefront sensor data } \\ \cdashline{1-4}
{\footnotesize CA-DM-CON-ICD-0016 } &
\multicolumn{3}{p{11.1cm}}{\footnotesize Provide guide sensor data } \\ \cdashline{1-4}
{\footnotesize CA-DM-CON-ICD-0017 } &
\multicolumn{3}{p{11.1cm}}{\footnotesize Data Management load on image data interfaces } \\ \cdashline{1-4}
{\footnotesize DM-TS-CON-ICD-0002 } &
\multicolumn{3}{p{11.1cm}}{\footnotesize Timing } \\ \cdashline{1-4}
{\footnotesize DM-TS-CON-ICD-0007 } &
\multicolumn{3}{p{11.1cm}}{\footnotesize Timing } \\ \cdashline{1-4}
{\footnotesize DMS-REQ-0004 } &
\multicolumn{3}{p{11.1cm}}{\footnotesize Nightly Data Accessible Within 24 hrs } \\ \cdashline{1-4}
{\footnotesize DMS-REQ-0008 } &
\multicolumn{3}{p{11.1cm}}{\footnotesize Pipeline Availability } \\ \cdashline{1-4}
{\footnotesize DMS-REQ-0022 } &
\multicolumn{3}{p{11.1cm}}{\footnotesize Crosstalk Corrected Science Image Data Acquisition } \\ \cdashline{1-4}
{\footnotesize DMS-REQ-0099 } &
\multicolumn{3}{p{11.1cm}}{\footnotesize Level 1 Performance Report Definition } \\ \cdashline{1-4}
{\footnotesize DMS-REQ-0131 } &
\multicolumn{3}{p{11.1cm}}{\footnotesize Calibration Images Available Within Specified Time } \\ \cdashline{1-4}
{\footnotesize DMS-REQ-0162 } &
\multicolumn{3}{p{11.1cm}}{\footnotesize Pipeline Throughput } \\ \cdashline{1-4}
{\footnotesize DMS-REQ-0167 } &
\multicolumn{3}{p{11.1cm}}{\footnotesize Incorporate Autonomics } \\ \cdashline{1-4}
{\footnotesize DMS-REQ-0284 } &
\multicolumn{3}{p{11.1cm}}{\footnotesize Level-1 Production Completeness } \\ \cdashline{1-4}
{\footnotesize DMS-REQ-0294 } &
\multicolumn{3}{p{11.1cm}}{\footnotesize Processing of Datasets } \\ \cdashline{1-4}
{\footnotesize DMS-REQ-0301 } &
\multicolumn{3}{p{11.1cm}}{\footnotesize Control of Level-1 Production } \\ \cdashline{1-4}
{\footnotesize DMS-REQ-0314 } &
\multicolumn{3}{p{11.1cm}}{\footnotesize Compute Platform Heterogeneity } \\ \cdashline{1-4}
{\footnotesize DMS-REQ-0315 } &
\multicolumn{3}{p{11.1cm}}{\footnotesize DMS Communication with OCS } \\ \cdashline{1-4}
{\footnotesize DMS-REQ-0318 } &
\multicolumn{3}{p{11.1cm}}{\footnotesize Data Management Unscheduled Downtime } \\ \cdashline{1-4}
{\footnotesize OCS-DM-COM-ICD-0003 } &
\multicolumn{3}{p{11.1cm}}{\footnotesize Data Management CSC Command Response Model } \\ \cdashline{1-4}
{\footnotesize OCS-DM-COM-ICD-0004 } &
\multicolumn{3}{p{11.1cm}}{\footnotesize Data Management Exposed CSCs } \\ \cdashline{1-4}
{\footnotesize OCS-DM-COM-ICD-0007 } &
\multicolumn{3}{p{11.1cm}}{\footnotesize Prompt Processing CSC } \\ \cdashline{1-4}
{\footnotesize OCS-DM-COM-ICD-0009 } &
\multicolumn{3}{p{11.1cm}}{\footnotesize Command Set Implementation by Data Management } \\ \cdashline{1-4}
{\footnotesize OCS-DM-COM-ICD-0012 } &
\multicolumn{3}{p{11.1cm}}{\footnotesize Start Command } \\ \cdashline{1-4}
{\footnotesize OCS-DM-COM-ICD-0013 } &
\multicolumn{3}{p{11.1cm}}{\footnotesize configure Successful Completion Response } \\ \cdashline{1-4}
{\footnotesize OCS-DM-COM-ICD-0014 } &
\multicolumn{3}{p{11.1cm}}{\footnotesize enable Command } \\ \cdashline{1-4}
{\footnotesize OCS-DM-COM-ICD-0015 } &
\multicolumn{3}{p{11.1cm}}{\footnotesize disable Command } \\ \cdashline{1-4}
{\footnotesize OCS-DM-COM-ICD-0036 } &
\multicolumn{3}{p{11.1cm}}{\footnotesize standby Command } \\ \cdashline{1-4}
{\footnotesize OCS-DM-COM-ICD-0037 } &
\multicolumn{3}{p{11.1cm}}{\footnotesize exit Command } \\ \cdashline{1-4}
{\footnotesize OCS-DM-COM-ICD-0038 } &
\multicolumn{3}{p{11.1cm}}{\footnotesize enterControl Command } \\ \cdashline{1-4}
{\footnotesize OCS-DM-COM-ICD-0039 } &
\multicolumn{3}{p{11.1cm}}{\footnotesize enterControl Successful Completion Response } \\ \cdashline{1-4}
{\footnotesize OCS-DM-COM-ICD-0040 } &
\multicolumn{3}{p{11.1cm}}{\footnotesize Command Completion Response } \\ \cdashline{1-4}
{\footnotesize OCS-DM-COM-ICD-0046 } &
\multicolumn{3}{p{11.1cm}}{\footnotesize Image Forwarded Event } \\ \cdashline{1-4}
{\footnotesize OCS-DM-COM-ICD-0056 } &
\multicolumn{3}{p{11.1cm}}{\footnotesize Prompt Processing Resource Availability } \\ \cdashline{1-4}
\end{longtable}

     \newpage
\begin{longtable}{p{3.7cm}p{3.7cm}p{3.7cm}p{3.7cm}}\toprule
\multicolumn{2}{l}{\large \textbf{ \hypertarget{prprsrv}{Prompt Processing} } }
& \multicolumn{2}{l}{(product in: Prompt Services
)}
\\ \hline
\textbf{\footnotesize Manager} & \textbf{\footnotesize Owner} &
\textbf{\footnotesize WBS} & \textbf{\footnotesize Team} \\ \hline
\parbox{3.5cm}{
Michelle Butler
\vspace{2mm}%
} &
\begin{tabular}{@{}l@{}}
\parbox{3.5cm}{
Eric Bellm
\vspace{2mm}%
} \\
\end{tabular} &
\begin{tabular}{@{}l@{}}
 \\
\end{tabular} & \begin{tabular}{@{}l@{}}
 \\
\end{tabular} \\ \hline
\multicolumn{4}{c}{
{\footnotesize ( Prmpt Processing
 - PRPRSRV ) }
}\\ \hline
\end{longtable}

(from LDM-148)

This service receives crosstalk-corrected images and metadata from the
Prompt Processing Ingest service at the Base and executes the Alert
Production science payload on them, generating ``online'' data products
that are stored in the Data Backbone. The Alert Production payload then
sends alerts to the Alert Distribution service.

~

The Prompt Processing service has calibration (including Collimated Beam
Projector images), science, and deep drilling modes. In calibration
mode, it executes a Raw Calibration Validation payload that provides
rapid feedback of raw calibration image quality. In normal science mode,
two consecutive exposures are grouped and processed as a single visit.
Definitions of exposure groupings to be processed as visits in deep
drilling and other modes are TBD. The service is required to deliver
Alerts within 60 seconds of the final camera readout of a standard
science visit with 98\% availability.

~

There is no Prompt Processing service instance for the Auxiliary
Telescope Spectrograph.

~


\begin{longtable}{p{3.7cm}p{3.7cm}p{3.7cm}p{3.7cm}}\hline
\textbf{\footnotesize Uses:}  & & \textbf{\footnotesize Used in:} & \\ \hline
\multicolumn{2}{c}{
\begin{tabular}{c}
\hyperlink{apprmpt}{Alert Production} \\ \hline
\hyperlink{dmcal}{Calibration SW} \\ \hline
\hyperlink{mops}{MOPS and Forced Photometry} \\ \hline
\end{tabular}
} &
\multicolumn{2}{c}{
\begin{tabular}{c}
\hyperlink{encprn}{Prompt NCSA Enclave} \\ \hline
\end{tabular}
} \\ \bottomrule
\multicolumn{4}{c}{\textbf{Related Requirements} } \\ \hline
{\footnotesize CA-DM-CON-ICD-0007 } &
\multicolumn{3}{p{11.1cm}}{\footnotesize Provide Data Management Conditions data } \\ \cdashline{1-4}
{\footnotesize CA-DM-CON-ICD-0008 } &
\multicolumn{3}{p{11.1cm}}{\footnotesize Data Management Conditions data latency } \\ \cdashline{1-4}
{\footnotesize DM-TS-CON-ICD-0002 } &
\multicolumn{3}{p{11.1cm}}{\footnotesize Timing } \\ \cdashline{1-4}
{\footnotesize DM-TS-CON-ICD-0006 } &
\multicolumn{3}{p{11.1cm}}{\footnotesize Data } \\ \cdashline{1-4}
{\footnotesize DM-TS-CON-ICD-0007 } &
\multicolumn{3}{p{11.1cm}}{\footnotesize Timing } \\ \cdashline{1-4}
{\footnotesize DM-TS-CON-ICD-0011 } &
\multicolumn{3}{p{11.1cm}}{\footnotesize Data Format } \\ \cdashline{1-4}
{\footnotesize DMS-REQ-0002 } &
\multicolumn{3}{p{11.1cm}}{\footnotesize Transient Alert Distribution } \\ \cdashline{1-4}
{\footnotesize DMS-REQ-0004 } &
\multicolumn{3}{p{11.1cm}}{\footnotesize Nightly Data Accessible Within 24 hrs } \\ \cdashline{1-4}
{\footnotesize DMS-REQ-0008 } &
\multicolumn{3}{p{11.1cm}}{\footnotesize Pipeline Availability } \\ \cdashline{1-4}
{\footnotesize DMS-REQ-0099 } &
\multicolumn{3}{p{11.1cm}}{\footnotesize Level 1 Performance Report Definition } \\ \cdashline{1-4}
{\footnotesize DMS-REQ-0131 } &
\multicolumn{3}{p{11.1cm}}{\footnotesize Calibration Images Available Within Specified Time } \\ \cdashline{1-4}
{\footnotesize DMS-REQ-0162 } &
\multicolumn{3}{p{11.1cm}}{\footnotesize Pipeline Throughput } \\ \cdashline{1-4}
{\footnotesize DMS-REQ-0167 } &
\multicolumn{3}{p{11.1cm}}{\footnotesize Incorporate Autonomics } \\ \cdashline{1-4}
{\footnotesize DMS-REQ-0284 } &
\multicolumn{3}{p{11.1cm}}{\footnotesize Level-1 Production Completeness } \\ \cdashline{1-4}
{\footnotesize DMS-REQ-0294 } &
\multicolumn{3}{p{11.1cm}}{\footnotesize Processing of Datasets } \\ \cdashline{1-4}
{\footnotesize DMS-REQ-0301 } &
\multicolumn{3}{p{11.1cm}}{\footnotesize Control of Level-1 Production } \\ \cdashline{1-4}
{\footnotesize DMS-REQ-0312 } &
\multicolumn{3}{p{11.1cm}}{\footnotesize Level 1 Data Product Access } \\ \cdashline{1-4}
{\footnotesize DMS-REQ-0314 } &
\multicolumn{3}{p{11.1cm}}{\footnotesize Compute Platform Heterogeneity } \\ \cdashline{1-4}
{\footnotesize DMS-REQ-0318 } &
\multicolumn{3}{p{11.1cm}}{\footnotesize Data Management Unscheduled Downtime } \\ \cdashline{1-4}
{\footnotesize DMS-REQ-0321 } &
\multicolumn{3}{p{11.1cm}}{\footnotesize Level 1 Processing of Special Programs Data } \\ \cdashline{1-4}
{\footnotesize DMS-REQ-0344 } &
\multicolumn{3}{p{11.1cm}}{\footnotesize Constraints on Level 1 Special Program Products Generation } \\ \cdashline{1-4}
{\footnotesize DMS-REQ-0346 } &
\multicolumn{3}{p{11.1cm}}{\footnotesize Data Availability } \\ \cdashline{1-4}
{\footnotesize OCS-DM-COM-ICD-0007 } &
\multicolumn{3}{p{11.1cm}}{\footnotesize Prompt Processing CSC } \\ \cdashline{1-4}
{\footnotesize OCS-DM-COM-ICD-0048 } &
\multicolumn{3}{p{11.1cm}}{\footnotesize Alert Production Complete Event } \\ \cdashline{1-4}
{\footnotesize OCS-DM-COM-ICD-0049 } &
\multicolumn{3}{p{11.1cm}}{\footnotesize WCS Information } \\ \cdashline{1-4}
{\footnotesize OCS-DM-COM-ICD-0050 } &
\multicolumn{3}{p{11.1cm}}{\footnotesize PSF Information } \\ \cdashline{1-4}
{\footnotesize OCS-DM-COM-ICD-0051 } &
\multicolumn{3}{p{11.1cm}}{\footnotesize Photometric Zeropoint Information } \\ \cdashline{1-4}
{\footnotesize OCS-DM-COM-ICD-0052 } &
\multicolumn{3}{p{11.1cm}}{\footnotesize Number of Alerts Information } \\ \cdashline{1-4}
{\footnotesize OCS-DM-COM-ICD-0056 } &
\multicolumn{3}{p{11.1cm}}{\footnotesize Prompt Processing Resource Availability } \\ \cdashline{1-4}
\end{longtable}

     \newpage
\begin{longtable}{p{3.7cm}p{3.7cm}p{3.7cm}p{3.7cm}}\toprule
\multicolumn{2}{l}{\large \textbf{ \hypertarget{oodssrv}{Observatory Operations Data} } }
& \multicolumn{2}{l}{(product in: Prompt Services
)}
\\ \hline
\textbf{\footnotesize Manager} & \textbf{\footnotesize Owner} &
\textbf{\footnotesize WBS} & \textbf{\footnotesize Team} \\ \hline
\parbox{3.5cm}{
Michelle Butler
\vspace{2mm}%
} &
\begin{tabular}{@{}l@{}}
\parbox{3.5cm}{
Steve Pietrowicz
\vspace{2mm}%
} \\
\end{tabular} &
\begin{tabular}{@{}l@{}}
1.02C.07.06.02 \\
\end{tabular} & \begin{tabular}{@{}l@{}}
LDF \\
\end{tabular} \\ \hline
\multicolumn{4}{c}{
{\footnotesize ( Obs Ops Data
 - OODSSRV ) }
}\\ \hline
\end{longtable}

Observatory Operations Data Service


\begin{longtable}{p{3.7cm}p{3.7cm}p{3.7cm}p{3.7cm}}\hline
\textbf{\footnotesize Uses:}  & & \textbf{\footnotesize Used in:} & \\ \hline
\multicolumn{2}{c}{
\begin{tabular}{c}
\hyperlink{oods}{Observatory Operations Data Service SW} \\ \hline
\end{tabular}
} &
\multicolumn{2}{c}{
\begin{tabular}{c}
\hyperlink{encprb}{Prompt Base Enclave} \\ \hline
\end{tabular}
} \\ \bottomrule
\multicolumn{4}{c}{\textbf{Related Requirements} } \\ \hline
{\footnotesize CA-DM-DAQ-ICD-0052 } &
\multicolumn{3}{p{11.1cm}}{\footnotesize Correction constants for science sensors sourced by Data Management } \\ \cdashline{1-4}
{\footnotesize DM-TS-CON-ICD-0003 } &
\multicolumn{3}{p{11.1cm}}{\footnotesize Wavefront image archive access } \\ \cdashline{1-4}
{\footnotesize DM-TS-CON-ICD-0009 } &
\multicolumn{3}{p{11.1cm}}{\footnotesize Calibration Data Products } \\ \cdashline{1-4}
{\footnotesize DMS-REQ-0167 } &
\multicolumn{3}{p{11.1cm}}{\footnotesize Incorporate Autonomics } \\ \cdashline{1-4}
{\footnotesize DMS-REQ-0314 } &
\multicolumn{3}{p{11.1cm}}{\footnotesize Compute Platform Heterogeneity } \\ \cdashline{1-4}
{\footnotesize DMS-REQ-0318 } &
\multicolumn{3}{p{11.1cm}}{\footnotesize Data Management Unscheduled Downtime } \\ \cdashline{1-4}
{\footnotesize DMS-REQ-0346 } &
\multicolumn{3}{p{11.1cm}}{\footnotesize Data Availability } \\ \cdashline{1-4}
\end{longtable}

     \newpage
\begin{longtable}{p{3.7cm}p{3.7cm}p{3.7cm}p{3.7cm}}\toprule
\multicolumn{2}{l}{\large \textbf{ \hypertarget{ocsbatsrv}{OCS-Driven Batch} } }
& \multicolumn{2}{l}{(product in: Prompt Services
)}
\\ \hline
\textbf{\footnotesize Manager} & \textbf{\footnotesize Owner} &
\textbf{\footnotesize WBS} & \textbf{\footnotesize Team} \\ \hline
\parbox{3.5cm}{
Michelle Butler
\vspace{2mm}%
} &
\begin{tabular}{@{}l@{}}
\parbox{3.5cm}{
Felipe Menanteau
\vspace{2mm}%
} \\
\end{tabular} &
\begin{tabular}{@{}l@{}}
1.02C.07.06.02 \\
\end{tabular} & \begin{tabular}{@{}l@{}}
LDF \\
\end{tabular} \\ \hline
\multicolumn{4}{c}{
{\footnotesize ( OCS Batch
 - OCSBATSRV ) }
}\\ \hline
\citeds{DMTN-133} &
\multicolumn{3}{l}{ OCS driven data processing }
\\ \cdashline{1-4}
\end{longtable}

OCS Driven Batch Processing Service


\begin{longtable}{p{3.7cm}p{3.7cm}p{3.7cm}p{3.7cm}}\hline
\textbf{\footnotesize Uses:}  & & \textbf{\footnotesize Used in:} & \\ \hline
\multicolumn{2}{c}{
\begin{tabular}{c}
\hyperlink{ocsbat}{OCS Batch SW} \\ \hline
\end{tabular}
} &
\multicolumn{2}{c}{
\begin{tabular}{c}
\hyperlink{encprb}{Prompt Base Enclave} \\ \hline
\end{tabular}
} \\ \bottomrule
\multicolumn{4}{c}{\textbf{Related Requirements} } \\ \hline
{\footnotesize DMS-REQ-0008 } &
\multicolumn{3}{p{11.1cm}}{\footnotesize Pipeline Availability } \\ \cdashline{1-4}
{\footnotesize DMS-REQ-0101 } &
\multicolumn{3}{p{11.1cm}}{\footnotesize Level 1 Calibration Report Definition } \\ \cdashline{1-4}
{\footnotesize DMS-REQ-0131 } &
\multicolumn{3}{p{11.1cm}}{\footnotesize Calibration Images Available Within Specified Time } \\ \cdashline{1-4}
{\footnotesize DMS-REQ-0162 } &
\multicolumn{3}{p{11.1cm}}{\footnotesize Pipeline Throughput } \\ \cdashline{1-4}
{\footnotesize DMS-REQ-0265 } &
\multicolumn{3}{p{11.1cm}}{\footnotesize Guider Calibration Data Acquisition } \\ \cdashline{1-4}
{\footnotesize DMS-REQ-0289 } &
\multicolumn{3}{p{11.1cm}}{\footnotesize Calibration Production Processing } \\ \cdashline{1-4}
{\footnotesize DMS-REQ-0294 } &
\multicolumn{3}{p{11.1cm}}{\footnotesize Processing of Datasets } \\ \cdashline{1-4}
{\footnotesize DMS-REQ-0301 } &
\multicolumn{3}{p{11.1cm}}{\footnotesize Control of Level-1 Production } \\ \cdashline{1-4}
{\footnotesize DMS-REQ-0314 } &
\multicolumn{3}{p{11.1cm}}{\footnotesize Compute Platform Heterogeneity } \\ \cdashline{1-4}
{\footnotesize DMS-REQ-0315 } &
\multicolumn{3}{p{11.1cm}}{\footnotesize DMS Communication with OCS } \\ \cdashline{1-4}
{\footnotesize DMS-REQ-0318 } &
\multicolumn{3}{p{11.1cm}}{\footnotesize Data Management Unscheduled Downtime } \\ \cdashline{1-4}
{\footnotesize OCS-DM-COM-ICD-0003 } &
\multicolumn{3}{p{11.1cm}}{\footnotesize Data Management CSC Command Response Model } \\ \cdashline{1-4}
{\footnotesize OCS-DM-COM-ICD-0009 } &
\multicolumn{3}{p{11.1cm}}{\footnotesize Command Set Implementation by Data Management } \\ \cdashline{1-4}
{\footnotesize OCS-DM-COM-ICD-0012 } &
\multicolumn{3}{p{11.1cm}}{\footnotesize Start Command } \\ \cdashline{1-4}
{\footnotesize OCS-DM-COM-ICD-0013 } &
\multicolumn{3}{p{11.1cm}}{\footnotesize configure Successful Completion Response } \\ \cdashline{1-4}
{\footnotesize OCS-DM-COM-ICD-0014 } &
\multicolumn{3}{p{11.1cm}}{\footnotesize enable Command } \\ \cdashline{1-4}
{\footnotesize OCS-DM-COM-ICD-0015 } &
\multicolumn{3}{p{11.1cm}}{\footnotesize disable Command } \\ \cdashline{1-4}
{\footnotesize OCS-DM-COM-ICD-0035 } &
\multicolumn{3}{p{11.1cm}}{\footnotesize OCS-Driven Batch CSC } \\ \cdashline{1-4}
{\footnotesize OCS-DM-COM-ICD-0036 } &
\multicolumn{3}{p{11.1cm}}{\footnotesize standby Command } \\ \cdashline{1-4}
{\footnotesize OCS-DM-COM-ICD-0037 } &
\multicolumn{3}{p{11.1cm}}{\footnotesize exit Command } \\ \cdashline{1-4}
{\footnotesize OCS-DM-COM-ICD-0038 } &
\multicolumn{3}{p{11.1cm}}{\footnotesize enterControl Command } \\ \cdashline{1-4}
{\footnotesize OCS-DM-COM-ICD-0039 } &
\multicolumn{3}{p{11.1cm}}{\footnotesize enterControl Successful Completion Response } \\ \cdashline{1-4}
{\footnotesize OCS-DM-COM-ICD-0040 } &
\multicolumn{3}{p{11.1cm}}{\footnotesize Command Completion Response } \\ \cdashline{1-4}
\end{longtable}

     \newpage
\begin{longtable}{p{3.7cm}p{3.7cm}p{3.7cm}p{3.7cm}}\toprule
\multicolumn{2}{l}{\large \textbf{ \hypertarget{tmgsrv}{Telemetry Gateway} } }
& \multicolumn{2}{l}{(product in: Prompt Services
)}
\\ \hline
\textbf{\footnotesize Manager} & \textbf{\footnotesize Owner} &
\textbf{\footnotesize WBS} & \textbf{\footnotesize Team} \\ \hline
\parbox{3.5cm}{
Michelle Butler
\vspace{2mm}%
} &
\begin{tabular}{@{}l@{}}
\parbox{3.5cm}{
Kian-Tat Lim
\vspace{2mm}%
} \\
\end{tabular} &
\begin{tabular}{@{}l@{}}
1.02C.07.06.02 \\
\end{tabular} & \begin{tabular}{@{}l@{}}
LDF \\
\end{tabular} \\ \hline
\multicolumn{4}{c}{
{\footnotesize ( Telem Gateway
 - TMGSRV ) }
}\\ \hline
\end{longtable}

This service is implemented by the DMCS, a more general system that
manages communication from and to the Observatory. The DMCS is developed
as part of the Image Ingest and Processing software product.


\begin{longtable}{p{3.7cm}p{3.7cm}p{3.7cm}p{3.7cm}}\hline
\textbf{\footnotesize Uses:}  & & \textbf{\footnotesize Used in:} & \\ \hline
\multicolumn{2}{c}{
\begin{tabular}{c}
\hyperlink{iip}{Image Ingest and Processing} \\ \hline
\end{tabular}
} &
\multicolumn{2}{c}{
\begin{tabular}{c}
\hyperlink{encprb}{Prompt Base Enclave} \\ \hline
\end{tabular}
} \\ \bottomrule
\multicolumn{4}{c}{\textbf{Related Requirements} } \\ \hline
{\footnotesize CA-DM-CON-ICD-0007 } &
\multicolumn{3}{p{11.1cm}}{\footnotesize Provide Data Management Conditions data } \\ \cdashline{1-4}
{\footnotesize CA-DM-CON-ICD-0008 } &
\multicolumn{3}{p{11.1cm}}{\footnotesize Data Management Conditions data latency } \\ \cdashline{1-4}
{\footnotesize DM-TS-CON-ICD-0002 } &
\multicolumn{3}{p{11.1cm}}{\footnotesize Timing } \\ \cdashline{1-4}
{\footnotesize DM-TS-CON-ICD-0004 } &
\multicolumn{3}{p{11.1cm}}{\footnotesize Use OCS for data transport } \\ \cdashline{1-4}
{\footnotesize DM-TS-CON-ICD-0006 } &
\multicolumn{3}{p{11.1cm}}{\footnotesize Data } \\ \cdashline{1-4}
{\footnotesize DM-TS-CON-ICD-0007 } &
\multicolumn{3}{p{11.1cm}}{\footnotesize Timing } \\ \cdashline{1-4}
{\footnotesize DM-TS-CON-ICD-0011 } &
\multicolumn{3}{p{11.1cm}}{\footnotesize Data Format } \\ \cdashline{1-4}
{\footnotesize DMS-REQ-0008 } &
\multicolumn{3}{p{11.1cm}}{\footnotesize Pipeline Availability } \\ \cdashline{1-4}
{\footnotesize DMS-REQ-0314 } &
\multicolumn{3}{p{11.1cm}}{\footnotesize Compute Platform Heterogeneity } \\ \cdashline{1-4}
{\footnotesize DMS-REQ-0315 } &
\multicolumn{3}{p{11.1cm}}{\footnotesize DMS Communication with OCS } \\ \cdashline{1-4}
{\footnotesize DMS-REQ-0318 } &
\multicolumn{3}{p{11.1cm}}{\footnotesize Data Management Unscheduled Downtime } \\ \cdashline{1-4}
{\footnotesize OCS-DM-COM-ICD-0017 } &
\multicolumn{3}{p{11.1cm}}{\footnotesize Data Management Telemetry Interface Model } \\ \cdashline{1-4}
{\footnotesize OCS-DM-COM-ICD-0018 } &
\multicolumn{3}{p{11.1cm}}{\footnotesize Data Management Telemetry Time Stamp } \\ \cdashline{1-4}
{\footnotesize OCS-DM-COM-ICD-0019 } &
\multicolumn{3}{p{11.1cm}}{\footnotesize Data Management Events and Telemetry Required by the OCS } \\ \cdashline{1-4}
{\footnotesize OCS-DM-COM-ICD-0020 } &
\multicolumn{3}{p{11.1cm}}{\footnotesize Image and Visit Processing and Archiving Status } \\ \cdashline{1-4}
{\footnotesize OCS-DM-COM-ICD-0021 } &
\multicolumn{3}{p{11.1cm}}{\footnotesize Data Quality Metrics } \\ \cdashline{1-4}
{\footnotesize OCS-DM-COM-ICD-0022 } &
\multicolumn{3}{p{11.1cm}}{\footnotesize System Health Metrics } \\ \cdashline{1-4}
{\footnotesize OCS-DM-COM-ICD-0043 } &
\multicolumn{3}{p{11.1cm}}{\footnotesize Image Retrieval for Archiving Event } \\ \cdashline{1-4}
{\footnotesize OCS-DM-COM-ICD-0044 } &
\multicolumn{3}{p{11.1cm}}{\footnotesize Image Retrieval For Processing Event } \\ \cdashline{1-4}
{\footnotesize OCS-DM-COM-ICD-0045 } &
\multicolumn{3}{p{11.1cm}}{\footnotesize Image in OODS Event } \\ \cdashline{1-4}
{\footnotesize OCS-DM-COM-ICD-0046 } &
\multicolumn{3}{p{11.1cm}}{\footnotesize Image Forwarded Event } \\ \cdashline{1-4}
{\footnotesize OCS-DM-COM-ICD-0047 } &
\multicolumn{3}{p{11.1cm}}{\footnotesize Image Archived Event } \\ \cdashline{1-4}
{\footnotesize OCS-DM-COM-ICD-0048 } &
\multicolumn{3}{p{11.1cm}}{\footnotesize Alert Production Complete Event } \\ \cdashline{1-4}
{\footnotesize OCS-DM-COM-ICD-0049 } &
\multicolumn{3}{p{11.1cm}}{\footnotesize WCS Information } \\ \cdashline{1-4}
{\footnotesize OCS-DM-COM-ICD-0050 } &
\multicolumn{3}{p{11.1cm}}{\footnotesize PSF Information } \\ \cdashline{1-4}
{\footnotesize OCS-DM-COM-ICD-0051 } &
\multicolumn{3}{p{11.1cm}}{\footnotesize Photometric Zeropoint Information } \\ \cdashline{1-4}
{\footnotesize OCS-DM-COM-ICD-0052 } &
\multicolumn{3}{p{11.1cm}}{\footnotesize Number of Alerts Information } \\ \cdashline{1-4}
{\footnotesize OCS-DM-COM-ICD-0053 } &
\multicolumn{3}{p{11.1cm}}{\footnotesize Summit-Base Network Utilization } \\ \cdashline{1-4}
{\footnotesize OCS-DM-COM-ICD-0054 } &
\multicolumn{3}{p{11.1cm}}{\footnotesize Base-Archive Network Utilization } \\ \cdashline{1-4}
{\footnotesize OCS-DM-COM-ICD-0055 } &
\multicolumn{3}{p{11.1cm}}{\footnotesize Archiver Resource Availability } \\ \cdashline{1-4}
{\footnotesize OCS-DM-COM-ICD-0056 } &
\multicolumn{3}{p{11.1cm}}{\footnotesize Prompt Processing Resource Availability } \\ \cdashline{1-4}
\end{longtable}

     \newpage
\begin{longtable}{p{3.7cm}p{3.7cm}p{3.7cm}p{3.7cm}}\toprule
\multicolumn{2}{l}{\large \textbf{ \hypertarget{alrtdstsrv}{Alert Distribution} } }
& \multicolumn{2}{l}{(product in: Prompt Services
)}
\\ \hline
\textbf{\footnotesize Manager} & \textbf{\footnotesize Owner} &
\textbf{\footnotesize WBS} & \textbf{\footnotesize Team} \\ \hline
\parbox{3.5cm}{
John Swinbank
\vspace{2mm}%
} &
\begin{tabular}{@{}l@{}}
\parbox{3.5cm}{
Eric Bellm
\vspace{2mm}%
} \\
\end{tabular} &
\begin{tabular}{@{}l@{}}
1.02C.03.03 \\
\end{tabular} & \begin{tabular}{@{}l@{}}
AP \\
\end{tabular} \\ \hline
\multicolumn{4}{c}{
{\footnotesize ( Alert Distrib
 - ALRTDSTSRV ) }
}\\ \hline
\end{longtable}

Alert Distribution and Filtering Service


\begin{longtable}{p{3.7cm}p{3.7cm}p{3.7cm}p{3.7cm}}\hline
\textbf{\footnotesize Uses:}  & & \textbf{\footnotesize Used in:} & \\ \hline
\multicolumn{2}{c}{
\begin{tabular}{c}
\hyperlink{alrtdstr}{Alert Distribution SW} \\ \hline
\end{tabular}
} &
\multicolumn{2}{c}{
\begin{tabular}{c}
\hyperlink{encprn}{Prompt NCSA Enclave} \\ \hline
\end{tabular}
} \\ \bottomrule
\multicolumn{4}{c}{\textbf{Related Requirements} } \\ \hline
{\footnotesize DMS-REQ-0002 } &
\multicolumn{3}{p{11.1cm}}{\footnotesize Transient Alert Distribution } \\ \cdashline{1-4}
{\footnotesize DMS-REQ-0004 } &
\multicolumn{3}{p{11.1cm}}{\footnotesize Nightly Data Accessible Within 24 hrs } \\ \cdashline{1-4}
{\footnotesize DMS-REQ-0008 } &
\multicolumn{3}{p{11.1cm}}{\footnotesize Pipeline Availability } \\ \cdashline{1-4}
{\footnotesize DMS-REQ-0161 } &
\multicolumn{3}{p{11.1cm}}{\footnotesize Optimization of Cost, Reliability and Availability in Order } \\ \cdashline{1-4}
{\footnotesize DMS-REQ-0162 } &
\multicolumn{3}{p{11.1cm}}{\footnotesize Pipeline Throughput } \\ \cdashline{1-4}
{\footnotesize DMS-REQ-0167 } &
\multicolumn{3}{p{11.1cm}}{\footnotesize Incorporate Autonomics } \\ \cdashline{1-4}
{\footnotesize DMS-REQ-0314 } &
\multicolumn{3}{p{11.1cm}}{\footnotesize Compute Platform Heterogeneity } \\ \cdashline{1-4}
{\footnotesize DMS-REQ-0318 } &
\multicolumn{3}{p{11.1cm}}{\footnotesize Data Management Unscheduled Downtime } \\ \cdashline{1-4}
{\footnotesize DMS-REQ-0342 } &
\multicolumn{3}{p{11.1cm}}{\footnotesize Alert Filtering Service } \\ \cdashline{1-4}
{\footnotesize DMS-REQ-0343 } &
\multicolumn{3}{p{11.1cm}}{\footnotesize Performance Requirements for LSST Alert Filtering Service } \\ \cdashline{1-4}
{\footnotesize DMS-REQ-0348 } &
\multicolumn{3}{p{11.1cm}}{\footnotesize Pre-defined alert filters } \\ \cdashline{1-4}
\end{longtable}

     \newpage
\begin{longtable}{p{3.7cm}p{3.7cm}p{3.7cm}p{3.7cm}}\toprule
\multicolumn{2}{l}{\large \textbf{ \hypertarget{prqcsrv}{Prompt Quality Control} } }
& \multicolumn{2}{l}{(product in: Prompt Services
)}
\\ \hline
\textbf{\footnotesize Manager} & \textbf{\footnotesize Owner} &
\textbf{\footnotesize WBS} & \textbf{\footnotesize Team} \\ \hline
\parbox{3.5cm}{
John Swinbank
\vspace{2mm}%
} &
\begin{tabular}{@{}l@{}}
\parbox{3.5cm}{
Eric Bellm
\vspace{2mm}%
} \\
\end{tabular} &
\begin{tabular}{@{}l@{}}
1.02C.03.08 \\
\end{tabular} & \begin{tabular}{@{}l@{}}
AP \\
\end{tabular} \\ \hline
\multicolumn{4}{c}{
{\footnotesize ( Prompt QC
 - PRQCSRV ) }
}\\ \hline
\end{longtable}

This service collects all KPI metrics calculated by the prompt
processing service, aggregate the information and provides reports in
order to ensure the monitoring of the quality of the processing and of
the data collected by the survey.


\begin{longtable}{p{3.7cm}p{3.7cm}p{3.7cm}p{3.7cm}}\hline
\textbf{\footnotesize Uses:}  & & \textbf{\footnotesize Used in:} & \\ \hline
\multicolumn{2}{c}{
\begin{tabular}{c}
\hyperlink{qcsw}{Quality Control SW} \\ \hline
\end{tabular}
} &
\multicolumn{2}{c}{
\begin{tabular}{c}
\hyperlink{encprn}{Prompt NCSA Enclave} \\ \hline
\end{tabular}
} \\ \bottomrule
\multicolumn{4}{c}{\textbf{Related Requirements} } \\ \hline
{\footnotesize DMS-REQ-0096 } &
\multicolumn{3}{p{11.1cm}}{\footnotesize Generate Data Quality Report Within Specified Time } \\ \cdashline{1-4}
{\footnotesize DMS-REQ-0097 } &
\multicolumn{3}{p{11.1cm}}{\footnotesize Level 1 Data Quality Report Definition } \\ \cdashline{1-4}
{\footnotesize DMS-REQ-0098 } &
\multicolumn{3}{p{11.1cm}}{\footnotesize Generate DMS Performance Report Within Specified Time } \\ \cdashline{1-4}
{\footnotesize DMS-REQ-0099 } &
\multicolumn{3}{p{11.1cm}}{\footnotesize Level 1 Performance Report Definition } \\ \cdashline{1-4}
{\footnotesize DMS-REQ-0100 } &
\multicolumn{3}{p{11.1cm}}{\footnotesize Generate Calibration Report Within Specified Time } \\ \cdashline{1-4}
{\footnotesize DMS-REQ-0101 } &
\multicolumn{3}{p{11.1cm}}{\footnotesize Level 1 Calibration Report Definition } \\ \cdashline{1-4}
{\footnotesize DMS-REQ-0314 } &
\multicolumn{3}{p{11.1cm}}{\footnotesize Compute Platform Heterogeneity } \\ \cdashline{1-4}
{\footnotesize DMS-REQ-0318 } &
\multicolumn{3}{p{11.1cm}}{\footnotesize Data Management Unscheduled Downtime } \\ \cdashline{1-4}
\end{longtable}

  \newpage
\paragraph{Archiving Services}\label{arcsrvs}\mbox{}\\
\begin{longtable}{p{3.7cm}p{3.7cm}p{3.7cm}p{3.7cm}}\hline
\textbf{Manager} & \textbf{Owner} & \textbf{WBS} & \textbf{Team} \\ \hline
\parbox{3.5cm}{
Michelle Butler
\vspace{2mm}%
} &
\begin{tabular}{@{}l@{}}
\parbox{3.5cm}{

\vspace{2mm}%
} \\
\end{tabular}
 &
\begin{tabular}{@{}l@{}}
 \\
\end{tabular} &
\begin{tabular}{@{}l@{}}
LDF \\
\end{tabular} \\ \hline
\multicolumn{4}{c}{
{\footnotesize ( Archive Services
 - ARCSRVS ) }
}\\ \hline
\end{longtable}

  Services involved in the acquisition of the data.


  \newpage
\begin{longtable}{p{3.7cm}p{3.7cm}p{3.7cm}p{3.7cm}}\toprule
\multicolumn{2}{l}{\large \textbf{ \hypertarget{imas}{Image Archiver} } }
& \multicolumn{2}{l}{(product in: Archive Services
)}
\\ \hline
\textbf{\footnotesize Manager} & \textbf{\footnotesize Owner} &
\textbf{\footnotesize WBS} & \textbf{\footnotesize Team} \\ \hline
\parbox{3.5cm}{
Michelle Butler
\vspace{2mm}%
} &
\begin{tabular}{@{}l@{}}
\parbox{3.5cm}{
Steve Pietrowicz
\vspace{2mm}%
} \\
\end{tabular} &
\begin{tabular}{@{}l@{}}
1.02C.07.06.02 \\
\end{tabular} & \begin{tabular}{@{}l@{}}
LDF \\
\end{tabular} \\ \hline
\multicolumn{4}{c}{
{\footnotesize ( Image Archiver
 - IMAS ) }
}\\ \hline
\end{longtable}

This service capture raw images taken by a camera, retrieving them from
the corresponding Camera Data System instance.\\[2\baselineskip]The
image pixels and metadata are passed to the Observatory Operations Data
Service (OODS), which serves as a buffer from which observing-critical
data can be retrieved. They are also passed to a staging area for
ingestion into the permanent archive in the Data Backbone.


\begin{longtable}{p{3.7cm}p{3.7cm}p{3.7cm}p{3.7cm}}\hline
\textbf{\footnotesize Uses:}  & & \textbf{\footnotesize Used in:} & \\ \hline
\multicolumn{2}{c}{
\begin{tabular}{c}
\hyperlink{iip}{Image Ingest and Processing} \\ \hline
\end{tabular}
} &
\multicolumn{2}{c}{
\begin{tabular}{c}
\hyperlink{encprb}{Prompt Base Enclave} \\ \hline
\end{tabular}
} \\ \bottomrule
\multicolumn{4}{c}{\textbf{Related Requirements} } \\ \hline
\end{longtable}

    \newpage
\begin{longtable}{p{3.7cm}p{3.7cm}p{3.7cm}p{3.7cm}}\toprule
\multicolumn{2}{l}{\large \textbf{ \hypertarget{heads}{Header Generator} } }
& \multicolumn{2}{l}{(product in: Archive Services
)}
\\ \hline
\textbf{\footnotesize Manager} & \textbf{\footnotesize Owner} &
\textbf{\footnotesize WBS} & \textbf{\footnotesize Team} \\ \hline
\parbox{3.5cm}{
Michelle Butler
\vspace{2mm}%
} &
\begin{tabular}{@{}l@{}}
\parbox{3.5cm}{
Steve Pietrowicz
\vspace{2mm}%
} \\
\end{tabular} &
\begin{tabular}{@{}l@{}}
1.02C.07.06.02 \\
\end{tabular} & \begin{tabular}{@{}l@{}}
LDF \\
\end{tabular} \\ \hline
\multicolumn{4}{c}{
{\footnotesize ( Header Generator
 - HEADS ) }
}\\ \hline
\end{longtable}

The Header Generator service, written by Data Management but operated by
the Observatory, captures specific sets of metadata associated with the
images, including telemetry values and event timings, from the OCS
publish/subscribe middleware and/or from the EFD. It formats these into
a metadata package that is recorded in the EFD Large File Annex. The
Archiver and CatchUp Archiver instances retrieve this metadata package
and attach it to the captured image pixels.\\


\begin{longtable}{p{3.7cm}p{3.7cm}p{3.7cm}p{3.7cm}}\hline
\textbf{\footnotesize Uses:}  & & \textbf{\footnotesize Used in:} & \\ \hline
\multicolumn{2}{c}{
\begin{tabular}{c}
\hyperlink{header}{Header Service SW} \\ \hline
\end{tabular}
} &
\multicolumn{2}{c}{
\begin{tabular}{c}
\hyperlink{encprb}{Prompt Base Enclave} \\ \hline
\end{tabular}
} \\ \bottomrule
\multicolumn{4}{c}{\textbf{Related Requirements} } \\ \hline
\end{longtable}

    \newpage
\begin{longtable}{p{3.7cm}p{3.7cm}p{3.7cm}p{3.7cm}}\toprule
\multicolumn{2}{l}{\large \textbf{ \hypertarget{efdts}{EFD Transformation} } }
& \multicolumn{2}{l}{(product in: Archive Services
)}
\\ \hline
\textbf{\footnotesize Manager} & \textbf{\footnotesize Owner} &
\textbf{\footnotesize WBS} & \textbf{\footnotesize Team} \\ \hline
\parbox{3.5cm}{
Michelle Butler
\vspace{2mm}%
} &
\begin{tabular}{@{}l@{}}
\parbox{3.5cm}{
Steve Pietrowicz
\vspace{2mm}%
} \\
\end{tabular} &
\begin{tabular}{@{}l@{}}
1.02C.07.06.02 \\
\end{tabular} & \begin{tabular}{@{}l@{}}
LDF \\
\end{tabular} \\ \hline
\multicolumn{4}{c}{
{\footnotesize ( EFD Transf.
 - EFDTS ) }
}\\ \hline
\end{longtable}

The ~EFD Transformation service extracts all information (including
telemetry, events, configurations, and commands) from the EFD and its
large file annex, transforms it into a form more suitable for querying
by image timestamp, and loads it into the permanently archived
``Transformed EFD'' database in the Data Backbone.


\begin{longtable}{p{3.7cm}p{3.7cm}p{3.7cm}p{3.7cm}}\hline
\textbf{\footnotesize Uses:}  & & \textbf{\footnotesize Used in:} & \\ \hline
\multicolumn{2}{c}{
\begin{tabular}{c}
\hyperlink{efdt}{EFD Transformation} \\ \hline
\end{tabular}
} &
\multicolumn{2}{c}{
\begin{tabular}{c}
\hyperlink{encprb}{Prompt Base Enclave} \\ \hline
\end{tabular}
} \\ \bottomrule
\multicolumn{4}{c}{\textbf{Related Requirements} } \\ \hline
\end{longtable}

    
     \newpage
\begin{longtable}{p{3.7cm}p{3.7cm}p{3.7cm}p{3.7cm}}\toprule
\multicolumn{2}{l}{\large \textbf{ \hypertarget{arcsrvo}{Archiving [Obsolete]} } }
& \multicolumn{2}{l}{(product in: Prompt Services
)}
\\ \hline
\textbf{\footnotesize Manager} & \textbf{\footnotesize Owner} &
\textbf{\footnotesize WBS} & \textbf{\footnotesize Team} \\ \hline
\parbox{3.5cm}{
Michelle Butler
\vspace{2mm}%
} &
\begin{tabular}{@{}l@{}}
\parbox{3.5cm}{
Steve Pietrowicz
\vspace{2mm}%
} \\
\end{tabular} &
\begin{tabular}{@{}l@{}}
1.02C.07.06.02 \\
\end{tabular} & \begin{tabular}{@{}l@{}}
LDF \\
\end{tabular} \\ \hline
\multicolumn{4}{c}{
{\footnotesize ( Archiving
 - ARCSRVo ) }
}\\ \hline
\end{longtable}

OBSOLETE: component split in 3 Archiving services.


\begin{longtable}{p{3.7cm}p{3.7cm}p{3.7cm}p{3.7cm}}\hline
\textbf{\footnotesize Uses:}  & & \textbf{\footnotesize Used in:} & \\ \hline
\multicolumn{2}{c}{
} &
\multicolumn{2}{c}{
\begin{tabular}{c}
\hyperlink{encprb}{Prompt Base Enclave} \\ \hline
\end{tabular}
} \\ \bottomrule
\multicolumn{4}{c}{\textbf{Related Requirements} } \\ \hline
{\footnotesize CA-DM-CON-ICD-0014 } &
\multicolumn{3}{p{11.1cm}}{\footnotesize Provide science sensor data } \\ \cdashline{1-4}
{\footnotesize CA-DM-CON-ICD-0015 } &
\multicolumn{3}{p{11.1cm}}{\footnotesize Provide wavefront sensor data } \\ \cdashline{1-4}
{\footnotesize CA-DM-CON-ICD-0016 } &
\multicolumn{3}{p{11.1cm}}{\footnotesize Provide guide sensor data } \\ \cdashline{1-4}
{\footnotesize CA-DM-CON-ICD-0017 } &
\multicolumn{3}{p{11.1cm}}{\footnotesize Data Management load on image data interfaces } \\ \cdashline{1-4}
{\footnotesize CA-DM-CON-ICD-0019 } &
\multicolumn{3}{p{11.1cm}}{\footnotesize Camera engineering image data archiving } \\ \cdashline{1-4}
{\footnotesize DMS-REQ-0008 } &
\multicolumn{3}{p{11.1cm}}{\footnotesize Pipeline Availability } \\ \cdashline{1-4}
{\footnotesize DMS-REQ-0018 } &
\multicolumn{3}{p{11.1cm}}{\footnotesize Raw Science Image Data Acquisition } \\ \cdashline{1-4}
{\footnotesize DMS-REQ-0020 } &
\multicolumn{3}{p{11.1cm}}{\footnotesize Wavefront Sensor Data Acquisition } \\ \cdashline{1-4}
{\footnotesize DMS-REQ-0024 } &
\multicolumn{3}{p{11.1cm}}{\footnotesize Raw Image Assembly } \\ \cdashline{1-4}
{\footnotesize DMS-REQ-0068 } &
\multicolumn{3}{p{11.1cm}}{\footnotesize Raw Science Image Metadata } \\ \cdashline{1-4}
{\footnotesize DMS-REQ-0099 } &
\multicolumn{3}{p{11.1cm}}{\footnotesize Level 1 Performance Report Definition } \\ \cdashline{1-4}
{\footnotesize DMS-REQ-0102 } &
\multicolumn{3}{p{11.1cm}}{\footnotesize Provide Engineering \&  Facility Database Archive } \\ \cdashline{1-4}
{\footnotesize DMS-REQ-0162 } &
\multicolumn{3}{p{11.1cm}}{\footnotesize Pipeline Throughput } \\ \cdashline{1-4}
{\footnotesize DMS-REQ-0165 } &
\multicolumn{3}{p{11.1cm}}{\footnotesize Infrastructure Sizing for "catching up" } \\ \cdashline{1-4}
{\footnotesize DMS-REQ-0167 } &
\multicolumn{3}{p{11.1cm}}{\footnotesize Incorporate Autonomics } \\ \cdashline{1-4}
{\footnotesize DMS-REQ-0265 } &
\multicolumn{3}{p{11.1cm}}{\footnotesize Guider Calibration Data Acquisition } \\ \cdashline{1-4}
{\footnotesize DMS-REQ-0301 } &
\multicolumn{3}{p{11.1cm}}{\footnotesize Control of Level-1 Production } \\ \cdashline{1-4}
{\footnotesize DMS-REQ-0309 } &
\multicolumn{3}{p{11.1cm}}{\footnotesize Raw Data Archiving Reliability } \\ \cdashline{1-4}
{\footnotesize DMS-REQ-0314 } &
\multicolumn{3}{p{11.1cm}}{\footnotesize Compute Platform Heterogeneity } \\ \cdashline{1-4}
{\footnotesize DMS-REQ-0315 } &
\multicolumn{3}{p{11.1cm}}{\footnotesize DMS Communication with OCS } \\ \cdashline{1-4}
{\footnotesize DMS-REQ-0318 } &
\multicolumn{3}{p{11.1cm}}{\footnotesize Data Management Unscheduled Downtime } \\ \cdashline{1-4}
{\footnotesize DMS-REQ-0346 } &
\multicolumn{3}{p{11.1cm}}{\footnotesize Data Availability } \\ \cdashline{1-4}
{\footnotesize OCS-DM-COM-ICD-0003 } &
\multicolumn{3}{p{11.1cm}}{\footnotesize Data Management CSC Command Response Model } \\ \cdashline{1-4}
{\footnotesize OCS-DM-COM-ICD-0004 } &
\multicolumn{3}{p{11.1cm}}{\footnotesize Data Management Exposed CSCs } \\ \cdashline{1-4}
{\footnotesize OCS-DM-COM-ICD-0005 } &
\multicolumn{3}{p{11.1cm}}{\footnotesize Main Camera Archiver } \\ \cdashline{1-4}
{\footnotesize OCS-DM-COM-ICD-0006 } &
\multicolumn{3}{p{11.1cm}}{\footnotesize Catch-up Archiver } \\ \cdashline{1-4}
{\footnotesize OCS-DM-COM-ICD-0008 } &
\multicolumn{3}{p{11.1cm}}{\footnotesize EFD Transformation Service CSC } \\ \cdashline{1-4}
{\footnotesize OCS-DM-COM-ICD-0009 } &
\multicolumn{3}{p{11.1cm}}{\footnotesize Command Set Implementation by Data Management } \\ \cdashline{1-4}
{\footnotesize OCS-DM-COM-ICD-0012 } &
\multicolumn{3}{p{11.1cm}}{\footnotesize Start Command } \\ \cdashline{1-4}
{\footnotesize OCS-DM-COM-ICD-0013 } &
\multicolumn{3}{p{11.1cm}}{\footnotesize configure Successful Completion Response } \\ \cdashline{1-4}
{\footnotesize OCS-DM-COM-ICD-0014 } &
\multicolumn{3}{p{11.1cm}}{\footnotesize enable Command } \\ \cdashline{1-4}
{\footnotesize OCS-DM-COM-ICD-0015 } &
\multicolumn{3}{p{11.1cm}}{\footnotesize disable Command } \\ \cdashline{1-4}
{\footnotesize OCS-DM-COM-ICD-0025 } &
\multicolumn{3}{p{11.1cm}}{\footnotesize Expected Load of Queries from DM } \\ \cdashline{1-4}
{\footnotesize OCS-DM-COM-ICD-0026 } &
\multicolumn{3}{p{11.1cm}}{\footnotesize Engineering and Facilities Database Archiving by Data Management } \\ \cdashline{1-4}
{\footnotesize OCS-DM-COM-ICD-0027 } &
\multicolumn{3}{p{11.1cm}}{\footnotesize Multiple Physically Separated Copies } \\ \cdashline{1-4}
{\footnotesize OCS-DM-COM-ICD-0028 } &
\multicolumn{3}{p{11.1cm}}{\footnotesize Expected Data Volume } \\ \cdashline{1-4}
{\footnotesize OCS-DM-COM-ICD-0029 } &
\multicolumn{3}{p{11.1cm}}{\footnotesize Archive Latency } \\ \cdashline{1-4}
{\footnotesize OCS-DM-COM-ICD-0030 } &
\multicolumn{3}{p{11.1cm}}{\footnotesize EFD Transformation Service Interface } \\ \cdashline{1-4}
{\footnotesize OCS-DM-COM-ICD-0032 } &
\multicolumn{3}{p{11.1cm}}{\footnotesize Auxiliary Telescope Archiver CSC } \\ \cdashline{1-4}
{\footnotesize OCS-DM-COM-ICD-0033 } &
\multicolumn{3}{p{11.1cm}}{\footnotesize Header Service CSC } \\ \cdashline{1-4}
{\footnotesize OCS-DM-COM-ICD-0034 } &
\multicolumn{3}{p{11.1cm}}{\footnotesize Auxiliary Header Service CSC } \\ \cdashline{1-4}
{\footnotesize OCS-DM-COM-ICD-0036 } &
\multicolumn{3}{p{11.1cm}}{\footnotesize standby Command } \\ \cdashline{1-4}
{\footnotesize OCS-DM-COM-ICD-0037 } &
\multicolumn{3}{p{11.1cm}}{\footnotesize exit Command } \\ \cdashline{1-4}
{\footnotesize OCS-DM-COM-ICD-0038 } &
\multicolumn{3}{p{11.1cm}}{\footnotesize enterControl Command } \\ \cdashline{1-4}
{\footnotesize OCS-DM-COM-ICD-0039 } &
\multicolumn{3}{p{11.1cm}}{\footnotesize enterControl Successful Completion Response } \\ \cdashline{1-4}
{\footnotesize OCS-DM-COM-ICD-0040 } &
\multicolumn{3}{p{11.1cm}}{\footnotesize Command Completion Response } \\ \cdashline{1-4}
{\footnotesize OCS-DM-COM-ICD-0041 } &
\multicolumn{3}{p{11.1cm}}{\footnotesize Large File Annex Replication Interface } \\ \cdashline{1-4}
{\footnotesize OCS-DM-COM-ICD-0042 } &
\multicolumn{3}{p{11.1cm}}{\footnotesize EFD Disaster Recovery by Data Management } \\ \cdashline{1-4}
{\footnotesize OCS-DM-COM-ICD-0043 } &
\multicolumn{3}{p{11.1cm}}{\footnotesize Image Retrieval for Archiving Event } \\ \cdashline{1-4}
{\footnotesize OCS-DM-COM-ICD-0044 } &
\multicolumn{3}{p{11.1cm}}{\footnotesize Image Retrieval For Processing Event } \\ \cdashline{1-4}
{\footnotesize OCS-DM-COM-ICD-0045 } &
\multicolumn{3}{p{11.1cm}}{\footnotesize Image in OODS Event } \\ \cdashline{1-4}
{\footnotesize OCS-DM-COM-ICD-0055 } &
\multicolumn{3}{p{11.1cm}}{\footnotesize Archiver Resource Availability } \\ \cdashline{1-4}
{\footnotesize OCS-EFD-HS-0001 } &
\multicolumn{3}{p{11.1cm}}{\footnotesize Fulfill requirements of a Commandable SAL Component (CSC) } \\ \cdashline{1-4}
{\footnotesize OCS-EFD-HS-0002 } &
\multicolumn{3}{p{11.1cm}}{\footnotesize Critical System } \\ \cdashline{1-4}
{\footnotesize OCS-EFD-HS-0003 } &
\multicolumn{3}{p{11.1cm}}{\footnotesize Write Headers for all images taken by all Cameras supported by LSST } \\ \cdashline{1-4}
{\footnotesize OCS-EFD-HS-0004 } &
\multicolumn{3}{p{11.1cm}}{\footnotesize Ability to capture metadata at the beginning of exposure } \\ \cdashline{1-4}
{\footnotesize OCS-EFD-HS-0005 } &
\multicolumn{3}{p{11.1cm}}{\footnotesize Ability to capture metadata during of exposure integration } \\ \cdashline{1-4}
{\footnotesize OCS-EFD-HS-0006 } &
\multicolumn{3}{p{11.1cm}}{\footnotesize Ability to capture metadata at end of readout } \\ \cdashline{1-4}
{\footnotesize OCS-EFD-HS-0007 } &
\multicolumn{3}{p{11.1cm}}{\footnotesize Write header and Publish Event after end of telemetry event } \\ \cdashline{1-4}
{\footnotesize OCS-EFD-HS-0008 } &
\multicolumn{3}{p{11.1cm}}{\footnotesize Write header and Publish Event within specified time of the end-of-telemetry Event } \\ \cdashline{1-4}
{\footnotesize OCS-EFD-HS-0009 } &
\multicolumn{3}{p{11.1cm}}{\footnotesize Adherence to the FITS Standard } \\ \cdashline{1-4}
{\footnotesize OCS-EFD-HS-0010 } &
\multicolumn{3}{p{11.1cm}}{\footnotesize Configuration of Header Keywords and source } \\ \cdashline{1-4}
{\footnotesize OCS-EFD-HS-0011 } &
\multicolumn{3}{p{11.1cm}}{\footnotesize Produce header even if some meta-data not avaiable } \\ \cdashline{1-4}
{\footnotesize OCS-EFD-HS-0012 } &
\multicolumn{3}{p{11.1cm}}{\footnotesize Publish an Event if monitoring detects any failure of the service. } \\ \cdashline{1-4}
{\footnotesize OCS-EFD-HS-0013 } &
\multicolumn{3}{p{11.1cm}}{\footnotesize Extract metadata from published configuration } \\ \cdashline{1-4}
{\footnotesize OCS-EFD-HS-0014 } &
\multicolumn{3}{p{11.1cm}}{\footnotesize Metadata Capture } \\ \cdashline{1-4}
{\footnotesize OCS-EFD-HS-0015 } &
\multicolumn{3}{p{11.1cm}}{\footnotesize Generate on-the-fly additional metadata as approved by the Project CCB. } \\ \cdashline{1-4}
\end{longtable}

    
   \newpage
\subsubsection{Offline Services}\label{offlsrv}
\begin{longtable}{p{3.7cm}p{3.7cm}p{3.7cm}p{3.7cm}}\hline
\textbf{Manager} & \textbf{Owner} & \textbf{WBS} & \textbf{Team} \\ \hline
\parbox{3.5cm}{

\vspace{2mm}%
} &
\begin{tabular}{@{}l@{}}
\parbox{3.5cm}{

\vspace{2mm}%
} \\
\end{tabular}
 &
\begin{tabular}{@{}l@{}}
 \\
\end{tabular} &
\begin{tabular}{@{}l@{}}
 \\
\end{tabular} \\ \hline
\multicolumn{4}{c}{
{\footnotesize ( Offline Services
 - OFFLSRV ) }
}\\ \hline
\citeds{LDM-534} &
\multicolumn{3}{l}{ LSST Level 2 System Test Specification }
\\ \cdashline{1-4}
\citeds{LDM-562} &
\multicolumn{3}{l}{ Data Management System (DMS) Level 2 System Requirements }
\\ \cdashline{1-4}
\end{longtable}

  DM services that do not need to run in a prompt manner.


   \newpage
\begin{longtable}{p{3.7cm}p{3.7cm}p{3.7cm}p{3.7cm}}\toprule
\multicolumn{2}{l}{\large \textbf{ \hypertarget{prodsrv}{Batch Production} } }
& \multicolumn{2}{l}{(product in: Offline Services
)}
\\ \hline
\textbf{\footnotesize Manager} & \textbf{\footnotesize Owner} &
\textbf{\footnotesize WBS} & \textbf{\footnotesize Team} \\ \hline
\parbox{3.5cm}{
Michelle Butler
\vspace{2mm}%
} &
\begin{tabular}{@{}l@{}}
\parbox{3.5cm}{
Michelle Gower
\vspace{2mm}%
} \\
\end{tabular} &
\begin{tabular}{@{}l@{}}
1.02C.07.06.02 \\
\end{tabular} & \begin{tabular}{@{}l@{}}
LDF \\
\end{tabular} \\ \hline
\multicolumn{4}{c}{
{\footnotesize ( Batch Production
 - PRODSRV ) }
}\\ \hline
\end{longtable}

Batch Production Service


\begin{longtable}{p{3.7cm}p{3.7cm}p{3.7cm}p{3.7cm}}\hline
\textbf{\footnotesize Uses:}  & & \textbf{\footnotesize Used in:} & \\ \hline
\multicolumn{2}{c}{
\begin{tabular}{c}
\hyperlink{wlwf}{Workload/ Workflow Management} \\ \hline
\hyperlink{htcondor}{HTCondor} \\ \hline
\hyperlink{dmcal}{Calibration SW} \\ \hline
\hyperlink{drp}{Data Release Production} \\ \hline
\hyperlink{mops}{MOPS and Forced Photometry} \\ \hline
\hyperlink{sp}{Special Programs Productions} \\ \hline
\end{tabular}
} &
\multicolumn{2}{c}{
\begin{tabular}{c}
\hyperlink{encoffl}{Offline Production Enclave} \\ \hline
\end{tabular}
} \\ \bottomrule
\multicolumn{4}{c}{\textbf{Related Requirements} } \\ \hline
{\footnotesize DMS-REQ-0008 } &
\multicolumn{3}{p{11.1cm}}{\footnotesize Pipeline Availability } \\ \cdashline{1-4}
{\footnotesize DMS-REQ-0156 } &
\multicolumn{3}{p{11.1cm}}{\footnotesize Provide Pipeline Execution Services } \\ \cdashline{1-4}
{\footnotesize DMS-REQ-0163 } &
\multicolumn{3}{p{11.1cm}}{\footnotesize Re-processing Capacity } \\ \cdashline{1-4}
{\footnotesize DMS-REQ-0167 } &
\multicolumn{3}{p{11.1cm}}{\footnotesize Incorporate Autonomics } \\ \cdashline{1-4}
{\footnotesize DMS-REQ-0294 } &
\multicolumn{3}{p{11.1cm}}{\footnotesize Processing of Datasets } \\ \cdashline{1-4}
{\footnotesize DMS-REQ-0302 } &
\multicolumn{3}{p{11.1cm}}{\footnotesize Production Orchestration } \\ \cdashline{1-4}
{\footnotesize DMS-REQ-0303 } &
\multicolumn{3}{p{11.1cm}}{\footnotesize Production Monitoring } \\ \cdashline{1-4}
{\footnotesize DMS-REQ-0304 } &
\multicolumn{3}{p{11.1cm}}{\footnotesize Production Fault Tolerance } \\ \cdashline{1-4}
{\footnotesize DMS-REQ-0307 } &
\multicolumn{3}{p{11.1cm}}{\footnotesize Unique Processing Coverage } \\ \cdashline{1-4}
{\footnotesize DMS-REQ-0308 } &
\multicolumn{3}{p{11.1cm}}{\footnotesize Software Architecture to Enable Community Re-Use } \\ \cdashline{1-4}
{\footnotesize DMS-REQ-0314 } &
\multicolumn{3}{p{11.1cm}}{\footnotesize Compute Platform Heterogeneity } \\ \cdashline{1-4}
{\footnotesize DMS-REQ-0318 } &
\multicolumn{3}{p{11.1cm}}{\footnotesize Data Management Unscheduled Downtime } \\ \cdashline{1-4}
{\footnotesize DMS-REQ-0388 } &
\multicolumn{3}{p{11.1cm}}{\footnotesize Provide Re-Run Tools } \\ \cdashline{1-4}
{\footnotesize DMS-REQ-0389 } &
\multicolumn{3}{p{11.1cm}}{\footnotesize Re-Runs on Similar Systems } \\ \cdashline{1-4}
{\footnotesize DMS-REQ-0390 } &
\multicolumn{3}{p{11.1cm}}{\footnotesize Re-Runs on Other Systems } \\ \cdashline{1-4}
\end{longtable}

     \newpage
\begin{longtable}{p{3.7cm}p{3.7cm}p{3.7cm}p{3.7cm}}\toprule
\multicolumn{2}{l}{\large \textbf{ \hypertarget{offlqcsrv}{Offline Quality Control} } }
& \multicolumn{2}{l}{(product in: Offline Services
)}
\\ \hline
\textbf{\footnotesize Manager} & \textbf{\footnotesize Owner} &
\textbf{\footnotesize WBS} & \textbf{\footnotesize Team} \\ \hline
\parbox{3.5cm}{
Yusra AlSayyad
\vspace{2mm}%
} &
\begin{tabular}{@{}l@{}}
\parbox{3.5cm}{
Jim Bosch
\vspace{2mm}%
} \\
\end{tabular} &
\begin{tabular}{@{}l@{}}
1.02C.04.07 \\
\end{tabular} & \begin{tabular}{@{}l@{}}
DRP \\
\end{tabular} \\ \hline
\multicolumn{4}{c}{
{\footnotesize ( Offline QC
 - OFFLQCSRV ) }
}\\ \hline
\end{longtable}

This service collects all KPI metrics calculated by the offline
processing activities, aggregate the information and provides reports in
order to ensure the monitoring of the quality of the processing and of
the data collected by the survey.


\begin{longtable}{p{3.7cm}p{3.7cm}p{3.7cm}p{3.7cm}}\hline
\textbf{\footnotesize Uses:}  & & \textbf{\footnotesize Used in:} & \\ \hline
\multicolumn{2}{c}{
\begin{tabular}{c}
\hyperlink{qcsw}{Quality Control SW} \\ \hline
\end{tabular}
} &
\multicolumn{2}{c}{
\begin{tabular}{c}
\hyperlink{encprn}{Prompt NCSA Enclave} \\ \hline
\hyperlink{encoffl}{Offline Production Enclave} \\ \hline
\end{tabular}
} \\ \bottomrule
\multicolumn{4}{c}{\textbf{Related Requirements} } \\ \hline
\end{longtable}

     \newpage
\begin{longtable}{p{3.7cm}p{3.7cm}p{3.7cm}p{3.7cm}}\toprule
\multicolumn{2}{l}{\large \textbf{ \hypertarget{bulkdsrv}{Bulk Distribution} } }
& \multicolumn{2}{l}{(product in: Offline Services
)}
\\ \hline
\textbf{\footnotesize Manager} & \textbf{\footnotesize Owner} &
\textbf{\footnotesize WBS} & \textbf{\footnotesize Team} \\ \hline
\parbox{3.5cm}{
Michelle Butler
\vspace{2mm}%
} &
\begin{tabular}{@{}l@{}}
\parbox{3.5cm}{
Michelle Gower
\vspace{2mm}%
} \\
\end{tabular} &
\begin{tabular}{@{}l@{}}
1.02C.07.06.02 \\
\end{tabular} & \begin{tabular}{@{}l@{}}
LDF \\
\end{tabular} \\ \hline
\multicolumn{4}{c}{
{\footnotesize ( Bulk Distrib
 - BULKDSRV ) }
}\\ \hline
\end{longtable}

Bulk Data Distribution Service


\begin{longtable}{p{3.7cm}p{3.7cm}p{3.7cm}p{3.7cm}}\hline
\textbf{\footnotesize Uses:}  & & \textbf{\footnotesize Used in:} & \\ \hline
\multicolumn{2}{c}{
\begin{tabular}{c}
\hyperlink{rucio}{Rucio} \\ \hline
\end{tabular}
} &
\multicolumn{2}{c}{
\begin{tabular}{c}
\hyperlink{encoffl}{Offline Production Enclave} \\ \hline
\end{tabular}
} \\ \bottomrule
\multicolumn{4}{c}{\textbf{Related Requirements} } \\ \hline
{\footnotesize DMS-REQ-0122 } &
\multicolumn{3}{p{11.1cm}}{\footnotesize Access to catalogs for external Level 3 processing } \\ \cdashline{1-4}
{\footnotesize DMS-REQ-0123 } &
\multicolumn{3}{p{11.1cm}}{\footnotesize Access to input catalogs for DAC-based Level 3 processing } \\ \cdashline{1-4}
{\footnotesize DMS-REQ-0126 } &
\multicolumn{3}{p{11.1cm}}{\footnotesize Access to images for external Level 3 processing } \\ \cdashline{1-4}
{\footnotesize DMS-REQ-0127 } &
\multicolumn{3}{p{11.1cm}}{\footnotesize Access to input images for DAC-based Level 3 processing } \\ \cdashline{1-4}
{\footnotesize DMS-REQ-0185 } &
\multicolumn{3}{p{11.1cm}}{\footnotesize Archive Center } \\ \cdashline{1-4}
{\footnotesize DMS-REQ-0186 } &
\multicolumn{3}{p{11.1cm}}{\footnotesize Archive Center Disaster Recovery } \\ \cdashline{1-4}
{\footnotesize DMS-REQ-0193 } &
\multicolumn{3}{p{11.1cm}}{\footnotesize Data Access Centers } \\ \cdashline{1-4}
{\footnotesize DMS-REQ-0300 } &
\multicolumn{3}{p{11.1cm}}{\footnotesize Bulk Download Service } \\ \cdashline{1-4}
{\footnotesize DMS-REQ-0314 } &
\multicolumn{3}{p{11.1cm}}{\footnotesize Compute Platform Heterogeneity } \\ \cdashline{1-4}
{\footnotesize DMS-REQ-0318 } &
\multicolumn{3}{p{11.1cm}}{\footnotesize Data Management Unscheduled Downtime } \\ \cdashline{1-4}
{\footnotesize DMS-REQ-0346 } &
\multicolumn{3}{p{11.1cm}}{\footnotesize Data Availability } \\ \cdashline{1-4}
{\footnotesize EP-DM-CON-ICD-0001 } &
\multicolumn{3}{p{11.1cm}}{\footnotesize US DAC Provides EPO Interface } \\ \cdashline{1-4}
{\footnotesize EP-DM-CON-ICD-0004 } &
\multicolumn{3}{p{11.1cm}}{\footnotesize DM Transfer of Catalog Data to EPO } \\ \cdashline{1-4}
{\footnotesize EP-DM-CON-ICD-0009 } &
\multicolumn{3}{p{11.1cm}}{\footnotesize Catalog Format } \\ \cdashline{1-4}
{\footnotesize EP-DM-CON-ICD-0014 } &
\multicolumn{3}{p{11.1cm}}{\footnotesize Color Co-Add Image Format } \\ \cdashline{1-4}
{\footnotesize EP-DM-CON-ICD-0019 } &
\multicolumn{3}{p{11.1cm}}{\footnotesize DM to EPO Data Transfer Cadence } \\ \cdashline{1-4}
{\footnotesize EP-DM-CON-ICD-0021 } &
\multicolumn{3}{p{11.1cm}}{\footnotesize DM Generation of a Color Hierarchical Progressive Survey for EPO } \\ \cdashline{1-4}
{\footnotesize EP-DM-CON-ICD-0022 } &
\multicolumn{3}{p{11.1cm}}{\footnotesize Annual DM Transfer of UGRIZY+Panchromatic Co-Add Images to EPO } \\ \cdashline{1-4}
{\footnotesize EP-DM-CON-ICD-0023 } &
\multicolumn{3}{p{11.1cm}}{\footnotesize Nightly DM Transfer of Processed Visit Images (PVI)-Based Images to EPO } \\ \cdashline{1-4}
\end{longtable}

    
   \newpage
\subsubsection{Backbone Services}\label{dbbsrv}
\begin{longtable}{p{3.7cm}p{3.7cm}p{3.7cm}p{3.7cm}}\hline
\textbf{Manager} & \textbf{Owner} & \textbf{WBS} & \textbf{Team} \\ \hline
\parbox{3.5cm}{
Michelle Butler
\vspace{2mm}%
} &
\begin{tabular}{@{}l@{}}
\parbox{3.5cm}{
Michelle Gower
\vspace{2mm}%
} \\
\end{tabular}
 &
\begin{tabular}{@{}l@{}}
 \\
\end{tabular} &
\begin{tabular}{@{}l@{}}
 \\
\end{tabular} \\ \hline
\multicolumn{4}{c}{
{\footnotesize ( DBB Services
 - DBBSRV ) }
}\\ \hline
\end{longtable}

  The Backbone services supply data to other services.\\
They are implemented in the Archive enclaves.\\[2\baselineskip]Detailed
concepts of operations for each service can be found in \citeds{LDM-230}


   \newpage
\begin{longtable}{p{3.7cm}p{3.7cm}p{3.7cm}p{3.7cm}}\toprule
\multicolumn{2}{l}{\large \textbf{ \hypertarget{dbbmdsrv}{DBB Ingest/ Metadata Management} } }
& \multicolumn{2}{l}{(product in: DBB Services
)}
\\ \hline
\textbf{\footnotesize Manager} & \textbf{\footnotesize Owner} &
\textbf{\footnotesize WBS} & \textbf{\footnotesize Team} \\ \hline
\parbox{3.5cm}{
Michelle Butler
\vspace{2mm}%
} &
\begin{tabular}{@{}l@{}}
\parbox{3.5cm}{
Michelle Gower
\vspace{2mm}%
} \\
\end{tabular} &
\begin{tabular}{@{}l@{}}
1.02C.07.07 \\
\end{tabular} & \begin{tabular}{@{}l@{}}
LDF \\
\end{tabular} \\ \hline
\multicolumn{4}{c}{
{\footnotesize ( DBB Metadata
 - DBBMDSRV ) }
}\\ \hline
\end{longtable}

This service within the Data Backbone is responsible for maintaining and
providing access to the metadata describing the location,
characteristics, and provenance of the data products it manages. Part of
this service involves creating the appropriate metadata during ingest
when data from external sources is incorporated into the DBB. The Batch
Production services will generally create the necessary DBB metadata as
part of their operation, so only a minimal ingest process is needed for
internally-generated data products. Metadata is kept in a database that
is a superset of the registry required by the Data Butler, allowing the
Butler to directly access data within the DBB.


\begin{longtable}{p{3.7cm}p{3.7cm}p{3.7cm}p{3.7cm}}\hline
\textbf{\footnotesize Uses:}  & & \textbf{\footnotesize Used in:} & \\ \hline
\multicolumn{2}{c}{
\begin{tabular}{c}
\hyperlink{dbbmd}{DBB Ingest/ Metadata Management SW} \\ \hline
\end{tabular}
} &
\multicolumn{2}{c}{
\begin{tabular}{c}
\hyperlink{encarcb}{Archive Base Enclave} \\ \hline
\hyperlink{encarcn}{Archive NCSA Enclave} \\ \hline
\end{tabular}
} \\ \bottomrule
\multicolumn{4}{c}{\textbf{Related Requirements} } \\ \hline
{\footnotesize DMS-REQ-0008 } &
\multicolumn{3}{p{11.1cm}}{\footnotesize Pipeline Availability } \\ \cdashline{1-4}
{\footnotesize DMS-REQ-0068 } &
\multicolumn{3}{p{11.1cm}}{\footnotesize Raw Science Image Metadata } \\ \cdashline{1-4}
{\footnotesize DMS-REQ-0074 } &
\multicolumn{3}{p{11.1cm}}{\footnotesize Difference Exposure Attributes } \\ \cdashline{1-4}
{\footnotesize DMS-REQ-0077 } &
\multicolumn{3}{p{11.1cm}}{\footnotesize Maintain Archive Publicly Accessible } \\ \cdashline{1-4}
{\footnotesize DMS-REQ-0089 } &
\multicolumn{3}{p{11.1cm}}{\footnotesize Solar System Objects Available Within Specified Time } \\ \cdashline{1-4}
{\footnotesize DMS-REQ-0094 } &
\multicolumn{3}{p{11.1cm}}{\footnotesize Keep Historical Alert Archive } \\ \cdashline{1-4}
{\footnotesize DMS-REQ-0102 } &
\multicolumn{3}{p{11.1cm}}{\footnotesize Provide Engineering \&  Facility Database Archive } \\ \cdashline{1-4}
{\footnotesize DMS-REQ-0120 } &
\multicolumn{3}{p{11.1cm}}{\footnotesize Level 3 Data Product Self Consistency } \\ \cdashline{1-4}
{\footnotesize DMS-REQ-0122 } &
\multicolumn{3}{p{11.1cm}}{\footnotesize Access to catalogs for external Level 3 processing } \\ \cdashline{1-4}
{\footnotesize DMS-REQ-0123 } &
\multicolumn{3}{p{11.1cm}}{\footnotesize Access to input catalogs for DAC-based Level 3 processing } \\ \cdashline{1-4}
{\footnotesize DMS-REQ-0126 } &
\multicolumn{3}{p{11.1cm}}{\footnotesize Access to images for external Level 3 processing } \\ \cdashline{1-4}
{\footnotesize DMS-REQ-0127 } &
\multicolumn{3}{p{11.1cm}}{\footnotesize Access to input images for DAC-based Level 3 processing } \\ \cdashline{1-4}
{\footnotesize DMS-REQ-0130 } &
\multicolumn{3}{p{11.1cm}}{\footnotesize Calibration Data Products } \\ \cdashline{1-4}
{\footnotesize DMS-REQ-0131 } &
\multicolumn{3}{p{11.1cm}}{\footnotesize Calibration Images Available Within Specified Time } \\ \cdashline{1-4}
{\footnotesize DMS-REQ-0132 } &
\multicolumn{3}{p{11.1cm}}{\footnotesize Calibration Image Provenance } \\ \cdashline{1-4}
{\footnotesize DMS-REQ-0161 } &
\multicolumn{3}{p{11.1cm}}{\footnotesize Optimization of Cost, Reliability and Availability in Order } \\ \cdashline{1-4}
{\footnotesize DMS-REQ-0162 } &
\multicolumn{3}{p{11.1cm}}{\footnotesize Pipeline Throughput } \\ \cdashline{1-4}
{\footnotesize DMS-REQ-0163 } &
\multicolumn{3}{p{11.1cm}}{\footnotesize Re-processing Capacity } \\ \cdashline{1-4}
{\footnotesize DMS-REQ-0164 } &
\multicolumn{3}{p{11.1cm}}{\footnotesize Temporary Storage for Communications Links } \\ \cdashline{1-4}
{\footnotesize DMS-REQ-0165 } &
\multicolumn{3}{p{11.1cm}}{\footnotesize Infrastructure Sizing for "catching up" } \\ \cdashline{1-4}
{\footnotesize DMS-REQ-0166 } &
\multicolumn{3}{p{11.1cm}}{\footnotesize Incorporate Fault-Tolerance } \\ \cdashline{1-4}
{\footnotesize DMS-REQ-0167 } &
\multicolumn{3}{p{11.1cm}}{\footnotesize Incorporate Autonomics } \\ \cdashline{1-4}
{\footnotesize DMS-REQ-0176 } &
\multicolumn{3}{p{11.1cm}}{\footnotesize Base Facility Infrastructure } \\ \cdashline{1-4}
{\footnotesize DMS-REQ-0185 } &
\multicolumn{3}{p{11.1cm}}{\footnotesize Archive Center } \\ \cdashline{1-4}
{\footnotesize DMS-REQ-0186 } &
\multicolumn{3}{p{11.1cm}}{\footnotesize Archive Center Disaster Recovery } \\ \cdashline{1-4}
{\footnotesize DMS-REQ-0197 } &
\multicolumn{3}{p{11.1cm}}{\footnotesize No Limit on Data Access Centers } \\ \cdashline{1-4}
{\footnotesize DMS-REQ-0266 } &
\multicolumn{3}{p{11.1cm}}{\footnotesize Exposure Catalog } \\ \cdashline{1-4}
{\footnotesize DMS-REQ-0269 } &
\multicolumn{3}{p{11.1cm}}{\footnotesize DIASource Catalog } \\ \cdashline{1-4}
{\footnotesize DMS-REQ-0271 } &
\multicolumn{3}{p{11.1cm}}{\footnotesize DIAObject Catalog } \\ \cdashline{1-4}
{\footnotesize DMS-REQ-0273 } &
\multicolumn{3}{p{11.1cm}}{\footnotesize SSObject Catalog } \\ \cdashline{1-4}
{\footnotesize DMS-REQ-0287 } &
\multicolumn{3}{p{11.1cm}}{\footnotesize DIASource Precovery } \\ \cdashline{1-4}
{\footnotesize DMS-REQ-0291 } &
\multicolumn{3}{p{11.1cm}}{\footnotesize Query Repeatability } \\ \cdashline{1-4}
{\footnotesize DMS-REQ-0292 } &
\multicolumn{3}{p{11.1cm}}{\footnotesize Uniqueness of IDs Across Data Releases } \\ \cdashline{1-4}
{\footnotesize DMS-REQ-0293 } &
\multicolumn{3}{p{11.1cm}}{\footnotesize Selection of Datasets } \\ \cdashline{1-4}
{\footnotesize DMS-REQ-0299 } &
\multicolumn{3}{p{11.1cm}}{\footnotesize Data Product Ingest } \\ \cdashline{1-4}
{\footnotesize DMS-REQ-0309 } &
\multicolumn{3}{p{11.1cm}}{\footnotesize Raw Data Archiving Reliability } \\ \cdashline{1-4}
{\footnotesize DMS-REQ-0310 } &
\multicolumn{3}{p{11.1cm}}{\footnotesize Un-Archived Data Product Cache } \\ \cdashline{1-4}
{\footnotesize DMS-REQ-0313 } &
\multicolumn{3}{p{11.1cm}}{\footnotesize Level 1 \&  2 Catalog Access } \\ \cdashline{1-4}
{\footnotesize DMS-REQ-0314 } &
\multicolumn{3}{p{11.1cm}}{\footnotesize Compute Platform Heterogeneity } \\ \cdashline{1-4}
{\footnotesize DMS-REQ-0317 } &
\multicolumn{3}{p{11.1cm}}{\footnotesize DIAForcedSource Catalog } \\ \cdashline{1-4}
{\footnotesize DMS-REQ-0318 } &
\multicolumn{3}{p{11.1cm}}{\footnotesize Data Management Unscheduled Downtime } \\ \cdashline{1-4}
{\footnotesize DMS-REQ-0322 } &
\multicolumn{3}{p{11.1cm}}{\footnotesize Special Programs Database } \\ \cdashline{1-4}
{\footnotesize DMS-REQ-0334 } &
\multicolumn{3}{p{11.1cm}}{\footnotesize Persisting Data Products } \\ \cdashline{1-4}
{\footnotesize DMS-REQ-0338 } &
\multicolumn{3}{p{11.1cm}}{\footnotesize Targeted Coadds } \\ \cdashline{1-4}
{\footnotesize DMS-REQ-0339 } &
\multicolumn{3}{p{11.1cm}}{\footnotesize Tracking Characterization Changes Between Data Releases } \\ \cdashline{1-4}
{\footnotesize DMS-REQ-0346 } &
\multicolumn{3}{p{11.1cm}}{\footnotesize Data Availability } \\ \cdashline{1-4}
{\footnotesize DMS-REQ-0363 } &
\multicolumn{3}{p{11.1cm}}{\footnotesize Access to Previous Data Releases } \\ \cdashline{1-4}
{\footnotesize DMS-REQ-0364 } &
\multicolumn{3}{p{11.1cm}}{\footnotesize Data Access Services } \\ \cdashline{1-4}
{\footnotesize DMS-REQ-0365 } &
\multicolumn{3}{p{11.1cm}}{\footnotesize Operations Subsets } \\ \cdashline{1-4}
{\footnotesize DMS-REQ-0366 } &
\multicolumn{3}{p{11.1cm}}{\footnotesize Subsets Support } \\ \cdashline{1-4}
{\footnotesize DMS-REQ-0369 } &
\multicolumn{3}{p{11.1cm}}{\footnotesize Evolution } \\ \cdashline{1-4}
{\footnotesize DMS-REQ-0370 } &
\multicolumn{3}{p{11.1cm}}{\footnotesize Older Release Behavior } \\ \cdashline{1-4}
{\footnotesize DMS-REQ-0372 } &
\multicolumn{3}{p{11.1cm}}{\footnotesize a Archiving Camera Test Data } \\ \cdashline{1-4}
{\footnotesize DMS-REQ-0386 } &
\multicolumn{3}{p{11.1cm}}{\footnotesize a Archive Processing Provenance } \\ \cdashline{1-4}
{\footnotesize DMS-REQ-0387 } &
\multicolumn{3}{p{11.1cm}}{\footnotesize b Serve Archived Provenance } \\ \cdashline{1-4}
{\footnotesize DMS-REQ-0388 } &
\multicolumn{3}{p{11.1cm}}{\footnotesize Provide Re-Run Tools } \\ \cdashline{1-4}
{\footnotesize OCS-DM-COM-ICD-0047 } &
\multicolumn{3}{p{11.1cm}}{\footnotesize Image Archived Event } \\ \cdashline{1-4}
\end{longtable}

     \newpage
\begin{longtable}{p{3.7cm}p{3.7cm}p{3.7cm}p{3.7cm}}\toprule
\multicolumn{2}{l}{\large \textbf{ \hypertarget{dbblifesrv}{DBB Lifetime Management} } }
& \multicolumn{2}{l}{(product in: DBB Services
)}
\\ \hline
\textbf{\footnotesize Manager} & \textbf{\footnotesize Owner} &
\textbf{\footnotesize WBS} & \textbf{\footnotesize Team} \\ \hline
\parbox{3.5cm}{
Michelle Butler
\vspace{2mm}%
} &
\begin{tabular}{@{}l@{}}
\parbox{3.5cm}{
Michelle Gower
\vspace{2mm}%
} \\
\end{tabular} &
\begin{tabular}{@{}l@{}}
1.02C.07.07 \\
\end{tabular} & \begin{tabular}{@{}l@{}}
LDF \\
\end{tabular} \\ \hline
\multicolumn{4}{c}{
{\footnotesize ( DBB Lifetime
 - DBBLIFESRV ) }
}\\ \hline
\end{longtable}

This service is responsible for managing the lifetimes of data products
within the DBB based on a set of policies. Data products may move from
high-speed storage to near-line or offline storage or may be deleted
completely. Some products are kept permanently. Some are kept for
defined time periods as specified in requirements. Intermediate data
products may be kept until all downstream products have been generated.


\begin{longtable}{p{3.7cm}p{3.7cm}p{3.7cm}p{3.7cm}}\hline
\textbf{\footnotesize Uses:}  & & \textbf{\footnotesize Used in:} & \\ \hline
\multicolumn{2}{c}{
\begin{tabular}{c}
\hyperlink{dbblife}{DBB Lifetime Management SW} \\ \hline
\end{tabular}
} &
\multicolumn{2}{c}{
\begin{tabular}{c}
\hyperlink{encarcb}{Archive Base Enclave} \\ \hline
\hyperlink{encarcn}{Archive NCSA Enclave} \\ \hline
\end{tabular}
} \\ \bottomrule
\multicolumn{4}{c}{\textbf{Related Requirements} } \\ \hline
{\footnotesize DMS-REQ-0310 } &
\multicolumn{3}{p{11.1cm}}{\footnotesize Un-Archived Data Product Cache } \\ \cdashline{1-4}
{\footnotesize DMS-REQ-0334 } &
\multicolumn{3}{p{11.1cm}}{\footnotesize Persisting Data Products } \\ \cdashline{1-4}
{\footnotesize DMS-REQ-0338 } &
\multicolumn{3}{p{11.1cm}}{\footnotesize Targeted Coadds } \\ \cdashline{1-4}
{\footnotesize DMS-REQ-0339 } &
\multicolumn{3}{p{11.1cm}}{\footnotesize Tracking Characterization Changes Between Data Releases } \\ \cdashline{1-4}
{\footnotesize DMS-REQ-0346 } &
\multicolumn{3}{p{11.1cm}}{\footnotesize Data Availability } \\ \cdashline{1-4}
\end{longtable}

     \newpage
\begin{longtable}{p{3.7cm}p{3.7cm}p{3.7cm}p{3.7cm}}\toprule
\multicolumn{2}{l}{\large \textbf{ \hypertarget{dbbtrsrv}{DBB Transport/ Replication/ Backup} } }
& \multicolumn{2}{l}{(product in: DBB Services
)}
\\ \hline
\textbf{\footnotesize Manager} & \textbf{\footnotesize Owner} &
\textbf{\footnotesize WBS} & \textbf{\footnotesize Team} \\ \hline
\parbox{3.5cm}{
Michelle Butler
\vspace{2mm}%
} &
\begin{tabular}{@{}l@{}}
\parbox{3.5cm}{
Michelle Gower
\vspace{2mm}%
} \\
\end{tabular} &
\begin{tabular}{@{}l@{}}
1.02C.07.07 \\
\end{tabular} & \begin{tabular}{@{}l@{}}
LDF \\
\end{tabular} \\ \hline
\multicolumn{4}{c}{
{\footnotesize ( DBB Transport
 - DBBTRSRV ) }
}\\ \hline
\end{longtable}

This service is responsible for moving data products from one Facility
to another and to backup and disaster recovery storage. It handles
recovery if a data product is found to be missing or corrupt.


\begin{longtable}{p{3.7cm}p{3.7cm}p{3.7cm}p{3.7cm}}\hline
\textbf{\footnotesize Uses:}  & & \textbf{\footnotesize Used in:} & \\ \hline
\multicolumn{2}{c}{
\begin{tabular}{c}
\hyperlink{dbbtr}{DBB Transport/ Replication/ Backup SW} \\ \hline
\end{tabular}
} &
\multicolumn{2}{c}{
\begin{tabular}{c}
\hyperlink{encarcb}{Archive Base Enclave} \\ \hline
\hyperlink{encarcn}{Archive NCSA Enclave} \\ \hline
\end{tabular}
} \\ \bottomrule
\multicolumn{4}{c}{\textbf{Related Requirements} } \\ \hline
{\footnotesize DMS-REQ-0008 } &
\multicolumn{3}{p{11.1cm}}{\footnotesize Pipeline Availability } \\ \cdashline{1-4}
{\footnotesize DMS-REQ-0089 } &
\multicolumn{3}{p{11.1cm}}{\footnotesize Solar System Objects Available Within Specified Time } \\ \cdashline{1-4}
{\footnotesize DMS-REQ-0102 } &
\multicolumn{3}{p{11.1cm}}{\footnotesize Provide Engineering \&  Facility Database Archive } \\ \cdashline{1-4}
{\footnotesize DMS-REQ-0122 } &
\multicolumn{3}{p{11.1cm}}{\footnotesize Access to catalogs for external Level 3 processing } \\ \cdashline{1-4}
{\footnotesize DMS-REQ-0123 } &
\multicolumn{3}{p{11.1cm}}{\footnotesize Access to input catalogs for DAC-based Level 3 processing } \\ \cdashline{1-4}
{\footnotesize DMS-REQ-0126 } &
\multicolumn{3}{p{11.1cm}}{\footnotesize Access to images for external Level 3 processing } \\ \cdashline{1-4}
{\footnotesize DMS-REQ-0127 } &
\multicolumn{3}{p{11.1cm}}{\footnotesize Access to input images for DAC-based Level 3 processing } \\ \cdashline{1-4}
{\footnotesize DMS-REQ-0131 } &
\multicolumn{3}{p{11.1cm}}{\footnotesize Calibration Images Available Within Specified Time } \\ \cdashline{1-4}
{\footnotesize DMS-REQ-0161 } &
\multicolumn{3}{p{11.1cm}}{\footnotesize Optimization of Cost, Reliability and Availability in Order } \\ \cdashline{1-4}
{\footnotesize DMS-REQ-0162 } &
\multicolumn{3}{p{11.1cm}}{\footnotesize Pipeline Throughput } \\ \cdashline{1-4}
{\footnotesize DMS-REQ-0163 } &
\multicolumn{3}{p{11.1cm}}{\footnotesize Re-processing Capacity } \\ \cdashline{1-4}
{\footnotesize DMS-REQ-0164 } &
\multicolumn{3}{p{11.1cm}}{\footnotesize Temporary Storage for Communications Links } \\ \cdashline{1-4}
{\footnotesize DMS-REQ-0165 } &
\multicolumn{3}{p{11.1cm}}{\footnotesize Infrastructure Sizing for "catching up" } \\ \cdashline{1-4}
{\footnotesize DMS-REQ-0166 } &
\multicolumn{3}{p{11.1cm}}{\footnotesize Incorporate Fault-Tolerance } \\ \cdashline{1-4}
{\footnotesize DMS-REQ-0167 } &
\multicolumn{3}{p{11.1cm}}{\footnotesize Incorporate Autonomics } \\ \cdashline{1-4}
{\footnotesize DMS-REQ-0185 } &
\multicolumn{3}{p{11.1cm}}{\footnotesize Archive Center } \\ \cdashline{1-4}
{\footnotesize DMS-REQ-0186 } &
\multicolumn{3}{p{11.1cm}}{\footnotesize Archive Center Disaster Recovery } \\ \cdashline{1-4}
{\footnotesize DMS-REQ-0197 } &
\multicolumn{3}{p{11.1cm}}{\footnotesize No Limit on Data Access Centers } \\ \cdashline{1-4}
{\footnotesize DMS-REQ-0287 } &
\multicolumn{3}{p{11.1cm}}{\footnotesize DIASource Precovery } \\ \cdashline{1-4}
{\footnotesize DMS-REQ-0309 } &
\multicolumn{3}{p{11.1cm}}{\footnotesize Raw Data Archiving Reliability } \\ \cdashline{1-4}
{\footnotesize DMS-REQ-0313 } &
\multicolumn{3}{p{11.1cm}}{\footnotesize Level 1 \&  2 Catalog Access } \\ \cdashline{1-4}
{\footnotesize DMS-REQ-0314 } &
\multicolumn{3}{p{11.1cm}}{\footnotesize Compute Platform Heterogeneity } \\ \cdashline{1-4}
{\footnotesize DMS-REQ-0318 } &
\multicolumn{3}{p{11.1cm}}{\footnotesize Data Management Unscheduled Downtime } \\ \cdashline{1-4}
{\footnotesize DMS-REQ-0344 } &
\multicolumn{3}{p{11.1cm}}{\footnotesize Constraints on Level 1 Special Program Products Generation } \\ \cdashline{1-4}
{\footnotesize DMS-REQ-0366 } &
\multicolumn{3}{p{11.1cm}}{\footnotesize Subsets Support } \\ \cdashline{1-4}
{\footnotesize DMS-REQ-0370 } &
\multicolumn{3}{p{11.1cm}}{\footnotesize Older Release Behavior } \\ \cdashline{1-4}
\end{longtable}

     \newpage
\begin{longtable}{p{3.7cm}p{3.7cm}p{3.7cm}p{3.7cm}}\toprule
\multicolumn{2}{l}{\large \textbf{ \hypertarget{dbbstrsrv}{DBB Storage} } }
& \multicolumn{2}{l}{(product in: DBB Services
)}
\\ \hline
\textbf{\footnotesize Manager} & \textbf{\footnotesize Owner} &
\textbf{\footnotesize WBS} & \textbf{\footnotesize Team} \\ \hline
\parbox{3.5cm}{
Michelle Butler
\vspace{2mm}%
} &
\begin{tabular}{@{}l@{}}
\parbox{3.5cm}{
Michelle Gower
\vspace{2mm}%
} \\
\end{tabular} &
\begin{tabular}{@{}l@{}}
1.02C.07.07 \\
\end{tabular} & \begin{tabular}{@{}l@{}}
LDF \\
\end{tabular} \\ \hline
\multicolumn{4}{c}{
{\footnotesize ( DBB Storage
 - DBBSTRSRV ) }
}\\ \hline
\end{longtable}

This service is responsible for storage of data products in the DBB. The
storage service provides an interface usable by the Data Butler as a
datastore.


\begin{longtable}{p{3.7cm}p{3.7cm}p{3.7cm}p{3.7cm}}\hline
\textbf{\footnotesize Uses:}  & & \textbf{\footnotesize Used in:} & \\ \hline
\multicolumn{2}{c}{
\begin{tabular}{c}
\hyperlink{dbbtr}{DBB Transport/ Replication/ Backup SW} \\ \hline
\end{tabular}
} &
\multicolumn{2}{c}{
\begin{tabular}{c}
\hyperlink{encarcb}{Archive Base Enclave} \\ \hline
\hyperlink{encarcn}{Archive NCSA Enclave} \\ \hline
\end{tabular}
} \\ \bottomrule
\multicolumn{4}{c}{\textbf{Related Requirements} } \\ \hline
{\footnotesize DMS-REQ-0008 } &
\multicolumn{3}{p{11.1cm}}{\footnotesize Pipeline Availability } \\ \cdashline{1-4}
{\footnotesize DMS-REQ-0077 } &
\multicolumn{3}{p{11.1cm}}{\footnotesize Maintain Archive Publicly Accessible } \\ \cdashline{1-4}
{\footnotesize DMS-REQ-0089 } &
\multicolumn{3}{p{11.1cm}}{\footnotesize Solar System Objects Available Within Specified Time } \\ \cdashline{1-4}
{\footnotesize DMS-REQ-0094 } &
\multicolumn{3}{p{11.1cm}}{\footnotesize Keep Historical Alert Archive } \\ \cdashline{1-4}
{\footnotesize DMS-REQ-0102 } &
\multicolumn{3}{p{11.1cm}}{\footnotesize Provide Engineering \&  Facility Database Archive } \\ \cdashline{1-4}
{\footnotesize DMS-REQ-0122 } &
\multicolumn{3}{p{11.1cm}}{\footnotesize Access to catalogs for external Level 3 processing } \\ \cdashline{1-4}
{\footnotesize DMS-REQ-0123 } &
\multicolumn{3}{p{11.1cm}}{\footnotesize Access to input catalogs for DAC-based Level 3 processing } \\ \cdashline{1-4}
{\footnotesize DMS-REQ-0126 } &
\multicolumn{3}{p{11.1cm}}{\footnotesize Access to images for external Level 3 processing } \\ \cdashline{1-4}
{\footnotesize DMS-REQ-0127 } &
\multicolumn{3}{p{11.1cm}}{\footnotesize Access to input images for DAC-based Level 3 processing } \\ \cdashline{1-4}
{\footnotesize DMS-REQ-0130 } &
\multicolumn{3}{p{11.1cm}}{\footnotesize Calibration Data Products } \\ \cdashline{1-4}
{\footnotesize DMS-REQ-0131 } &
\multicolumn{3}{p{11.1cm}}{\footnotesize Calibration Images Available Within Specified Time } \\ \cdashline{1-4}
{\footnotesize DMS-REQ-0161 } &
\multicolumn{3}{p{11.1cm}}{\footnotesize Optimization of Cost, Reliability and Availability in Order } \\ \cdashline{1-4}
{\footnotesize DMS-REQ-0162 } &
\multicolumn{3}{p{11.1cm}}{\footnotesize Pipeline Throughput } \\ \cdashline{1-4}
{\footnotesize DMS-REQ-0163 } &
\multicolumn{3}{p{11.1cm}}{\footnotesize Re-processing Capacity } \\ \cdashline{1-4}
{\footnotesize DMS-REQ-0164 } &
\multicolumn{3}{p{11.1cm}}{\footnotesize Temporary Storage for Communications Links } \\ \cdashline{1-4}
{\footnotesize DMS-REQ-0165 } &
\multicolumn{3}{p{11.1cm}}{\footnotesize Infrastructure Sizing for "catching up" } \\ \cdashline{1-4}
{\footnotesize DMS-REQ-0166 } &
\multicolumn{3}{p{11.1cm}}{\footnotesize Incorporate Fault-Tolerance } \\ \cdashline{1-4}
{\footnotesize DMS-REQ-0167 } &
\multicolumn{3}{p{11.1cm}}{\footnotesize Incorporate Autonomics } \\ \cdashline{1-4}
{\footnotesize DMS-REQ-0185 } &
\multicolumn{3}{p{11.1cm}}{\footnotesize Archive Center } \\ \cdashline{1-4}
{\footnotesize DMS-REQ-0186 } &
\multicolumn{3}{p{11.1cm}}{\footnotesize Archive Center Disaster Recovery } \\ \cdashline{1-4}
{\footnotesize DMS-REQ-0197 } &
\multicolumn{3}{p{11.1cm}}{\footnotesize No Limit on Data Access Centers } \\ \cdashline{1-4}
{\footnotesize DMS-REQ-0287 } &
\multicolumn{3}{p{11.1cm}}{\footnotesize DIASource Precovery } \\ \cdashline{1-4}
{\footnotesize DMS-REQ-0309 } &
\multicolumn{3}{p{11.1cm}}{\footnotesize Raw Data Archiving Reliability } \\ \cdashline{1-4}
{\footnotesize DMS-REQ-0310 } &
\multicolumn{3}{p{11.1cm}}{\footnotesize Un-Archived Data Product Cache } \\ \cdashline{1-4}
{\footnotesize DMS-REQ-0313 } &
\multicolumn{3}{p{11.1cm}}{\footnotesize Level 1 \&  2 Catalog Access } \\ \cdashline{1-4}
{\footnotesize DMS-REQ-0314 } &
\multicolumn{3}{p{11.1cm}}{\footnotesize Compute Platform Heterogeneity } \\ \cdashline{1-4}
{\footnotesize DMS-REQ-0318 } &
\multicolumn{3}{p{11.1cm}}{\footnotesize Data Management Unscheduled Downtime } \\ \cdashline{1-4}
{\footnotesize DMS-REQ-0334 } &
\multicolumn{3}{p{11.1cm}}{\footnotesize Persisting Data Products } \\ \cdashline{1-4}
{\footnotesize DMS-REQ-0338 } &
\multicolumn{3}{p{11.1cm}}{\footnotesize Targeted Coadds } \\ \cdashline{1-4}
{\footnotesize DMS-REQ-0344 } &
\multicolumn{3}{p{11.1cm}}{\footnotesize Constraints on Level 1 Special Program Products Generation } \\ \cdashline{1-4}
{\footnotesize DMS-REQ-0346 } &
\multicolumn{3}{p{11.1cm}}{\footnotesize Data Availability } \\ \cdashline{1-4}
{\footnotesize DMS-REQ-0363 } &
\multicolumn{3}{p{11.1cm}}{\footnotesize Access to Previous Data Releases } \\ \cdashline{1-4}
{\footnotesize DMS-REQ-0364 } &
\multicolumn{3}{p{11.1cm}}{\footnotesize Data Access Services } \\ \cdashline{1-4}
{\footnotesize DMS-REQ-0365 } &
\multicolumn{3}{p{11.1cm}}{\footnotesize Operations Subsets } \\ \cdashline{1-4}
{\footnotesize DMS-REQ-0366 } &
\multicolumn{3}{p{11.1cm}}{\footnotesize Subsets Support } \\ \cdashline{1-4}
{\footnotesize DMS-REQ-0367 } &
\multicolumn{3}{p{11.1cm}}{\footnotesize Access Services Performance } \\ \cdashline{1-4}
{\footnotesize DMS-REQ-0368 } &
\multicolumn{3}{p{11.1cm}}{\footnotesize Implementation Provisions } \\ \cdashline{1-4}
{\footnotesize DMS-REQ-0370 } &
\multicolumn{3}{p{11.1cm}}{\footnotesize Older Release Behavior } \\ \cdashline{1-4}
{\footnotesize DMS-REQ-0372 } &
\multicolumn{3}{p{11.1cm}}{\footnotesize a Archiving Camera Test Data } \\ \cdashline{1-4}
{\footnotesize DMS-REQ-0386 } &
\multicolumn{3}{p{11.1cm}}{\footnotesize a Archive Processing Provenance } \\ \cdashline{1-4}
{\footnotesize DMS-REQ-0387 } &
\multicolumn{3}{p{11.1cm}}{\footnotesize b Serve Archived Provenance } \\ \cdashline{1-4}
\end{longtable}

    
   \newpage
\subsubsection{LSP Services}\label{lspsrv}
\begin{longtable}{p{3.7cm}p{3.7cm}p{3.7cm}p{3.7cm}}\hline
\textbf{Manager} & \textbf{Owner} & \textbf{WBS} & \textbf{Team} \\ \hline
\parbox{3.5cm}{
Frossie Economou
\vspace{2mm}%
} &
\begin{tabular}{@{}l@{}}
\parbox{3.5cm}{
Gregory Dubois-Felsmann
\vspace{2mm}%
} \\
\end{tabular}
 &
\begin{tabular}{@{}l@{}}
 \\
\end{tabular} &
\begin{tabular}{@{}l@{}}
 \\
\end{tabular} \\ \hline
\multicolumn{4}{c}{
{\footnotesize ( LSP Services
 - LSPSRV ) }
}\\ \hline
\citeds{LDM-540} &
\multicolumn{3}{l}{ LSST Science Platform Test
Specification }
\\ \cdashline{1-4}
\citeds{LDM-542} &
\multicolumn{3}{l}{ Science Platform Design }
\\ \cdashline{1-4}
\citeds{DMTN-103} &
\multicolumn{3}{l}{ LSST Science Platform Deployments }
\\ \cdashline{1-4}
\citeds{LDM-554} &
\multicolumn{3}{l}{ Data Management LSST Science Platform Requirements }
\\ \cdashline{1-4}
\end{longtable}

  This group of services provide an exploratory analysis environment for
the end user. These services are usually referred as ``Aspects'',
identified as follows:

\begin{itemize}
\tightlist
\item
  LSP Portal Aspect
\item
  LSP JupiterLab Aspect
\item
  LSP WebAPIs Aspect
\end{itemize}

These services permit external users access to project resources, such
as data and processing SW.\\[2\baselineskip]More details available in
\citeds{LDM-542}and \citeds{LSE-319}


   \newpage
\begin{longtable}{p{3.7cm}p{3.7cm}p{3.7cm}p{3.7cm}}\toprule
\multicolumn{2}{l}{\large \textbf{ \hypertarget{lspprtlsrv}{LSP Portal} } }
& \multicolumn{2}{l}{(product in: LSP Services
)}
\\ \hline
\textbf{\footnotesize Manager} & \textbf{\footnotesize Owner} &
\textbf{\footnotesize WBS} & \textbf{\footnotesize Team} \\ \hline
\parbox{3.5cm}{
Xiuqin Wu
\vspace{2mm}%
} &
\begin{tabular}{@{}l@{}}
\parbox{3.5cm}{
Gregory Dubois-Felsmann
\vspace{2mm}%
} \\
\end{tabular} &
\begin{tabular}{@{}l@{}}
1.02C.05.09 \\
1.02C.05.08 \\
1.02C.05.07 \\
\end{tabular} & \begin{tabular}{@{}l@{}}
SUIT \\
\end{tabular} \\ \hline
\multicolumn{4}{c}{
{\footnotesize ( LSP Portal
 - LSPPRTLSRV ) }
}\\ \hline
\end{longtable}

This service provides Web-based query and visualization tools for all
the LSST data products.


\begin{longtable}{p{3.7cm}p{3.7cm}p{3.7cm}p{3.7cm}}\hline
\multicolumn{2}{r}{\textbf{GutHub Packages:}} &
\multicolumn{2}{l}{\href{https://github.com/lsst-sqre/lsp-deploy}{lsst-sqre/lsp-deploy} }\ref{lsst-sqre/lsp-deploy}
\\ \hline \\ \hline
\textbf{\footnotesize Uses:}  & & \textbf{\footnotesize Used in:} & \\ \hline
\multicolumn{2}{c}{
\begin{tabular}{c}
\hyperlink{suit}{SUIT} \\ \hline
\hyperlink{suitoh}{SUIT Online Help} \\ \hline
\end{tabular}
} &
\multicolumn{2}{c}{
\begin{tabular}{c}
\hyperlink{encdacu}{DAC US Enclave} \\ \hline
\hyperlink{encdacc}{DAC Chile Enclave} \\ \hline
\hyperlink{enccomm}{Commissioning Cluster Enclave} \\ \hline
\end{tabular}
} \\ \bottomrule
\multicolumn{4}{c}{\textbf{Related Requirements} } \\ \hline
{\footnotesize DMS-REQ-0119 } &
\multicolumn{3}{p{11.1cm}}{\footnotesize DAC resource allocation for Level 3 processing } \\ \cdashline{1-4}
{\footnotesize DMS-REQ-0123 } &
\multicolumn{3}{p{11.1cm}}{\footnotesize Access to input catalogs for DAC-based Level 3 processing } \\ \cdashline{1-4}
{\footnotesize DMS-REQ-0124 } &
\multicolumn{3}{p{11.1cm}}{\footnotesize Federation with external catalogs } \\ \cdashline{1-4}
{\footnotesize DMS-REQ-0127 } &
\multicolumn{3}{p{11.1cm}}{\footnotesize Access to input images for DAC-based Level 3 processing } \\ \cdashline{1-4}
{\footnotesize DMS-REQ-0160 } &
\multicolumn{3}{p{11.1cm}}{\footnotesize Provide User Interface Services } \\ \cdashline{1-4}
{\footnotesize DMS-REQ-0161 } &
\multicolumn{3}{p{11.1cm}}{\footnotesize Optimization of Cost, Reliability and Availability in Order } \\ \cdashline{1-4}
{\footnotesize DMS-REQ-0193 } &
\multicolumn{3}{p{11.1cm}}{\footnotesize Data Access Centers } \\ \cdashline{1-4}
{\footnotesize DMS-REQ-0194 } &
\multicolumn{3}{p{11.1cm}}{\footnotesize Data Access Center Simultaneous Connections } \\ \cdashline{1-4}
{\footnotesize DMS-REQ-0197 } &
\multicolumn{3}{p{11.1cm}}{\footnotesize No Limit on Data Access Centers } \\ \cdashline{1-4}
{\footnotesize DMS-REQ-0314 } &
\multicolumn{3}{p{11.1cm}}{\footnotesize Compute Platform Heterogeneity } \\ \cdashline{1-4}
{\footnotesize DMS-REQ-0318 } &
\multicolumn{3}{p{11.1cm}}{\footnotesize Data Management Unscheduled Downtime } \\ \cdashline{1-4}
{\footnotesize DMS-REQ-0341 } &
\multicolumn{3}{p{11.1cm}}{\footnotesize Providing a Precovery Service } \\ \cdashline{1-4}
{\footnotesize DMS-REQ-0363 } &
\multicolumn{3}{p{11.1cm}}{\footnotesize Access to Previous Data Releases } \\ \cdashline{1-4}
{\footnotesize DMS-REQ-0364 } &
\multicolumn{3}{p{11.1cm}}{\footnotesize Data Access Services } \\ \cdashline{1-4}
{\footnotesize DMS-REQ-0365 } &
\multicolumn{3}{p{11.1cm}}{\footnotesize Operations Subsets } \\ \cdashline{1-4}
{\footnotesize DMS-REQ-0366 } &
\multicolumn{3}{p{11.1cm}}{\footnotesize Subsets Support } \\ \cdashline{1-4}
{\footnotesize DMS-REQ-0367 } &
\multicolumn{3}{p{11.1cm}}{\footnotesize Access Services Performance } \\ \cdashline{1-4}
{\footnotesize DMS-REQ-0368 } &
\multicolumn{3}{p{11.1cm}}{\footnotesize Implementation Provisions } \\ \cdashline{1-4}
{\footnotesize DMS-REQ-0369 } &
\multicolumn{3}{p{11.1cm}}{\footnotesize Evolution } \\ \cdashline{1-4}
{\footnotesize DMS-REQ-0370 } &
\multicolumn{3}{p{11.1cm}}{\footnotesize Older Release Behavior } \\ \cdashline{1-4}
{\footnotesize DMS-REQ-0371 } &
\multicolumn{3}{p{11.1cm}}{\footnotesize Query Availability } \\ \cdashline{1-4}
{\footnotesize DMS-REQ-0382 } &
\multicolumn{3}{p{11.1cm}}{\footnotesize HiPS Visualization } \\ \cdashline{1-4}
{\footnotesize DMS-REQ-0385 } &
\multicolumn{3}{p{11.1cm}}{\footnotesize MOC Visualization } \\ \cdashline{1-4}
\end{longtable}

     \newpage
\begin{longtable}{p{3.7cm}p{3.7cm}p{3.7cm}p{3.7cm}}\toprule
\multicolumn{2}{l}{\large \textbf{ \hypertarget{lspnbl}{LSP Nublado} } }
& \multicolumn{2}{l}{(product in: LSP Services
)}
\\ \hline
\textbf{\footnotesize Manager} & \textbf{\footnotesize Owner} &
\textbf{\footnotesize WBS} & \textbf{\footnotesize Team} \\ \hline
\parbox{3.5cm}{
Frossie Economou
\vspace{2mm}%
} &
\begin{tabular}{@{}l@{}}
\parbox{3.5cm}{
Simon Krughoff
\vspace{2mm}%
} \\
\end{tabular} &
\begin{tabular}{@{}l@{}}
1.02C.10.02.02 \\
\end{tabular} & \begin{tabular}{@{}l@{}}
SQuaRE \\
\end{tabular} \\ \hline
\multicolumn{4}{c}{
{\footnotesize ( LSP JupyterLab
 - LSPNBL ) }
}\\ \hline
\end{longtable}

This service provides access to a Python-oriented computational
environment, hosted at the LSST Data Access Centers. Through a Web-based
notebook interface, users are able to run Python code in close proximity
to the LSST data archive, accessing and analyzing the data and
generating derived data products.


\begin{longtable}{p{3.7cm}p{3.7cm}p{3.7cm}p{3.7cm}}\hline
\multicolumn{2}{r}{\textbf{GutHub Packages:}} &
\multicolumn{2}{l}{\href{https://github.com/lsst-sqre/nublado}{lsst-sqre/nublado} }\ref{lsst-sqre/nublado}
\\ \hline \\ \hline
\textbf{\footnotesize Uses:}  & & \textbf{\footnotesize Used in:} & \\ \hline
\multicolumn{2}{c}{
\begin{tabular}{c}
\hyperlink{lspjl}{LSP JupyterLab SW} \\ \hline
\hyperlink{spdist}{Science Pipelines Distribution} \\ \hline
\end{tabular}
} &
\multicolumn{2}{c}{
\begin{tabular}{c}
\hyperlink{encdacu}{DAC US Enclave} \\ \hline
\hyperlink{encdacc}{DAC Chile Enclave} \\ \hline
\hyperlink{enccomm}{Commissioning Cluster Enclave} \\ \hline
\end{tabular}
} \\ \bottomrule
\multicolumn{4}{c}{\textbf{Related Requirements} } \\ \hline
{\footnotesize DMS-REQ-0119 } &
\multicolumn{3}{p{11.1cm}}{\footnotesize DAC resource allocation for Level 3 processing } \\ \cdashline{1-4}
{\footnotesize DMS-REQ-0123 } &
\multicolumn{3}{p{11.1cm}}{\footnotesize Access to input catalogs for DAC-based Level 3 processing } \\ \cdashline{1-4}
{\footnotesize DMS-REQ-0124 } &
\multicolumn{3}{p{11.1cm}}{\footnotesize Federation with external catalogs } \\ \cdashline{1-4}
{\footnotesize DMS-REQ-0127 } &
\multicolumn{3}{p{11.1cm}}{\footnotesize Access to input images for DAC-based Level 3 processing } \\ \cdashline{1-4}
{\footnotesize DMS-REQ-0161 } &
\multicolumn{3}{p{11.1cm}}{\footnotesize Optimization of Cost, Reliability and Availability in Order } \\ \cdashline{1-4}
{\footnotesize DMS-REQ-0193 } &
\multicolumn{3}{p{11.1cm}}{\footnotesize Data Access Centers } \\ \cdashline{1-4}
{\footnotesize DMS-REQ-0194 } &
\multicolumn{3}{p{11.1cm}}{\footnotesize Data Access Center Simultaneous Connections } \\ \cdashline{1-4}
{\footnotesize DMS-REQ-0197 } &
\multicolumn{3}{p{11.1cm}}{\footnotesize No Limit on Data Access Centers } \\ \cdashline{1-4}
{\footnotesize DMS-REQ-0314 } &
\multicolumn{3}{p{11.1cm}}{\footnotesize Compute Platform Heterogeneity } \\ \cdashline{1-4}
{\footnotesize DMS-REQ-0318 } &
\multicolumn{3}{p{11.1cm}}{\footnotesize Data Management Unscheduled Downtime } \\ \cdashline{1-4}
\end{longtable}

  \newpage
\paragraph{LSP Web API}\label{lspwapi}\mbox{}\\
\begin{longtable}{p{3.7cm}p{3.7cm}p{3.7cm}p{3.7cm}}\hline
\textbf{Manager} & \textbf{Owner} & \textbf{WBS} & \textbf{Team} \\ \hline
\parbox{3.5cm}{
Frossie Economou
\vspace{2mm}%
} &
\begin{tabular}{@{}l@{}}
\parbox{3.5cm}{
Simon Krughoff
\vspace{2mm}%
} \\
\end{tabular}
 &
\begin{tabular}{@{}l@{}}
 \\
\end{tabular} &
\begin{tabular}{@{}l@{}}
 \\
\end{tabular} \\ \hline
\multicolumn{4}{c}{
{\footnotesize ( LSP Web API
 - LSPWAPI ) }
}\\ \hline
\end{longtable}

  The API Aspect services provide remote access to the LSST data, user
data, and user computing resources, through a set of Web APIs (many
based on VO standards). The Web APIs will deliver data in
community-standard formats, including, e.g., VOTable, CSV, and FITS. The
same Web APIs are used internally in the Portal Aspect, and are also
available in the Notebook Aspect.


  \newpage
\begin{longtable}{p{3.7cm}p{3.7cm}p{3.7cm}p{3.7cm}}\toprule
\multicolumn{2}{l}{\large \textbf{ \hypertarget{tapsev}{TAP API} } }
& \multicolumn{2}{l}{(product in: LSP Web API
)}
\\ \hline
\textbf{\footnotesize Manager} & \textbf{\footnotesize Owner} &
\textbf{\footnotesize WBS} & \textbf{\footnotesize Team} \\ \hline
\parbox{3.5cm}{
Frossie Economou
\vspace{2mm}%
} &
\begin{tabular}{@{}l@{}}
\parbox{3.5cm}{
Simon Krughoff
\vspace{2mm}%
} \\
\end{tabular} &
\begin{tabular}{@{}l@{}}
 \\
\end{tabular} & \begin{tabular}{@{}l@{}}
 \\
\end{tabular} \\ \hline
\multicolumn{4}{c}{
{\footnotesize ( TAP
 - TAPSEV ) }
}\\ \hline
\end{longtable}

TAP Compliant Interface web-service.\\[2\baselineskip]See \citeds{DMTN-090}for
details (dmtn-090.lsst.io).


\begin{longtable}{p{3.7cm}p{3.7cm}p{3.7cm}p{3.7cm}}\hline
\multicolumn{2}{r}{\textbf{GutHub Packages:}} &
\multicolumn{2}{l}{\href{https://github.com/lsst-sqre/lsst-tap-service}{lsst-sqre/lsst-tap-service} }\ref{lsst-sqre/lsst-tap-service}
\\ \hline \\ \hline
\textbf{\footnotesize Uses:}  & & \textbf{\footnotesize Used in:} & \\ \hline
\multicolumn{2}{c}{
\begin{tabular}{c}
\hyperlink{tapsw}{TAP SW} \\ \hline
\end{tabular}
} &
\multicolumn{2}{c}{
\begin{tabular}{c}
\hyperlink{enccomm}{Commissioning Cluster Enclave} \\ \hline
\hyperlink{encdacc}{DAC Chile Enclave} \\ \hline
\hyperlink{encdacu}{DAC US Enclave} \\ \hline
\end{tabular}
} \\ \bottomrule
\multicolumn{4}{c}{\textbf{Related Requirements} } \\ \hline
\end{longtable}

    \newpage
\begin{longtable}{p{3.7cm}p{3.7cm}p{3.7cm}p{3.7cm}}\toprule
\multicolumn{2}{l}{\large \textbf{ \hypertarget{soda}{SODA API} } }
& \multicolumn{2}{l}{(product in: LSP Web API
)}
\\ \hline
\textbf{\footnotesize Manager} & \textbf{\footnotesize Owner} &
\textbf{\footnotesize WBS} & \textbf{\footnotesize Team} \\ \hline
\parbox{3.5cm}{
Frossie Economou
\vspace{2mm}%
} &
\begin{tabular}{@{}l@{}}
\parbox{3.5cm}{
Simon Krughoff
\vspace{2mm}%
} \\
\end{tabular} &
\begin{tabular}{@{}l@{}}
 \\
\end{tabular} & \begin{tabular}{@{}l@{}}
 \\
\end{tabular} \\ \hline
\multicolumn{4}{c}{
{\footnotesize ( SODA
 - SODA ) }
}\\ \hline
\end{longtable}

Service that allows server-side image processing, as per IVOA's SODA
standard.

~

See \citeds{DMTN-090}for details (dmtn-090.lsst.io).


\begin{longtable}{p{3.7cm}p{3.7cm}p{3.7cm}p{3.7cm}}\hline
\textbf{\footnotesize Uses:}  & & \textbf{\footnotesize Used in:} & \\ \hline
\multicolumn{2}{c}{
\begin{tabular}{c}
\hyperlink{daximg}{Image/ Cutout Server} \\ \hline
\end{tabular}
} &
\multicolumn{2}{c}{
\begin{tabular}{c}
\hyperlink{enccomm}{Commissioning Cluster Enclave} \\ \hline
\hyperlink{encdacc}{DAC Chile Enclave} \\ \hline
\hyperlink{encdacu}{DAC US Enclave} \\ \hline
\end{tabular}
} \\ \bottomrule
\multicolumn{4}{c}{\textbf{Related Requirements} } \\ \hline
\end{longtable}

    \newpage
\begin{longtable}{p{3.7cm}p{3.7cm}p{3.7cm}p{3.7cm}}\toprule
\multicolumn{2}{l}{\large \textbf{ \hypertarget{siasrv}{SIA API} } }
& \multicolumn{2}{l}{(product in: LSP Web API
)}
\\ \hline
\textbf{\footnotesize Manager} & \textbf{\footnotesize Owner} &
\textbf{\footnotesize WBS} & \textbf{\footnotesize Team} \\ \hline
\parbox{3.5cm}{
Frossie Economou
\vspace{2mm}%
} &
\begin{tabular}{@{}l@{}}
\parbox{3.5cm}{
Simon Krughoff
\vspace{2mm}%
} \\
\end{tabular} &
\begin{tabular}{@{}l@{}}
 \\
\end{tabular} & \begin{tabular}{@{}l@{}}
 \\
\end{tabular} \\ \hline
\multicolumn{4}{c}{
{\footnotesize ( SIA
 - SIASRV ) }
}\\ \hline
\end{longtable}

Service implementing SIA access to LSST images.

~

See \citeds{DMTN-090}for details (dmtn-090.lsst.io).


\begin{longtable}{p{3.7cm}p{3.7cm}p{3.7cm}p{3.7cm}}\hline
\textbf{\footnotesize Uses:}  & & \textbf{\footnotesize Used in:} & \\ \hline
\multicolumn{2}{c}{
} &
\multicolumn{2}{c}{
\begin{tabular}{c}
\hyperlink{enccomm}{Commissioning Cluster Enclave} \\ \hline
\hyperlink{encdacc}{DAC Chile Enclave} \\ \hline
\hyperlink{encdacu}{DAC US Enclave} \\ \hline
\end{tabular}
} \\ \bottomrule
\multicolumn{4}{c}{\textbf{Related Requirements} } \\ \hline
\end{longtable}

    \newpage
\begin{longtable}{p{3.7cm}p{3.7cm}p{3.7cm}p{3.7cm}}\toprule
\multicolumn{2}{l}{\large \textbf{ \hypertarget{wdav}{WebDAV API} } }
& \multicolumn{2}{l}{(product in: LSP Web API
)}
\\ \hline
\textbf{\footnotesize Manager} & \textbf{\footnotesize Owner} &
\textbf{\footnotesize WBS} & \textbf{\footnotesize Team} \\ \hline
\parbox{3.5cm}{
Frossie Economou
\vspace{2mm}%
} &
\begin{tabular}{@{}l@{}}
\parbox{3.5cm}{
Simon Krughoff
\vspace{2mm}%
} \\
\end{tabular} &
\begin{tabular}{@{}l@{}}
 \\
\end{tabular} & \begin{tabular}{@{}l@{}}
 \\
\end{tabular} \\ \hline
\multicolumn{4}{c}{
{\footnotesize ( WebDAV
 - WDAV ) }
}\\ \hline
\end{longtable}

This service permits the use of filesystem over http.


\begin{longtable}{p{3.7cm}p{3.7cm}p{3.7cm}p{3.7cm}}\hline
\multicolumn{2}{r}{\textbf{GutHub Packages:}} &
\multicolumn{2}{l}{\href{https://github.com/lsst/davt}{davt} }\ref{davt}
\\ \hline \\ \hline
\textbf{\footnotesize Uses:}  & & \textbf{\footnotesize Used in:} & \\ \hline
\multicolumn{2}{c}{
} &
\multicolumn{2}{c}{
\begin{tabular}{c}
\hyperlink{enccomm}{Commissioning Cluster Enclave} \\ \hline
\hyperlink{encdacc}{DAC Chile Enclave} \\ \hline
\hyperlink{encdacu}{DAC US Enclave} \\ \hline
\end{tabular}
} \\ \bottomrule
\multicolumn{4}{c}{\textbf{Related Requirements} } \\ \hline
\end{longtable}

    
     \newpage
\begin{longtable}{p{3.7cm}p{3.7cm}p{3.7cm}p{3.7cm}}\toprule
\multicolumn{2}{l}{\large \textbf{ \hypertarget{lspwebsrv}{LSP Web API (obsolete product)} } }
& \multicolumn{2}{l}{(product in: LSP Services
)}
\\ \hline
\textbf{\footnotesize Manager} & \textbf{\footnotesize Owner} &
\textbf{\footnotesize WBS} & \textbf{\footnotesize Team} \\ \hline
\parbox{3.5cm}{

\vspace{2mm}%
} &
\begin{tabular}{@{}l@{}}
\parbox{3.5cm}{

\vspace{2mm}%
} \\
\end{tabular} &
\begin{tabular}{@{}l@{}}
1.02C.06.02 \\
\end{tabular} & \begin{tabular}{@{}l@{}}
DAX \\
\end{tabular} \\ \hline
\multicolumn{4}{c}{
{\footnotesize ( LSP Web API
 - LSPWEBSRV ) }
}\\ \hline
\end{longtable}

{[}OBSOLETE{]} replaced by theLSP WebAPI package and corresponding
services in it. Before removing this obsolete product, requirements
traced to it need to be relocated to the corresponding WebAPI service.


\begin{longtable}{p{3.7cm}p{3.7cm}p{3.7cm}p{3.7cm}}\hline
\textbf{\footnotesize Uses:}  & & \textbf{\footnotesize Used in:} & \\ \hline
\multicolumn{2}{c}{
} &
\multicolumn{2}{c}{
} \\ \bottomrule
\multicolumn{4}{c}{\textbf{Related Requirements} } \\ \hline
{\footnotesize DMS-REQ-0065 } &
\multicolumn{3}{p{11.1cm}}{\footnotesize Provide Image Access Services } \\ \cdashline{1-4}
{\footnotesize DMS-REQ-0075 } &
\multicolumn{3}{p{11.1cm}}{\footnotesize Catalog Queries } \\ \cdashline{1-4}
{\footnotesize DMS-REQ-0078 } &
\multicolumn{3}{p{11.1cm}}{\footnotesize Catalog Export Formats } \\ \cdashline{1-4}
{\footnotesize DMS-REQ-0089 } &
\multicolumn{3}{p{11.1cm}}{\footnotesize Solar System Objects Available Within Specified Time } \\ \cdashline{1-4}
{\footnotesize DMS-REQ-0119 } &
\multicolumn{3}{p{11.1cm}}{\footnotesize DAC resource allocation for Level 3 processing } \\ \cdashline{1-4}
{\footnotesize DMS-REQ-0120 } &
\multicolumn{3}{p{11.1cm}}{\footnotesize Level 3 Data Product Self Consistency } \\ \cdashline{1-4}
{\footnotesize DMS-REQ-0121 } &
\multicolumn{3}{p{11.1cm}}{\footnotesize Provenance for Level 3 processing at DACs } \\ \cdashline{1-4}
{\footnotesize DMS-REQ-0123 } &
\multicolumn{3}{p{11.1cm}}{\footnotesize Access to input catalogs for DAC-based Level 3 processing } \\ \cdashline{1-4}
{\footnotesize DMS-REQ-0124 } &
\multicolumn{3}{p{11.1cm}}{\footnotesize Federation with external catalogs } \\ \cdashline{1-4}
{\footnotesize DMS-REQ-0127 } &
\multicolumn{3}{p{11.1cm}}{\footnotesize Access to input images for DAC-based Level 3 processing } \\ \cdashline{1-4}
{\footnotesize DMS-REQ-0155 } &
\multicolumn{3}{p{11.1cm}}{\footnotesize Provide Data Access Services } \\ \cdashline{1-4}
{\footnotesize DMS-REQ-0161 } &
\multicolumn{3}{p{11.1cm}}{\footnotesize Optimization of Cost, Reliability and Availability in Order } \\ \cdashline{1-4}
{\footnotesize DMS-REQ-0193 } &
\multicolumn{3}{p{11.1cm}}{\footnotesize Data Access Centers } \\ \cdashline{1-4}
{\footnotesize DMS-REQ-0194 } &
\multicolumn{3}{p{11.1cm}}{\footnotesize Data Access Center Simultaneous Connections } \\ \cdashline{1-4}
{\footnotesize DMS-REQ-0197 } &
\multicolumn{3}{p{11.1cm}}{\footnotesize No Limit on Data Access Centers } \\ \cdashline{1-4}
{\footnotesize DMS-REQ-0287 } &
\multicolumn{3}{p{11.1cm}}{\footnotesize DIASource Precovery } \\ \cdashline{1-4}
{\footnotesize DMS-REQ-0290 } &
\multicolumn{3}{p{11.1cm}}{\footnotesize Level 3 Data Import } \\ \cdashline{1-4}
{\footnotesize DMS-REQ-0291 } &
\multicolumn{3}{p{11.1cm}}{\footnotesize Query Repeatability } \\ \cdashline{1-4}
{\footnotesize DMS-REQ-0292 } &
\multicolumn{3}{p{11.1cm}}{\footnotesize Uniqueness of IDs Across Data Releases } \\ \cdashline{1-4}
{\footnotesize DMS-REQ-0293 } &
\multicolumn{3}{p{11.1cm}}{\footnotesize Selection of Datasets } \\ \cdashline{1-4}
{\footnotesize DMS-REQ-0295 } &
\multicolumn{3}{p{11.1cm}}{\footnotesize Transparent Data Access } \\ \cdashline{1-4}
{\footnotesize DMS-REQ-0298 } &
\multicolumn{3}{p{11.1cm}}{\footnotesize Data Product and Raw Data Access } \\ \cdashline{1-4}
{\footnotesize DMS-REQ-0311 } &
\multicolumn{3}{p{11.1cm}}{\footnotesize Regenerate Un-archived Data Products } \\ \cdashline{1-4}
{\footnotesize DMS-REQ-0312 } &
\multicolumn{3}{p{11.1cm}}{\footnotesize Level 1 Data Product Access } \\ \cdashline{1-4}
{\footnotesize DMS-REQ-0313 } &
\multicolumn{3}{p{11.1cm}}{\footnotesize Level 1 \&  2 Catalog Access } \\ \cdashline{1-4}
{\footnotesize DMS-REQ-0314 } &
\multicolumn{3}{p{11.1cm}}{\footnotesize Compute Platform Heterogeneity } \\ \cdashline{1-4}
{\footnotesize DMS-REQ-0318 } &
\multicolumn{3}{p{11.1cm}}{\footnotesize Data Management Unscheduled Downtime } \\ \cdashline{1-4}
{\footnotesize DMS-REQ-0322 } &
\multicolumn{3}{p{11.1cm}}{\footnotesize Special Programs Database } \\ \cdashline{1-4}
{\footnotesize DMS-REQ-0323 } &
\multicolumn{3}{p{11.1cm}}{\footnotesize Calculating SSObject Parameters } \\ \cdashline{1-4}
{\footnotesize DMS-REQ-0324 } &
\multicolumn{3}{p{11.1cm}}{\footnotesize Matching DIASources to Objects } \\ \cdashline{1-4}
{\footnotesize DMS-REQ-0331 } &
\multicolumn{3}{p{11.1cm}}{\footnotesize Computing Derived Quantities } \\ \cdashline{1-4}
{\footnotesize DMS-REQ-0332 } &
\multicolumn{3}{p{11.1cm}}{\footnotesize Denormalizing Database Tables } \\ \cdashline{1-4}
{\footnotesize DMS-REQ-0334 } &
\multicolumn{3}{p{11.1cm}}{\footnotesize Persisting Data Products } \\ \cdashline{1-4}
{\footnotesize DMS-REQ-0336 } &
\multicolumn{3}{p{11.1cm}}{\footnotesize b Regenerating Data Products from Previous Data Releases } \\ \cdashline{1-4}
{\footnotesize DMS-REQ-0338 } &
\multicolumn{3}{p{11.1cm}}{\footnotesize Targeted Coadds } \\ \cdashline{1-4}
{\footnotesize DMS-REQ-0339 } &
\multicolumn{3}{p{11.1cm}}{\footnotesize Tracking Characterization Changes Between Data Releases } \\ \cdashline{1-4}
{\footnotesize DMS-REQ-0344 } &
\multicolumn{3}{p{11.1cm}}{\footnotesize Constraints on Level 1 Special Program Products Generation } \\ \cdashline{1-4}
{\footnotesize DMS-REQ-0345 } &
\multicolumn{3}{p{11.1cm}}{\footnotesize Logging of catalog queries } \\ \cdashline{1-4}
{\footnotesize DMS-REQ-0346 } &
\multicolumn{3}{p{11.1cm}}{\footnotesize Data Availability } \\ \cdashline{1-4}
{\footnotesize DMS-REQ-0347 } &
\multicolumn{3}{p{11.1cm}}{\footnotesize Measurements in catalogs } \\ \cdashline{1-4}
{\footnotesize DMS-REQ-0363 } &
\multicolumn{3}{p{11.1cm}}{\footnotesize Access to Previous Data Releases } \\ \cdashline{1-4}
{\footnotesize DMS-REQ-0364 } &
\multicolumn{3}{p{11.1cm}}{\footnotesize Data Access Services } \\ \cdashline{1-4}
{\footnotesize DMS-REQ-0365 } &
\multicolumn{3}{p{11.1cm}}{\footnotesize Operations Subsets } \\ \cdashline{1-4}
{\footnotesize DMS-REQ-0366 } &
\multicolumn{3}{p{11.1cm}}{\footnotesize Subsets Support } \\ \cdashline{1-4}
{\footnotesize DMS-REQ-0367 } &
\multicolumn{3}{p{11.1cm}}{\footnotesize Access Services Performance } \\ \cdashline{1-4}
{\footnotesize DMS-REQ-0368 } &
\multicolumn{3}{p{11.1cm}}{\footnotesize Implementation Provisions } \\ \cdashline{1-4}
{\footnotesize DMS-REQ-0369 } &
\multicolumn{3}{p{11.1cm}}{\footnotesize Evolution } \\ \cdashline{1-4}
{\footnotesize DMS-REQ-0370 } &
\multicolumn{3}{p{11.1cm}}{\footnotesize Older Release Behavior } \\ \cdashline{1-4}
{\footnotesize DMS-REQ-0371 } &
\multicolumn{3}{p{11.1cm}}{\footnotesize Query Availability } \\ \cdashline{1-4}
{\footnotesize DMS-REQ-0380 } &
\multicolumn{3}{p{11.1cm}}{\footnotesize HiPS Service } \\ \cdashline{1-4}
{\footnotesize DMS-REQ-0381 } &
\multicolumn{3}{p{11.1cm}}{\footnotesize HiPS Linkage to Coadds } \\ \cdashline{1-4}
{\footnotesize DMS-REQ-0384 } &
\multicolumn{3}{p{11.1cm}}{\footnotesize Export MOCs As FITS } \\ \cdashline{1-4}
{\footnotesize DMS-REQ-0387 } &
\multicolumn{3}{p{11.1cm}}{\footnotesize b Serve Archived Provenance } \\ \cdashline{1-4}
{\footnotesize EP-DM-CON-ICD-0013 } &
\multicolumn{3}{p{11.1cm}}{\footnotesize Visualization Image Metadata Standard } \\ \cdashline{1-4}
{\footnotesize EP-DM-CON-ICD-0034 } &
\multicolumn{3}{p{11.1cm}}{\footnotesize Citizen Science Data } \\ \cdashline{1-4}
{\footnotesize EP-DM-CON-ICD-0036 } &
\multicolumn{3}{p{11.1cm}}{\footnotesize DM Services } \\ \cdashline{1-4}
\end{longtable}

    
   \newpage
\subsubsection{IT Services}\label{itsrv}
\begin{longtable}{p{3.7cm}p{3.7cm}p{3.7cm}p{3.7cm}}\hline
\textbf{Manager} & \textbf{Owner} & \textbf{WBS} & \textbf{Team} \\ \hline
\parbox{3.5cm}{
Wil O'Mullane
\vspace{2mm}%
} &
\begin{tabular}{@{}l@{}}
\parbox{3.5cm}{
Wil O'Mullane
\vspace{2mm}%
} \\
\end{tabular}
 &
\begin{tabular}{@{}l@{}}
 \\
\end{tabular} &
\begin{tabular}{@{}l@{}}
 \\
\end{tabular} \\ \hline
\multicolumn{4}{c}{
{\footnotesize ( IT Service
 - ITSRV ) }
}\\ \hline
\end{longtable}

  The services grouped under this package are meant to be used to support
operational activities.\\

Examples are:

\begin{itemize}
\tightlist
\item
  database instances available for operations
\item
  operational resource (i.e. networks) management services
\end{itemize}


   \newpage
\begin{longtable}{p{3.7cm}p{3.7cm}p{3.7cm}p{3.7cm}}\toprule
\multicolumn{2}{l}{\large \textbf{ \hypertarget{netmgmt}{Network Management} } }
& \multicolumn{2}{l}{(product in: IT Service
)}
\\ \hline
\textbf{\footnotesize Manager} & \textbf{\footnotesize Owner} &
\textbf{\footnotesize WBS} & \textbf{\footnotesize Team} \\ \hline
\parbox{3.5cm}{
Jeff Kantor
\vspace{2mm}%
} &
\begin{tabular}{@{}l@{}}
\parbox{3.5cm}{
Wil O'Mullane
\vspace{2mm}%
} \\
\end{tabular} &
\begin{tabular}{@{}l@{}}
1.02C.08.03 \\
\end{tabular} & \begin{tabular}{@{}l@{}}
Net/Base \\
\end{tabular} \\ \hline
\multicolumn{4}{c}{
{\footnotesize ( Net Mgmt
 - NETMGMT ) }
}\\ \hline
\end{longtable}

This service provides monitor capability on networks, including for
example failover control, bandwidth allocation management.\\


\begin{longtable}{p{3.7cm}p{3.7cm}p{3.7cm}p{3.7cm}}\hline
\textbf{\footnotesize Uses:}  & & \textbf{\footnotesize Used in:} & \\ \hline
\multicolumn{2}{c}{
} &
\multicolumn{2}{c}{
} \\ \bottomrule
\multicolumn{4}{c}{\textbf{Related Requirements} } \\ \hline
{\footnotesize DMS-REQ-0008 } &
\multicolumn{3}{p{11.1cm}}{\footnotesize Pipeline Availability } \\ \cdashline{1-4}
{\footnotesize DMS-REQ-0166 } &
\multicolumn{3}{p{11.1cm}}{\footnotesize Incorporate Fault-Tolerance } \\ \cdashline{1-4}
{\footnotesize DMS-REQ-0167 } &
\multicolumn{3}{p{11.1cm}}{\footnotesize Incorporate Autonomics } \\ \cdashline{1-4}
{\footnotesize DMS-REQ-0175 } &
\multicolumn{3}{p{11.1cm}}{\footnotesize Summit to Base Network Ownership and Operation } \\ \cdashline{1-4}
\end{longtable}

     \newpage
\begin{longtable}{p{3.7cm}p{3.7cm}p{3.7cm}p{3.7cm}}\toprule
\multicolumn{2}{l}{\large \textbf{ \hypertarget{lspdb}{LSP Database} } }
& \multicolumn{2}{l}{(product in: IT Service
)}
\\ \hline
\textbf{\footnotesize Manager} & \textbf{\footnotesize Owner} &
\textbf{\footnotesize WBS} & \textbf{\footnotesize Team} \\ \hline
\parbox{3.5cm}{
Michelle Butler
\vspace{2mm}%
} &
\begin{tabular}{@{}l@{}}
\parbox{3.5cm}{
Michelle Butler
\vspace{2mm}%
} \\
\end{tabular} &
\begin{tabular}{@{}l@{}}
 \\
\end{tabular} & \begin{tabular}{@{}l@{}}
 \\
\end{tabular} \\ \hline
\multicolumn{4}{c}{
{\footnotesize ( LSP Database
 - LSPDB ) }
}\\ \hline
\end{longtable}

Database instance, required by the LSP components instantiated in a DAC,
used for:

\begin{itemize}
\tightlist
\item
  storing user data
\item
  reading available catalogs.
\end{itemize}


\begin{longtable}{p{3.7cm}p{3.7cm}p{3.7cm}p{3.7cm}}\hline
\textbf{\footnotesize Uses:}  & & \textbf{\footnotesize Used in:} & \\ \hline
\multicolumn{2}{c}{
\begin{tabular}{c}
\hyperlink{qserv}{Distributed Database} \\ \hline
\end{tabular}
} &
\multicolumn{2}{c}{
\begin{tabular}{c}
\hyperlink{enccomm}{Commissioning Cluster Enclave} \\ \hline
\hyperlink{encdacc}{DAC Chile Enclave} \\ \hline
\hyperlink{encdacu}{DAC US Enclave} \\ \hline
\end{tabular}
} \\ \bottomrule
\multicolumn{4}{c}{\textbf{Related Requirements} } \\ \hline
\end{longtable}

     \newpage
\begin{longtable}{p{3.7cm}p{3.7cm}p{3.7cm}p{3.7cm}}\toprule
\multicolumn{2}{l}{\large \textbf{ \hypertarget{efdb}{EFD Cache} } }
& \multicolumn{2}{l}{(product in: IT Service
)}
\\ \hline
\textbf{\footnotesize Manager} & \textbf{\footnotesize Owner} &
\textbf{\footnotesize WBS} & \textbf{\footnotesize Team} \\ \hline
\parbox{3.5cm}{
Frossie Economou
\vspace{2mm}%
} &
\begin{tabular}{@{}l@{}}
\parbox{3.5cm}{
Simon Krughoff
\vspace{2mm}%
} \\
\end{tabular} &
\begin{tabular}{@{}l@{}}
 \\
\end{tabular} & \begin{tabular}{@{}l@{}}
 \\
\end{tabular} \\ \hline
\multicolumn{4}{c}{
{\footnotesize ( EFD Cache
 - EFDB ) }
}\\ \hline
\end{longtable}




\begin{longtable}{p{3.7cm}p{3.7cm}p{3.7cm}p{3.7cm}}\hline
\textbf{\footnotesize Uses:}  & & \textbf{\footnotesize Used in:} & \\ \hline
\multicolumn{2}{c}{
\begin{tabular}{c}
\hyperlink{efdt}{EFD Transformation} \\ \hline
\end{tabular}
} &
\multicolumn{2}{c}{
\begin{tabular}{c}
\hyperlink{encprb}{Prompt Base Enclave} \\ \hline
\end{tabular}
} \\ \bottomrule
\multicolumn{4}{c}{\textbf{Related Requirements} } \\ \hline
\end{longtable}

     \newpage
\begin{longtable}{p{3.7cm}p{3.7cm}p{3.7cm}p{3.7cm}}\toprule
\multicolumn{2}{l}{\large \textbf{ \hypertarget{apdb}{APDB} } }
& \multicolumn{2}{l}{(product in: IT Service
)}
\\ \hline
\textbf{\footnotesize Manager} & \textbf{\footnotesize Owner} &
\textbf{\footnotesize WBS} & \textbf{\footnotesize Team} \\ \hline
\parbox{3.5cm}{
Fritz Mueller
\vspace{2mm}%
} &
\begin{tabular}{@{}l@{}}
\parbox{3.5cm}{
Colin Slater
\vspace{2mm}%
} \\
\end{tabular} &
\begin{tabular}{@{}l@{}}
1.02C.06.01.01 \\
\end{tabular} & \begin{tabular}{@{}l@{}}
DAX \\
\end{tabular} \\ \hline
\multicolumn{4}{c}{
{\footnotesize ( APDB
 - APDB ) }
}\\ \hline
\citeds{DMTN-018} &
\multicolumn{3}{l}{ Re-visiting L1 Database Design }
\\ \cdashline{1-4}
\end{longtable}

Performance critical internal database used to support Alert Production;
will not support end-user queries. Until October 2019 this product was
known as L1 Database.\\[2\baselineskip]Implementation details still
under evaluation.

~

\emph{{{[}last review: F.Mueller - Jan 2020{]}}}


\begin{longtable}{p{3.7cm}p{3.7cm}p{3.7cm}p{3.7cm}}\hline
\textbf{\footnotesize Uses:}  & & \textbf{\footnotesize Used in:} & \\ \hline
\multicolumn{2}{c}{
} &
\multicolumn{2}{c}{
\begin{tabular}{c}
\hyperlink{encprn}{Prompt NCSA Enclave} \\ \hline
\end{tabular}
} \\ \bottomrule
\multicolumn{4}{c}{\textbf{Related Requirements} } \\ \hline
\end{longtable}

    
     \newpage
\subsection{Software Products}\label{dmsw}
\begin{longtable}{p{3.7cm}p{3.7cm}p{3.7cm}p{3.7cm}}\hline
\textbf{Manager} & \textbf{Owner} & \textbf{WBS} & \textbf{Team} \\ \hline
\parbox{3.5cm}{
Wil O'Mullane
\vspace{2mm}%
} &
\begin{tabular}{@{}l@{}}
\parbox{3.5cm}{

\vspace{2mm}%
} \\
\end{tabular}
 &
\begin{tabular}{@{}l@{}}
 \\
\end{tabular} &
\begin{tabular}{@{}l@{}}
 \\
\end{tabular} \\ \hline
\multicolumn{4}{c}{
{\footnotesize ( Software Products
 - DMSW ) }
}\\ \hline
\end{longtable}

  DM operational SW products. This include all SW products implemented by
the DM team in order to implement the operational services.\\

Dependencies should be derived from the corresponding Git packages
definition.


\includegraphics[max width=\linewidth]{subtrees/Main_DMSW.pdf}

\newpage
\subsubsection{Prompt SW Products}\label{pr}
\begin{longtable}{p{3.7cm}p{3.7cm}p{3.7cm}p{3.7cm}}\hline
\textbf{Manager} & \textbf{Owner} & \textbf{WBS} & \textbf{Team} \\ \hline
\parbox{3.5cm}{
Michelle Butler
\vspace{2mm}%
} &
\begin{tabular}{@{}l@{}}
\parbox{3.5cm}{

\vspace{2mm}%
} \\
\end{tabular}
 &
\begin{tabular}{@{}l@{}}
 \\
\end{tabular} &
\begin{tabular}{@{}l@{}}
 \\
\end{tabular} \\ \hline
\multicolumn{4}{c}{
{\footnotesize ( Prompt SW
 - PR ) }
}\\ \hline
\citeds{LDM-533} &
\multicolumn{3}{l}{ LSST Level 1 System Test Specification }
\\ \cdashline{1-4}
\end{longtable}

  DM software products required to implement the DM Prompt Services.


   \newpage
\begin{longtable}{p{3.7cm}p{3.7cm}p{3.7cm}p{3.7cm}}\toprule
\multicolumn{2}{l}{\large \textbf{ \hypertarget{alrtdstr}{Alert Distribution SW} } }
& \multicolumn{2}{l}{(product in: Prompt SW
)}
\\ \hline
\textbf{\footnotesize Manager} & \textbf{\footnotesize Owner} &
\textbf{\footnotesize WBS} & \textbf{\footnotesize Team} \\ \hline
\parbox{3.5cm}{
John Swinbank
\vspace{2mm}%
} &
\begin{tabular}{@{}l@{}}
\parbox{3.5cm}{
Eric Bellm
\vspace{2mm}%
} \\
\end{tabular} &
\begin{tabular}{@{}l@{}}
1.02C.03.03 \\
\end{tabular} & \begin{tabular}{@{}l@{}}
AP \\
\end{tabular} \\ \hline
\multicolumn{4}{c}{
{\footnotesize ( Alert Distrib SW
 - ALRTDSTR ) }
}\\ \hline
\end{longtable}

SW product for alert distribution and filtering.


\begin{longtable}{p{3.7cm}p{3.7cm}p{3.7cm}p{3.7cm}}\hline
\multicolumn{2}{r}{\textbf{GutHub Packages:}} &
\multicolumn{2}{l}{\href{https://github.com/lsst-dm/alert_stream}{lsst-dm/alert\_stream} }\ref{lsst-dm/alert_stream}
\\ \hline \\ \hline
\textbf{\footnotesize Uses:}  & & \textbf{\footnotesize Used in:} & \\ \hline
\multicolumn{2}{c}{
} &
\multicolumn{2}{c}{
\begin{tabular}{c}
\hyperlink{alrtdstsrv}{Alert Distribution} \\ \hline
\end{tabular}
} \\ \bottomrule
\multicolumn{4}{c}{\textbf{Related Requirements} } \\ \hline
{\footnotesize DMS-REQ-0002 } &
\multicolumn{3}{p{11.1cm}}{\footnotesize Transient Alert Distribution } \\ \cdashline{1-4}
{\footnotesize DMS-REQ-0004 } &
\multicolumn{3}{p{11.1cm}}{\footnotesize Nightly Data Accessible Within 24 hrs } \\ \cdashline{1-4}
{\footnotesize DMS-REQ-0008 } &
\multicolumn{3}{p{11.1cm}}{\footnotesize Pipeline Availability } \\ \cdashline{1-4}
{\footnotesize DMS-REQ-0161 } &
\multicolumn{3}{p{11.1cm}}{\footnotesize Optimization of Cost, Reliability and Availability in Order } \\ \cdashline{1-4}
{\footnotesize DMS-REQ-0162 } &
\multicolumn{3}{p{11.1cm}}{\footnotesize Pipeline Throughput } \\ \cdashline{1-4}
{\footnotesize DMS-REQ-0167 } &
\multicolumn{3}{p{11.1cm}}{\footnotesize Incorporate Autonomics } \\ \cdashline{1-4}
{\footnotesize DMS-REQ-0314 } &
\multicolumn{3}{p{11.1cm}}{\footnotesize Compute Platform Heterogeneity } \\ \cdashline{1-4}
{\footnotesize DMS-REQ-0318 } &
\multicolumn{3}{p{11.1cm}}{\footnotesize Data Management Unscheduled Downtime } \\ \cdashline{1-4}
{\footnotesize DMS-REQ-0342 } &
\multicolumn{3}{p{11.1cm}}{\footnotesize Alert Filtering Service } \\ \cdashline{1-4}
{\footnotesize DMS-REQ-0343 } &
\multicolumn{3}{p{11.1cm}}{\footnotesize Performance Requirements for LSST Alert Filtering Service } \\ \cdashline{1-4}
{\footnotesize DMS-REQ-0348 } &
\multicolumn{3}{p{11.1cm}}{\footnotesize Pre-defined alert filters } \\ \cdashline{1-4}
{\footnotesize OCS-DM-COM-ICD-0048 } &
\multicolumn{3}{p{11.1cm}}{\footnotesize Alert Production Complete Event } \\ \cdashline{1-4}
\end{longtable}

     \newpage
\begin{longtable}{p{3.7cm}p{3.7cm}p{3.7cm}p{3.7cm}}\toprule
\multicolumn{2}{l}{\large \textbf{ \hypertarget{efdt}{EFD Transformation} } }
& \multicolumn{2}{l}{(product in: Prompt SW
)}
\\ \hline
\textbf{\footnotesize Manager} & \textbf{\footnotesize Owner} &
\textbf{\footnotesize WBS} & \textbf{\footnotesize Team} \\ \hline
\parbox{3.5cm}{
Frossie Economou
\vspace{2mm}%
} &
\begin{tabular}{@{}l@{}}
\parbox{3.5cm}{
Simon Krughoff
\vspace{2mm}%
} \\
\end{tabular} &
\begin{tabular}{@{}l@{}}
1.02C.07.08 \\
\end{tabular} & \begin{tabular}{@{}l@{}}
LDF \\
\end{tabular} \\ \hline
\multicolumn{4}{c}{
{\footnotesize ( EFD Transform
 - EFDT ) }
}\\ \hline
\end{longtable}

This SW is the one used for Image and EFD Archiving in the Archive
services. (EFD Extract-Tranform-Load) Software still to be implemented.


\begin{longtable}{p{3.7cm}p{3.7cm}p{3.7cm}p{3.7cm}}\hline
\textbf{\footnotesize Uses:}  & & \textbf{\footnotesize Used in:} & \\ \hline
\multicolumn{2}{c}{
} &
\multicolumn{2}{c}{
\begin{tabular}{c}
\hyperlink{efdb}{EFD Cache} \\ \hline
\hyperlink{efdts}{EFD Transformation} \\ \hline
\end{tabular}
} \\ \bottomrule
\multicolumn{4}{c}{\textbf{Related Requirements} } \\ \hline
{\footnotesize DMS-REQ-0004 } &
\multicolumn{3}{p{11.1cm}}{\footnotesize Nightly Data Accessible Within 24 hrs } \\ \cdashline{1-4}
{\footnotesize DMS-REQ-0008 } &
\multicolumn{3}{p{11.1cm}}{\footnotesize Pipeline Availability } \\ \cdashline{1-4}
{\footnotesize DMS-REQ-0102 } &
\multicolumn{3}{p{11.1cm}}{\footnotesize Provide Engineering \&  Facility Database Archive } \\ \cdashline{1-4}
{\footnotesize DMS-REQ-0165 } &
\multicolumn{3}{p{11.1cm}}{\footnotesize Infrastructure Sizing for "catching up" } \\ \cdashline{1-4}
{\footnotesize DMS-REQ-0167 } &
\multicolumn{3}{p{11.1cm}}{\footnotesize Incorporate Autonomics } \\ \cdashline{1-4}
{\footnotesize DMS-REQ-0309 } &
\multicolumn{3}{p{11.1cm}}{\footnotesize Raw Data Archiving Reliability } \\ \cdashline{1-4}
{\footnotesize DMS-REQ-0314 } &
\multicolumn{3}{p{11.1cm}}{\footnotesize Compute Platform Heterogeneity } \\ \cdashline{1-4}
{\footnotesize DMS-REQ-0315 } &
\multicolumn{3}{p{11.1cm}}{\footnotesize DMS Communication with OCS } \\ \cdashline{1-4}
{\footnotesize DMS-REQ-0318 } &
\multicolumn{3}{p{11.1cm}}{\footnotesize Data Management Unscheduled Downtime } \\ \cdashline{1-4}
{\footnotesize DMS-REQ-0346 } &
\multicolumn{3}{p{11.1cm}}{\footnotesize Data Availability } \\ \cdashline{1-4}
{\footnotesize OCS-DM-COM-ICD-0003 } &
\multicolumn{3}{p{11.1cm}}{\footnotesize Data Management CSC Command Response Model } \\ \cdashline{1-4}
{\footnotesize OCS-DM-COM-ICD-0004 } &
\multicolumn{3}{p{11.1cm}}{\footnotesize Data Management Exposed CSCs } \\ \cdashline{1-4}
{\footnotesize OCS-DM-COM-ICD-0008 } &
\multicolumn{3}{p{11.1cm}}{\footnotesize EFD Transformation Service CSC } \\ \cdashline{1-4}
{\footnotesize OCS-DM-COM-ICD-0025 } &
\multicolumn{3}{p{11.1cm}}{\footnotesize Expected Load of Queries from DM } \\ \cdashline{1-4}
{\footnotesize OCS-DM-COM-ICD-0026 } &
\multicolumn{3}{p{11.1cm}}{\footnotesize Engineering and Facilities Database Archiving by Data Management } \\ \cdashline{1-4}
{\footnotesize OCS-DM-COM-ICD-0027 } &
\multicolumn{3}{p{11.1cm}}{\footnotesize Multiple Physically Separated Copies } \\ \cdashline{1-4}
{\footnotesize OCS-DM-COM-ICD-0028 } &
\multicolumn{3}{p{11.1cm}}{\footnotesize Expected Data Volume } \\ \cdashline{1-4}
{\footnotesize OCS-DM-COM-ICD-0029 } &
\multicolumn{3}{p{11.1cm}}{\footnotesize Archive Latency } \\ \cdashline{1-4}
{\footnotesize OCS-DM-COM-ICD-0030 } &
\multicolumn{3}{p{11.1cm}}{\footnotesize EFD Transformation Service Interface } \\ \cdashline{1-4}
\end{longtable}

     \newpage
\begin{longtable}{p{3.7cm}p{3.7cm}p{3.7cm}p{3.7cm}}\toprule
\multicolumn{2}{l}{\large \textbf{ \hypertarget{header}{Header Service SW} } }
& \multicolumn{2}{l}{(product in: Prompt SW
)}
\\ \hline
\textbf{\footnotesize Manager} & \textbf{\footnotesize Owner} &
\textbf{\footnotesize WBS} & \textbf{\footnotesize Team} \\ \hline
\parbox{3.5cm}{
Michelle Butler
\vspace{2mm}%
} &
\begin{tabular}{@{}l@{}}
\parbox{3.5cm}{
Kian-Tat Lim
\vspace{2mm}%
} \\
\end{tabular} &
\begin{tabular}{@{}l@{}}
1.02C.07.08 \\
\end{tabular} & \begin{tabular}{@{}l@{}}
LDF \\
\end{tabular} \\ \hline
\multicolumn{4}{c}{
{\footnotesize ( Header Srv SW
 - HEADER ) }
}\\ \hline
\end{longtable}

This software product is developed by DM but is inteded to be used by
external users. It is developed at University of Illinois (Urbana)


\begin{longtable}{p{3.7cm}p{3.7cm}p{3.7cm}p{3.7cm}}\hline
\multicolumn{2}{r}{\textbf{GutHub Packages:}} &
\multicolumn{2}{l}{\href{https://github.com/lsst-dm/HeaderService}{lsst-dm/HeaderService} }\ref{lsst-dm/headerservice}
\\ \hline \\ \hline
\textbf{\footnotesize Uses:}  & & \textbf{\footnotesize Used in:} & \\ \hline
\multicolumn{2}{c}{
} &
\multicolumn{2}{c}{
\begin{tabular}{c}
\hyperlink{heads}{Header Generator} \\ \hline
\end{tabular}
} \\ \bottomrule
\multicolumn{4}{c}{\textbf{Related Requirements} } \\ \hline
{\footnotesize DMS-REQ-0266 } &
\multicolumn{3}{p{11.1cm}}{\footnotesize Exposure Catalog } \\ \cdashline{1-4}
{\footnotesize OCS-DM-COM-ICD-0003 } &
\multicolumn{3}{p{11.1cm}}{\footnotesize Data Management CSC Command Response Model } \\ \cdashline{1-4}
{\footnotesize OCS-DM-COM-ICD-0009 } &
\multicolumn{3}{p{11.1cm}}{\footnotesize Command Set Implementation by Data Management } \\ \cdashline{1-4}
{\footnotesize OCS-DM-COM-ICD-0012 } &
\multicolumn{3}{p{11.1cm}}{\footnotesize Start Command } \\ \cdashline{1-4}
{\footnotesize OCS-DM-COM-ICD-0013 } &
\multicolumn{3}{p{11.1cm}}{\footnotesize configure Successful Completion Response } \\ \cdashline{1-4}
{\footnotesize OCS-DM-COM-ICD-0014 } &
\multicolumn{3}{p{11.1cm}}{\footnotesize enable Command } \\ \cdashline{1-4}
{\footnotesize OCS-DM-COM-ICD-0015 } &
\multicolumn{3}{p{11.1cm}}{\footnotesize disable Command } \\ \cdashline{1-4}
{\footnotesize OCS-DM-COM-ICD-0033 } &
\multicolumn{3}{p{11.1cm}}{\footnotesize Header Service CSC } \\ \cdashline{1-4}
{\footnotesize OCS-DM-COM-ICD-0034 } &
\multicolumn{3}{p{11.1cm}}{\footnotesize Auxiliary Header Service CSC } \\ \cdashline{1-4}
{\footnotesize OCS-DM-COM-ICD-0036 } &
\multicolumn{3}{p{11.1cm}}{\footnotesize standby Command } \\ \cdashline{1-4}
{\footnotesize OCS-DM-COM-ICD-0037 } &
\multicolumn{3}{p{11.1cm}}{\footnotesize exit Command } \\ \cdashline{1-4}
{\footnotesize OCS-DM-COM-ICD-0038 } &
\multicolumn{3}{p{11.1cm}}{\footnotesize enterControl Command } \\ \cdashline{1-4}
{\footnotesize OCS-DM-COM-ICD-0039 } &
\multicolumn{3}{p{11.1cm}}{\footnotesize enterControl Successful Completion Response } \\ \cdashline{1-4}
{\footnotesize OCS-DM-COM-ICD-0040 } &
\multicolumn{3}{p{11.1cm}}{\footnotesize Command Completion Response } \\ \cdashline{1-4}
{\footnotesize OCS-EFD-HS-0001 } &
\multicolumn{3}{p{11.1cm}}{\footnotesize Fulfill requirements of a Commandable SAL Component (CSC) } \\ \cdashline{1-4}
{\footnotesize OCS-EFD-HS-0002 } &
\multicolumn{3}{p{11.1cm}}{\footnotesize Critical System } \\ \cdashline{1-4}
{\footnotesize OCS-EFD-HS-0003 } &
\multicolumn{3}{p{11.1cm}}{\footnotesize Write Headers for all images taken by all Cameras supported by LSST } \\ \cdashline{1-4}
{\footnotesize OCS-EFD-HS-0004 } &
\multicolumn{3}{p{11.1cm}}{\footnotesize Ability to capture metadata at the beginning of exposure } \\ \cdashline{1-4}
{\footnotesize OCS-EFD-HS-0005 } &
\multicolumn{3}{p{11.1cm}}{\footnotesize Ability to capture metadata during of exposure integration } \\ \cdashline{1-4}
{\footnotesize OCS-EFD-HS-0006 } &
\multicolumn{3}{p{11.1cm}}{\footnotesize Ability to capture metadata at end of readout } \\ \cdashline{1-4}
{\footnotesize OCS-EFD-HS-0007 } &
\multicolumn{3}{p{11.1cm}}{\footnotesize Write header and Publish Event after end of telemetry event } \\ \cdashline{1-4}
{\footnotesize OCS-EFD-HS-0008 } &
\multicolumn{3}{p{11.1cm}}{\footnotesize Write header and Publish Event within specified time of the end-of-telemetry Event } \\ \cdashline{1-4}
{\footnotesize OCS-EFD-HS-0009 } &
\multicolumn{3}{p{11.1cm}}{\footnotesize Adherence to the FITS Standard } \\ \cdashline{1-4}
{\footnotesize OCS-EFD-HS-0010 } &
\multicolumn{3}{p{11.1cm}}{\footnotesize Configuration of Header Keywords and source } \\ \cdashline{1-4}
{\footnotesize OCS-EFD-HS-0011 } &
\multicolumn{3}{p{11.1cm}}{\footnotesize Produce header even if some meta-data not avaiable } \\ \cdashline{1-4}
{\footnotesize OCS-EFD-HS-0012 } &
\multicolumn{3}{p{11.1cm}}{\footnotesize Publish an Event if monitoring detects any failure of the service. } \\ \cdashline{1-4}
{\footnotesize OCS-EFD-HS-0013 } &
\multicolumn{3}{p{11.1cm}}{\footnotesize Extract metadata from published configuration } \\ \cdashline{1-4}
{\footnotesize OCS-EFD-HS-0014 } &
\multicolumn{3}{p{11.1cm}}{\footnotesize Metadata Capture } \\ \cdashline{1-4}
{\footnotesize OCS-EFD-HS-0015 } &
\multicolumn{3}{p{11.1cm}}{\footnotesize Generate on-the-fly additional metadata as approved by the Project CCB. } \\ \cdashline{1-4}
\end{longtable}

     \newpage
\begin{longtable}{p{3.7cm}p{3.7cm}p{3.7cm}p{3.7cm}}\toprule
\multicolumn{2}{l}{\large \textbf{ \hypertarget{iip}{Image Ingest and Processing} } }
& \multicolumn{2}{l}{(product in: Prompt SW
)}
\\ \hline
\textbf{\footnotesize Manager} & \textbf{\footnotesize Owner} &
\textbf{\footnotesize WBS} & \textbf{\footnotesize Team} \\ \hline
\parbox{3.5cm}{
Michelle Butler
\vspace{2mm}%
} &
\begin{tabular}{@{}l@{}}
\parbox{3.5cm}{
Kian-Tat Lim
\vspace{2mm}%
} \\
\parbox{3.5cm}{
Steve Pietrowicz
\vspace{2mm}%
} \\
\end{tabular} &
\begin{tabular}{@{}l@{}}
1.02C.07.08 \\
\end{tabular} & \begin{tabular}{@{}l@{}}
LDF \\
\end{tabular} \\ \hline
\multicolumn{4}{c}{
{\footnotesize ( Image Ingest SW
 - IIP ) }
}\\ \hline
\end{longtable}

Image ingest and processing SW product.\\[2\baselineskip]This includes
the DMCS, used to implements the Telemetry Gateway service.


\begin{longtable}{p{3.7cm}p{3.7cm}p{3.7cm}p{3.7cm}}\hline
\multicolumn{2}{r}{\textbf{GutHub Packages:}} &
\multicolumn{2}{l}{\href{https://github.com/lsst/ctrl_iip}{ctrl\_iip} }\ref{ctrl_iip}
\\ \hline \\ \hline
\textbf{\footnotesize Uses:}  & & \textbf{\footnotesize Used in:} & \\ \hline
\multicolumn{2}{c}{
} &
\multicolumn{2}{c}{
\begin{tabular}{c}
\hyperlink{prpingsrv}{Prompt Processing Ingest} \\ \hline
\hyperlink{tmgsrv}{Telemetry Gateway} \\ \hline
\hyperlink{imas}{Image Archiver} \\ \hline
\end{tabular}
} \\ \bottomrule
\multicolumn{4}{c}{\textbf{Related Requirements} } \\ \hline
{\footnotesize CA-DM-CON-ICD-0014 } &
\multicolumn{3}{p{11.1cm}}{\footnotesize Provide science sensor data } \\ \cdashline{1-4}
{\footnotesize CA-DM-CON-ICD-0015 } &
\multicolumn{3}{p{11.1cm}}{\footnotesize Provide wavefront sensor data } \\ \cdashline{1-4}
{\footnotesize CA-DM-CON-ICD-0016 } &
\multicolumn{3}{p{11.1cm}}{\footnotesize Provide guide sensor data } \\ \cdashline{1-4}
{\footnotesize CA-DM-CON-ICD-0017 } &
\multicolumn{3}{p{11.1cm}}{\footnotesize Data Management load on image data interfaces } \\ \cdashline{1-4}
{\footnotesize CA-DM-CON-ICD-0019 } &
\multicolumn{3}{p{11.1cm}}{\footnotesize Camera engineering image data archiving } \\ \cdashline{1-4}
{\footnotesize DM-TS-CON-ICD-0002 } &
\multicolumn{3}{p{11.1cm}}{\footnotesize Timing } \\ \cdashline{1-4}
{\footnotesize DM-TS-CON-ICD-0007 } &
\multicolumn{3}{p{11.1cm}}{\footnotesize Timing } \\ \cdashline{1-4}
{\footnotesize DMS-REQ-0004 } &
\multicolumn{3}{p{11.1cm}}{\footnotesize Nightly Data Accessible Within 24 hrs } \\ \cdashline{1-4}
{\footnotesize DMS-REQ-0008 } &
\multicolumn{3}{p{11.1cm}}{\footnotesize Pipeline Availability } \\ \cdashline{1-4}
{\footnotesize DMS-REQ-0018 } &
\multicolumn{3}{p{11.1cm}}{\footnotesize Raw Science Image Data Acquisition } \\ \cdashline{1-4}
{\footnotesize DMS-REQ-0020 } &
\multicolumn{3}{p{11.1cm}}{\footnotesize Wavefront Sensor Data Acquisition } \\ \cdashline{1-4}
{\footnotesize DMS-REQ-0022 } &
\multicolumn{3}{p{11.1cm}}{\footnotesize Crosstalk Corrected Science Image Data Acquisition } \\ \cdashline{1-4}
{\footnotesize DMS-REQ-0024 } &
\multicolumn{3}{p{11.1cm}}{\footnotesize Raw Image Assembly } \\ \cdashline{1-4}
{\footnotesize DMS-REQ-0068 } &
\multicolumn{3}{p{11.1cm}}{\footnotesize Raw Science Image Metadata } \\ \cdashline{1-4}
{\footnotesize DMS-REQ-0099 } &
\multicolumn{3}{p{11.1cm}}{\footnotesize Level 1 Performance Report Definition } \\ \cdashline{1-4}
{\footnotesize DMS-REQ-0131 } &
\multicolumn{3}{p{11.1cm}}{\footnotesize Calibration Images Available Within Specified Time } \\ \cdashline{1-4}
{\footnotesize DMS-REQ-0165 } &
\multicolumn{3}{p{11.1cm}}{\footnotesize Infrastructure Sizing for "catching up" } \\ \cdashline{1-4}
{\footnotesize DMS-REQ-0167 } &
\multicolumn{3}{p{11.1cm}}{\footnotesize Incorporate Autonomics } \\ \cdashline{1-4}
{\footnotesize DMS-REQ-0265 } &
\multicolumn{3}{p{11.1cm}}{\footnotesize Guider Calibration Data Acquisition } \\ \cdashline{1-4}
{\footnotesize DMS-REQ-0284 } &
\multicolumn{3}{p{11.1cm}}{\footnotesize Level-1 Production Completeness } \\ \cdashline{1-4}
{\footnotesize DMS-REQ-0294 } &
\multicolumn{3}{p{11.1cm}}{\footnotesize Processing of Datasets } \\ \cdashline{1-4}
{\footnotesize DMS-REQ-0301 } &
\multicolumn{3}{p{11.1cm}}{\footnotesize Control of Level-1 Production } \\ \cdashline{1-4}
{\footnotesize DMS-REQ-0309 } &
\multicolumn{3}{p{11.1cm}}{\footnotesize Raw Data Archiving Reliability } \\ \cdashline{1-4}
{\footnotesize DMS-REQ-0314 } &
\multicolumn{3}{p{11.1cm}}{\footnotesize Compute Platform Heterogeneity } \\ \cdashline{1-4}
{\footnotesize DMS-REQ-0315 } &
\multicolumn{3}{p{11.1cm}}{\footnotesize DMS Communication with OCS } \\ \cdashline{1-4}
{\footnotesize DMS-REQ-0318 } &
\multicolumn{3}{p{11.1cm}}{\footnotesize Data Management Unscheduled Downtime } \\ \cdashline{1-4}
{\footnotesize DMS-REQ-0321 } &
\multicolumn{3}{p{11.1cm}}{\footnotesize Level 1 Processing of Special Programs Data } \\ \cdashline{1-4}
{\footnotesize DMS-REQ-0344 } &
\multicolumn{3}{p{11.1cm}}{\footnotesize Constraints on Level 1 Special Program Products Generation } \\ \cdashline{1-4}
{\footnotesize DMS-REQ-0346 } &
\multicolumn{3}{p{11.1cm}}{\footnotesize Data Availability } \\ \cdashline{1-4}
{\footnotesize OCS-DM-COM-ICD-0003 } &
\multicolumn{3}{p{11.1cm}}{\footnotesize Data Management CSC Command Response Model } \\ \cdashline{1-4}
{\footnotesize OCS-DM-COM-ICD-0004 } &
\multicolumn{3}{p{11.1cm}}{\footnotesize Data Management Exposed CSCs } \\ \cdashline{1-4}
{\footnotesize OCS-DM-COM-ICD-0005 } &
\multicolumn{3}{p{11.1cm}}{\footnotesize Main Camera Archiver } \\ \cdashline{1-4}
{\footnotesize OCS-DM-COM-ICD-0006 } &
\multicolumn{3}{p{11.1cm}}{\footnotesize Catch-up Archiver } \\ \cdashline{1-4}
{\footnotesize OCS-DM-COM-ICD-0007 } &
\multicolumn{3}{p{11.1cm}}{\footnotesize Prompt Processing CSC } \\ \cdashline{1-4}
{\footnotesize OCS-DM-COM-ICD-0009 } &
\multicolumn{3}{p{11.1cm}}{\footnotesize Command Set Implementation by Data Management } \\ \cdashline{1-4}
{\footnotesize OCS-DM-COM-ICD-0012 } &
\multicolumn{3}{p{11.1cm}}{\footnotesize Start Command } \\ \cdashline{1-4}
{\footnotesize OCS-DM-COM-ICD-0013 } &
\multicolumn{3}{p{11.1cm}}{\footnotesize configure Successful Completion Response } \\ \cdashline{1-4}
{\footnotesize OCS-DM-COM-ICD-0014 } &
\multicolumn{3}{p{11.1cm}}{\footnotesize enable Command } \\ \cdashline{1-4}
{\footnotesize OCS-DM-COM-ICD-0015 } &
\multicolumn{3}{p{11.1cm}}{\footnotesize disable Command } \\ \cdashline{1-4}
{\footnotesize OCS-DM-COM-ICD-0032 } &
\multicolumn{3}{p{11.1cm}}{\footnotesize Auxiliary Telescope Archiver CSC } \\ \cdashline{1-4}
{\footnotesize OCS-DM-COM-ICD-0036 } &
\multicolumn{3}{p{11.1cm}}{\footnotesize standby Command } \\ \cdashline{1-4}
{\footnotesize OCS-DM-COM-ICD-0037 } &
\multicolumn{3}{p{11.1cm}}{\footnotesize exit Command } \\ \cdashline{1-4}
{\footnotesize OCS-DM-COM-ICD-0038 } &
\multicolumn{3}{p{11.1cm}}{\footnotesize enterControl Command } \\ \cdashline{1-4}
{\footnotesize OCS-DM-COM-ICD-0039 } &
\multicolumn{3}{p{11.1cm}}{\footnotesize enterControl Successful Completion Response } \\ \cdashline{1-4}
{\footnotesize OCS-DM-COM-ICD-0040 } &
\multicolumn{3}{p{11.1cm}}{\footnotesize Command Completion Response } \\ \cdashline{1-4}
{\footnotesize OCS-DM-COM-ICD-0043 } &
\multicolumn{3}{p{11.1cm}}{\footnotesize Image Retrieval for Archiving Event } \\ \cdashline{1-4}
{\footnotesize OCS-DM-COM-ICD-0044 } &
\multicolumn{3}{p{11.1cm}}{\footnotesize Image Retrieval For Processing Event } \\ \cdashline{1-4}
{\footnotesize OCS-DM-COM-ICD-0045 } &
\multicolumn{3}{p{11.1cm}}{\footnotesize Image in OODS Event } \\ \cdashline{1-4}
{\footnotesize OCS-DM-COM-ICD-0055 } &
\multicolumn{3}{p{11.1cm}}{\footnotesize Archiver Resource Availability } \\ \cdashline{1-4}
{\footnotesize OCS-DM-COM-ICD-0056 } &
\multicolumn{3}{p{11.1cm}}{\footnotesize Prompt Processing Resource Availability } \\ \cdashline{1-4}
\end{longtable}

     \newpage
\begin{longtable}{p{3.7cm}p{3.7cm}p{3.7cm}p{3.7cm}}\toprule
\multicolumn{2}{l}{\large \textbf{ \hypertarget{obspub}{Planned Observation Publication SW} } }
& \multicolumn{2}{l}{(product in: Prompt SW
)}
\\ \hline
\textbf{\footnotesize Manager} & \textbf{\footnotesize Owner} &
\textbf{\footnotesize WBS} & \textbf{\footnotesize Team} \\ \hline
\parbox{3.5cm}{
Michelle Butler
\vspace{2mm}%
} &
\begin{tabular}{@{}l@{}}
\parbox{3.5cm}{
Tim Jenness
\vspace{2mm}%
} \\
\parbox{3.5cm}{
Kian-Tat Lim
\vspace{2mm}%
} \\
\end{tabular} &
\begin{tabular}{@{}l@{}}
1.02C.07.08 \\
\end{tabular} & \begin{tabular}{@{}l@{}}
LDF \\
\end{tabular} \\ \hline
\multicolumn{4}{c}{
{\footnotesize ( Plan Obs Pub SW
 - OBSPUB ) }
}\\ \hline
\end{longtable}

Pointing Prediction Publishing software products.


\begin{longtable}{p{3.7cm}p{3.7cm}p{3.7cm}p{3.7cm}}\hline
\textbf{\footnotesize Uses:}  & & \textbf{\footnotesize Used in:} & \\ \hline
\multicolumn{2}{c}{
} &
\multicolumn{2}{c}{
\begin{tabular}{c}
\hyperlink{popsrv}{Planned Observation Publication} \\ \hline
\end{tabular}
} \\ \bottomrule
\multicolumn{4}{c}{\textbf{Related Requirements} } \\ \hline
{\footnotesize DMS-REQ-0353 } &
\multicolumn{3}{p{11.1cm}}{\footnotesize Publishing predicted visit schedule } \\ \cdashline{1-4}
\end{longtable}

     \newpage
\begin{longtable}{p{3.7cm}p{3.7cm}p{3.7cm}p{3.7cm}}\toprule
\multicolumn{2}{l}{\large \textbf{ \hypertarget{ocsbat}{OCS Batch SW} } }
& \multicolumn{2}{l}{(product in: Prompt SW
)}
\\ \hline
\textbf{\footnotesize Manager} & \textbf{\footnotesize Owner} &
\textbf{\footnotesize WBS} & \textbf{\footnotesize Team} \\ \hline
\parbox{3.5cm}{
Michelle Butler
\vspace{2mm}%
} &
\begin{tabular}{@{}l@{}}
\parbox{3.5cm}{
Felipe Menanteau
\vspace{2mm}%
} \\
\end{tabular} &
\begin{tabular}{@{}l@{}}
1.02C.07.08 \\
\end{tabular} & \begin{tabular}{@{}l@{}}
LDF \\
\end{tabular} \\ \hline
\multicolumn{4}{c}{
{\footnotesize ( OCS Batch SW
 - OCSBAT ) }
}\\ \hline
\citeds{DMTN-133} &
\multicolumn{3}{l}{ OCS driven data processing }
\\ \cdashline{1-4}
\end{longtable}

Batch commandable SAL component.


\begin{longtable}{p{3.7cm}p{3.7cm}p{3.7cm}p{3.7cm}}\hline
\textbf{\footnotesize Uses:}  & & \textbf{\footnotesize Used in:} & \\ \hline
\multicolumn{2}{c}{
} &
\multicolumn{2}{c}{
\begin{tabular}{c}
\hyperlink{ocsbatsrv}{OCS-Driven Batch} \\ \hline
\end{tabular}
} \\ \bottomrule
\multicolumn{4}{c}{\textbf{Related Requirements} } \\ \hline
{\footnotesize DMS-REQ-0008 } &
\multicolumn{3}{p{11.1cm}}{\footnotesize Pipeline Availability } \\ \cdashline{1-4}
{\footnotesize DMS-REQ-0101 } &
\multicolumn{3}{p{11.1cm}}{\footnotesize Level 1 Calibration Report Definition } \\ \cdashline{1-4}
{\footnotesize DMS-REQ-0131 } &
\multicolumn{3}{p{11.1cm}}{\footnotesize Calibration Images Available Within Specified Time } \\ \cdashline{1-4}
{\footnotesize DMS-REQ-0162 } &
\multicolumn{3}{p{11.1cm}}{\footnotesize Pipeline Throughput } \\ \cdashline{1-4}
{\footnotesize DMS-REQ-0265 } &
\multicolumn{3}{p{11.1cm}}{\footnotesize Guider Calibration Data Acquisition } \\ \cdashline{1-4}
{\footnotesize DMS-REQ-0289 } &
\multicolumn{3}{p{11.1cm}}{\footnotesize Calibration Production Processing } \\ \cdashline{1-4}
{\footnotesize DMS-REQ-0294 } &
\multicolumn{3}{p{11.1cm}}{\footnotesize Processing of Datasets } \\ \cdashline{1-4}
{\footnotesize DMS-REQ-0301 } &
\multicolumn{3}{p{11.1cm}}{\footnotesize Control of Level-1 Production } \\ \cdashline{1-4}
{\footnotesize DMS-REQ-0314 } &
\multicolumn{3}{p{11.1cm}}{\footnotesize Compute Platform Heterogeneity } \\ \cdashline{1-4}
{\footnotesize DMS-REQ-0315 } &
\multicolumn{3}{p{11.1cm}}{\footnotesize DMS Communication with OCS } \\ \cdashline{1-4}
{\footnotesize DMS-REQ-0318 } &
\multicolumn{3}{p{11.1cm}}{\footnotesize Data Management Unscheduled Downtime } \\ \cdashline{1-4}
{\footnotesize OCS-DM-COM-ICD-0003 } &
\multicolumn{3}{p{11.1cm}}{\footnotesize Data Management CSC Command Response Model } \\ \cdashline{1-4}
{\footnotesize OCS-DM-COM-ICD-0009 } &
\multicolumn{3}{p{11.1cm}}{\footnotesize Command Set Implementation by Data Management } \\ \cdashline{1-4}
{\footnotesize OCS-DM-COM-ICD-0012 } &
\multicolumn{3}{p{11.1cm}}{\footnotesize Start Command } \\ \cdashline{1-4}
{\footnotesize OCS-DM-COM-ICD-0013 } &
\multicolumn{3}{p{11.1cm}}{\footnotesize configure Successful Completion Response } \\ \cdashline{1-4}
{\footnotesize OCS-DM-COM-ICD-0014 } &
\multicolumn{3}{p{11.1cm}}{\footnotesize enable Command } \\ \cdashline{1-4}
{\footnotesize OCS-DM-COM-ICD-0015 } &
\multicolumn{3}{p{11.1cm}}{\footnotesize disable Command } \\ \cdashline{1-4}
{\footnotesize OCS-DM-COM-ICD-0035 } &
\multicolumn{3}{p{11.1cm}}{\footnotesize OCS-Driven Batch CSC } \\ \cdashline{1-4}
{\footnotesize OCS-DM-COM-ICD-0036 } &
\multicolumn{3}{p{11.1cm}}{\footnotesize standby Command } \\ \cdashline{1-4}
{\footnotesize OCS-DM-COM-ICD-0037 } &
\multicolumn{3}{p{11.1cm}}{\footnotesize exit Command } \\ \cdashline{1-4}
{\footnotesize OCS-DM-COM-ICD-0038 } &
\multicolumn{3}{p{11.1cm}}{\footnotesize enterControl Command } \\ \cdashline{1-4}
{\footnotesize OCS-DM-COM-ICD-0039 } &
\multicolumn{3}{p{11.1cm}}{\footnotesize enterControl Successful Completion Response } \\ \cdashline{1-4}
{\footnotesize OCS-DM-COM-ICD-0040 } &
\multicolumn{3}{p{11.1cm}}{\footnotesize Command Completion Response } \\ \cdashline{1-4}
\end{longtable}

     \newpage
\begin{longtable}{p{3.7cm}p{3.7cm}p{3.7cm}p{3.7cm}}\toprule
\multicolumn{2}{l}{\large \textbf{ \hypertarget{oods}{Observatory Operations Data Service SW} } }
& \multicolumn{2}{l}{(product in: Prompt SW
)}
\\ \hline
\textbf{\footnotesize Manager} & \textbf{\footnotesize Owner} &
\textbf{\footnotesize WBS} & \textbf{\footnotesize Team} \\ \hline
\parbox{3.5cm}{
Michelle Butler
\vspace{2mm}%
} &
\begin{tabular}{@{}l@{}}
\parbox{3.5cm}{
Steve Pietrowicz
\vspace{2mm}%
} \\
\end{tabular} &
\begin{tabular}{@{}l@{}}
1.02C.07.08 \\
\end{tabular} & \begin{tabular}{@{}l@{}}
LDF \\
\end{tabular} \\ \hline
\multicolumn{4}{c}{
{\footnotesize ( Obs Ops Data SW
 - OODS ) }
}\\ \hline
\end{longtable}

SW Product to use in the OODS service.


\begin{longtable}{p{3.7cm}p{3.7cm}p{3.7cm}p{3.7cm}}\hline
\multicolumn{2}{r}{\textbf{GutHub Packages:}} &
\multicolumn{2}{l}{\href{https://github.com/lsst-dm/ctrl_oods}{lsst-dm/ctrl\_oods} }\ref{lsst-dm/ctrl_oods}
\\ \hline \\ \hline
\textbf{\footnotesize Uses:}  & & \textbf{\footnotesize Used in:} & \\ \hline
\multicolumn{2}{c}{
} &
\multicolumn{2}{c}{
\begin{tabular}{c}
\hyperlink{oodssrv}{Observatory Operations Data} \\ \hline
\end{tabular}
} \\ \bottomrule
\multicolumn{4}{c}{\textbf{Related Requirements} } \\ \hline
{\footnotesize CA-DM-DAQ-ICD-0052 } &
\multicolumn{3}{p{11.1cm}}{\footnotesize Correction constants for science sensors sourced by Data Management } \\ \cdashline{1-4}
{\footnotesize DM-TS-CON-ICD-0003 } &
\multicolumn{3}{p{11.1cm}}{\footnotesize Wavefront image archive access } \\ \cdashline{1-4}
{\footnotesize DMS-REQ-0167 } &
\multicolumn{3}{p{11.1cm}}{\footnotesize Incorporate Autonomics } \\ \cdashline{1-4}
{\footnotesize DMS-REQ-0314 } &
\multicolumn{3}{p{11.1cm}}{\footnotesize Compute Platform Heterogeneity } \\ \cdashline{1-4}
{\footnotesize DMS-REQ-0318 } &
\multicolumn{3}{p{11.1cm}}{\footnotesize Data Management Unscheduled Downtime } \\ \cdashline{1-4}
{\footnotesize DMS-REQ-0346 } &
\multicolumn{3}{p{11.1cm}}{\footnotesize Data Availability } \\ \cdashline{1-4}
\end{longtable}

    
   \newpage
\subsubsection{Batch Production Products}\label{bpp}
\begin{longtable}{p{3.7cm}p{3.7cm}p{3.7cm}p{3.7cm}}\hline
\textbf{Manager} & \textbf{Owner} & \textbf{WBS} & \textbf{Team} \\ \hline
\parbox{3.5cm}{
Michelle Butler
\vspace{2mm}%
} &
\begin{tabular}{@{}l@{}}
\parbox{3.5cm}{

\vspace{2mm}%
} \\
\end{tabular}
 &
\begin{tabular}{@{}l@{}}
 \\
\end{tabular} &
\begin{tabular}{@{}l@{}}
 \\
\end{tabular} \\ \hline
\multicolumn{4}{c}{
{\footnotesize ( Batch Prod SW
 - BPP ) }
}\\ \hline
\citeds{LDM-534} &
\multicolumn{3}{l}{ LSST Level 2 System Test Specification }
\\ \cdashline{1-4}
\citeds{LDM-562} &
\multicolumn{3}{l}{ Data Management System (DMS) Level 2 System Requirements }
\\ \cdashline{1-4}
\end{longtable}

  Software products to orchestrate the long term data processing.


   \newpage
\begin{longtable}{p{3.7cm}p{3.7cm}p{3.7cm}p{3.7cm}}\toprule
\multicolumn{2}{l}{\large \textbf{ \hypertarget{cmpgn}{Campaign Management} } }
& \multicolumn{2}{l}{(product in: Batch Prod SW
)}
\\ \hline
\textbf{\footnotesize Manager} & \textbf{\footnotesize Owner} &
\textbf{\footnotesize WBS} & \textbf{\footnotesize Team} \\ \hline
\parbox{3.5cm}{
Michelle Butler
\vspace{2mm}%
} &
\begin{tabular}{@{}l@{}}
\parbox{3.5cm}{
Michelle Gower
\vspace{2mm}%
} \\
\end{tabular} &
\begin{tabular}{@{}l@{}}
1.02C.07.08 \\
\end{tabular} & \begin{tabular}{@{}l@{}}
LDF \\
\end{tabular} \\ \hline
\multicolumn{4}{c}{
{\footnotesize ( Campaign Mgmt
 - CMPGN ) }
}\\ \hline
\end{longtable}

This software product orchestrate the run of the science pipeline in a
docker.


\begin{longtable}{p{3.7cm}p{3.7cm}p{3.7cm}p{3.7cm}}\hline
\textbf{\footnotesize Uses:}  & & \textbf{\footnotesize Used in:} & \\ \hline
\multicolumn{2}{c}{
} &
\multicolumn{2}{c}{
} \\ \bottomrule
\multicolumn{4}{c}{\textbf{Related Requirements} } \\ \hline
{\footnotesize DMS-REQ-0008 } &
\multicolumn{3}{p{11.1cm}}{\footnotesize Pipeline Availability } \\ \cdashline{1-4}
{\footnotesize DMS-REQ-0156 } &
\multicolumn{3}{p{11.1cm}}{\footnotesize Provide Pipeline Execution Services } \\ \cdashline{1-4}
{\footnotesize DMS-REQ-0163 } &
\multicolumn{3}{p{11.1cm}}{\footnotesize Re-processing Capacity } \\ \cdashline{1-4}
{\footnotesize DMS-REQ-0167 } &
\multicolumn{3}{p{11.1cm}}{\footnotesize Incorporate Autonomics } \\ \cdashline{1-4}
{\footnotesize DMS-REQ-0294 } &
\multicolumn{3}{p{11.1cm}}{\footnotesize Processing of Datasets } \\ \cdashline{1-4}
{\footnotesize DMS-REQ-0302 } &
\multicolumn{3}{p{11.1cm}}{\footnotesize Production Orchestration } \\ \cdashline{1-4}
{\footnotesize DMS-REQ-0303 } &
\multicolumn{3}{p{11.1cm}}{\footnotesize Production Monitoring } \\ \cdashline{1-4}
{\footnotesize DMS-REQ-0304 } &
\multicolumn{3}{p{11.1cm}}{\footnotesize Production Fault Tolerance } \\ \cdashline{1-4}
{\footnotesize DMS-REQ-0314 } &
\multicolumn{3}{p{11.1cm}}{\footnotesize Compute Platform Heterogeneity } \\ \cdashline{1-4}
{\footnotesize DMS-REQ-0318 } &
\multicolumn{3}{p{11.1cm}}{\footnotesize Data Management Unscheduled Downtime } \\ \cdashline{1-4}
\end{longtable}

     \newpage
\begin{longtable}{p{3.7cm}p{3.7cm}p{3.7cm}p{3.7cm}}\toprule
\multicolumn{2}{l}{\large \textbf{ \hypertarget{wlwf}{Workload/ Workflow Management} } }
& \multicolumn{2}{l}{(product in: Batch Prod SW
)}
\\ \hline
\textbf{\footnotesize Manager} & \textbf{\footnotesize Owner} &
\textbf{\footnotesize WBS} & \textbf{\footnotesize Team} \\ \hline
\parbox{3.5cm}{
Michelle Butler
\vspace{2mm}%
} &
\begin{tabular}{@{}l@{}}
\parbox{3.5cm}{
Michelle Gower
\vspace{2mm}%
} \\
\end{tabular} &
\begin{tabular}{@{}l@{}}
1.02C.07.08 \\
\end{tabular} & \begin{tabular}{@{}l@{}}
LDF \\
\end{tabular} \\ \hline
\multicolumn{4}{c}{
{\footnotesize ( Workload/flow
 - WLWF ) }
}\\ \hline
\end{longtable}

This software product orchestrates the run of the science pipeline in a
docker.


\begin{longtable}{p{3.7cm}p{3.7cm}p{3.7cm}p{3.7cm}}\hline
\textbf{\footnotesize Uses:}  & & \textbf{\footnotesize Used in:} & \\ \hline
\multicolumn{2}{c}{
} &
\multicolumn{2}{c}{
\begin{tabular}{c}
\hyperlink{prodsrv}{Batch Production} \\ \hline
\end{tabular}
} \\ \bottomrule
\multicolumn{4}{c}{\textbf{Related Requirements} } \\ \hline
{\footnotesize DMS-REQ-0008 } &
\multicolumn{3}{p{11.1cm}}{\footnotesize Pipeline Availability } \\ \cdashline{1-4}
{\footnotesize DMS-REQ-0156 } &
\multicolumn{3}{p{11.1cm}}{\footnotesize Provide Pipeline Execution Services } \\ \cdashline{1-4}
{\footnotesize DMS-REQ-0163 } &
\multicolumn{3}{p{11.1cm}}{\footnotesize Re-processing Capacity } \\ \cdashline{1-4}
{\footnotesize DMS-REQ-0167 } &
\multicolumn{3}{p{11.1cm}}{\footnotesize Incorporate Autonomics } \\ \cdashline{1-4}
{\footnotesize DMS-REQ-0294 } &
\multicolumn{3}{p{11.1cm}}{\footnotesize Processing of Datasets } \\ \cdashline{1-4}
{\footnotesize DMS-REQ-0302 } &
\multicolumn{3}{p{11.1cm}}{\footnotesize Production Orchestration } \\ \cdashline{1-4}
{\footnotesize DMS-REQ-0303 } &
\multicolumn{3}{p{11.1cm}}{\footnotesize Production Monitoring } \\ \cdashline{1-4}
{\footnotesize DMS-REQ-0304 } &
\multicolumn{3}{p{11.1cm}}{\footnotesize Production Fault Tolerance } \\ \cdashline{1-4}
{\footnotesize DMS-REQ-0308 } &
\multicolumn{3}{p{11.1cm}}{\footnotesize Software Architecture to Enable Community Re-Use } \\ \cdashline{1-4}
{\footnotesize DMS-REQ-0314 } &
\multicolumn{3}{p{11.1cm}}{\footnotesize Compute Platform Heterogeneity } \\ \cdashline{1-4}
{\footnotesize DMS-REQ-0318 } &
\multicolumn{3}{p{11.1cm}}{\footnotesize Data Management Unscheduled Downtime } \\ \cdashline{1-4}
{\footnotesize DMS-REQ-0386 } &
\multicolumn{3}{p{11.1cm}}{\footnotesize a Archive Processing Provenance } \\ \cdashline{1-4}
{\footnotesize DMS-REQ-0388 } &
\multicolumn{3}{p{11.1cm}}{\footnotesize Provide Re-Run Tools } \\ \cdashline{1-4}
{\footnotesize DMS-REQ-0389 } &
\multicolumn{3}{p{11.1cm}}{\footnotesize Re-Runs on Similar Systems } \\ \cdashline{1-4}
{\footnotesize DMS-REQ-0390 } &
\multicolumn{3}{p{11.1cm}}{\footnotesize Re-Runs on Other Systems } \\ \cdashline{1-4}
\end{longtable}

    
   \newpage
\subsubsection{Quality Control Products}\label{qc}
\begin{longtable}{p{3.7cm}p{3.7cm}p{3.7cm}p{3.7cm}}\hline
\textbf{Manager} & \textbf{Owner} & \textbf{WBS} & \textbf{Team} \\ \hline
\parbox{3.5cm}{
Frossie Economou
\vspace{2mm}%
} &
\begin{tabular}{@{}l@{}}
\parbox{3.5cm}{

\vspace{2mm}%
} \\
\end{tabular}
 &
\begin{tabular}{@{}l@{}}
 \\
\end{tabular} &
\begin{tabular}{@{}l@{}}
 \\
\end{tabular} \\ \hline
\multicolumn{4}{c}{
{\footnotesize ( QC Products
 - QC ) }
}\\ \hline
\end{longtable}

  SW products for quality control.


   \newpage
\begin{longtable}{p{3.7cm}p{3.7cm}p{3.7cm}p{3.7cm}}\toprule
\multicolumn{2}{l}{\large \textbf{ \hypertarget{qcsw}{Quality Control SW} } }
& \multicolumn{2}{l}{(product in: QC Products
)}
\\ \hline
\textbf{\footnotesize Manager} & \textbf{\footnotesize Owner} &
\textbf{\footnotesize WBS} & \textbf{\footnotesize Team} \\ \hline
\parbox{3.5cm}{
Frossie Economou
\vspace{2mm}%
} &
\begin{tabular}{@{}l@{}}
\parbox{3.5cm}{
Simon Krughoff
\vspace{2mm}%
} \\
\end{tabular} &
\begin{tabular}{@{}l@{}}
1.02C.10.02.01 \\
\end{tabular} & \begin{tabular}{@{}l@{}}
SQuaRE \\
\end{tabular} \\ \hline
\multicolumn{4}{c}{
{\footnotesize ( Quality Ctrl SW
 - QCSW ) }
}\\ \hline
\end{longtable}

This software product is used to instantiate the quality control
services.


\begin{longtable}{p{3.7cm}p{3.7cm}p{3.7cm}p{3.7cm}}\hline
\multicolumn{2}{r}{\textbf{GutHub Packages:}} &
\multicolumn{2}{l}{\href{https://github.com/lsst-sqre/squash}{lsst-sqre/squash} }\ref{lsst-sqre/squash}
\\ \hline \\ \hline
\textbf{\footnotesize Uses:}  & & \textbf{\footnotesize Used in:} & \\ \hline
\multicolumn{2}{c}{
} &
\multicolumn{2}{c}{
\begin{tabular}{c}
\hyperlink{offlqcsrv}{Offline Quality Control} \\ \hline
\hyperlink{prqcsrv}{Prompt Quality Control} \\ \hline
\end{tabular}
} \\ \bottomrule
\multicolumn{4}{c}{\textbf{Related Requirements} } \\ \hline
{\footnotesize DMS-REQ-0096 } &
\multicolumn{3}{p{11.1cm}}{\footnotesize Generate Data Quality Report Within Specified Time } \\ \cdashline{1-4}
{\footnotesize DMS-REQ-0097 } &
\multicolumn{3}{p{11.1cm}}{\footnotesize Level 1 Data Quality Report Definition } \\ \cdashline{1-4}
{\footnotesize DMS-REQ-0098 } &
\multicolumn{3}{p{11.1cm}}{\footnotesize Generate DMS Performance Report Within Specified Time } \\ \cdashline{1-4}
{\footnotesize DMS-REQ-0099 } &
\multicolumn{3}{p{11.1cm}}{\footnotesize Level 1 Performance Report Definition } \\ \cdashline{1-4}
{\footnotesize DMS-REQ-0100 } &
\multicolumn{3}{p{11.1cm}}{\footnotesize Generate Calibration Report Within Specified Time } \\ \cdashline{1-4}
{\footnotesize DMS-REQ-0101 } &
\multicolumn{3}{p{11.1cm}}{\footnotesize Level 1 Calibration Report Definition } \\ \cdashline{1-4}
{\footnotesize DMS-REQ-0314 } &
\multicolumn{3}{p{11.1cm}}{\footnotesize Compute Platform Heterogeneity } \\ \cdashline{1-4}
{\footnotesize DMS-REQ-0318 } &
\multicolumn{3}{p{11.1cm}}{\footnotesize Data Management Unscheduled Downtime } \\ \cdashline{1-4}
\end{longtable}

    
   \newpage
\subsubsection{Backbone SW Products}\label{dbb}
\begin{longtable}{p{3.7cm}p{3.7cm}p{3.7cm}p{3.7cm}}\hline
\textbf{Manager} & \textbf{Owner} & \textbf{WBS} & \textbf{Team} \\ \hline
\parbox{3.5cm}{
Michelle Butler
\vspace{2mm}%
} &
\begin{tabular}{@{}l@{}}
\parbox{3.5cm}{

\vspace{2mm}%
} \\
\end{tabular}
 &
\begin{tabular}{@{}l@{}}
 \\
\end{tabular} &
\begin{tabular}{@{}l@{}}
 \\
\end{tabular} \\ \hline
\multicolumn{4}{c}{
{\footnotesize ( DBB SW
 - DBB ) }
}\\ \hline
\end{longtable}

  Software products that implement the Data Backbone services


   \newpage
\begin{longtable}{p{3.7cm}p{3.7cm}p{3.7cm}p{3.7cm}}\toprule
\multicolumn{2}{l}{\large \textbf{ \hypertarget{dbblife}{DBB Lifetime Management SW} } }
& \multicolumn{2}{l}{(product in: DBB SW
)}
\\ \hline
\textbf{\footnotesize Manager} & \textbf{\footnotesize Owner} &
\textbf{\footnotesize WBS} & \textbf{\footnotesize Team} \\ \hline
\parbox{3.5cm}{
Michelle Butler
\vspace{2mm}%
} &
\begin{tabular}{@{}l@{}}
\parbox{3.5cm}{
Michelle Gower
\vspace{2mm}%
} \\
\end{tabular} &
\begin{tabular}{@{}l@{}}
1.02C.07.08 \\
\end{tabular} & \begin{tabular}{@{}l@{}}
LDF \\
\end{tabular} \\ \hline
\multicolumn{4}{c}{
{\footnotesize ( DBB Lifetime SW
 - DBBLIFE ) }
}\\ \hline
\end{longtable}

This software product implements the management of the data in the
sciences storage.


\begin{longtable}{p{3.7cm}p{3.7cm}p{3.7cm}p{3.7cm}}\hline
\textbf{\footnotesize Uses:}  & & \textbf{\footnotesize Used in:} & \\ \hline
\multicolumn{2}{c}{
} &
\multicolumn{2}{c}{
\begin{tabular}{c}
\hyperlink{dbblifesrv}{DBB Lifetime Management} \\ \hline
\end{tabular}
} \\ \bottomrule
\multicolumn{4}{c}{\textbf{Related Requirements} } \\ \hline
{\footnotesize DMS-REQ-0310 } &
\multicolumn{3}{p{11.1cm}}{\footnotesize Un-Archived Data Product Cache } \\ \cdashline{1-4}
{\footnotesize DMS-REQ-0334 } &
\multicolumn{3}{p{11.1cm}}{\footnotesize Persisting Data Products } \\ \cdashline{1-4}
{\footnotesize DMS-REQ-0338 } &
\multicolumn{3}{p{11.1cm}}{\footnotesize Targeted Coadds } \\ \cdashline{1-4}
{\footnotesize DMS-REQ-0339 } &
\multicolumn{3}{p{11.1cm}}{\footnotesize Tracking Characterization Changes Between Data Releases } \\ \cdashline{1-4}
{\footnotesize DMS-REQ-0346 } &
\multicolumn{3}{p{11.1cm}}{\footnotesize Data Availability } \\ \cdashline{1-4}
\end{longtable}

     \newpage
\begin{longtable}{p{3.7cm}p{3.7cm}p{3.7cm}p{3.7cm}}\toprule
\multicolumn{2}{l}{\large \textbf{ \hypertarget{dbbmd}{DBB Ingest/ Metadata Management SW} } }
& \multicolumn{2}{l}{(product in: DBB SW
)}
\\ \hline
\textbf{\footnotesize Manager} & \textbf{\footnotesize Owner} &
\textbf{\footnotesize WBS} & \textbf{\footnotesize Team} \\ \hline
\parbox{3.5cm}{
Michelle Butler
\vspace{2mm}%
} &
\begin{tabular}{@{}l@{}}
\parbox{3.5cm}{
Michelle Gower
\vspace{2mm}%
} \\
\end{tabular} &
\begin{tabular}{@{}l@{}}
1.02C.07.08 \\
\end{tabular} & \begin{tabular}{@{}l@{}}
LDF \\
\end{tabular} \\ \hline
\multicolumn{4}{c}{
{\footnotesize ( DBB Meta SW
 - DBBMD ) }
}\\ \hline
\end{longtable}

Listener in the Endpoint Data Backbone Enclave to provide ingestion
services from the Facilities enclaves to the backbone.


\begin{longtable}{p{3.7cm}p{3.7cm}p{3.7cm}p{3.7cm}}\hline
\multicolumn{2}{r}{\textbf{GutHub Packages:}} &
\multicolumn{2}{l}{\href{https://github.com/lsst-dm/dbb_gwclient}{lsst-dm/dbb\_gwclient} }\ref{lsst-dm/dbb_gwclient}
\\ \cline{3-4}
& & \multicolumn{2}{l}{\href{https://github.com/lsst-dm/dbb_gateway}{lsst-dm/dbb\_gateway} }\ref{lsst-dm/dbb_gateway}
\\ \hline \\ \hline
\textbf{\footnotesize Uses:}  & & \textbf{\footnotesize Used in:} & \\ \hline
\multicolumn{2}{c}{
} &
\multicolumn{2}{c}{
\begin{tabular}{c}
\hyperlink{dbbmdsrv}{DBB Ingest/ Metadata Management} \\ \hline
\end{tabular}
} \\ \bottomrule
\multicolumn{4}{c}{\textbf{Related Requirements} } \\ \hline
{\footnotesize DMS-REQ-0008 } &
\multicolumn{3}{p{11.1cm}}{\footnotesize Pipeline Availability } \\ \cdashline{1-4}
{\footnotesize DMS-REQ-0068 } &
\multicolumn{3}{p{11.1cm}}{\footnotesize Raw Science Image Metadata } \\ \cdashline{1-4}
{\footnotesize DMS-REQ-0074 } &
\multicolumn{3}{p{11.1cm}}{\footnotesize Difference Exposure Attributes } \\ \cdashline{1-4}
{\footnotesize DMS-REQ-0077 } &
\multicolumn{3}{p{11.1cm}}{\footnotesize Maintain Archive Publicly Accessible } \\ \cdashline{1-4}
{\footnotesize DMS-REQ-0089 } &
\multicolumn{3}{p{11.1cm}}{\footnotesize Solar System Objects Available Within Specified Time } \\ \cdashline{1-4}
{\footnotesize DMS-REQ-0094 } &
\multicolumn{3}{p{11.1cm}}{\footnotesize Keep Historical Alert Archive } \\ \cdashline{1-4}
{\footnotesize DMS-REQ-0102 } &
\multicolumn{3}{p{11.1cm}}{\footnotesize Provide Engineering \&  Facility Database Archive } \\ \cdashline{1-4}
{\footnotesize DMS-REQ-0120 } &
\multicolumn{3}{p{11.1cm}}{\footnotesize Level 3 Data Product Self Consistency } \\ \cdashline{1-4}
{\footnotesize DMS-REQ-0122 } &
\multicolumn{3}{p{11.1cm}}{\footnotesize Access to catalogs for external Level 3 processing } \\ \cdashline{1-4}
{\footnotesize DMS-REQ-0123 } &
\multicolumn{3}{p{11.1cm}}{\footnotesize Access to input catalogs for DAC-based Level 3 processing } \\ \cdashline{1-4}
{\footnotesize DMS-REQ-0126 } &
\multicolumn{3}{p{11.1cm}}{\footnotesize Access to images for external Level 3 processing } \\ \cdashline{1-4}
{\footnotesize DMS-REQ-0127 } &
\multicolumn{3}{p{11.1cm}}{\footnotesize Access to input images for DAC-based Level 3 processing } \\ \cdashline{1-4}
{\footnotesize DMS-REQ-0130 } &
\multicolumn{3}{p{11.1cm}}{\footnotesize Calibration Data Products } \\ \cdashline{1-4}
{\footnotesize DMS-REQ-0131 } &
\multicolumn{3}{p{11.1cm}}{\footnotesize Calibration Images Available Within Specified Time } \\ \cdashline{1-4}
{\footnotesize DMS-REQ-0132 } &
\multicolumn{3}{p{11.1cm}}{\footnotesize Calibration Image Provenance } \\ \cdashline{1-4}
{\footnotesize DMS-REQ-0161 } &
\multicolumn{3}{p{11.1cm}}{\footnotesize Optimization of Cost, Reliability and Availability in Order } \\ \cdashline{1-4}
{\footnotesize DMS-REQ-0162 } &
\multicolumn{3}{p{11.1cm}}{\footnotesize Pipeline Throughput } \\ \cdashline{1-4}
{\footnotesize DMS-REQ-0163 } &
\multicolumn{3}{p{11.1cm}}{\footnotesize Re-processing Capacity } \\ \cdashline{1-4}
{\footnotesize DMS-REQ-0164 } &
\multicolumn{3}{p{11.1cm}}{\footnotesize Temporary Storage for Communications Links } \\ \cdashline{1-4}
{\footnotesize DMS-REQ-0165 } &
\multicolumn{3}{p{11.1cm}}{\footnotesize Infrastructure Sizing for "catching up" } \\ \cdashline{1-4}
{\footnotesize DMS-REQ-0166 } &
\multicolumn{3}{p{11.1cm}}{\footnotesize Incorporate Fault-Tolerance } \\ \cdashline{1-4}
{\footnotesize DMS-REQ-0167 } &
\multicolumn{3}{p{11.1cm}}{\footnotesize Incorporate Autonomics } \\ \cdashline{1-4}
{\footnotesize DMS-REQ-0176 } &
\multicolumn{3}{p{11.1cm}}{\footnotesize Base Facility Infrastructure } \\ \cdashline{1-4}
{\footnotesize DMS-REQ-0185 } &
\multicolumn{3}{p{11.1cm}}{\footnotesize Archive Center } \\ \cdashline{1-4}
{\footnotesize DMS-REQ-0186 } &
\multicolumn{3}{p{11.1cm}}{\footnotesize Archive Center Disaster Recovery } \\ \cdashline{1-4}
{\footnotesize DMS-REQ-0197 } &
\multicolumn{3}{p{11.1cm}}{\footnotesize No Limit on Data Access Centers } \\ \cdashline{1-4}
{\footnotesize DMS-REQ-0266 } &
\multicolumn{3}{p{11.1cm}}{\footnotesize Exposure Catalog } \\ \cdashline{1-4}
{\footnotesize DMS-REQ-0269 } &
\multicolumn{3}{p{11.1cm}}{\footnotesize DIASource Catalog } \\ \cdashline{1-4}
{\footnotesize DMS-REQ-0271 } &
\multicolumn{3}{p{11.1cm}}{\footnotesize DIAObject Catalog } \\ \cdashline{1-4}
{\footnotesize DMS-REQ-0273 } &
\multicolumn{3}{p{11.1cm}}{\footnotesize SSObject Catalog } \\ \cdashline{1-4}
{\footnotesize DMS-REQ-0287 } &
\multicolumn{3}{p{11.1cm}}{\footnotesize DIASource Precovery } \\ \cdashline{1-4}
{\footnotesize DMS-REQ-0291 } &
\multicolumn{3}{p{11.1cm}}{\footnotesize Query Repeatability } \\ \cdashline{1-4}
{\footnotesize DMS-REQ-0292 } &
\multicolumn{3}{p{11.1cm}}{\footnotesize Uniqueness of IDs Across Data Releases } \\ \cdashline{1-4}
{\footnotesize DMS-REQ-0293 } &
\multicolumn{3}{p{11.1cm}}{\footnotesize Selection of Datasets } \\ \cdashline{1-4}
{\footnotesize DMS-REQ-0299 } &
\multicolumn{3}{p{11.1cm}}{\footnotesize Data Product Ingest } \\ \cdashline{1-4}
{\footnotesize DMS-REQ-0309 } &
\multicolumn{3}{p{11.1cm}}{\footnotesize Raw Data Archiving Reliability } \\ \cdashline{1-4}
{\footnotesize DMS-REQ-0310 } &
\multicolumn{3}{p{11.1cm}}{\footnotesize Un-Archived Data Product Cache } \\ \cdashline{1-4}
{\footnotesize DMS-REQ-0313 } &
\multicolumn{3}{p{11.1cm}}{\footnotesize Level 1 \&  2 Catalog Access } \\ \cdashline{1-4}
{\footnotesize DMS-REQ-0314 } &
\multicolumn{3}{p{11.1cm}}{\footnotesize Compute Platform Heterogeneity } \\ \cdashline{1-4}
{\footnotesize DMS-REQ-0317 } &
\multicolumn{3}{p{11.1cm}}{\footnotesize DIAForcedSource Catalog } \\ \cdashline{1-4}
{\footnotesize DMS-REQ-0318 } &
\multicolumn{3}{p{11.1cm}}{\footnotesize Data Management Unscheduled Downtime } \\ \cdashline{1-4}
{\footnotesize DMS-REQ-0322 } &
\multicolumn{3}{p{11.1cm}}{\footnotesize Special Programs Database } \\ \cdashline{1-4}
{\footnotesize DMS-REQ-0334 } &
\multicolumn{3}{p{11.1cm}}{\footnotesize Persisting Data Products } \\ \cdashline{1-4}
{\footnotesize DMS-REQ-0338 } &
\multicolumn{3}{p{11.1cm}}{\footnotesize Targeted Coadds } \\ \cdashline{1-4}
{\footnotesize DMS-REQ-0339 } &
\multicolumn{3}{p{11.1cm}}{\footnotesize Tracking Characterization Changes Between Data Releases } \\ \cdashline{1-4}
{\footnotesize DMS-REQ-0346 } &
\multicolumn{3}{p{11.1cm}}{\footnotesize Data Availability } \\ \cdashline{1-4}
{\footnotesize DMS-REQ-0363 } &
\multicolumn{3}{p{11.1cm}}{\footnotesize Access to Previous Data Releases } \\ \cdashline{1-4}
{\footnotesize DMS-REQ-0364 } &
\multicolumn{3}{p{11.1cm}}{\footnotesize Data Access Services } \\ \cdashline{1-4}
{\footnotesize DMS-REQ-0365 } &
\multicolumn{3}{p{11.1cm}}{\footnotesize Operations Subsets } \\ \cdashline{1-4}
{\footnotesize DMS-REQ-0366 } &
\multicolumn{3}{p{11.1cm}}{\footnotesize Subsets Support } \\ \cdashline{1-4}
{\footnotesize DMS-REQ-0369 } &
\multicolumn{3}{p{11.1cm}}{\footnotesize Evolution } \\ \cdashline{1-4}
{\footnotesize DMS-REQ-0370 } &
\multicolumn{3}{p{11.1cm}}{\footnotesize Older Release Behavior } \\ \cdashline{1-4}
{\footnotesize OCS-DM-COM-ICD-0047 } &
\multicolumn{3}{p{11.1cm}}{\footnotesize Image Archived Event } \\ \cdashline{1-4}
\end{longtable}

     \newpage
\begin{longtable}{p{3.7cm}p{3.7cm}p{3.7cm}p{3.7cm}}\toprule
\multicolumn{2}{l}{\large \textbf{ \hypertarget{dbbtr}{DBB Transport/ Replication/ Backup SW} } }
& \multicolumn{2}{l}{(product in: DBB SW
)}
\\ \hline
\textbf{\footnotesize Manager} & \textbf{\footnotesize Owner} &
\textbf{\footnotesize WBS} & \textbf{\footnotesize Team} \\ \hline
\parbox{3.5cm}{
Michelle Butler
\vspace{2mm}%
} &
\begin{tabular}{@{}l@{}}
\parbox{3.5cm}{
Michelle Gower
\vspace{2mm}%
} \\
\end{tabular} &
\begin{tabular}{@{}l@{}}
1.02C.07.08 \\
\end{tabular} & \begin{tabular}{@{}l@{}}
LDF \\
\end{tabular} \\ \hline
\multicolumn{4}{c}{
{\footnotesize ( DBB Transport SW
 - DBBTR ) }
}\\ \hline
\end{longtable}

This software products will be used in the services to replicate data
from different facilities and to send it to tape for long time
preservation.


\begin{longtable}{p{3.7cm}p{3.7cm}p{3.7cm}p{3.7cm}}\hline
\textbf{\footnotesize Uses:}  & & \textbf{\footnotesize Used in:} & \\ \hline
\multicolumn{2}{c}{
} &
\multicolumn{2}{c}{
\begin{tabular}{c}
\hyperlink{dbbtrsrv}{DBB Transport/ Replication/ Backup} \\ \hline
\hyperlink{dbbstrsrv}{DBB Storage} \\ \hline
\end{tabular}
} \\ \bottomrule
\multicolumn{4}{c}{\textbf{Related Requirements} } \\ \hline
{\footnotesize DMS-REQ-0008 } &
\multicolumn{3}{p{11.1cm}}{\footnotesize Pipeline Availability } \\ \cdashline{1-4}
{\footnotesize DMS-REQ-0089 } &
\multicolumn{3}{p{11.1cm}}{\footnotesize Solar System Objects Available Within Specified Time } \\ \cdashline{1-4}
{\footnotesize DMS-REQ-0102 } &
\multicolumn{3}{p{11.1cm}}{\footnotesize Provide Engineering \&  Facility Database Archive } \\ \cdashline{1-4}
{\footnotesize DMS-REQ-0122 } &
\multicolumn{3}{p{11.1cm}}{\footnotesize Access to catalogs for external Level 3 processing } \\ \cdashline{1-4}
{\footnotesize DMS-REQ-0123 } &
\multicolumn{3}{p{11.1cm}}{\footnotesize Access to input catalogs for DAC-based Level 3 processing } \\ \cdashline{1-4}
{\footnotesize DMS-REQ-0126 } &
\multicolumn{3}{p{11.1cm}}{\footnotesize Access to images for external Level 3 processing } \\ \cdashline{1-4}
{\footnotesize DMS-REQ-0127 } &
\multicolumn{3}{p{11.1cm}}{\footnotesize Access to input images for DAC-based Level 3 processing } \\ \cdashline{1-4}
{\footnotesize DMS-REQ-0131 } &
\multicolumn{3}{p{11.1cm}}{\footnotesize Calibration Images Available Within Specified Time } \\ \cdashline{1-4}
{\footnotesize DMS-REQ-0161 } &
\multicolumn{3}{p{11.1cm}}{\footnotesize Optimization of Cost, Reliability and Availability in Order } \\ \cdashline{1-4}
{\footnotesize DMS-REQ-0162 } &
\multicolumn{3}{p{11.1cm}}{\footnotesize Pipeline Throughput } \\ \cdashline{1-4}
{\footnotesize DMS-REQ-0163 } &
\multicolumn{3}{p{11.1cm}}{\footnotesize Re-processing Capacity } \\ \cdashline{1-4}
{\footnotesize DMS-REQ-0164 } &
\multicolumn{3}{p{11.1cm}}{\footnotesize Temporary Storage for Communications Links } \\ \cdashline{1-4}
{\footnotesize DMS-REQ-0165 } &
\multicolumn{3}{p{11.1cm}}{\footnotesize Infrastructure Sizing for "catching up" } \\ \cdashline{1-4}
{\footnotesize DMS-REQ-0166 } &
\multicolumn{3}{p{11.1cm}}{\footnotesize Incorporate Fault-Tolerance } \\ \cdashline{1-4}
{\footnotesize DMS-REQ-0167 } &
\multicolumn{3}{p{11.1cm}}{\footnotesize Incorporate Autonomics } \\ \cdashline{1-4}
{\footnotesize DMS-REQ-0185 } &
\multicolumn{3}{p{11.1cm}}{\footnotesize Archive Center } \\ \cdashline{1-4}
{\footnotesize DMS-REQ-0186 } &
\multicolumn{3}{p{11.1cm}}{\footnotesize Archive Center Disaster Recovery } \\ \cdashline{1-4}
{\footnotesize DMS-REQ-0197 } &
\multicolumn{3}{p{11.1cm}}{\footnotesize No Limit on Data Access Centers } \\ \cdashline{1-4}
{\footnotesize DMS-REQ-0287 } &
\multicolumn{3}{p{11.1cm}}{\footnotesize DIASource Precovery } \\ \cdashline{1-4}
{\footnotesize DMS-REQ-0309 } &
\multicolumn{3}{p{11.1cm}}{\footnotesize Raw Data Archiving Reliability } \\ \cdashline{1-4}
{\footnotesize DMS-REQ-0313 } &
\multicolumn{3}{p{11.1cm}}{\footnotesize Level 1 \&  2 Catalog Access } \\ \cdashline{1-4}
{\footnotesize DMS-REQ-0314 } &
\multicolumn{3}{p{11.1cm}}{\footnotesize Compute Platform Heterogeneity } \\ \cdashline{1-4}
{\footnotesize DMS-REQ-0318 } &
\multicolumn{3}{p{11.1cm}}{\footnotesize Data Management Unscheduled Downtime } \\ \cdashline{1-4}
{\footnotesize DMS-REQ-0344 } &
\multicolumn{3}{p{11.1cm}}{\footnotesize Constraints on Level 1 Special Program Products Generation } \\ \cdashline{1-4}
{\footnotesize DMS-REQ-0366 } &
\multicolumn{3}{p{11.1cm}}{\footnotesize Subsets Support } \\ \cdashline{1-4}
{\footnotesize DMS-REQ-0370 } &
\multicolumn{3}{p{11.1cm}}{\footnotesize Older Release Behavior } \\ \cdashline{1-4}
\end{longtable}

    
   \newpage
\subsubsection{LSP SW Products}\label{lsp}
\begin{longtable}{p{3.7cm}p{3.7cm}p{3.7cm}p{3.7cm}}\hline
\textbf{Manager} & \textbf{Owner} & \textbf{WBS} & \textbf{Team} \\ \hline
\parbox{3.5cm}{
Frossie Economou
\vspace{2mm}%
} &
\begin{tabular}{@{}l@{}}
\parbox{3.5cm}{
Gregory Dubois-Felsmann
\vspace{2mm}%
} \\
\end{tabular}
 &
\begin{tabular}{@{}l@{}}
 \\
\end{tabular} &
\begin{tabular}{@{}l@{}}
 \\
\end{tabular} \\ \hline
\multicolumn{4}{c}{
{\footnotesize ( LSP SW
 - LSP ) }
}\\ \hline
\citeds{LDM-542} &
\multicolumn{3}{l}{ Science Platform Design }
\\ \cdashline{1-4}
\citeds{LDM-554} &
\multicolumn{3}{l}{ Data Management LSST Science Platform Requirements }
\\ \cdashline{1-4}
\citeds{LDM-540} &
\multicolumn{3}{l}{ LSST Science Platform Test Specification }
\\ \cdashline{1-4}
\end{longtable}

  Software products that implement the LSP services.


   \newpage
\begin{longtable}{p{3.7cm}p{3.7cm}p{3.7cm}p{3.7cm}}\toprule
\multicolumn{2}{l}{\large \textbf{ \hypertarget{suit}{SUIT} } }
& \multicolumn{2}{l}{(product in: LSP SW
)}
\\ \hline
\textbf{\footnotesize Manager} & \textbf{\footnotesize Owner} &
\textbf{\footnotesize WBS} & \textbf{\footnotesize Team} \\ \hline
\parbox{3.5cm}{
Xiuqin Wu
\vspace{2mm}%
} &
\begin{tabular}{@{}l@{}}
\parbox{3.5cm}{
Gregory Dubois-Felsmann
\vspace{2mm}%
} \\
\end{tabular} &
\begin{tabular}{@{}l@{}}
1.02C.05.09 \\
1.02C.05.07 \\
1.02C.05.08 \\
\end{tabular} & \begin{tabular}{@{}l@{}}
SUIT \\
\end{tabular} \\ \hline
\multicolumn{4}{c}{
{\footnotesize ( SUIT
 - SUIT ) }
}\\ \hline
\end{longtable}

Implements LSST Portal Aspect-specific behaviors added to the core
Firefly library. Defines the Portal web application. Primarily
JavaScript but also contains Java extensions (search processors) to the
Firefly server side.


\begin{longtable}{p{3.7cm}p{3.7cm}p{3.7cm}p{3.7cm}}\hline
\multicolumn{2}{r}{\textbf{GutHub Packages:}} &
\multicolumn{2}{l}{\href{https://github.com/lsst/suit}{suit} }\ref{suit}
\\ \hline \\ \hline
\textbf{\footnotesize Uses:}  & & \textbf{\footnotesize Used in:} & \\ \hline
\multicolumn{2}{c}{
\begin{tabular}{c}
\hyperlink{firefly}{Firefly} \\ \hline
\end{tabular}
} &
\multicolumn{2}{c}{
\begin{tabular}{c}
\hyperlink{lspprtlsrv}{LSP Portal} \\ \hline
\end{tabular}
} \\ \bottomrule
\multicolumn{4}{c}{\textbf{Related Requirements} } \\ \hline
{\footnotesize DMS-REQ-0119 } &
\multicolumn{3}{p{11.1cm}}{\footnotesize DAC resource allocation for Level 3 processing } \\ \cdashline{1-4}
{\footnotesize DMS-REQ-0123 } &
\multicolumn{3}{p{11.1cm}}{\footnotesize Access to input catalogs for DAC-based Level 3 processing } \\ \cdashline{1-4}
{\footnotesize DMS-REQ-0124 } &
\multicolumn{3}{p{11.1cm}}{\footnotesize Federation with external catalogs } \\ \cdashline{1-4}
{\footnotesize DMS-REQ-0127 } &
\multicolumn{3}{p{11.1cm}}{\footnotesize Access to input images for DAC-based Level 3 processing } \\ \cdashline{1-4}
{\footnotesize DMS-REQ-0160 } &
\multicolumn{3}{p{11.1cm}}{\footnotesize Provide User Interface Services } \\ \cdashline{1-4}
{\footnotesize DMS-REQ-0161 } &
\multicolumn{3}{p{11.1cm}}{\footnotesize Optimization of Cost, Reliability and Availability in Order } \\ \cdashline{1-4}
{\footnotesize DMS-REQ-0193 } &
\multicolumn{3}{p{11.1cm}}{\footnotesize Data Access Centers } \\ \cdashline{1-4}
{\footnotesize DMS-REQ-0194 } &
\multicolumn{3}{p{11.1cm}}{\footnotesize Data Access Center Simultaneous Connections } \\ \cdashline{1-4}
{\footnotesize DMS-REQ-0197 } &
\multicolumn{3}{p{11.1cm}}{\footnotesize No Limit on Data Access Centers } \\ \cdashline{1-4}
{\footnotesize DMS-REQ-0314 } &
\multicolumn{3}{p{11.1cm}}{\footnotesize Compute Platform Heterogeneity } \\ \cdashline{1-4}
{\footnotesize DMS-REQ-0318 } &
\multicolumn{3}{p{11.1cm}}{\footnotesize Data Management Unscheduled Downtime } \\ \cdashline{1-4}
{\footnotesize DMS-REQ-0341 } &
\multicolumn{3}{p{11.1cm}}{\footnotesize Providing a Precovery Service } \\ \cdashline{1-4}
{\footnotesize DMS-REQ-0363 } &
\multicolumn{3}{p{11.1cm}}{\footnotesize Access to Previous Data Releases } \\ \cdashline{1-4}
{\footnotesize DMS-REQ-0364 } &
\multicolumn{3}{p{11.1cm}}{\footnotesize Data Access Services } \\ \cdashline{1-4}
{\footnotesize DMS-REQ-0365 } &
\multicolumn{3}{p{11.1cm}}{\footnotesize Operations Subsets } \\ \cdashline{1-4}
{\footnotesize DMS-REQ-0366 } &
\multicolumn{3}{p{11.1cm}}{\footnotesize Subsets Support } \\ \cdashline{1-4}
{\footnotesize DMS-REQ-0367 } &
\multicolumn{3}{p{11.1cm}}{\footnotesize Access Services Performance } \\ \cdashline{1-4}
{\footnotesize DMS-REQ-0368 } &
\multicolumn{3}{p{11.1cm}}{\footnotesize Implementation Provisions } \\ \cdashline{1-4}
{\footnotesize DMS-REQ-0369 } &
\multicolumn{3}{p{11.1cm}}{\footnotesize Evolution } \\ \cdashline{1-4}
{\footnotesize DMS-REQ-0370 } &
\multicolumn{3}{p{11.1cm}}{\footnotesize Older Release Behavior } \\ \cdashline{1-4}
{\footnotesize DMS-REQ-0371 } &
\multicolumn{3}{p{11.1cm}}{\footnotesize Query Availability } \\ \cdashline{1-4}
{\footnotesize DMS-REQ-0382 } &
\multicolumn{3}{p{11.1cm}}{\footnotesize HiPS Visualization } \\ \cdashline{1-4}
\end{longtable}

     \newpage
\begin{longtable}{p{3.7cm}p{3.7cm}p{3.7cm}p{3.7cm}}\toprule
\multicolumn{2}{l}{\large \textbf{ \hypertarget{lspjl}{LSP JupyterLab SW} } }
& \multicolumn{2}{l}{(product in: LSP SW
)}
\\ \hline
\textbf{\footnotesize Manager} & \textbf{\footnotesize Owner} &
\textbf{\footnotesize WBS} & \textbf{\footnotesize Team} \\ \hline
\parbox{3.5cm}{
Frossie Economou
\vspace{2mm}%
} &
\begin{tabular}{@{}l@{}}
\parbox{3.5cm}{
Simon Krughoff
\vspace{2mm}%
} \\
\end{tabular} &
\begin{tabular}{@{}l@{}}
1.02C.10.02.02 \\
\end{tabular} & \begin{tabular}{@{}l@{}}
SQuaRE \\
\end{tabular} \\ \hline
\multicolumn{4}{c}{
{\footnotesize ( LSP JL SW
 - LSPJL ) }
}\\ \hline
\end{longtable}

LSP Jupiter Notebook Product


\begin{longtable}{p{3.7cm}p{3.7cm}p{3.7cm}p{3.7cm}}\hline
\multicolumn{2}{r}{\textbf{GutHub Packages:}} &
\multicolumn{2}{l}{\href{https://github.com/lsst-sqre/jupyterhubutils}{lsst-sqre/jupyterhubutils} }\ref{lsst-sqre/jupyterhubutils}
\\ \cline{3-4}
& & \multicolumn{2}{l}{\href{https://github.com/lsst-sqre/jupyterlabutils}{lsst-sqre/jupyterlabutils} }\ref{lsst-sqre/jupyterlabutils}
\\ \hline \\ \hline
\textbf{\footnotesize Uses:}  & & \textbf{\footnotesize Used in:} & \\ \hline
\multicolumn{2}{c}{
\begin{tabular}{c}
\hyperlink{jl3}{Jupyterlab} \\ \hline
\hyperlink{jh3}{Jupyterhub} \\ \hline
\end{tabular}
} &
\multicolumn{2}{c}{
\begin{tabular}{c}
\hyperlink{lspnbl}{LSP Nublado} \\ \hline
\end{tabular}
} \\ \bottomrule
\multicolumn{4}{c}{\textbf{Related Requirements} } \\ \hline
{\footnotesize DMS-REQ-0119 } &
\multicolumn{3}{p{11.1cm}}{\footnotesize DAC resource allocation for Level 3 processing } \\ \cdashline{1-4}
{\footnotesize DMS-REQ-0123 } &
\multicolumn{3}{p{11.1cm}}{\footnotesize Access to input catalogs for DAC-based Level 3 processing } \\ \cdashline{1-4}
{\footnotesize DMS-REQ-0124 } &
\multicolumn{3}{p{11.1cm}}{\footnotesize Federation with external catalogs } \\ \cdashline{1-4}
{\footnotesize DMS-REQ-0127 } &
\multicolumn{3}{p{11.1cm}}{\footnotesize Access to input images for DAC-based Level 3 processing } \\ \cdashline{1-4}
{\footnotesize DMS-REQ-0161 } &
\multicolumn{3}{p{11.1cm}}{\footnotesize Optimization of Cost, Reliability and Availability in Order } \\ \cdashline{1-4}
{\footnotesize DMS-REQ-0193 } &
\multicolumn{3}{p{11.1cm}}{\footnotesize Data Access Centers } \\ \cdashline{1-4}
{\footnotesize DMS-REQ-0194 } &
\multicolumn{3}{p{11.1cm}}{\footnotesize Data Access Center Simultaneous Connections } \\ \cdashline{1-4}
{\footnotesize DMS-REQ-0197 } &
\multicolumn{3}{p{11.1cm}}{\footnotesize No Limit on Data Access Centers } \\ \cdashline{1-4}
{\footnotesize DMS-REQ-0314 } &
\multicolumn{3}{p{11.1cm}}{\footnotesize Compute Platform Heterogeneity } \\ \cdashline{1-4}
{\footnotesize DMS-REQ-0318 } &
\multicolumn{3}{p{11.1cm}}{\footnotesize Data Management Unscheduled Downtime } \\ \cdashline{1-4}
\end{longtable}

     \newpage
\begin{longtable}{p{3.7cm}p{3.7cm}p{3.7cm}p{3.7cm}}\toprule
\multicolumn{2}{l}{\large \textbf{ \hypertarget{lspweb}{LSP Web API SW (obsolete product)} } }
& \multicolumn{2}{l}{(product in: LSP SW
)}
\\ \hline
\textbf{\footnotesize Manager} & \textbf{\footnotesize Owner} &
\textbf{\footnotesize WBS} & \textbf{\footnotesize Team} \\ \hline
\parbox{3.5cm}{

\vspace{2mm}%
} &
\begin{tabular}{@{}l@{}}
\parbox{3.5cm}{

\vspace{2mm}%
} \\
\end{tabular} &
\begin{tabular}{@{}l@{}}
1.02C.06.02 \\
\end{tabular} & \begin{tabular}{@{}l@{}}
DAX \\
\end{tabular} \\ \hline
\multicolumn{4}{c}{
{\footnotesize ( LSP Web SW
 - LSPWEB ) }
}\\ \hline
\end{longtable}

{[}OBSOLETE{]}

LSP Web API software product


\begin{longtable}{p{3.7cm}p{3.7cm}p{3.7cm}p{3.7cm}}\hline
\multicolumn{2}{r}{\textbf{GutHub Packages:}} &
\multicolumn{2}{l}{\href{https://github.com/lsst/dax_webserv}{dax\_webserv} }\ref{dax_webserv}
\\ \hline \\ \hline
\textbf{\footnotesize Uses:}  & & \textbf{\footnotesize Used in:} & \\ \hline
\multicolumn{2}{c}{
} &
\multicolumn{2}{c}{
} \\ \bottomrule
\multicolumn{4}{c}{\textbf{Related Requirements} } \\ \hline
{\footnotesize DMS-REQ-0065 } &
\multicolumn{3}{p{11.1cm}}{\footnotesize Provide Image Access Services } \\ \cdashline{1-4}
{\footnotesize DMS-REQ-0075 } &
\multicolumn{3}{p{11.1cm}}{\footnotesize Catalog Queries } \\ \cdashline{1-4}
{\footnotesize DMS-REQ-0078 } &
\multicolumn{3}{p{11.1cm}}{\footnotesize Catalog Export Formats } \\ \cdashline{1-4}
{\footnotesize DMS-REQ-0089 } &
\multicolumn{3}{p{11.1cm}}{\footnotesize Solar System Objects Available Within Specified Time } \\ \cdashline{1-4}
{\footnotesize DMS-REQ-0119 } &
\multicolumn{3}{p{11.1cm}}{\footnotesize DAC resource allocation for Level 3 processing } \\ \cdashline{1-4}
{\footnotesize DMS-REQ-0120 } &
\multicolumn{3}{p{11.1cm}}{\footnotesize Level 3 Data Product Self Consistency } \\ \cdashline{1-4}
{\footnotesize DMS-REQ-0121 } &
\multicolumn{3}{p{11.1cm}}{\footnotesize Provenance for Level 3 processing at DACs } \\ \cdashline{1-4}
{\footnotesize DMS-REQ-0123 } &
\multicolumn{3}{p{11.1cm}}{\footnotesize Access to input catalogs for DAC-based Level 3 processing } \\ \cdashline{1-4}
{\footnotesize DMS-REQ-0124 } &
\multicolumn{3}{p{11.1cm}}{\footnotesize Federation with external catalogs } \\ \cdashline{1-4}
{\footnotesize DMS-REQ-0127 } &
\multicolumn{3}{p{11.1cm}}{\footnotesize Access to input images for DAC-based Level 3 processing } \\ \cdashline{1-4}
{\footnotesize DMS-REQ-0155 } &
\multicolumn{3}{p{11.1cm}}{\footnotesize Provide Data Access Services } \\ \cdashline{1-4}
{\footnotesize DMS-REQ-0161 } &
\multicolumn{3}{p{11.1cm}}{\footnotesize Optimization of Cost, Reliability and Availability in Order } \\ \cdashline{1-4}
{\footnotesize DMS-REQ-0193 } &
\multicolumn{3}{p{11.1cm}}{\footnotesize Data Access Centers } \\ \cdashline{1-4}
{\footnotesize DMS-REQ-0194 } &
\multicolumn{3}{p{11.1cm}}{\footnotesize Data Access Center Simultaneous Connections } \\ \cdashline{1-4}
{\footnotesize DMS-REQ-0197 } &
\multicolumn{3}{p{11.1cm}}{\footnotesize No Limit on Data Access Centers } \\ \cdashline{1-4}
{\footnotesize DMS-REQ-0287 } &
\multicolumn{3}{p{11.1cm}}{\footnotesize DIASource Precovery } \\ \cdashline{1-4}
{\footnotesize DMS-REQ-0290 } &
\multicolumn{3}{p{11.1cm}}{\footnotesize Level 3 Data Import } \\ \cdashline{1-4}
{\footnotesize DMS-REQ-0291 } &
\multicolumn{3}{p{11.1cm}}{\footnotesize Query Repeatability } \\ \cdashline{1-4}
{\footnotesize DMS-REQ-0292 } &
\multicolumn{3}{p{11.1cm}}{\footnotesize Uniqueness of IDs Across Data Releases } \\ \cdashline{1-4}
{\footnotesize DMS-REQ-0293 } &
\multicolumn{3}{p{11.1cm}}{\footnotesize Selection of Datasets } \\ \cdashline{1-4}
{\footnotesize DMS-REQ-0295 } &
\multicolumn{3}{p{11.1cm}}{\footnotesize Transparent Data Access } \\ \cdashline{1-4}
{\footnotesize DMS-REQ-0298 } &
\multicolumn{3}{p{11.1cm}}{\footnotesize Data Product and Raw Data Access } \\ \cdashline{1-4}
{\footnotesize DMS-REQ-0299 } &
\multicolumn{3}{p{11.1cm}}{\footnotesize Data Product Ingest } \\ \cdashline{1-4}
{\footnotesize DMS-REQ-0311 } &
\multicolumn{3}{p{11.1cm}}{\footnotesize Regenerate Un-archived Data Products } \\ \cdashline{1-4}
{\footnotesize DMS-REQ-0312 } &
\multicolumn{3}{p{11.1cm}}{\footnotesize Level 1 Data Product Access } \\ \cdashline{1-4}
{\footnotesize DMS-REQ-0313 } &
\multicolumn{3}{p{11.1cm}}{\footnotesize Level 1 \&  2 Catalog Access } \\ \cdashline{1-4}
{\footnotesize DMS-REQ-0314 } &
\multicolumn{3}{p{11.1cm}}{\footnotesize Compute Platform Heterogeneity } \\ \cdashline{1-4}
{\footnotesize DMS-REQ-0318 } &
\multicolumn{3}{p{11.1cm}}{\footnotesize Data Management Unscheduled Downtime } \\ \cdashline{1-4}
{\footnotesize DMS-REQ-0322 } &
\multicolumn{3}{p{11.1cm}}{\footnotesize Special Programs Database } \\ \cdashline{1-4}
{\footnotesize DMS-REQ-0323 } &
\multicolumn{3}{p{11.1cm}}{\footnotesize Calculating SSObject Parameters } \\ \cdashline{1-4}
{\footnotesize DMS-REQ-0324 } &
\multicolumn{3}{p{11.1cm}}{\footnotesize Matching DIASources to Objects } \\ \cdashline{1-4}
{\footnotesize DMS-REQ-0331 } &
\multicolumn{3}{p{11.1cm}}{\footnotesize Computing Derived Quantities } \\ \cdashline{1-4}
{\footnotesize DMS-REQ-0332 } &
\multicolumn{3}{p{11.1cm}}{\footnotesize Denormalizing Database Tables } \\ \cdashline{1-4}
{\footnotesize DMS-REQ-0334 } &
\multicolumn{3}{p{11.1cm}}{\footnotesize Persisting Data Products } \\ \cdashline{1-4}
{\footnotesize DMS-REQ-0336 } &
\multicolumn{3}{p{11.1cm}}{\footnotesize b Regenerating Data Products from Previous Data Releases } \\ \cdashline{1-4}
{\footnotesize DMS-REQ-0338 } &
\multicolumn{3}{p{11.1cm}}{\footnotesize Targeted Coadds } \\ \cdashline{1-4}
{\footnotesize DMS-REQ-0339 } &
\multicolumn{3}{p{11.1cm}}{\footnotesize Tracking Characterization Changes Between Data Releases } \\ \cdashline{1-4}
{\footnotesize DMS-REQ-0344 } &
\multicolumn{3}{p{11.1cm}}{\footnotesize Constraints on Level 1 Special Program Products Generation } \\ \cdashline{1-4}
{\footnotesize DMS-REQ-0345 } &
\multicolumn{3}{p{11.1cm}}{\footnotesize Logging of catalog queries } \\ \cdashline{1-4}
{\footnotesize DMS-REQ-0346 } &
\multicolumn{3}{p{11.1cm}}{\footnotesize Data Availability } \\ \cdashline{1-4}
{\footnotesize DMS-REQ-0347 } &
\multicolumn{3}{p{11.1cm}}{\footnotesize Measurements in catalogs } \\ \cdashline{1-4}
{\footnotesize DMS-REQ-0363 } &
\multicolumn{3}{p{11.1cm}}{\footnotesize Access to Previous Data Releases } \\ \cdashline{1-4}
{\footnotesize DMS-REQ-0364 } &
\multicolumn{3}{p{11.1cm}}{\footnotesize Data Access Services } \\ \cdashline{1-4}
{\footnotesize DMS-REQ-0365 } &
\multicolumn{3}{p{11.1cm}}{\footnotesize Operations Subsets } \\ \cdashline{1-4}
{\footnotesize DMS-REQ-0366 } &
\multicolumn{3}{p{11.1cm}}{\footnotesize Subsets Support } \\ \cdashline{1-4}
{\footnotesize DMS-REQ-0367 } &
\multicolumn{3}{p{11.1cm}}{\footnotesize Access Services Performance } \\ \cdashline{1-4}
{\footnotesize DMS-REQ-0368 } &
\multicolumn{3}{p{11.1cm}}{\footnotesize Implementation Provisions } \\ \cdashline{1-4}
{\footnotesize DMS-REQ-0369 } &
\multicolumn{3}{p{11.1cm}}{\footnotesize Evolution } \\ \cdashline{1-4}
{\footnotesize DMS-REQ-0370 } &
\multicolumn{3}{p{11.1cm}}{\footnotesize Older Release Behavior } \\ \cdashline{1-4}
{\footnotesize DMS-REQ-0371 } &
\multicolumn{3}{p{11.1cm}}{\footnotesize Query Availability } \\ \cdashline{1-4}
{\footnotesize EP-DM-CON-ICD-0013 } &
\multicolumn{3}{p{11.1cm}}{\footnotesize Visualization Image Metadata Standard } \\ \cdashline{1-4}
{\footnotesize EP-DM-CON-ICD-0034 } &
\multicolumn{3}{p{11.1cm}}{\footnotesize Citizen Science Data } \\ \cdashline{1-4}
\end{longtable}

     \newpage
\begin{longtable}{p{3.7cm}p{3.7cm}p{3.7cm}p{3.7cm}}\toprule
\multicolumn{2}{l}{\large \textbf{ \hypertarget{suitoh}{SUIT Online Help} } }
& \multicolumn{2}{l}{(product in: LSP SW
)}
\\ \hline
\textbf{\footnotesize Manager} & \textbf{\footnotesize Owner} &
\textbf{\footnotesize WBS} & \textbf{\footnotesize Team} \\ \hline
\parbox{3.5cm}{
Xiuqin Wu
\vspace{2mm}%
} &
\begin{tabular}{@{}l@{}}
\parbox{3.5cm}{
Gregory Dubois-Felsmann
\vspace{2mm}%
} \\
\end{tabular} &
\begin{tabular}{@{}l@{}}
 \\
\end{tabular} & \begin{tabular}{@{}l@{}}
SUIT \\
\end{tabular} \\ \hline
\multicolumn{4}{c}{
{\footnotesize ( SUIT OnlineHelp
 - SUITOH ) }
}\\ \hline
\end{longtable}

Contains help content, as HTML, and a small amount of GWT GUI code to
implement a help application that provides a navigation sidebar for the
HTML content.


\begin{longtable}{p{3.7cm}p{3.7cm}p{3.7cm}p{3.7cm}}\hline
\multicolumn{2}{r}{\textbf{GutHub Packages:}} &
\multicolumn{2}{l}{\href{https://github.com/lsst/suit-onlinehelp}{suit-onlinehelp} }\ref{suit-onlinehelp}
\\ \hline \\ \hline
\textbf{\footnotesize Uses:}  & & \textbf{\footnotesize Used in:} & \\ \hline
\multicolumn{2}{c}{
} &
\multicolumn{2}{c}{
\begin{tabular}{c}
\hyperlink{lspprtlsrv}{LSP Portal} \\ \hline
\end{tabular}
} \\ \bottomrule
\multicolumn{4}{c}{\textbf{Related Requirements} } \\ \hline
\end{longtable}

     \newpage
\begin{longtable}{p{3.7cm}p{3.7cm}p{3.7cm}p{3.7cm}}\toprule
\multicolumn{2}{l}{\large \textbf{ \hypertarget{daximg}{Image/ Cutout Server} } }
& \multicolumn{2}{l}{(product in: LSP SW
)}
\\ \hline
\textbf{\footnotesize Manager} & \textbf{\footnotesize Owner} &
\textbf{\footnotesize WBS} & \textbf{\footnotesize Team} \\ \hline
\parbox{3.5cm}{
Frossie Economou
\vspace{2mm}%
} &
\begin{tabular}{@{}l@{}}
\parbox{3.5cm}{
Colin Slater
\vspace{2mm}%
} \\
\end{tabular} &
\begin{tabular}{@{}l@{}}
1.02C.06.02.04 \\
\end{tabular} & \begin{tabular}{@{}l@{}}
DAX \\
\end{tabular} \\ \hline
\multicolumn{4}{c}{
{\footnotesize ( Image Server
 - DAXIMG ) }
}\\ \hline
\end{longtable}

Web Interface for LSST Image Services software product.


\begin{longtable}{p{3.7cm}p{3.7cm}p{3.7cm}p{3.7cm}}\hline
\multicolumn{2}{r}{\textbf{GutHub Packages:}} &
\multicolumn{2}{l}{\href{https://github.com/lsst/dax_imgserv}{dax\_imgserv} }\ref{dax_imgserv}
\\ \hline \\ \hline
\textbf{\footnotesize Uses:}  & & \textbf{\footnotesize Used in:} & \\ \hline
\multicolumn{2}{c}{
} &
\multicolumn{2}{c}{
\begin{tabular}{c}
\hyperlink{soda}{SODA API} \\ \hline
\end{tabular}
} \\ \bottomrule
\multicolumn{4}{c}{\textbf{Related Requirements} } \\ \hline
{\footnotesize DMS-REQ-0065 } &
\multicolumn{3}{p{11.1cm}}{\footnotesize Provide Image Access Services } \\ \cdashline{1-4}
{\footnotesize DMS-REQ-0127 } &
\multicolumn{3}{p{11.1cm}}{\footnotesize Access to input images for DAC-based Level 3 processing } \\ \cdashline{1-4}
{\footnotesize DMS-REQ-0155 } &
\multicolumn{3}{p{11.1cm}}{\footnotesize Provide Data Access Services } \\ \cdashline{1-4}
{\footnotesize DMS-REQ-0293 } &
\multicolumn{3}{p{11.1cm}}{\footnotesize Selection of Datasets } \\ \cdashline{1-4}
{\footnotesize DMS-REQ-0298 } &
\multicolumn{3}{p{11.1cm}}{\footnotesize Data Product and Raw Data Access } \\ \cdashline{1-4}
{\footnotesize DMS-REQ-0311 } &
\multicolumn{3}{p{11.1cm}}{\footnotesize Regenerate Un-archived Data Products } \\ \cdashline{1-4}
{\footnotesize DMS-REQ-0334 } &
\multicolumn{3}{p{11.1cm}}{\footnotesize Persisting Data Products } \\ \cdashline{1-4}
{\footnotesize DMS-REQ-0336 } &
\multicolumn{3}{p{11.1cm}}{\footnotesize b Regenerating Data Products from Previous Data Releases } \\ \cdashline{1-4}
{\footnotesize DMS-REQ-0338 } &
\multicolumn{3}{p{11.1cm}}{\footnotesize Targeted Coadds } \\ \cdashline{1-4}
{\footnotesize DMS-REQ-0339 } &
\multicolumn{3}{p{11.1cm}}{\footnotesize Tracking Characterization Changes Between Data Releases } \\ \cdashline{1-4}
{\footnotesize DMS-REQ-0346 } &
\multicolumn{3}{p{11.1cm}}{\footnotesize Data Availability } \\ \cdashline{1-4}
{\footnotesize DMS-REQ-0368 } &
\multicolumn{3}{p{11.1cm}}{\footnotesize Implementation Provisions } \\ \cdashline{1-4}
{\footnotesize EP-DM-CON-ICD-0013 } &
\multicolumn{3}{p{11.1cm}}{\footnotesize Visualization Image Metadata Standard } \\ \cdashline{1-4}
{\footnotesize EP-DM-CON-ICD-0034 } &
\multicolumn{3}{p{11.1cm}}{\footnotesize Citizen Science Data } \\ \cdashline{1-4}
\end{longtable}

     \newpage
\begin{longtable}{p{3.7cm}p{3.7cm}p{3.7cm}p{3.7cm}}\toprule
\multicolumn{2}{l}{\large \textbf{ \hypertarget{tapsw}{TAP SW} } }
& \multicolumn{2}{l}{(product in: LSP SW
)}
\\ \hline
\textbf{\footnotesize Manager} & \textbf{\footnotesize Owner} &
\textbf{\footnotesize WBS} & \textbf{\footnotesize Team} \\ \hline
\parbox{3.5cm}{
Frossie Economou
\vspace{2mm}%
} &
\begin{tabular}{@{}l@{}}
\parbox{3.5cm}{
Simon Krughoff
\vspace{2mm}%
} \\
\end{tabular} &
\begin{tabular}{@{}l@{}}
 \\
\end{tabular} & \begin{tabular}{@{}l@{}}
 \\
\end{tabular} \\ \hline
\multicolumn{4}{c}{
{\footnotesize ( TAP
 - TAPSW ) }
}\\ \hline
\end{longtable}




\begin{longtable}{p{3.7cm}p{3.7cm}p{3.7cm}p{3.7cm}}\hline
\textbf{\footnotesize Uses:}  & & \textbf{\footnotesize Used in:} & \\ \hline
\multicolumn{2}{c}{
} &
\multicolumn{2}{c}{
\begin{tabular}{c}
\hyperlink{tapsev}{TAP API} \\ \hline
\end{tabular}
} \\ \bottomrule
\multicolumn{4}{c}{\textbf{Related Requirements} } \\ \hline
\end{longtable}

    
   \newpage
\subsubsection{Science Pipeline SW Products}\label{prodn}
\begin{longtable}{p{3.7cm}p{3.7cm}p{3.7cm}p{3.7cm}}\hline
\textbf{Manager} & \textbf{Owner} & \textbf{WBS} & \textbf{Team} \\ \hline
\parbox{3.5cm}{
John Swinbank
\vspace{2mm}%
} &
\begin{tabular}{@{}l@{}}
\parbox{3.5cm}{
Leanne Guy
\vspace{2mm}%
} \\
\end{tabular}
 &
\begin{tabular}{@{}l@{}}
 \\
\end{tabular} &
\begin{tabular}{@{}l@{}}
 \\
\end{tabular} \\ \hline
\multicolumn{4}{c}{
{\footnotesize ( Sci Pipelines SW
 - PRODN ) }
}\\ \hline
\citeds{LDM-151} &
\multicolumn{3}{l}{ Data Management Science Pipelines Design }
\\ \cdashline{1-4}
\end{longtable}

  DM Software products that implements the Science Pipelines payloads.


   \newpage
\begin{longtable}{p{3.7cm}p{3.7cm}p{3.7cm}p{3.7cm}}\toprule
\multicolumn{2}{l}{\large \textbf{ \hypertarget{apprmpt}{Alert Production} } }
& \multicolumn{2}{l}{(product in: Sci Pipelines SW
)}
\\ \hline
\textbf{\footnotesize Manager} & \textbf{\footnotesize Owner} &
\textbf{\footnotesize WBS} & \textbf{\footnotesize Team} \\ \hline
\parbox{3.5cm}{
John Swinbank
\vspace{2mm}%
} &
\begin{tabular}{@{}l@{}}
\parbox{3.5cm}{
Eric Bellm
\vspace{2mm}%
} \\
\end{tabular} &
\begin{tabular}{@{}l@{}}
1.02C.03 \\
\end{tabular} & \begin{tabular}{@{}l@{}}
AP \\
\end{tabular} \\ \hline
\multicolumn{4}{c}{
{\footnotesize ( Alert Prod SW
 - APPRMPT ) }
}\\ \hline
\end{longtable}

Software product for Allert Production processing.


\begin{longtable}{p{3.7cm}p{3.7cm}p{3.7cm}p{3.7cm}}\hline
\textbf{\footnotesize Uses:}  & & \textbf{\footnotesize Used in:} & \\ \hline
\multicolumn{2}{c}{
\begin{tabular}{c}
\hyperlink{scipipe}{Science Pipelines Libraries} \\ \hline
\end{tabular}
} &
\multicolumn{2}{c}{
\begin{tabular}{c}
\hyperlink{prprsrv}{Prompt Processing} \\ \hline
\hyperlink{spdist}{Science Pipelines Distribution} \\ \hline
\end{tabular}
} \\ \bottomrule
\multicolumn{4}{c}{\textbf{Related Requirements} } \\ \hline
{\footnotesize DMS-REQ-0002 } &
\multicolumn{3}{p{11.1cm}}{\footnotesize Transient Alert Distribution } \\ \cdashline{1-4}
{\footnotesize DMS-REQ-0004 } &
\multicolumn{3}{p{11.1cm}}{\footnotesize Nightly Data Accessible Within 24 hrs } \\ \cdashline{1-4}
{\footnotesize DMS-REQ-0009 } &
\multicolumn{3}{p{11.1cm}}{\footnotesize Simulated Data } \\ \cdashline{1-4}
{\footnotesize DMS-REQ-0010 } &
\multicolumn{3}{p{11.1cm}}{\footnotesize Difference Exposures } \\ \cdashline{1-4}
{\footnotesize DMS-REQ-0029 } &
\multicolumn{3}{p{11.1cm}}{\footnotesize Generate Photometric Zeropoint for Visit Image } \\ \cdashline{1-4}
{\footnotesize DMS-REQ-0030 } &
\multicolumn{3}{p{11.1cm}}{\footnotesize Generate WCS for Visit Images } \\ \cdashline{1-4}
{\footnotesize DMS-REQ-0032 } &
\multicolumn{3}{p{11.1cm}}{\footnotesize Image Differencing } \\ \cdashline{1-4}
{\footnotesize DMS-REQ-0033 } &
\multicolumn{3}{p{11.1cm}}{\footnotesize Provide Source Detection Software } \\ \cdashline{1-4}
{\footnotesize DMS-REQ-0042 } &
\multicolumn{3}{p{11.1cm}}{\footnotesize Provide Astrometric Model } \\ \cdashline{1-4}
{\footnotesize DMS-REQ-0043 } &
\multicolumn{3}{p{11.1cm}}{\footnotesize Provide Calibrated Photometry } \\ \cdashline{1-4}
{\footnotesize DMS-REQ-0052 } &
\multicolumn{3}{p{11.1cm}}{\footnotesize Enable a Range of Shape Measurement Approaches } \\ \cdashline{1-4}
{\footnotesize DMS-REQ-0069 } &
\multicolumn{3}{p{11.1cm}}{\footnotesize Processed Visit Images } \\ \cdashline{1-4}
{\footnotesize DMS-REQ-0070 } &
\multicolumn{3}{p{11.1cm}}{\footnotesize Generate PSF for Visit Images } \\ \cdashline{1-4}
{\footnotesize DMS-REQ-0072 } &
\multicolumn{3}{p{11.1cm}}{\footnotesize Processed Visit Image Content } \\ \cdashline{1-4}
{\footnotesize DMS-REQ-0074 } &
\multicolumn{3}{p{11.1cm}}{\footnotesize Difference Exposure Attributes } \\ \cdashline{1-4}
{\footnotesize DMS-REQ-0097 } &
\multicolumn{3}{p{11.1cm}}{\footnotesize Level 1 Data Quality Report Definition } \\ \cdashline{1-4}
{\footnotesize DMS-REQ-0266 } &
\multicolumn{3}{p{11.1cm}}{\footnotesize Exposure Catalog } \\ \cdashline{1-4}
{\footnotesize DMS-REQ-0269 } &
\multicolumn{3}{p{11.1cm}}{\footnotesize DIASource Catalog } \\ \cdashline{1-4}
{\footnotesize DMS-REQ-0270 } &
\multicolumn{3}{p{11.1cm}}{\footnotesize Faint DIASource Measurements } \\ \cdashline{1-4}
{\footnotesize DMS-REQ-0271 } &
\multicolumn{3}{p{11.1cm}}{\footnotesize DIAObject Catalog } \\ \cdashline{1-4}
{\footnotesize DMS-REQ-0272 } &
\multicolumn{3}{p{11.1cm}}{\footnotesize DIAObject Attributes } \\ \cdashline{1-4}
{\footnotesize DMS-REQ-0274 } &
\multicolumn{3}{p{11.1cm}}{\footnotesize Alert Content } \\ \cdashline{1-4}
{\footnotesize DMS-REQ-0285 } &
\multicolumn{3}{p{11.1cm}}{\footnotesize Level 1 Source Association } \\ \cdashline{1-4}
{\footnotesize DMS-REQ-0288 } &
\multicolumn{3}{p{11.1cm}}{\footnotesize Use of External Orbit Catalogs } \\ \cdashline{1-4}
{\footnotesize DMS-REQ-0308 } &
\multicolumn{3}{p{11.1cm}}{\footnotesize Software Architecture to Enable Community Re-Use } \\ \cdashline{1-4}
{\footnotesize DMS-REQ-0312 } &
\multicolumn{3}{p{11.1cm}}{\footnotesize Level 1 Data Product Access } \\ \cdashline{1-4}
{\footnotesize DMS-REQ-0317 } &
\multicolumn{3}{p{11.1cm}}{\footnotesize DIAForcedSource Catalog } \\ \cdashline{1-4}
{\footnotesize DMS-REQ-0319 } &
\multicolumn{3}{p{11.1cm}}{\footnotesize Characterizing Variability } \\ \cdashline{1-4}
{\footnotesize DMS-REQ-0321 } &
\multicolumn{3}{p{11.1cm}}{\footnotesize Level 1 Processing of Special Programs Data } \\ \cdashline{1-4}
{\footnotesize DMS-REQ-0324 } &
\multicolumn{3}{p{11.1cm}}{\footnotesize Matching DIASources to Objects } \\ \cdashline{1-4}
{\footnotesize DMS-REQ-0327 } &
\multicolumn{3}{p{11.1cm}}{\footnotesize Background Model Calculation } \\ \cdashline{1-4}
{\footnotesize DMS-REQ-0328 } &
\multicolumn{3}{p{11.1cm}}{\footnotesize Documenting Image Characterization } \\ \cdashline{1-4}
{\footnotesize DMS-REQ-0333 } &
\multicolumn{3}{p{11.1cm}}{\footnotesize Maximum Likelihood Values and Covariances } \\ \cdashline{1-4}
{\footnotesize DMS-REQ-0344 } &
\multicolumn{3}{p{11.1cm}}{\footnotesize Constraints on Level 1 Special Program Products Generation } \\ \cdashline{1-4}
{\footnotesize DMS-REQ-0347 } &
\multicolumn{3}{p{11.1cm}}{\footnotesize Measurements in catalogs } \\ \cdashline{1-4}
{\footnotesize EP-DM-CON-ICD-0013 } &
\multicolumn{3}{p{11.1cm}}{\footnotesize Visualization Image Metadata Standard } \\ \cdashline{1-4}
{\footnotesize EP-DM-CON-ICD-0023 } &
\multicolumn{3}{p{11.1cm}}{\footnotesize Nightly DM Transfer of Processed Visit Images (PVI)-Based Images to EPO } \\ \cdashline{1-4}
{\footnotesize OCS-DM-COM-ICD-0049 } &
\multicolumn{3}{p{11.1cm}}{\footnotesize WCS Information } \\ \cdashline{1-4}
{\footnotesize OCS-DM-COM-ICD-0050 } &
\multicolumn{3}{p{11.1cm}}{\footnotesize PSF Information } \\ \cdashline{1-4}
{\footnotesize OCS-DM-COM-ICD-0051 } &
\multicolumn{3}{p{11.1cm}}{\footnotesize Photometric Zeropoint Information } \\ \cdashline{1-4}
{\footnotesize OCS-DM-COM-ICD-0052 } &
\multicolumn{3}{p{11.1cm}}{\footnotesize Number of Alerts Information } \\ \cdashline{1-4}
\end{longtable}

     \newpage
\begin{longtable}{p{3.7cm}p{3.7cm}p{3.7cm}p{3.7cm}}\toprule
\multicolumn{2}{l}{\large \textbf{ \hypertarget{dmcal}{Calibration SW} } }
& \multicolumn{2}{l}{(product in: Sci Pipelines SW
)}
\\ \hline
\textbf{\footnotesize Manager} & \textbf{\footnotesize Owner} &
\textbf{\footnotesize WBS} & \textbf{\footnotesize Team} \\ \hline
\parbox{3.5cm}{
John Swinbank
\vspace{2mm}%
} &
\begin{tabular}{@{}l@{}}
\parbox{3.5cm}{
Robert Lupton
\vspace{2mm}%
} \\
\end{tabular} &
\begin{tabular}{@{}l@{}}
1.02C.04.02 \\
\end{tabular} & \begin{tabular}{@{}l@{}}
DRP \\
\end{tabular} \\ \hline
\multicolumn{4}{c}{
{\footnotesize ( Calibration SW
 - DMCAL ) }
}\\ \hline
\end{longtable}

Software product for generating calibration data products.


\begin{longtable}{p{3.7cm}p{3.7cm}p{3.7cm}p{3.7cm}}\hline
\textbf{\footnotesize Uses:}  & & \textbf{\footnotesize Used in:} & \\ \hline
\multicolumn{2}{c}{
\begin{tabular}{c}
\hyperlink{scipipe}{Science Pipelines Libraries} \\ \hline
\end{tabular}
} &
\multicolumn{2}{c}{
\begin{tabular}{c}
\hyperlink{prprsrv}{Prompt Processing} \\ \hline
\hyperlink{prodsrv}{Batch Production} \\ \hline
\hyperlink{spdist}{Science Pipelines Distribution} \\ \hline
\end{tabular}
} \\ \bottomrule
\multicolumn{4}{c}{\textbf{Related Requirements} } \\ \hline
{\footnotesize DMS-REQ-0101 } &
\multicolumn{3}{p{11.1cm}}{\footnotesize Level 1 Calibration Report Definition } \\ \cdashline{1-4}
{\footnotesize DMS-REQ-0131 } &
\multicolumn{3}{p{11.1cm}}{\footnotesize Calibration Images Available Within Specified Time } \\ \cdashline{1-4}
{\footnotesize DMS-REQ-0265 } &
\multicolumn{3}{p{11.1cm}}{\footnotesize Guider Calibration Data Acquisition } \\ \cdashline{1-4}
{\footnotesize DMS-REQ-0308 } &
\multicolumn{3}{p{11.1cm}}{\footnotesize Software Architecture to Enable Community Re-Use } \\ \cdashline{1-4}
\end{longtable}

     \newpage
\begin{longtable}{p{3.7cm}p{3.7cm}p{3.7cm}p{3.7cm}}\toprule
\multicolumn{2}{l}{\large \textbf{ \hypertarget{drp}{Data Release Production} } }
& \multicolumn{2}{l}{(product in: Sci Pipelines SW
)}
\\ \hline
\textbf{\footnotesize Manager} & \textbf{\footnotesize Owner} &
\textbf{\footnotesize WBS} & \textbf{\footnotesize Team} \\ \hline
\parbox{3.5cm}{
Yusra AlSayyad
\vspace{2mm}%
} &
\begin{tabular}{@{}l@{}}
\parbox{3.5cm}{
Jim Bosch
\vspace{2mm}%
} \\
\end{tabular} &
\begin{tabular}{@{}l@{}}
1.02C.04 \\
\end{tabular} & \begin{tabular}{@{}l@{}}
DRP \\
\end{tabular} \\ \hline
\multicolumn{4}{c}{
{\footnotesize ( DR Prod SW
 - DRP ) }
}\\ \hline
\end{longtable}

Software product for data release production.


\begin{longtable}{p{3.7cm}p{3.7cm}p{3.7cm}p{3.7cm}}\hline
\textbf{\footnotesize Uses:}  & & \textbf{\footnotesize Used in:} & \\ \hline
\multicolumn{2}{c}{
\begin{tabular}{c}
\hyperlink{scipipe}{Science Pipelines Libraries} \\ \hline
\end{tabular}
} &
\multicolumn{2}{c}{
\begin{tabular}{c}
\hyperlink{prodsrv}{Batch Production} \\ \hline
\hyperlink{spdist}{Science Pipelines Distribution} \\ \hline
\end{tabular}
} \\ \bottomrule
\multicolumn{4}{c}{\textbf{Related Requirements} } \\ \hline
{\footnotesize DMS-REQ-0009 } &
\multicolumn{3}{p{11.1cm}}{\footnotesize Simulated Data } \\ \cdashline{1-4}
{\footnotesize DMS-REQ-0032 } &
\multicolumn{3}{p{11.1cm}}{\footnotesize Image Differencing } \\ \cdashline{1-4}
{\footnotesize DMS-REQ-0033 } &
\multicolumn{3}{p{11.1cm}}{\footnotesize Provide Source Detection Software } \\ \cdashline{1-4}
{\footnotesize DMS-REQ-0034 } &
\multicolumn{3}{p{11.1cm}}{\footnotesize Associate Sources to Objects } \\ \cdashline{1-4}
{\footnotesize DMS-REQ-0042 } &
\multicolumn{3}{p{11.1cm}}{\footnotesize Provide Astrometric Model } \\ \cdashline{1-4}
{\footnotesize DMS-REQ-0043 } &
\multicolumn{3}{p{11.1cm}}{\footnotesize Provide Calibrated Photometry } \\ \cdashline{1-4}
{\footnotesize DMS-REQ-0046 } &
\multicolumn{3}{p{11.1cm}}{\footnotesize Provide Photometric Redshifts of Galaxies } \\ \cdashline{1-4}
{\footnotesize DMS-REQ-0047 } &
\multicolumn{3}{p{11.1cm}}{\footnotesize Provide PSF for Coadded Images } \\ \cdashline{1-4}
{\footnotesize DMS-REQ-0052 } &
\multicolumn{3}{p{11.1cm}}{\footnotesize Enable a Range of Shape Measurement Approaches } \\ \cdashline{1-4}
{\footnotesize DMS-REQ-0103 } &
\multicolumn{3}{p{11.1cm}}{\footnotesize Produce Images for EPO } \\ \cdashline{1-4}
{\footnotesize DMS-REQ-0106 } &
\multicolumn{3}{p{11.1cm}}{\footnotesize Coadded Image Provenance } \\ \cdashline{1-4}
{\footnotesize DMS-REQ-0267 } &
\multicolumn{3}{p{11.1cm}}{\footnotesize Source Catalog } \\ \cdashline{1-4}
{\footnotesize DMS-REQ-0268 } &
\multicolumn{3}{p{11.1cm}}{\footnotesize Forced-Source Catalog } \\ \cdashline{1-4}
{\footnotesize DMS-REQ-0275 } &
\multicolumn{3}{p{11.1cm}}{\footnotesize Object Catalog } \\ \cdashline{1-4}
{\footnotesize DMS-REQ-0276 } &
\multicolumn{3}{p{11.1cm}}{\footnotesize Object Characterization } \\ \cdashline{1-4}
{\footnotesize DMS-REQ-0277 } &
\multicolumn{3}{p{11.1cm}}{\footnotesize Coadd Source Catalog } \\ \cdashline{1-4}
{\footnotesize DMS-REQ-0278 } &
\multicolumn{3}{p{11.1cm}}{\footnotesize Coadd Image Method Constraints } \\ \cdashline{1-4}
{\footnotesize DMS-REQ-0279 } &
\multicolumn{3}{p{11.1cm}}{\footnotesize Deep Detection Coadds } \\ \cdashline{1-4}
{\footnotesize DMS-REQ-0280 } &
\multicolumn{3}{p{11.1cm}}{\footnotesize Template Coadds } \\ \cdashline{1-4}
{\footnotesize DMS-REQ-0281 } &
\multicolumn{3}{p{11.1cm}}{\footnotesize Multi-band Coadds } \\ \cdashline{1-4}
{\footnotesize DMS-REQ-0308 } &
\multicolumn{3}{p{11.1cm}}{\footnotesize Software Architecture to Enable Community Re-Use } \\ \cdashline{1-4}
{\footnotesize DMS-REQ-0320 } &
\multicolumn{3}{p{11.1cm}}{\footnotesize Processing of Data From Special Programs } \\ \cdashline{1-4}
{\footnotesize DMS-REQ-0325 } &
\multicolumn{3}{p{11.1cm}}{\footnotesize Regenerating L1 Data Products During Data Release Processing } \\ \cdashline{1-4}
{\footnotesize DMS-REQ-0326 } &
\multicolumn{3}{p{11.1cm}}{\footnotesize Storing Approximations of Per-pixel Metadata } \\ \cdashline{1-4}
{\footnotesize DMS-REQ-0329 } &
\multicolumn{3}{p{11.1cm}}{\footnotesize All-Sky Visualization of Data Releases } \\ \cdashline{1-4}
{\footnotesize DMS-REQ-0330 } &
\multicolumn{3}{p{11.1cm}}{\footnotesize Best Seeing Coadds } \\ \cdashline{1-4}
{\footnotesize DMS-REQ-0331 } &
\multicolumn{3}{p{11.1cm}}{\footnotesize Computing Derived Quantities } \\ \cdashline{1-4}
{\footnotesize DMS-REQ-0333 } &
\multicolumn{3}{p{11.1cm}}{\footnotesize Maximum Likelihood Values and Covariances } \\ \cdashline{1-4}
{\footnotesize DMS-REQ-0335 } &
\multicolumn{3}{p{11.1cm}}{\footnotesize PSF-Matched Coadds } \\ \cdashline{1-4}
{\footnotesize DMS-REQ-0337 } &
\multicolumn{3}{p{11.1cm}}{\footnotesize Detecting faint variable objects } \\ \cdashline{1-4}
{\footnotesize DMS-REQ-0347 } &
\multicolumn{3}{p{11.1cm}}{\footnotesize Measurements in catalogs } \\ \cdashline{1-4}
{\footnotesize DMS-REQ-0349 } &
\multicolumn{3}{p{11.1cm}}{\footnotesize Detecting extended  low surface brightness objects } \\ \cdashline{1-4}
{\footnotesize DMS-REQ-0350 } &
\multicolumn{3}{p{11.1cm}}{\footnotesize Associating Objects across data releases } \\ \cdashline{1-4}
{\footnotesize DMS-REQ-0379 } &
\multicolumn{3}{p{11.1cm}}{\footnotesize Produce All-Sky HiPS Map } \\ \cdashline{1-4}
{\footnotesize DMS-REQ-0381 } &
\multicolumn{3}{p{11.1cm}}{\footnotesize HiPS Linkage to Coadds } \\ \cdashline{1-4}
{\footnotesize DMS-REQ-0383 } &
\multicolumn{3}{p{11.1cm}}{\footnotesize Produce MOC Maps } \\ \cdashline{1-4}
{\footnotesize EP-DM-CON-ICD-0013 } &
\multicolumn{3}{p{11.1cm}}{\footnotesize Visualization Image Metadata Standard } \\ \cdashline{1-4}
{\footnotesize EP-DM-CON-ICD-0014 } &
\multicolumn{3}{p{11.1cm}}{\footnotesize Color Co-Add Image Format } \\ \cdashline{1-4}
\end{longtable}

     \newpage
\begin{longtable}{p{3.7cm}p{3.7cm}p{3.7cm}p{3.7cm}}\toprule
\multicolumn{2}{l}{\large \textbf{ \hypertarget{mops}{MOPS and Forced Photometry} } }
& \multicolumn{2}{l}{(product in: Sci Pipelines SW
)}
\\ \hline
\textbf{\footnotesize Manager} & \textbf{\footnotesize Owner} &
\textbf{\footnotesize WBS} & \textbf{\footnotesize Team} \\ \hline
\parbox{3.5cm}{
John Swinbank
\vspace{2mm}%
} &
\begin{tabular}{@{}l@{}}
\parbox{3.5cm}{
Eric Bellm
\vspace{2mm}%
} \\
\end{tabular} &
\begin{tabular}{@{}l@{}}
1.02C.03.06 \\
\end{tabular} & \begin{tabular}{@{}l@{}}
AP \\
\end{tabular} \\ \hline
\multicolumn{4}{c}{
{\footnotesize ( MOPS SW
 - MOPS ) }
}\\ \hline
\end{longtable}

Software product for MOPS and Forced Photometry data processing.


\begin{longtable}{p{3.7cm}p{3.7cm}p{3.7cm}p{3.7cm}}\hline
\multicolumn{2}{r}{\textbf{GutHub Packages:}} &
\multicolumn{2}{l}{\href{https://github.com/lsst/mops_daymops}{mops\_daymops} }\ref{mops_daymops}
\\ \hline \\ \hline
\textbf{\footnotesize Uses:}  & & \textbf{\footnotesize Used in:} & \\ \hline
\multicolumn{2}{c}{
\begin{tabular}{c}
\hyperlink{scipipe}{Science Pipelines Libraries} \\ \hline
\end{tabular}
} &
\multicolumn{2}{c}{
\begin{tabular}{c}
\hyperlink{prprsrv}{Prompt Processing} \\ \hline
\hyperlink{prodsrv}{Batch Production} \\ \hline
\hyperlink{spdist}{Science Pipelines Distribution} \\ \hline
\end{tabular}
} \\ \bottomrule
\multicolumn{4}{c}{\textbf{Related Requirements} } \\ \hline
{\footnotesize DMS-REQ-0004 } &
\multicolumn{3}{p{11.1cm}}{\footnotesize Nightly Data Accessible Within 24 hrs } \\ \cdashline{1-4}
{\footnotesize DMS-REQ-0089 } &
\multicolumn{3}{p{11.1cm}}{\footnotesize Solar System Objects Available Within Specified Time } \\ \cdashline{1-4}
{\footnotesize DMS-REQ-0273 } &
\multicolumn{3}{p{11.1cm}}{\footnotesize SSObject Catalog } \\ \cdashline{1-4}
{\footnotesize DMS-REQ-0286 } &
\multicolumn{3}{p{11.1cm}}{\footnotesize SSObject Precovery } \\ \cdashline{1-4}
{\footnotesize DMS-REQ-0287 } &
\multicolumn{3}{p{11.1cm}}{\footnotesize DIASource Precovery } \\ \cdashline{1-4}
{\footnotesize DMS-REQ-0288 } &
\multicolumn{3}{p{11.1cm}}{\footnotesize Use of External Orbit Catalogs } \\ \cdashline{1-4}
{\footnotesize DMS-REQ-0308 } &
\multicolumn{3}{p{11.1cm}}{\footnotesize Software Architecture to Enable Community Re-Use } \\ \cdashline{1-4}
{\footnotesize DMS-REQ-0317 } &
\multicolumn{3}{p{11.1cm}}{\footnotesize DIAForcedSource Catalog } \\ \cdashline{1-4}
{\footnotesize DMS-REQ-0319 } &
\multicolumn{3}{p{11.1cm}}{\footnotesize Characterizing Variability } \\ \cdashline{1-4}
{\footnotesize DMS-REQ-0321 } &
\multicolumn{3}{p{11.1cm}}{\footnotesize Level 1 Processing of Special Programs Data } \\ \cdashline{1-4}
{\footnotesize DMS-REQ-0341 } &
\multicolumn{3}{p{11.1cm}}{\footnotesize Providing a Precovery Service } \\ \cdashline{1-4}
{\footnotesize DMS-REQ-0344 } &
\multicolumn{3}{p{11.1cm}}{\footnotesize Constraints on Level 1 Special Program Products Generation } \\ \cdashline{1-4}
\end{longtable}

     \newpage
\begin{longtable}{p{3.7cm}p{3.7cm}p{3.7cm}p{3.7cm}}\toprule
\multicolumn{2}{l}{\large \textbf{ \hypertarget{sp}{Special Programs Productions} } }
& \multicolumn{2}{l}{(product in: Sci Pipelines SW
)}
\\ \hline
\textbf{\footnotesize Manager} & \textbf{\footnotesize Owner} &
\textbf{\footnotesize WBS} & \textbf{\footnotesize Team} \\ \hline
\parbox{3.5cm}{
John Swinbank
\vspace{2mm}%
} &
\begin{tabular}{@{}l@{}}
\parbox{3.5cm}{
Leanne Guy
\vspace{2mm}%
} \\
\end{tabular} &
\begin{tabular}{@{}l@{}}
1.02C.03 \\
1.02C.04 \\
\end{tabular} & \begin{tabular}{@{}l@{}}
AP \\
DRP \\
\end{tabular} \\ \hline
\multicolumn{4}{c}{
{\footnotesize ( Spec Prog SW
 - SP ) }
}\\ \hline
\end{longtable}

Software product for special programs data processing.


\begin{longtable}{p{3.7cm}p{3.7cm}p{3.7cm}p{3.7cm}}\hline
\textbf{\footnotesize Uses:}  & & \textbf{\footnotesize Used in:} & \\ \hline
\multicolumn{2}{c}{
\begin{tabular}{c}
\hyperlink{scipipe}{Science Pipelines Libraries} \\ \hline
\end{tabular}
} &
\multicolumn{2}{c}{
\begin{tabular}{c}
\hyperlink{prodsrv}{Batch Production} \\ \hline
\hyperlink{spdist}{Science Pipelines Distribution} \\ \hline
\end{tabular}
} \\ \bottomrule
\multicolumn{4}{c}{\textbf{Related Requirements} } \\ \hline
{\footnotesize DMS-REQ-0320 } &
\multicolumn{3}{p{11.1cm}}{\footnotesize Processing of Data From Special Programs } \\ \cdashline{1-4}
{\footnotesize DMS-REQ-0322 } &
\multicolumn{3}{p{11.1cm}}{\footnotesize Special Programs Database } \\ \cdashline{1-4}
\end{longtable}

     \newpage
\begin{longtable}{p{3.7cm}p{3.7cm}p{3.7cm}p{3.7cm}}\toprule
\multicolumn{2}{l}{\large \textbf{ \hypertarget{tmplgen}{Template Generation} } }
& \multicolumn{2}{l}{(product in: Sci Pipelines SW
)}
\\ \hline
\textbf{\footnotesize Manager} & \textbf{\footnotesize Owner} &
\textbf{\footnotesize WBS} & \textbf{\footnotesize Team} \\ \hline
\parbox{3.5cm}{
Yusra AlSayyad
\vspace{2mm}%
} &
\begin{tabular}{@{}l@{}}
\parbox{3.5cm}{
Jim Bosch
\vspace{2mm}%
} \\
\end{tabular} &
\begin{tabular}{@{}l@{}}
1.02C.04.04 \\
\end{tabular} & \begin{tabular}{@{}l@{}}
DRP \\
\end{tabular} \\ \hline
\multicolumn{4}{c}{
{\footnotesize ( Tmpl Gen SW
 - TMPLGEN ) }
}\\ \hline
\end{longtable}




\begin{longtable}{p{3.7cm}p{3.7cm}p{3.7cm}p{3.7cm}}\hline
\textbf{\footnotesize Uses:}  & & \textbf{\footnotesize Used in:} & \\ \hline
\multicolumn{2}{c}{
} &
\multicolumn{2}{c}{
\begin{tabular}{c}
\hyperlink{spdist}{Science Pipelines Distribution} \\ \hline
\end{tabular}
} \\ \bottomrule
\multicolumn{4}{c}{\textbf{Related Requirements} } \\ \hline
{\footnotesize DMS-REQ-0280 } &
\multicolumn{3}{p{11.1cm}}{\footnotesize Template Coadds } \\ \cdashline{1-4}
{\footnotesize DMS-REQ-0308 } &
\multicolumn{3}{p{11.1cm}}{\footnotesize Software Architecture to Enable Community Re-Use } \\ \cdashline{1-4}
\end{longtable}

     \newpage
\begin{longtable}{p{3.7cm}p{3.7cm}p{3.7cm}p{3.7cm}}\toprule
\multicolumn{2}{l}{\large \textbf{ \hypertarget{splug}{Science Plugins} } }
& \multicolumn{2}{l}{(product in: Sci Pipelines SW
)}
\\ \hline
\textbf{\footnotesize Manager} & \textbf{\footnotesize Owner} &
\textbf{\footnotesize WBS} & \textbf{\footnotesize Team} \\ \hline
\parbox{3.5cm}{
John Swinbank
\vspace{2mm}%
} &
\begin{tabular}{@{}l@{}}
\parbox{3.5cm}{
Leanne Guy
\vspace{2mm}%
} \\
\end{tabular} &
\begin{tabular}{@{}l@{}}
1.02C.04.02 \\
\end{tabular} & \begin{tabular}{@{}l@{}}
DRP \\
\end{tabular} \\ \hline
\multicolumn{4}{c}{
{\footnotesize ( Science Plugins
 - SPLUG ) }
}\\ \hline
\end{longtable}

This software product includes all Science Pipelines plugin packages. A
top level meta-package need to be defined, and all corresponding plugins
shall to be a dependency in it. Definition: a plugin is any piece of
software that meets a well-defined interface standard and can be
configured in or out of the processing by the user.


\begin{longtable}{p{3.7cm}p{3.7cm}p{3.7cm}p{3.7cm}}\hline
\textbf{\footnotesize Uses:}  & & \textbf{\footnotesize Used in:} & \\ \hline
\multicolumn{2}{c}{
} &
\multicolumn{2}{c}{
\begin{tabular}{c}
\hyperlink{spdist}{Science Pipelines Distribution} \\ \hline
\end{tabular}
} \\ \bottomrule
\multicolumn{4}{c}{\textbf{Related Requirements} } \\ \hline
{\footnotesize DMS-REQ-0059 } &
\multicolumn{3}{p{11.1cm}}{\footnotesize Bad Pixel Map } \\ \cdashline{1-4}
{\footnotesize DMS-REQ-0060 } &
\multicolumn{3}{p{11.1cm}}{\footnotesize Bias Residual Image } \\ \cdashline{1-4}
{\footnotesize DMS-REQ-0061 } &
\multicolumn{3}{p{11.1cm}}{\footnotesize Crosstalk Correction Matrix } \\ \cdashline{1-4}
{\footnotesize DMS-REQ-0062 } &
\multicolumn{3}{p{11.1cm}}{\footnotesize Illumination Correction Frame } \\ \cdashline{1-4}
{\footnotesize DMS-REQ-0063 } &
\multicolumn{3}{p{11.1cm}}{\footnotesize Monochromatic Flatfield Data Cube } \\ \cdashline{1-4}
{\footnotesize DMS-REQ-0100 } &
\multicolumn{3}{p{11.1cm}}{\footnotesize Generate Calibration Report Within Specified Time } \\ \cdashline{1-4}
{\footnotesize DMS-REQ-0101 } &
\multicolumn{3}{p{11.1cm}}{\footnotesize Level 1 Calibration Report Definition } \\ \cdashline{1-4}
{\footnotesize DMS-REQ-0130 } &
\multicolumn{3}{p{11.1cm}}{\footnotesize Calibration Data Products } \\ \cdashline{1-4}
{\footnotesize DMS-REQ-0131 } &
\multicolumn{3}{p{11.1cm}}{\footnotesize Calibration Images Available Within Specified Time } \\ \cdashline{1-4}
{\footnotesize DMS-REQ-0132 } &
\multicolumn{3}{p{11.1cm}}{\footnotesize Calibration Image Provenance } \\ \cdashline{1-4}
{\footnotesize DMS-REQ-0265 } &
\multicolumn{3}{p{11.1cm}}{\footnotesize Guider Calibration Data Acquisition } \\ \cdashline{1-4}
{\footnotesize DMS-REQ-0282 } &
\multicolumn{3}{p{11.1cm}}{\footnotesize Dark Current Correction Frame } \\ \cdashline{1-4}
{\footnotesize DMS-REQ-0283 } &
\multicolumn{3}{p{11.1cm}}{\footnotesize Fringe Correction Frame } \\ \cdashline{1-4}
{\footnotesize DMS-REQ-0289 } &
\multicolumn{3}{p{11.1cm}}{\footnotesize Calibration Production Processing } \\ \cdashline{1-4}
{\footnotesize DMS-REQ-0308 } &
\multicolumn{3}{p{11.1cm}}{\footnotesize Software Architecture to Enable Community Re-Use } \\ \cdashline{1-4}
\end{longtable}

     \newpage
\begin{longtable}{p{3.7cm}p{3.7cm}p{3.7cm}p{3.7cm}}\toprule
\multicolumn{2}{l}{\large \textbf{ \hypertarget{spdist}{Science Pipelines Distribution} } }
& \multicolumn{2}{l}{(product in: Sci Pipelines SW
)}
\\ \hline
\textbf{\footnotesize Manager} & \textbf{\footnotesize Owner} &
\textbf{\footnotesize WBS} & \textbf{\footnotesize Team} \\ \hline
\parbox{3.5cm}{
John Swinbank
\vspace{2mm}%
} &
\begin{tabular}{@{}l@{}}
\parbox{3.5cm}{
Leanne Guy
\vspace{2mm}%
} \\
\end{tabular} &
\begin{tabular}{@{}l@{}}
1.02C.04.02 \\
\end{tabular} & \begin{tabular}{@{}l@{}}
DRP \\
\end{tabular} \\ \hline
\multicolumn{4}{c}{
{\footnotesize ( Science P. Dist.
 - SPDIST ) }
}\\ \hline
\end{longtable}

Top level distribution product for the science pipelines


\begin{longtable}{p{3.7cm}p{3.7cm}p{3.7cm}p{3.7cm}}\hline
\multicolumn{2}{r}{\textbf{GutHub Packages:}} &
\multicolumn{2}{l}{\href{https://github.com/lsst/lsst_distrib}{lsst\_distrib} }\ref{lsst_distrib}
\\ \hline \\ \hline
\textbf{\footnotesize Uses:}  & & \textbf{\footnotesize Used in:} & \\ \hline
\multicolumn{2}{c}{
\begin{tabular}{c}
\hyperlink{apprmpt}{Alert Production} \\ \hline
\hyperlink{dmcal}{Calibration SW} \\ \hline
\hyperlink{drp}{Data Release Production} \\ \hline
\hyperlink{mops}{MOPS and Forced Photometry} \\ \hline
\hyperlink{sp}{Special Programs Productions} \\ \hline
\hyperlink{splug}{Science Plugins} \\ \hline
\hyperlink{tmplgen}{Template Generation} \\ \hline
\hyperlink{butler}{Data Butler} \\ \hline
\hyperlink{scipipe}{Science Pipelines Libraries} \\ \hline
\hyperlink{txf}{Task Framework} \\ \hline
\hyperlink{and}{Astrometry.net Data} \\ \hline
\end{tabular}
} &
\multicolumn{2}{c}{
\begin{tabular}{c}
\hyperlink{lspnbl}{LSP Nublado} \\ \hline
\end{tabular}
} \\ \bottomrule
\multicolumn{4}{c}{\textbf{Related Requirements} } \\ \hline
{\footnotesize CA-DM-DAQ-ICD-0052 } &
\multicolumn{3}{p{11.1cm}}{\footnotesize Correction constants for science sensors sourced by Data Management } \\ \cdashline{1-4}
{\footnotesize DMS-REQ-0059 } &
\multicolumn{3}{p{11.1cm}}{\footnotesize Bad Pixel Map } \\ \cdashline{1-4}
{\footnotesize DMS-REQ-0060 } &
\multicolumn{3}{p{11.1cm}}{\footnotesize Bias Residual Image } \\ \cdashline{1-4}
{\footnotesize DMS-REQ-0061 } &
\multicolumn{3}{p{11.1cm}}{\footnotesize Crosstalk Correction Matrix } \\ \cdashline{1-4}
{\footnotesize DMS-REQ-0062 } &
\multicolumn{3}{p{11.1cm}}{\footnotesize Illumination Correction Frame } \\ \cdashline{1-4}
{\footnotesize DMS-REQ-0063 } &
\multicolumn{3}{p{11.1cm}}{\footnotesize Monochromatic Flatfield Data Cube } \\ \cdashline{1-4}
{\footnotesize DMS-REQ-0130 } &
\multicolumn{3}{p{11.1cm}}{\footnotesize Calibration Data Products } \\ \cdashline{1-4}
{\footnotesize DMS-REQ-0132 } &
\multicolumn{3}{p{11.1cm}}{\footnotesize Calibration Image Provenance } \\ \cdashline{1-4}
{\footnotesize DMS-REQ-0265 } &
\multicolumn{3}{p{11.1cm}}{\footnotesize Guider Calibration Data Acquisition } \\ \cdashline{1-4}
{\footnotesize DMS-REQ-0282 } &
\multicolumn{3}{p{11.1cm}}{\footnotesize Dark Current Correction Frame } \\ \cdashline{1-4}
{\footnotesize DMS-REQ-0283 } &
\multicolumn{3}{p{11.1cm}}{\footnotesize Fringe Correction Frame } \\ \cdashline{1-4}
{\footnotesize DMS-REQ-0289 } &
\multicolumn{3}{p{11.1cm}}{\footnotesize Calibration Production Processing } \\ \cdashline{1-4}
{\footnotesize DMS-REQ-0308 } &
\multicolumn{3}{p{11.1cm}}{\footnotesize Software Architecture to Enable Community Re-Use } \\ \cdashline{1-4}
\end{longtable}

    
   \newpage
\subsubsection{Supporting SW Products}\label{suppsw}
\begin{longtable}{p{3.7cm}p{3.7cm}p{3.7cm}p{3.7cm}}\hline
\textbf{Manager} & \textbf{Owner} & \textbf{WBS} & \textbf{Team} \\ \hline
\parbox{3.5cm}{
John Swinbank
\vspace{2mm}%
} &
\begin{tabular}{@{}l@{}}
\parbox{3.5cm}{

\vspace{2mm}%
} \\
\end{tabular}
 &
\begin{tabular}{@{}l@{}}
 \\
\end{tabular} &
\begin{tabular}{@{}l@{}}
 \\
\end{tabular} \\ \hline
\multicolumn{4}{c}{
{\footnotesize ( Supporting SW
 - SUPPSW ) }
}\\ \hline
\end{longtable}

  DM software products used as libraries and shared between different
other DM software products.


   \newpage
\begin{longtable}{p{3.7cm}p{3.7cm}p{3.7cm}p{3.7cm}}\toprule
\multicolumn{2}{l}{\large \textbf{ \hypertarget{scipipe}{Science Pipelines Libraries} } }
& \multicolumn{2}{l}{(product in: Supporting SW
)}
\\ \hline
\textbf{\footnotesize Manager} & \textbf{\footnotesize Owner} &
\textbf{\footnotesize WBS} & \textbf{\footnotesize Team} \\ \hline
\parbox{3.5cm}{
John Swinbank
\vspace{2mm}%
} &
\begin{tabular}{@{}l@{}}
\parbox{3.5cm}{
Jim Bosch
\vspace{2mm}%
} \\
\end{tabular} &
\begin{tabular}{@{}l@{}}
1.02C.04 \\
1.02C.03 \\
\end{tabular} & \begin{tabular}{@{}l@{}}
DRP \\
AP \\
\end{tabular} \\ \hline
\multicolumn{4}{c}{
{\footnotesize ( Sci Pipelines Libs
 - SCIPIPE ) }
}\\ \hline
\end{longtable}

Science Pipeline Top Level Software product.


\begin{longtable}{p{3.7cm}p{3.7cm}p{3.7cm}p{3.7cm}}\hline
\multicolumn{2}{r}{\textbf{GutHub Packages:}} &
\multicolumn{2}{l}{\href{https://github.com/lsst/lsst_apps}{lsst\_apps} }\ref{lsst_apps}
\\ \hline \\ \hline
\textbf{\footnotesize Uses:}  & & \textbf{\footnotesize Used in:} & \\ \hline
\multicolumn{2}{c}{
} &
\multicolumn{2}{c}{
\begin{tabular}{c}
\hyperlink{apprmpt}{Alert Production} \\ \hline
\hyperlink{dmcal}{Calibration SW} \\ \hline
\hyperlink{drp}{Data Release Production} \\ \hline
\hyperlink{mops}{MOPS and Forced Photometry} \\ \hline
\hyperlink{sp}{Special Programs Productions} \\ \hline
\hyperlink{spdist}{Science Pipelines Distribution} \\ \hline
\end{tabular}
} \\ \bottomrule
\multicolumn{4}{c}{\textbf{Related Requirements} } \\ \hline
{\footnotesize DM-TS-CON-ICD-0008 } &
\multicolumn{3}{p{11.1cm}}{\footnotesize LSST Stack Availability } \\ \cdashline{1-4}
{\footnotesize DMS-REQ-0032 } &
\multicolumn{3}{p{11.1cm}}{\footnotesize Image Differencing } \\ \cdashline{1-4}
{\footnotesize DMS-REQ-0033 } &
\multicolumn{3}{p{11.1cm}}{\footnotesize Provide Source Detection Software } \\ \cdashline{1-4}
{\footnotesize DMS-REQ-0042 } &
\multicolumn{3}{p{11.1cm}}{\footnotesize Provide Astrometric Model } \\ \cdashline{1-4}
{\footnotesize DMS-REQ-0043 } &
\multicolumn{3}{p{11.1cm}}{\footnotesize Provide Calibrated Photometry } \\ \cdashline{1-4}
{\footnotesize DMS-REQ-0052 } &
\multicolumn{3}{p{11.1cm}}{\footnotesize Enable a Range of Shape Measurement Approaches } \\ \cdashline{1-4}
{\footnotesize DMS-REQ-0296 } &
\multicolumn{3}{p{11.1cm}}{\footnotesize Pre-cursor, and Real Data } \\ \cdashline{1-4}
{\footnotesize DMS-REQ-0308 } &
\multicolumn{3}{p{11.1cm}}{\footnotesize Software Architecture to Enable Community Re-Use } \\ \cdashline{1-4}
{\footnotesize DMS-REQ-0351 } &
\multicolumn{3}{p{11.1cm}}{\footnotesize Provide Beam Projector Coordinate Calculation Software } \\ \cdashline{1-4}
\end{longtable}

     \newpage
\begin{longtable}{p{3.7cm}p{3.7cm}p{3.7cm}p{3.7cm}}\toprule
\multicolumn{2}{l}{\large \textbf{ \hypertarget{butler}{Data Butler} } }
& \multicolumn{2}{l}{(product in: Supporting SW
)}
\\ \hline
\textbf{\footnotesize Manager} & \textbf{\footnotesize Owner} &
\textbf{\footnotesize WBS} & \textbf{\footnotesize Team} \\ \hline
\parbox{3.5cm}{
Michelle Butler
\vspace{2mm}%
} &
\begin{tabular}{@{}l@{}}
\parbox{3.5cm}{
Tim Jenness
\vspace{2mm}%
} \\
\end{tabular} &
\begin{tabular}{@{}l@{}}
1.02C.06.02.01 \\
\end{tabular} & \begin{tabular}{@{}l@{}}
DAX \\
\end{tabular} \\ \hline
\multicolumn{4}{c}{
{\footnotesize ( Data Butler
 - BUTLER ) }
}\\ \hline
\citeds{LDM-556} &
\multicolumn{3}{l}{ Data Management Middleware Requirements }
\\ \cdashline{1-4}
\citeds{LDM-152} &
\multicolumn{3}{l}{ Data Management Middleware Design }
\\ \cdashline{1-4}
\end{longtable}

Butler middleware software product.


\begin{longtable}{p{3.7cm}p{3.7cm}p{3.7cm}p{3.7cm}}\hline
\multicolumn{2}{r}{\textbf{GutHub Packages:}} &
\multicolumn{2}{l}{\href{https://github.com/lsst/daf_butler}{daf\_butler} }\ref{daf_butler}
\\ \hline \\ \hline
\textbf{\footnotesize Uses:}  & & \textbf{\footnotesize Used in:} & \\ \hline
\multicolumn{2}{c}{
} &
\multicolumn{2}{c}{
\begin{tabular}{c}
\hyperlink{spdist}{Science Pipelines Distribution} \\ \hline
\end{tabular}
} \\ \bottomrule
\multicolumn{4}{c}{\textbf{Related Requirements} } \\ \hline
{\footnotesize DMS-REQ-0121 } &
\multicolumn{3}{p{11.1cm}}{\footnotesize Provenance for Level 3 processing at DACs } \\ \cdashline{1-4}
{\footnotesize DMS-REQ-0125 } &
\multicolumn{3}{p{11.1cm}}{\footnotesize Software framework for Level 3 catalog processing } \\ \cdashline{1-4}
{\footnotesize DMS-REQ-0128 } &
\multicolumn{3}{p{11.1cm}}{\footnotesize Software framework for Level 3 image processing } \\ \cdashline{1-4}
{\footnotesize DMS-REQ-0293 } &
\multicolumn{3}{p{11.1cm}}{\footnotesize Selection of Datasets } \\ \cdashline{1-4}
{\footnotesize DMS-REQ-0295 } &
\multicolumn{3}{p{11.1cm}}{\footnotesize Transparent Data Access } \\ \cdashline{1-4}
{\footnotesize DMS-REQ-0296 } &
\multicolumn{3}{p{11.1cm}}{\footnotesize Pre-cursor, and Real Data } \\ \cdashline{1-4}
{\footnotesize DMS-REQ-0308 } &
\multicolumn{3}{p{11.1cm}}{\footnotesize Software Architecture to Enable Community Re-Use } \\ \cdashline{1-4}
{\footnotesize DMS-REQ-0314 } &
\multicolumn{3}{p{11.1cm}}{\footnotesize Compute Platform Heterogeneity } \\ \cdashline{1-4}
{\footnotesize EP-DM-CON-ICD-0035 } &
\multicolumn{3}{p{11.1cm}}{\footnotesize DM Software } \\ \cdashline{1-4}
\end{longtable}

     \newpage
\begin{longtable}{p{3.7cm}p{3.7cm}p{3.7cm}p{3.7cm}}\toprule
\multicolumn{2}{l}{\large \textbf{ \hypertarget{txf}{Task Framework} } }
& \multicolumn{2}{l}{(product in: Supporting SW
)}
\\ \hline
\textbf{\footnotesize Manager} & \textbf{\footnotesize Owner} &
\textbf{\footnotesize WBS} & \textbf{\footnotesize Team} \\ \hline
\parbox{3.5cm}{
Michelle Butler
\vspace{2mm}%
} &
\begin{tabular}{@{}l@{}}
\parbox{3.5cm}{
Tim Jenness
\vspace{2mm}%
} \\
\end{tabular} &
\begin{tabular}{@{}l@{}}
1.02C.06.03 \\
\end{tabular} & \begin{tabular}{@{}l@{}}
DAX \\
\end{tabular} \\ \hline
\multicolumn{4}{c}{
{\footnotesize ( Task Framework
 - TXF ) }
}\\ \hline
\end{longtable}

SuperTask middleware software product.


\begin{longtable}{p{3.7cm}p{3.7cm}p{3.7cm}p{3.7cm}}\hline
\multicolumn{2}{r}{\textbf{GutHub Packages:}} &
\multicolumn{2}{l}{\href{https://github.com/lsst/pipe_supertask}{pipe\_supertask} }\ref{pipe_supertask}
\\ \hline \\ \hline
\textbf{\footnotesize Uses:}  & & \textbf{\footnotesize Used in:} & \\ \hline
\multicolumn{2}{c}{
} &
\multicolumn{2}{c}{
\begin{tabular}{c}
\hyperlink{spdist}{Science Pipelines Distribution} \\ \hline
\end{tabular}
} \\ \bottomrule
\multicolumn{4}{c}{\textbf{Related Requirements} } \\ \hline
{\footnotesize DMS-REQ-0121 } &
\multicolumn{3}{p{11.1cm}}{\footnotesize Provenance for Level 3 processing at DACs } \\ \cdashline{1-4}
{\footnotesize DMS-REQ-0125 } &
\multicolumn{3}{p{11.1cm}}{\footnotesize Software framework for Level 3 catalog processing } \\ \cdashline{1-4}
{\footnotesize DMS-REQ-0128 } &
\multicolumn{3}{p{11.1cm}}{\footnotesize Software framework for Level 3 image processing } \\ \cdashline{1-4}
{\footnotesize DMS-REQ-0158 } &
\multicolumn{3}{p{11.1cm}}{\footnotesize Provide Pipeline Construction Services } \\ \cdashline{1-4}
{\footnotesize DMS-REQ-0294 } &
\multicolumn{3}{p{11.1cm}}{\footnotesize Processing of Datasets } \\ \cdashline{1-4}
{\footnotesize DMS-REQ-0304 } &
\multicolumn{3}{p{11.1cm}}{\footnotesize Production Fault Tolerance } \\ \cdashline{1-4}
{\footnotesize DMS-REQ-0305 } &
\multicolumn{3}{p{11.1cm}}{\footnotesize Task Specification } \\ \cdashline{1-4}
{\footnotesize DMS-REQ-0306 } &
\multicolumn{3}{p{11.1cm}}{\footnotesize Task Configuration } \\ \cdashline{1-4}
{\footnotesize DMS-REQ-0308 } &
\multicolumn{3}{p{11.1cm}}{\footnotesize Software Architecture to Enable Community Re-Use } \\ \cdashline{1-4}
{\footnotesize DMS-REQ-0314 } &
\multicolumn{3}{p{11.1cm}}{\footnotesize Compute Platform Heterogeneity } \\ \cdashline{1-4}
{\footnotesize DMS-REQ-0320 } &
\multicolumn{3}{p{11.1cm}}{\footnotesize Processing of Data From Special Programs } \\ \cdashline{1-4}
{\footnotesize EP-DM-CON-ICD-0035 } &
\multicolumn{3}{p{11.1cm}}{\footnotesize DM Software } \\ \cdashline{1-4}
\end{longtable}

     \newpage
\begin{longtable}{p{3.7cm}p{3.7cm}p{3.7cm}p{3.7cm}}\toprule
\multicolumn{2}{l}{\large \textbf{ \hypertarget{qserv}{Distributed Database} } }
& \multicolumn{2}{l}{(product in: Supporting SW
)}
\\ \hline
\textbf{\footnotesize Manager} & \textbf{\footnotesize Owner} &
\textbf{\footnotesize WBS} & \textbf{\footnotesize Team} \\ \hline
\parbox{3.5cm}{
Fritz Mueller
\vspace{2mm}%
} &
\begin{tabular}{@{}l@{}}
\parbox{3.5cm}{
Colin Slater
\vspace{2mm}%
} \\
\end{tabular} &
\begin{tabular}{@{}l@{}}
1.02C.06.02.03 \\
\end{tabular} & \begin{tabular}{@{}l@{}}
DAX \\
\end{tabular} \\ \hline
\multicolumn{4}{c}{
{\footnotesize ( Distrib Database
 - QSERV ) }
}\\ \hline
\citeds{DMTN-022} &
\multicolumn{3}{l}{ Tracks to optimize Qserv containers deployment and orchestration }
\\ \cdashline{1-4}
\citeds{LDM-135} &
\multicolumn{3}{l}{ Data Management Database Design }
\\ \cdashline{1-4}
\citeds{LDM-555} &
\multicolumn{3}{l}{ Data Management Database Requirements }
\\ \cdashline{1-4}
\end{longtable}

Distributed database software product.\\
It will be used to implement the database used by the LSP to serve the
release catalog data.\\
\hspace*{0.333em}\hspace*{0.333em}\hspace*{0.333em}\hspace*{0.333em}\hspace*{0.333em}\hspace*{0.333em}\\
\emph{{{[}last review: F.Mueller - Jan 2020{]} }}\\
\hspace*{0.333em}\hspace*{0.333em}\hspace*{0.333em}\hspace*{0.333em}\hspace*{0.333em}\\


\begin{longtable}{p{3.7cm}p{3.7cm}p{3.7cm}p{3.7cm}}\hline
\multicolumn{2}{r}{\textbf{GutHub Packages:}} &
\multicolumn{2}{l}{\href{https://github.com/lsst/qserv}{qserv} }\ref{qserv}
\\ \hline \\ \hline
\textbf{\footnotesize Uses:}  & & \textbf{\footnotesize Used in:} & \\ \hline
\multicolumn{2}{c}{
} &
\multicolumn{2}{c}{
\begin{tabular}{c}
\hyperlink{lspdb}{LSP Database} \\ \hline
\end{tabular}
} \\ \bottomrule
\multicolumn{4}{c}{\textbf{Related Requirements} } \\ \hline
{\footnotesize DMS-REQ-0046 } &
\multicolumn{3}{p{11.1cm}}{\footnotesize Provide Photometric Redshifts of Galaxies } \\ \cdashline{1-4}
{\footnotesize DMS-REQ-0075 } &
\multicolumn{3}{p{11.1cm}}{\footnotesize Catalog Queries } \\ \cdashline{1-4}
{\footnotesize DMS-REQ-0077 } &
\multicolumn{3}{p{11.1cm}}{\footnotesize Maintain Archive Publicly Accessible } \\ \cdashline{1-4}
{\footnotesize DMS-REQ-0267 } &
\multicolumn{3}{p{11.1cm}}{\footnotesize Source Catalog } \\ \cdashline{1-4}
{\footnotesize DMS-REQ-0268 } &
\multicolumn{3}{p{11.1cm}}{\footnotesize Forced-Source Catalog } \\ \cdashline{1-4}
{\footnotesize DMS-REQ-0275 } &
\multicolumn{3}{p{11.1cm}}{\footnotesize Object Catalog } \\ \cdashline{1-4}
{\footnotesize DMS-REQ-0276 } &
\multicolumn{3}{p{11.1cm}}{\footnotesize Object Characterization } \\ \cdashline{1-4}
{\footnotesize DMS-REQ-0277 } &
\multicolumn{3}{p{11.1cm}}{\footnotesize Coadd Source Catalog } \\ \cdashline{1-4}
{\footnotesize DMS-REQ-0290 } &
\multicolumn{3}{p{11.1cm}}{\footnotesize Level 3 Data Import } \\ \cdashline{1-4}
{\footnotesize DMS-REQ-0291 } &
\multicolumn{3}{p{11.1cm}}{\footnotesize Query Repeatability } \\ \cdashline{1-4}
{\footnotesize DMS-REQ-0292 } &
\multicolumn{3}{p{11.1cm}}{\footnotesize Uniqueness of IDs Across Data Releases } \\ \cdashline{1-4}
{\footnotesize DMS-REQ-0293 } &
\multicolumn{3}{p{11.1cm}}{\footnotesize Selection of Datasets } \\ \cdashline{1-4}
{\footnotesize DMS-REQ-0313 } &
\multicolumn{3}{p{11.1cm}}{\footnotesize Level 1 \&  2 Catalog Access } \\ \cdashline{1-4}
{\footnotesize DMS-REQ-0322 } &
\multicolumn{3}{p{11.1cm}}{\footnotesize Special Programs Database } \\ \cdashline{1-4}
{\footnotesize DMS-REQ-0331 } &
\multicolumn{3}{p{11.1cm}}{\footnotesize Computing Derived Quantities } \\ \cdashline{1-4}
{\footnotesize DMS-REQ-0332 } &
\multicolumn{3}{p{11.1cm}}{\footnotesize Denormalizing Database Tables } \\ \cdashline{1-4}
{\footnotesize DMS-REQ-0333 } &
\multicolumn{3}{p{11.1cm}}{\footnotesize Maximum Likelihood Values and Covariances } \\ \cdashline{1-4}
{\footnotesize DMS-REQ-0340 } &
\multicolumn{3}{p{11.1cm}}{\footnotesize Access Controls of Level 3 Data Products } \\ \cdashline{1-4}
{\footnotesize DMS-REQ-0345 } &
\multicolumn{3}{p{11.1cm}}{\footnotesize Logging of catalog queries } \\ \cdashline{1-4}
{\footnotesize DMS-REQ-0347 } &
\multicolumn{3}{p{11.1cm}}{\footnotesize Measurements in catalogs } \\ \cdashline{1-4}
{\footnotesize DMS-REQ-0350 } &
\multicolumn{3}{p{11.1cm}}{\footnotesize Associating Objects across data releases } \\ \cdashline{1-4}
{\footnotesize DMS-REQ-0363 } &
\multicolumn{3}{p{11.1cm}}{\footnotesize Access to Previous Data Releases } \\ \cdashline{1-4}
{\footnotesize DMS-REQ-0365 } &
\multicolumn{3}{p{11.1cm}}{\footnotesize Operations Subsets } \\ \cdashline{1-4}
{\footnotesize DMS-REQ-0366 } &
\multicolumn{3}{p{11.1cm}}{\footnotesize Subsets Support } \\ \cdashline{1-4}
{\footnotesize DMS-REQ-0367 } &
\multicolumn{3}{p{11.1cm}}{\footnotesize Access Services Performance } \\ \cdashline{1-4}
{\footnotesize DMS-REQ-0368 } &
\multicolumn{3}{p{11.1cm}}{\footnotesize Implementation Provisions } \\ \cdashline{1-4}
{\footnotesize DMS-REQ-0369 } &
\multicolumn{3}{p{11.1cm}}{\footnotesize Evolution } \\ \cdashline{1-4}
{\footnotesize DMS-REQ-0370 } &
\multicolumn{3}{p{11.1cm}}{\footnotesize Older Release Behavior } \\ \cdashline{1-4}
{\footnotesize DMS-REQ-0371 } &
\multicolumn{3}{p{11.1cm}}{\footnotesize Query Availability } \\ \cdashline{1-4}
\end{longtable}

     \newpage
\begin{longtable}{p{3.7cm}p{3.7cm}p{3.7cm}p{3.7cm}}\toprule
\multicolumn{2}{l}{\large \textbf{ \hypertarget{adql}{ADQL Translator} } }
& \multicolumn{2}{l}{(product in: Supporting SW
)}
\\ \hline
\textbf{\footnotesize Manager} & \textbf{\footnotesize Owner} &
\textbf{\footnotesize WBS} & \textbf{\footnotesize Team} \\ \hline
\parbox{3.5cm}{
Frossie Economou
\vspace{2mm}%
} &
\begin{tabular}{@{}l@{}}
\parbox{3.5cm}{
Simon Krughoff
\vspace{2mm}%
} \\
\end{tabular} &
\begin{tabular}{@{}l@{}}
1.02C.06.02.05 \\
\end{tabular} & \begin{tabular}{@{}l@{}}
DAX \\
\end{tabular} \\ \hline
\multicolumn{4}{c}{
{\footnotesize ( ADQL Translator
 - ADQL ) }
}\\ \hline
\end{longtable}

{[}OBSOLETE{]}\\
DAX Query Services in Kotlin.\\[2\baselineskip]


\begin{longtable}{p{3.7cm}p{3.7cm}p{3.7cm}p{3.7cm}}\hline
\multicolumn{2}{r}{\textbf{GutHub Packages:}} &
\multicolumn{2}{l}{\href{https://github.com/lsst/albuquery}{albuquery} }\ref{albuquery}
\\ \hline \\ \hline
\textbf{\footnotesize Uses:}  & & \textbf{\footnotesize Used in:} & \\ \hline
\multicolumn{2}{c}{
} &
\multicolumn{2}{c}{
} \\ \bottomrule
\multicolumn{4}{c}{\textbf{Related Requirements} } \\ \hline
{\footnotesize DMS-REQ-0075 } &
\multicolumn{3}{p{11.1cm}}{\footnotesize Catalog Queries } \\ \cdashline{1-4}
{\footnotesize DMS-REQ-0078 } &
\multicolumn{3}{p{11.1cm}}{\footnotesize Catalog Export Formats } \\ \cdashline{1-4}
{\footnotesize DMS-REQ-0123 } &
\multicolumn{3}{p{11.1cm}}{\footnotesize Access to input catalogs for DAC-based Level 3 processing } \\ \cdashline{1-4}
{\footnotesize DMS-REQ-0155 } &
\multicolumn{3}{p{11.1cm}}{\footnotesize Provide Data Access Services } \\ \cdashline{1-4}
{\footnotesize DMS-REQ-0291 } &
\multicolumn{3}{p{11.1cm}}{\footnotesize Query Repeatability } \\ \cdashline{1-4}
{\footnotesize DMS-REQ-0298 } &
\multicolumn{3}{p{11.1cm}}{\footnotesize Data Product and Raw Data Access } \\ \cdashline{1-4}
{\footnotesize DMS-REQ-0322 } &
\multicolumn{3}{p{11.1cm}}{\footnotesize Special Programs Database } \\ \cdashline{1-4}
{\footnotesize DMS-REQ-0323 } &
\multicolumn{3}{p{11.1cm}}{\footnotesize Calculating SSObject Parameters } \\ \cdashline{1-4}
{\footnotesize DMS-REQ-0331 } &
\multicolumn{3}{p{11.1cm}}{\footnotesize Computing Derived Quantities } \\ \cdashline{1-4}
{\footnotesize DMS-REQ-0340 } &
\multicolumn{3}{p{11.1cm}}{\footnotesize Access Controls of Level 3 Data Products } \\ \cdashline{1-4}
{\footnotesize DMS-REQ-0345 } &
\multicolumn{3}{p{11.1cm}}{\footnotesize Logging of catalog queries } \\ \cdashline{1-4}
{\footnotesize DMS-REQ-0364 } &
\multicolumn{3}{p{11.1cm}}{\footnotesize Data Access Services } \\ \cdashline{1-4}
{\footnotesize DMS-REQ-0368 } &
\multicolumn{3}{p{11.1cm}}{\footnotesize Implementation Provisions } \\ \cdashline{1-4}
{\footnotesize DMS-REQ-0369 } &
\multicolumn{3}{p{11.1cm}}{\footnotesize Evolution } \\ \cdashline{1-4}
{\footnotesize EP-DM-CON-ICD-0034 } &
\multicolumn{3}{p{11.1cm}}{\footnotesize Citizen Science Data } \\ \cdashline{1-4}
\end{longtable}

    
     \newpage
\subsection{Infrastructure Products}\label{infra}
\begin{longtable}{p{3.7cm}p{3.7cm}p{3.7cm}p{3.7cm}}\hline
\textbf{Manager} & \textbf{Owner} & \textbf{WBS} & \textbf{Team} \\ \hline
\parbox{3.5cm}{
Wil O'Mullane
\vspace{2mm}%
} &
\begin{tabular}{@{}l@{}}
\parbox{3.5cm}{

\vspace{2mm}%
} \\
\end{tabular}
 &
\begin{tabular}{@{}l@{}}
 \\
\end{tabular} &
\begin{tabular}{@{}l@{}}
 \\
\end{tabular} \\ \hline
\multicolumn{4}{c}{
{\footnotesize ( Infrastructure
 - INFRA ) }
}\\ \hline
\citeds{LDM-129} &
\multicolumn{3}{l}{ LSST Data Facility Logical Information Technology and Communications Design }
\\ \cdashline{1-4}
\end{longtable}

  DM infrastructural products.

~

These products are mainly physical facilities or logical groups
services.

~


\includegraphics[max width=\linewidth]{subtrees/Main_INFRA.pdf}

\newpage
\subsubsection{Networks}\label{net}
\begin{longtable}{p{3.7cm}p{3.7cm}p{3.7cm}p{3.7cm}}\hline
\textbf{Manager} & \textbf{Owner} & \textbf{WBS} & \textbf{Team} \\ \hline
\parbox{3.5cm}{
Jeff Kantor
\vspace{2mm}%
} &
\begin{tabular}{@{}l@{}}
\parbox{3.5cm}{

\vspace{2mm}%
} \\
\end{tabular}
 &
\begin{tabular}{@{}l@{}}
 \\
\end{tabular} &
\begin{tabular}{@{}l@{}}
 \\
\end{tabular} \\ \hline
\multicolumn{4}{c}{
{\footnotesize ( Networks
 - NET ) }
}\\ \hline
\citeds{LSE-78} &
\multicolumn{3}{l}{ LSST Observatory Network Design }
\\ \cdashline{1-4}
\end{longtable}

  High level physical DM networks definition.


   \newpage
\begin{longtable}{p{3.7cm}p{3.7cm}p{3.7cm}p{3.7cm}}\toprule
\multicolumn{2}{l}{\large \textbf{ \hypertarget{netsb}{Summit to Base Network} } }
& \multicolumn{2}{l}{(product in: Networks
)}
\\ \hline
\textbf{\footnotesize Manager} & \textbf{\footnotesize Owner} &
\textbf{\footnotesize WBS} & \textbf{\footnotesize Team} \\ \hline
\parbox{3.5cm}{
Jeff Kantor
\vspace{2mm}%
} &
\begin{tabular}{@{}l@{}}
\parbox{3.5cm}{
Jeff Kantor
\vspace{2mm}%
} \\
\end{tabular} &
\begin{tabular}{@{}l@{}}
1.02C.08.03 \\
\end{tabular} & \begin{tabular}{@{}l@{}}
Net/Base \\
\end{tabular} \\ \hline
\multicolumn{4}{c}{
{\footnotesize ( Sum/Base Net
 - NETSB ) }
}\\ \hline
\end{longtable}

La Serena - AURA Gatehouse Network


\begin{longtable}{p{3.7cm}p{3.7cm}p{3.7cm}p{3.7cm}}\hline
\textbf{\footnotesize Uses:}  & & \textbf{\footnotesize Used in:} & \\ \hline
\multicolumn{2}{c}{
} &
\multicolumn{2}{c}{
} \\ \bottomrule
\multicolumn{4}{c}{\textbf{Related Requirements} } \\ \hline
{\footnotesize DMS-REQ-0165 } &
\multicolumn{3}{p{11.1cm}}{\footnotesize Infrastructure Sizing for "catching up" } \\ \cdashline{1-4}
{\footnotesize DMS-REQ-0166 } &
\multicolumn{3}{p{11.1cm}}{\footnotesize Incorporate Fault-Tolerance } \\ \cdashline{1-4}
{\footnotesize DMS-REQ-0167 } &
\multicolumn{3}{p{11.1cm}}{\footnotesize Incorporate Autonomics } \\ \cdashline{1-4}
{\footnotesize DMS-REQ-0168 } &
\multicolumn{3}{p{11.1cm}}{\footnotesize Summit Facility Data Communications } \\ \cdashline{1-4}
{\footnotesize DMS-REQ-0171 } &
\multicolumn{3}{p{11.1cm}}{\footnotesize Summit to Base Network } \\ \cdashline{1-4}
{\footnotesize DMS-REQ-0172 } &
\multicolumn{3}{p{11.1cm}}{\footnotesize Summit to Base Network Availability } \\ \cdashline{1-4}
{\footnotesize DMS-REQ-0173 } &
\multicolumn{3}{p{11.1cm}}{\footnotesize Summit to Base Network Reliability } \\ \cdashline{1-4}
{\footnotesize DMS-REQ-0174 } &
\multicolumn{3}{p{11.1cm}}{\footnotesize Summit to Base Network Secondary Link } \\ \cdashline{1-4}
{\footnotesize DMS-REQ-0175 } &
\multicolumn{3}{p{11.1cm}}{\footnotesize Summit to Base Network Ownership and Operation } \\ \cdashline{1-4}
{\footnotesize OCS-DM-COM-ICD-0053 } &
\multicolumn{3}{p{11.1cm}}{\footnotesize Summit-Base Network Utilization } \\ \cdashline{1-4}
\end{longtable}

     \newpage
\begin{longtable}{p{3.7cm}p{3.7cm}p{3.7cm}p{3.7cm}}\toprule
\multicolumn{2}{l}{\large \textbf{ \hypertarget{netba}{Base to Archive Network} } }
& \multicolumn{2}{l}{(product in: Networks
)}
\\ \hline
\textbf{\footnotesize Manager} & \textbf{\footnotesize Owner} &
\textbf{\footnotesize WBS} & \textbf{\footnotesize Team} \\ \hline
\parbox{3.5cm}{
Jeff Kantor
\vspace{2mm}%
} &
\begin{tabular}{@{}l@{}}
\parbox{3.5cm}{
Jeff Kantor
\vspace{2mm}%
} \\
\end{tabular} &
\begin{tabular}{@{}l@{}}
1.02C.08.03 \\
\end{tabular} & \begin{tabular}{@{}l@{}}
Net/Base \\
\end{tabular} \\ \hline
\multicolumn{4}{c}{
{\footnotesize ( Base/Arch Net
 - NETBA ) }
}\\ \hline
\end{longtable}




\begin{longtable}{p{3.7cm}p{3.7cm}p{3.7cm}p{3.7cm}}\hline
\textbf{\footnotesize Uses:}  & & \textbf{\footnotesize Used in:} & \\ \hline
\multicolumn{2}{c}{
} &
\multicolumn{2}{c}{
} \\ \bottomrule
\multicolumn{4}{c}{\textbf{Related Requirements} } \\ \hline
{\footnotesize DMS-REQ-0165 } &
\multicolumn{3}{p{11.1cm}}{\footnotesize Infrastructure Sizing for "catching up" } \\ \cdashline{1-4}
{\footnotesize DMS-REQ-0166 } &
\multicolumn{3}{p{11.1cm}}{\footnotesize Incorporate Fault-Tolerance } \\ \cdashline{1-4}
{\footnotesize DMS-REQ-0167 } &
\multicolumn{3}{p{11.1cm}}{\footnotesize Incorporate Autonomics } \\ \cdashline{1-4}
{\footnotesize DMS-REQ-0180 } &
\multicolumn{3}{p{11.1cm}}{\footnotesize Base to Archive Network } \\ \cdashline{1-4}
{\footnotesize DMS-REQ-0181 } &
\multicolumn{3}{p{11.1cm}}{\footnotesize Base to Archive Network Availability } \\ \cdashline{1-4}
{\footnotesize DMS-REQ-0182 } &
\multicolumn{3}{p{11.1cm}}{\footnotesize Base to Archive Network Reliability } \\ \cdashline{1-4}
{\footnotesize DMS-REQ-0183 } &
\multicolumn{3}{p{11.1cm}}{\footnotesize Base to Archive Network Secondary Link } \\ \cdashline{1-4}
{\footnotesize DMS-REQ-0188 } &
\multicolumn{3}{p{11.1cm}}{\footnotesize Archive to Data Access Center Network } \\ \cdashline{1-4}
{\footnotesize DMS-REQ-0189 } &
\multicolumn{3}{p{11.1cm}}{\footnotesize Archive to Data Access Center Network Availability } \\ \cdashline{1-4}
{\footnotesize DMS-REQ-0190 } &
\multicolumn{3}{p{11.1cm}}{\footnotesize Archive to Data Access Center Network Reliability } \\ \cdashline{1-4}
{\footnotesize DMS-REQ-0191 } &
\multicolumn{3}{p{11.1cm}}{\footnotesize Archive to Data Access Center Network Secondary Link } \\ \cdashline{1-4}
{\footnotesize OCS-DM-COM-ICD-0054 } &
\multicolumn{3}{p{11.1cm}}{\footnotesize Base-Archive Network Utilization } \\ \cdashline{1-4}
\end{longtable}

     \newpage
\begin{longtable}{p{3.7cm}p{3.7cm}p{3.7cm}p{3.7cm}}\toprule
\multicolumn{2}{l}{\large \textbf{ \hypertarget{netbase}{Base LAN Network} } }
& \multicolumn{2}{l}{(product in: Networks
)}
\\ \hline
\textbf{\footnotesize Manager} & \textbf{\footnotesize Owner} &
\textbf{\footnotesize WBS} & \textbf{\footnotesize Team} \\ \hline
\parbox{3.5cm}{
Michelle Butler
\vspace{2mm}%
} &
\begin{tabular}{@{}l@{}}
\parbox{3.5cm}{
Michelle Butler
\vspace{2mm}%
} \\
\end{tabular} &
\begin{tabular}{@{}l@{}}
1.02C.07.08 \\
\end{tabular} & \begin{tabular}{@{}l@{}}
LDF \\
\end{tabular} \\ \hline
\multicolumn{4}{c}{
{\footnotesize ( Base LAN
 - NETBASE ) }
}\\ \hline
\end{longtable}

Base Local Area Network.


\begin{longtable}{p{3.7cm}p{3.7cm}p{3.7cm}p{3.7cm}}\hline
\textbf{\footnotesize Uses:}  & & \textbf{\footnotesize Used in:} & \\ \hline
\multicolumn{2}{c}{
} &
\multicolumn{2}{c}{
} \\ \bottomrule
\multicolumn{4}{c}{\textbf{Related Requirements} } \\ \hline
{\footnotesize DMS-REQ-0165 } &
\multicolumn{3}{p{11.1cm}}{\footnotesize Infrastructure Sizing for "catching up" } \\ \cdashline{1-4}
{\footnotesize DMS-REQ-0166 } &
\multicolumn{3}{p{11.1cm}}{\footnotesize Incorporate Fault-Tolerance } \\ \cdashline{1-4}
{\footnotesize DMS-REQ-0167 } &
\multicolumn{3}{p{11.1cm}}{\footnotesize Incorporate Autonomics } \\ \cdashline{1-4}
{\footnotesize DMS-REQ-0193 } &
\multicolumn{3}{p{11.1cm}}{\footnotesize Data Access Centers } \\ \cdashline{1-4}
{\footnotesize DMS-REQ-0194 } &
\multicolumn{3}{p{11.1cm}}{\footnotesize Data Access Center Simultaneous Connections } \\ \cdashline{1-4}
{\footnotesize DMS-REQ-0352 } &
\multicolumn{3}{p{11.1cm}}{\footnotesize Base Wireless LAN (WiFi) } \\ \cdashline{1-4}
\end{longtable}

     \newpage
\begin{longtable}{p{3.7cm}p{3.7cm}p{3.7cm}p{3.7cm}}\toprule
\multicolumn{2}{l}{\large \textbf{ \hypertarget{netncsa}{NCSA LAN Network} } }
& \multicolumn{2}{l}{(product in: Networks
)}
\\ \hline
\textbf{\footnotesize Manager} & \textbf{\footnotesize Owner} &
\textbf{\footnotesize WBS} & \textbf{\footnotesize Team} \\ \hline
\parbox{3.5cm}{
Michelle Butler
\vspace{2mm}%
} &
\begin{tabular}{@{}l@{}}
\parbox{3.5cm}{
Michelle Butler
\vspace{2mm}%
} \\
\end{tabular} &
\begin{tabular}{@{}l@{}}
1.02C.07.09 \\
\end{tabular} & \begin{tabular}{@{}l@{}}
LDF \\
\end{tabular} \\ \hline
\multicolumn{4}{c}{
{\footnotesize ( NCSA LAN
 - NETNCSA ) }
}\\ \hline
\end{longtable}

NCSA Local Area Network.


\begin{longtable}{p{3.7cm}p{3.7cm}p{3.7cm}p{3.7cm}}\hline
\textbf{\footnotesize Uses:}  & & \textbf{\footnotesize Used in:} & \\ \hline
\multicolumn{2}{c}{
} &
\multicolumn{2}{c}{
} \\ \bottomrule
\multicolumn{4}{c}{\textbf{Related Requirements} } \\ \hline
{\footnotesize DMS-REQ-0163 } &
\multicolumn{3}{p{11.1cm}}{\footnotesize Re-processing Capacity } \\ \cdashline{1-4}
{\footnotesize DMS-REQ-0165 } &
\multicolumn{3}{p{11.1cm}}{\footnotesize Infrastructure Sizing for "catching up" } \\ \cdashline{1-4}
{\footnotesize DMS-REQ-0166 } &
\multicolumn{3}{p{11.1cm}}{\footnotesize Incorporate Fault-Tolerance } \\ \cdashline{1-4}
{\footnotesize DMS-REQ-0167 } &
\multicolumn{3}{p{11.1cm}}{\footnotesize Incorporate Autonomics } \\ \cdashline{1-4}
{\footnotesize DMS-REQ-0188 } &
\multicolumn{3}{p{11.1cm}}{\footnotesize Archive to Data Access Center Network } \\ \cdashline{1-4}
{\footnotesize DMS-REQ-0189 } &
\multicolumn{3}{p{11.1cm}}{\footnotesize Archive to Data Access Center Network Availability } \\ \cdashline{1-4}
{\footnotesize DMS-REQ-0190 } &
\multicolumn{3}{p{11.1cm}}{\footnotesize Archive to Data Access Center Network Reliability } \\ \cdashline{1-4}
{\footnotesize DMS-REQ-0191 } &
\multicolumn{3}{p{11.1cm}}{\footnotesize Archive to Data Access Center Network Secondary Link } \\ \cdashline{1-4}
{\footnotesize DMS-REQ-0193 } &
\multicolumn{3}{p{11.1cm}}{\footnotesize Data Access Centers } \\ \cdashline{1-4}
{\footnotesize DMS-REQ-0194 } &
\multicolumn{3}{p{11.1cm}}{\footnotesize Data Access Center Simultaneous Connections } \\ \cdashline{1-4}
\end{longtable}

    
   \newpage
\subsubsection{Facilities}\label{fac}
\begin{longtable}{p{3.7cm}p{3.7cm}p{3.7cm}p{3.7cm}}\hline
\textbf{Manager} & \textbf{Owner} & \textbf{WBS} & \textbf{Team} \\ \hline
\parbox{3.5cm}{
Wil O'Mullane
\vspace{2mm}%
} &
\begin{tabular}{@{}l@{}}
\parbox{3.5cm}{

\vspace{2mm}%
} \\
\end{tabular}
 &
\begin{tabular}{@{}l@{}}
 \\
\end{tabular} &
\begin{tabular}{@{}l@{}}
 \\
\end{tabular} \\ \hline
\multicolumn{4}{c}{
{\footnotesize ( Facilities
 - FAC ) }
}\\ \hline
\end{longtable}

  Physical facilities where operational DM activities take place.


   \newpage
\begin{longtable}{p{3.7cm}p{3.7cm}p{3.7cm}p{3.7cm}}\toprule
\multicolumn{2}{l}{\large \textbf{ \hypertarget{facbase}{Base Facility} } }
& \multicolumn{2}{l}{(product in: Facilities
)}
\\ \hline
\textbf{\footnotesize Manager} & \textbf{\footnotesize Owner} &
\textbf{\footnotesize WBS} & \textbf{\footnotesize Team} \\ \hline
\parbox{3.5cm}{
Wil O'Mullane
\vspace{2mm}%
} &
\begin{tabular}{@{}l@{}}
\parbox{3.5cm}{
Jeff Kantor
\vspace{2mm}%
} \\
\end{tabular} &
\begin{tabular}{@{}l@{}}
1.02C.08.01 \\
1.02C.08.02 \\
\end{tabular} & \begin{tabular}{@{}l@{}}
Net/Base \\
\end{tabular} \\ \hline
\multicolumn{4}{c}{
{\footnotesize ( Base Facility
 - FACBASE ) }
}\\ \hline
\end{longtable}

Base facility located at La Serena, Chile.


\begin{longtable}{p{3.7cm}p{3.7cm}p{3.7cm}p{3.7cm}}\hline
\textbf{\footnotesize Uses:}  & & \textbf{\footnotesize Used in:} & \\ \hline
\multicolumn{2}{c}{
\begin{tabular}{c}
\hyperlink{encarcb}{Archive Base Enclave} \\ \hline
\hyperlink{encdacc}{DAC Chile Enclave} \\ \hline
\hyperlink{enccomm}{Commissioning Cluster Enclave} \\ \hline
\hyperlink{encprb}{Prompt Base Enclave} \\ \hline
\end{tabular}
} &
\multicolumn{2}{c}{
} \\ \bottomrule
\multicolumn{4}{c}{\textbf{Related Requirements} } \\ \hline
{\footnotesize DM-TS-CON-ICD-0003 } &
\multicolumn{3}{p{11.1cm}}{\footnotesize Wavefront image archive access } \\ \cdashline{1-4}
{\footnotesize DMS-REQ-0008 } &
\multicolumn{3}{p{11.1cm}}{\footnotesize Pipeline Availability } \\ \cdashline{1-4}
{\footnotesize DMS-REQ-0161 } &
\multicolumn{3}{p{11.1cm}}{\footnotesize Optimization of Cost, Reliability and Availability in Order } \\ \cdashline{1-4}
{\footnotesize DMS-REQ-0162 } &
\multicolumn{3}{p{11.1cm}}{\footnotesize Pipeline Throughput } \\ \cdashline{1-4}
{\footnotesize DMS-REQ-0163 } &
\multicolumn{3}{p{11.1cm}}{\footnotesize Re-processing Capacity } \\ \cdashline{1-4}
{\footnotesize DMS-REQ-0164 } &
\multicolumn{3}{p{11.1cm}}{\footnotesize Temporary Storage for Communications Links } \\ \cdashline{1-4}
{\footnotesize DMS-REQ-0165 } &
\multicolumn{3}{p{11.1cm}}{\footnotesize Infrastructure Sizing for "catching up" } \\ \cdashline{1-4}
{\footnotesize DMS-REQ-0166 } &
\multicolumn{3}{p{11.1cm}}{\footnotesize Incorporate Fault-Tolerance } \\ \cdashline{1-4}
{\footnotesize DMS-REQ-0167 } &
\multicolumn{3}{p{11.1cm}}{\footnotesize Incorporate Autonomics } \\ \cdashline{1-4}
{\footnotesize DMS-REQ-0170 } &
\multicolumn{3}{p{11.1cm}}{\footnotesize Prefer Computing and Storage Down } \\ \cdashline{1-4}
{\footnotesize DMS-REQ-0176 } &
\multicolumn{3}{p{11.1cm}}{\footnotesize Base Facility Infrastructure } \\ \cdashline{1-4}
{\footnotesize DMS-REQ-0178 } &
\multicolumn{3}{p{11.1cm}}{\footnotesize Base Facility Co-Location with Existing Facility } \\ \cdashline{1-4}
{\footnotesize DMS-REQ-0193 } &
\multicolumn{3}{p{11.1cm}}{\footnotesize Data Access Centers } \\ \cdashline{1-4}
{\footnotesize DMS-REQ-0196 } &
\multicolumn{3}{p{11.1cm}}{\footnotesize Data Access Center Geographical Distribution } \\ \cdashline{1-4}
{\footnotesize DMS-REQ-0297 } &
\multicolumn{3}{p{11.1cm}}{\footnotesize DMS Initialization Component } \\ \cdashline{1-4}
{\footnotesize DMS-REQ-0314 } &
\multicolumn{3}{p{11.1cm}}{\footnotesize Compute Platform Heterogeneity } \\ \cdashline{1-4}
{\footnotesize DMS-REQ-0315 } &
\multicolumn{3}{p{11.1cm}}{\footnotesize DMS Communication with OCS } \\ \cdashline{1-4}
{\footnotesize DMS-REQ-0316 } &
\multicolumn{3}{p{11.1cm}}{\footnotesize Commissioning Cluster } \\ \cdashline{1-4}
{\footnotesize DMS-REQ-0318 } &
\multicolumn{3}{p{11.1cm}}{\footnotesize Data Management Unscheduled Downtime } \\ \cdashline{1-4}
{\footnotesize DMS-REQ-0352 } &
\multicolumn{3}{p{11.1cm}}{\footnotesize Base Wireless LAN (WiFi) } \\ \cdashline{1-4}
{\footnotesize OCS-DM-COM-ICD-0027 } &
\multicolumn{3}{p{11.1cm}}{\footnotesize Multiple Physically Separated Copies } \\ \cdashline{1-4}
\end{longtable}

     \newpage
\begin{longtable}{p{3.7cm}p{3.7cm}p{3.7cm}p{3.7cm}}\toprule
\multicolumn{2}{l}{\large \textbf{ \hypertarget{facncsa}{NCSA Facility} } }
& \multicolumn{2}{l}{(product in: Facilities
)}
\\ \hline
\textbf{\footnotesize Manager} & \textbf{\footnotesize Owner} &
\textbf{\footnotesize WBS} & \textbf{\footnotesize Team} \\ \hline
\parbox{3.5cm}{
Wil O'Mullane
\vspace{2mm}%
} &
\begin{tabular}{@{}l@{}}
\parbox{3.5cm}{
Michelle Butler
\vspace{2mm}%
} \\
\end{tabular} &
\begin{tabular}{@{}l@{}}
1.02C.07.09 \\
\end{tabular} & \begin{tabular}{@{}l@{}}
LDF \\
\end{tabular} \\ \hline
\multicolumn{4}{c}{
{\footnotesize ( NCSA Facility
 - FACNCSA ) }
}\\ \hline
\end{longtable}

NCSA processing facility located at Urbana, Illinois(USA).


\begin{longtable}{p{3.7cm}p{3.7cm}p{3.7cm}p{3.7cm}}\hline
\textbf{\footnotesize Uses:}  & & \textbf{\footnotesize Used in:} & \\ \hline
\multicolumn{2}{c}{
\begin{tabular}{c}
\hyperlink{encdacu}{DAC US Enclave} \\ \hline
\hyperlink{encoffl}{Offline Production Enclave} \\ \hline
\hyperlink{}{Development and Integration E.} \\ \hline
\hyperlink{encprn}{Prompt NCSA Enclave} \\ \hline
\hyperlink{encarcn}{Archive NCSA Enclave} \\ \hline
\end{tabular}
} &
\multicolumn{2}{c}{
} \\ \bottomrule
\multicolumn{4}{c}{\textbf{Related Requirements} } \\ \hline
{\footnotesize DM-TS-CON-ICD-0003 } &
\multicolumn{3}{p{11.1cm}}{\footnotesize Wavefront image archive access } \\ \cdashline{1-4}
{\footnotesize DMS-REQ-0008 } &
\multicolumn{3}{p{11.1cm}}{\footnotesize Pipeline Availability } \\ \cdashline{1-4}
{\footnotesize DMS-REQ-0161 } &
\multicolumn{3}{p{11.1cm}}{\footnotesize Optimization of Cost, Reliability and Availability in Order } \\ \cdashline{1-4}
{\footnotesize DMS-REQ-0162 } &
\multicolumn{3}{p{11.1cm}}{\footnotesize Pipeline Throughput } \\ \cdashline{1-4}
{\footnotesize DMS-REQ-0163 } &
\multicolumn{3}{p{11.1cm}}{\footnotesize Re-processing Capacity } \\ \cdashline{1-4}
{\footnotesize DMS-REQ-0164 } &
\multicolumn{3}{p{11.1cm}}{\footnotesize Temporary Storage for Communications Links } \\ \cdashline{1-4}
{\footnotesize DMS-REQ-0165 } &
\multicolumn{3}{p{11.1cm}}{\footnotesize Infrastructure Sizing for "catching up" } \\ \cdashline{1-4}
{\footnotesize DMS-REQ-0166 } &
\multicolumn{3}{p{11.1cm}}{\footnotesize Incorporate Fault-Tolerance } \\ \cdashline{1-4}
{\footnotesize DMS-REQ-0167 } &
\multicolumn{3}{p{11.1cm}}{\footnotesize Incorporate Autonomics } \\ \cdashline{1-4}
{\footnotesize DMS-REQ-0170 } &
\multicolumn{3}{p{11.1cm}}{\footnotesize Prefer Computing and Storage Down } \\ \cdashline{1-4}
{\footnotesize DMS-REQ-0185 } &
\multicolumn{3}{p{11.1cm}}{\footnotesize Archive Center } \\ \cdashline{1-4}
{\footnotesize DMS-REQ-0186 } &
\multicolumn{3}{p{11.1cm}}{\footnotesize Archive Center Disaster Recovery } \\ \cdashline{1-4}
{\footnotesize DMS-REQ-0187 } &
\multicolumn{3}{p{11.1cm}}{\footnotesize Archive Center Co-Location with Existing Facility } \\ \cdashline{1-4}
{\footnotesize DMS-REQ-0193 } &
\multicolumn{3}{p{11.1cm}}{\footnotesize Data Access Centers } \\ \cdashline{1-4}
{\footnotesize DMS-REQ-0196 } &
\multicolumn{3}{p{11.1cm}}{\footnotesize Data Access Center Geographical Distribution } \\ \cdashline{1-4}
{\footnotesize DMS-REQ-0297 } &
\multicolumn{3}{p{11.1cm}}{\footnotesize DMS Initialization Component } \\ \cdashline{1-4}
{\footnotesize DMS-REQ-0314 } &
\multicolumn{3}{p{11.1cm}}{\footnotesize Compute Platform Heterogeneity } \\ \cdashline{1-4}
{\footnotesize DMS-REQ-0318 } &
\multicolumn{3}{p{11.1cm}}{\footnotesize Data Management Unscheduled Downtime } \\ \cdashline{1-4}
{\footnotesize EP-DM-CON-ICD-0001 } &
\multicolumn{3}{p{11.1cm}}{\footnotesize US DAC Provides EPO Interface } \\ \cdashline{1-4}
{\footnotesize OCS-DM-COM-ICD-0027 } &
\multicolumn{3}{p{11.1cm}}{\footnotesize Multiple Physically Separated Copies } \\ \cdashline{1-4}
\end{longtable}

    
   \newpage
\subsubsection{Enclaves}\label{enc}
\begin{longtable}{p{3.7cm}p{3.7cm}p{3.7cm}p{3.7cm}}\hline
\textbf{Manager} & \textbf{Owner} & \textbf{WBS} & \textbf{Team} \\ \hline
\parbox{3.5cm}{

\vspace{2mm}%
} &
\begin{tabular}{@{}l@{}}
\parbox{3.5cm}{
Multiple
\vspace{2mm}%
} \\
\end{tabular}
 &
\begin{tabular}{@{}l@{}}
 \\
\end{tabular} &
\begin{tabular}{@{}l@{}}
 \\
\end{tabular} \\ \hline
\multicolumn{4}{c}{
{\footnotesize ( Enclaves
 - ENC ) }
}\\ \hline
\end{longtable}

  Logical group of DM Services.


   \newpage
\begin{longtable}{p{3.7cm}p{3.7cm}p{3.7cm}p{3.7cm}}\toprule
\multicolumn{2}{l}{\large \textbf{ \hypertarget{encarcb}{Archive Base Enclave} } }
& \multicolumn{2}{l}{(product in: Enclaves
)}
\\ \hline
\textbf{\footnotesize Manager} & \textbf{\footnotesize Owner} &
\textbf{\footnotesize WBS} & \textbf{\footnotesize Team} \\ \hline
\parbox{3.5cm}{
Michelle Butler
\vspace{2mm}%
} &
\begin{tabular}{@{}l@{}}
\parbox{3.5cm}{
Michelle Butler
\vspace{2mm}%
} \\
\end{tabular} &
\begin{tabular}{@{}l@{}}
1.02C.08.01 \\
\end{tabular} & \begin{tabular}{@{}l@{}}
LDF \\
\end{tabular} \\ \hline
\multicolumn{4}{c}{
{\footnotesize ( Arch Base Encl
 - ENCARCB ) }
}\\ \hline
\end{longtable}

This product groups all archive services (Data BackBone) located at the
Base facility.


\begin{longtable}{p{3.7cm}p{3.7cm}p{3.7cm}p{3.7cm}}\hline
\textbf{\footnotesize Uses:}  & & \textbf{\footnotesize Used in:} & \\ \hline
\multicolumn{2}{c}{
\begin{tabular}{c}
\hyperlink{dbbmdsrv}{DBB Ingest/ Metadata Management} \\ \hline
\hyperlink{dbbtrsrv}{DBB Transport/ Replication/ Backup} \\ \hline
\hyperlink{dbblifesrv}{DBB Lifetime Management} \\ \hline
\hyperlink{dbbstrsrv}{DBB Storage} \\ \hline
\end{tabular}
} &
\multicolumn{2}{c}{
\begin{tabular}{c}
\hyperlink{facbase}{Base Facility} \\ \hline
\end{tabular}
} \\ \bottomrule
\multicolumn{4}{c}{\textbf{Related Requirements} } \\ \hline
{\footnotesize CA-DM-CON-ICD-0019 } &
\multicolumn{3}{p{11.1cm}}{\footnotesize Camera engineering image data archiving } \\ \cdashline{1-4}
{\footnotesize DM-TS-CON-ICD-0003 } &
\multicolumn{3}{p{11.1cm}}{\footnotesize Wavefront image archive access } \\ \cdashline{1-4}
{\footnotesize DMS-REQ-0004 } &
\multicolumn{3}{p{11.1cm}}{\footnotesize Nightly Data Accessible Within 24 hrs } \\ \cdashline{1-4}
{\footnotesize DMS-REQ-0010 } &
\multicolumn{3}{p{11.1cm}}{\footnotesize Difference Exposures } \\ \cdashline{1-4}
{\footnotesize DMS-REQ-0029 } &
\multicolumn{3}{p{11.1cm}}{\footnotesize Generate Photometric Zeropoint for Visit Image } \\ \cdashline{1-4}
{\footnotesize DMS-REQ-0030 } &
\multicolumn{3}{p{11.1cm}}{\footnotesize Generate WCS for Visit Images } \\ \cdashline{1-4}
{\footnotesize DMS-REQ-0069 } &
\multicolumn{3}{p{11.1cm}}{\footnotesize Processed Visit Images } \\ \cdashline{1-4}
{\footnotesize DMS-REQ-0077 } &
\multicolumn{3}{p{11.1cm}}{\footnotesize Maintain Archive Publicly Accessible } \\ \cdashline{1-4}
{\footnotesize DMS-REQ-0078 } &
\multicolumn{3}{p{11.1cm}}{\footnotesize Catalog Export Formats } \\ \cdashline{1-4}
{\footnotesize DMS-REQ-0089 } &
\multicolumn{3}{p{11.1cm}}{\footnotesize Solar System Objects Available Within Specified Time } \\ \cdashline{1-4}
{\footnotesize DMS-REQ-0094 } &
\multicolumn{3}{p{11.1cm}}{\footnotesize Keep Historical Alert Archive } \\ \cdashline{1-4}
{\footnotesize DMS-REQ-0102 } &
\multicolumn{3}{p{11.1cm}}{\footnotesize Provide Engineering \&  Facility Database Archive } \\ \cdashline{1-4}
{\footnotesize DMS-REQ-0106 } &
\multicolumn{3}{p{11.1cm}}{\footnotesize Coadded Image Provenance } \\ \cdashline{1-4}
{\footnotesize DMS-REQ-0131 } &
\multicolumn{3}{p{11.1cm}}{\footnotesize Calibration Images Available Within Specified Time } \\ \cdashline{1-4}
{\footnotesize DMS-REQ-0161 } &
\multicolumn{3}{p{11.1cm}}{\footnotesize Optimization of Cost, Reliability and Availability in Order } \\ \cdashline{1-4}
{\footnotesize DMS-REQ-0162 } &
\multicolumn{3}{p{11.1cm}}{\footnotesize Pipeline Throughput } \\ \cdashline{1-4}
{\footnotesize DMS-REQ-0163 } &
\multicolumn{3}{p{11.1cm}}{\footnotesize Re-processing Capacity } \\ \cdashline{1-4}
{\footnotesize DMS-REQ-0166 } &
\multicolumn{3}{p{11.1cm}}{\footnotesize Incorporate Fault-Tolerance } \\ \cdashline{1-4}
{\footnotesize DMS-REQ-0167 } &
\multicolumn{3}{p{11.1cm}}{\footnotesize Incorporate Autonomics } \\ \cdashline{1-4}
{\footnotesize DMS-REQ-0176 } &
\multicolumn{3}{p{11.1cm}}{\footnotesize Base Facility Infrastructure } \\ \cdashline{1-4}
{\footnotesize DMS-REQ-0186 } &
\multicolumn{3}{p{11.1cm}}{\footnotesize Archive Center Disaster Recovery } \\ \cdashline{1-4}
{\footnotesize DMS-REQ-0266 } &
\multicolumn{3}{p{11.1cm}}{\footnotesize Exposure Catalog } \\ \cdashline{1-4}
{\footnotesize DMS-REQ-0267 } &
\multicolumn{3}{p{11.1cm}}{\footnotesize Source Catalog } \\ \cdashline{1-4}
{\footnotesize DMS-REQ-0268 } &
\multicolumn{3}{p{11.1cm}}{\footnotesize Forced-Source Catalog } \\ \cdashline{1-4}
{\footnotesize DMS-REQ-0269 } &
\multicolumn{3}{p{11.1cm}}{\footnotesize DIASource Catalog } \\ \cdashline{1-4}
{\footnotesize DMS-REQ-0270 } &
\multicolumn{3}{p{11.1cm}}{\footnotesize Faint DIASource Measurements } \\ \cdashline{1-4}
{\footnotesize DMS-REQ-0271 } &
\multicolumn{3}{p{11.1cm}}{\footnotesize DIAObject Catalog } \\ \cdashline{1-4}
{\footnotesize DMS-REQ-0272 } &
\multicolumn{3}{p{11.1cm}}{\footnotesize DIAObject Attributes } \\ \cdashline{1-4}
{\footnotesize DMS-REQ-0273 } &
\multicolumn{3}{p{11.1cm}}{\footnotesize SSObject Catalog } \\ \cdashline{1-4}
{\footnotesize DMS-REQ-0274 } &
\multicolumn{3}{p{11.1cm}}{\footnotesize Alert Content } \\ \cdashline{1-4}
{\footnotesize DMS-REQ-0275 } &
\multicolumn{3}{p{11.1cm}}{\footnotesize Object Catalog } \\ \cdashline{1-4}
{\footnotesize DMS-REQ-0284 } &
\multicolumn{3}{p{11.1cm}}{\footnotesize Level-1 Production Completeness } \\ \cdashline{1-4}
{\footnotesize DMS-REQ-0287 } &
\multicolumn{3}{p{11.1cm}}{\footnotesize DIASource Precovery } \\ \cdashline{1-4}
{\footnotesize DMS-REQ-0291 } &
\multicolumn{3}{p{11.1cm}}{\footnotesize Query Repeatability } \\ \cdashline{1-4}
{\footnotesize DMS-REQ-0309 } &
\multicolumn{3}{p{11.1cm}}{\footnotesize Raw Data Archiving Reliability } \\ \cdashline{1-4}
{\footnotesize DMS-REQ-0310 } &
\multicolumn{3}{p{11.1cm}}{\footnotesize Un-Archived Data Product Cache } \\ \cdashline{1-4}
{\footnotesize DMS-REQ-0312 } &
\multicolumn{3}{p{11.1cm}}{\footnotesize Level 1 Data Product Access } \\ \cdashline{1-4}
{\footnotesize DMS-REQ-0314 } &
\multicolumn{3}{p{11.1cm}}{\footnotesize Compute Platform Heterogeneity } \\ \cdashline{1-4}
{\footnotesize DMS-REQ-0317 } &
\multicolumn{3}{p{11.1cm}}{\footnotesize DIAForcedSource Catalog } \\ \cdashline{1-4}
{\footnotesize DMS-REQ-0318 } &
\multicolumn{3}{p{11.1cm}}{\footnotesize Data Management Unscheduled Downtime } \\ \cdashline{1-4}
{\footnotesize DMS-REQ-0322 } &
\multicolumn{3}{p{11.1cm}}{\footnotesize Special Programs Database } \\ \cdashline{1-4}
{\footnotesize DMS-REQ-0324 } &
\multicolumn{3}{p{11.1cm}}{\footnotesize Matching DIASources to Objects } \\ \cdashline{1-4}
{\footnotesize DMS-REQ-0327 } &
\multicolumn{3}{p{11.1cm}}{\footnotesize Background Model Calculation } \\ \cdashline{1-4}
{\footnotesize DMS-REQ-0334 } &
\multicolumn{3}{p{11.1cm}}{\footnotesize Persisting Data Products } \\ \cdashline{1-4}
{\footnotesize DMS-REQ-0338 } &
\multicolumn{3}{p{11.1cm}}{\footnotesize Targeted Coadds } \\ \cdashline{1-4}
{\footnotesize DMS-REQ-0344 } &
\multicolumn{3}{p{11.1cm}}{\footnotesize Constraints on Level 1 Special Program Products Generation } \\ \cdashline{1-4}
{\footnotesize DMS-REQ-0363 } &
\multicolumn{3}{p{11.1cm}}{\footnotesize Access to Previous Data Releases } \\ \cdashline{1-4}
{\footnotesize DMS-REQ-0364 } &
\multicolumn{3}{p{11.1cm}}{\footnotesize Data Access Services } \\ \cdashline{1-4}
{\footnotesize DMS-REQ-0365 } &
\multicolumn{3}{p{11.1cm}}{\footnotesize Operations Subsets } \\ \cdashline{1-4}
{\footnotesize DMS-REQ-0366 } &
\multicolumn{3}{p{11.1cm}}{\footnotesize Subsets Support } \\ \cdashline{1-4}
{\footnotesize DMS-REQ-0367 } &
\multicolumn{3}{p{11.1cm}}{\footnotesize Access Services Performance } \\ \cdashline{1-4}
{\footnotesize DMS-REQ-0368 } &
\multicolumn{3}{p{11.1cm}}{\footnotesize Implementation Provisions } \\ \cdashline{1-4}
{\footnotesize DMS-REQ-0369 } &
\multicolumn{3}{p{11.1cm}}{\footnotesize Evolution } \\ \cdashline{1-4}
{\footnotesize DMS-REQ-0370 } &
\multicolumn{3}{p{11.1cm}}{\footnotesize Older Release Behavior } \\ \cdashline{1-4}
{\footnotesize OCS-DM-COM-ICD-0027 } &
\multicolumn{3}{p{11.1cm}}{\footnotesize Multiple Physically Separated Copies } \\ \cdashline{1-4}
{\footnotesize OCS-DM-COM-ICD-0028 } &
\multicolumn{3}{p{11.1cm}}{\footnotesize Expected Data Volume } \\ \cdashline{1-4}
{\footnotesize OCS-DM-COM-ICD-0029 } &
\multicolumn{3}{p{11.1cm}}{\footnotesize Archive Latency } \\ \cdashline{1-4}
{\footnotesize OCS-DM-COM-ICD-0030 } &
\multicolumn{3}{p{11.1cm}}{\footnotesize EFD Transformation Service Interface } \\ \cdashline{1-4}
\end{longtable}

     \newpage
\begin{longtable}{p{3.7cm}p{3.7cm}p{3.7cm}p{3.7cm}}\toprule
\multicolumn{2}{l}{\large \textbf{ \hypertarget{encarcn}{Archive NCSA Enclave} } }
& \multicolumn{2}{l}{(product in: Enclaves
)}
\\ \hline
\textbf{\footnotesize Manager} & \textbf{\footnotesize Owner} &
\textbf{\footnotesize WBS} & \textbf{\footnotesize Team} \\ \hline
\parbox{3.5cm}{
Michelle Butler
\vspace{2mm}%
} &
\begin{tabular}{@{}l@{}}
\parbox{3.5cm}{
Michelle Butler
\vspace{2mm}%
} \\
\end{tabular} &
\begin{tabular}{@{}l@{}}
1.02C.07.09 \\
\end{tabular} & \begin{tabular}{@{}l@{}}
LDF \\
\end{tabular} \\ \hline
\multicolumn{4}{c}{
{\footnotesize ( Arch NCSA Encl
 - ENCARCN ) }
}\\ \hline
\end{longtable}

This product groups all archive services (Data BackBone) located at the
NCSA facility.


\begin{longtable}{p{3.7cm}p{3.7cm}p{3.7cm}p{3.7cm}}\hline
\textbf{\footnotesize Uses:}  & & \textbf{\footnotesize Used in:} & \\ \hline
\multicolumn{2}{c}{
\begin{tabular}{c}
\hyperlink{dbbmdsrv}{DBB Ingest/ Metadata Management} \\ \hline
\hyperlink{dbblifesrv}{DBB Lifetime Management} \\ \hline
\hyperlink{dbbstrsrv}{DBB Storage} \\ \hline
\hyperlink{dbbtrsrv}{DBB Transport/ Replication/ Backup} \\ \hline
\end{tabular}
} &
\multicolumn{2}{c}{
\begin{tabular}{c}
\hyperlink{facncsa}{NCSA Facility} \\ \hline
\end{tabular}
} \\ \bottomrule
\multicolumn{4}{c}{\textbf{Related Requirements} } \\ \hline
{\footnotesize CA-DM-CON-ICD-0019 } &
\multicolumn{3}{p{11.1cm}}{\footnotesize Camera engineering image data archiving } \\ \cdashline{1-4}
{\footnotesize DM-TS-CON-ICD-0003 } &
\multicolumn{3}{p{11.1cm}}{\footnotesize Wavefront image archive access } \\ \cdashline{1-4}
{\footnotesize DMS-REQ-0004 } &
\multicolumn{3}{p{11.1cm}}{\footnotesize Nightly Data Accessible Within 24 hrs } \\ \cdashline{1-4}
{\footnotesize DMS-REQ-0010 } &
\multicolumn{3}{p{11.1cm}}{\footnotesize Difference Exposures } \\ \cdashline{1-4}
{\footnotesize DMS-REQ-0029 } &
\multicolumn{3}{p{11.1cm}}{\footnotesize Generate Photometric Zeropoint for Visit Image } \\ \cdashline{1-4}
{\footnotesize DMS-REQ-0030 } &
\multicolumn{3}{p{11.1cm}}{\footnotesize Generate WCS for Visit Images } \\ \cdashline{1-4}
{\footnotesize DMS-REQ-0069 } &
\multicolumn{3}{p{11.1cm}}{\footnotesize Processed Visit Images } \\ \cdashline{1-4}
{\footnotesize DMS-REQ-0077 } &
\multicolumn{3}{p{11.1cm}}{\footnotesize Maintain Archive Publicly Accessible } \\ \cdashline{1-4}
{\footnotesize DMS-REQ-0078 } &
\multicolumn{3}{p{11.1cm}}{\footnotesize Catalog Export Formats } \\ \cdashline{1-4}
{\footnotesize DMS-REQ-0089 } &
\multicolumn{3}{p{11.1cm}}{\footnotesize Solar System Objects Available Within Specified Time } \\ \cdashline{1-4}
{\footnotesize DMS-REQ-0094 } &
\multicolumn{3}{p{11.1cm}}{\footnotesize Keep Historical Alert Archive } \\ \cdashline{1-4}
{\footnotesize DMS-REQ-0102 } &
\multicolumn{3}{p{11.1cm}}{\footnotesize Provide Engineering \&  Facility Database Archive } \\ \cdashline{1-4}
{\footnotesize DMS-REQ-0106 } &
\multicolumn{3}{p{11.1cm}}{\footnotesize Coadded Image Provenance } \\ \cdashline{1-4}
{\footnotesize DMS-REQ-0131 } &
\multicolumn{3}{p{11.1cm}}{\footnotesize Calibration Images Available Within Specified Time } \\ \cdashline{1-4}
{\footnotesize DMS-REQ-0161 } &
\multicolumn{3}{p{11.1cm}}{\footnotesize Optimization of Cost, Reliability and Availability in Order } \\ \cdashline{1-4}
{\footnotesize DMS-REQ-0162 } &
\multicolumn{3}{p{11.1cm}}{\footnotesize Pipeline Throughput } \\ \cdashline{1-4}
{\footnotesize DMS-REQ-0163 } &
\multicolumn{3}{p{11.1cm}}{\footnotesize Re-processing Capacity } \\ \cdashline{1-4}
{\footnotesize DMS-REQ-0166 } &
\multicolumn{3}{p{11.1cm}}{\footnotesize Incorporate Fault-Tolerance } \\ \cdashline{1-4}
{\footnotesize DMS-REQ-0167 } &
\multicolumn{3}{p{11.1cm}}{\footnotesize Incorporate Autonomics } \\ \cdashline{1-4}
{\footnotesize DMS-REQ-0185 } &
\multicolumn{3}{p{11.1cm}}{\footnotesize Archive Center } \\ \cdashline{1-4}
{\footnotesize DMS-REQ-0186 } &
\multicolumn{3}{p{11.1cm}}{\footnotesize Archive Center Disaster Recovery } \\ \cdashline{1-4}
{\footnotesize DMS-REQ-0266 } &
\multicolumn{3}{p{11.1cm}}{\footnotesize Exposure Catalog } \\ \cdashline{1-4}
{\footnotesize DMS-REQ-0267 } &
\multicolumn{3}{p{11.1cm}}{\footnotesize Source Catalog } \\ \cdashline{1-4}
{\footnotesize DMS-REQ-0268 } &
\multicolumn{3}{p{11.1cm}}{\footnotesize Forced-Source Catalog } \\ \cdashline{1-4}
{\footnotesize DMS-REQ-0269 } &
\multicolumn{3}{p{11.1cm}}{\footnotesize DIASource Catalog } \\ \cdashline{1-4}
{\footnotesize DMS-REQ-0270 } &
\multicolumn{3}{p{11.1cm}}{\footnotesize Faint DIASource Measurements } \\ \cdashline{1-4}
{\footnotesize DMS-REQ-0271 } &
\multicolumn{3}{p{11.1cm}}{\footnotesize DIAObject Catalog } \\ \cdashline{1-4}
{\footnotesize DMS-REQ-0272 } &
\multicolumn{3}{p{11.1cm}}{\footnotesize DIAObject Attributes } \\ \cdashline{1-4}
{\footnotesize DMS-REQ-0273 } &
\multicolumn{3}{p{11.1cm}}{\footnotesize SSObject Catalog } \\ \cdashline{1-4}
{\footnotesize DMS-REQ-0274 } &
\multicolumn{3}{p{11.1cm}}{\footnotesize Alert Content } \\ \cdashline{1-4}
{\footnotesize DMS-REQ-0275 } &
\multicolumn{3}{p{11.1cm}}{\footnotesize Object Catalog } \\ \cdashline{1-4}
{\footnotesize DMS-REQ-0284 } &
\multicolumn{3}{p{11.1cm}}{\footnotesize Level-1 Production Completeness } \\ \cdashline{1-4}
{\footnotesize DMS-REQ-0287 } &
\multicolumn{3}{p{11.1cm}}{\footnotesize DIASource Precovery } \\ \cdashline{1-4}
{\footnotesize DMS-REQ-0291 } &
\multicolumn{3}{p{11.1cm}}{\footnotesize Query Repeatability } \\ \cdashline{1-4}
{\footnotesize DMS-REQ-0309 } &
\multicolumn{3}{p{11.1cm}}{\footnotesize Raw Data Archiving Reliability } \\ \cdashline{1-4}
{\footnotesize DMS-REQ-0310 } &
\multicolumn{3}{p{11.1cm}}{\footnotesize Un-Archived Data Product Cache } \\ \cdashline{1-4}
{\footnotesize DMS-REQ-0312 } &
\multicolumn{3}{p{11.1cm}}{\footnotesize Level 1 Data Product Access } \\ \cdashline{1-4}
{\footnotesize DMS-REQ-0314 } &
\multicolumn{3}{p{11.1cm}}{\footnotesize Compute Platform Heterogeneity } \\ \cdashline{1-4}
{\footnotesize DMS-REQ-0317 } &
\multicolumn{3}{p{11.1cm}}{\footnotesize DIAForcedSource Catalog } \\ \cdashline{1-4}
{\footnotesize DMS-REQ-0318 } &
\multicolumn{3}{p{11.1cm}}{\footnotesize Data Management Unscheduled Downtime } \\ \cdashline{1-4}
{\footnotesize DMS-REQ-0322 } &
\multicolumn{3}{p{11.1cm}}{\footnotesize Special Programs Database } \\ \cdashline{1-4}
{\footnotesize DMS-REQ-0324 } &
\multicolumn{3}{p{11.1cm}}{\footnotesize Matching DIASources to Objects } \\ \cdashline{1-4}
{\footnotesize DMS-REQ-0327 } &
\multicolumn{3}{p{11.1cm}}{\footnotesize Background Model Calculation } \\ \cdashline{1-4}
{\footnotesize DMS-REQ-0334 } &
\multicolumn{3}{p{11.1cm}}{\footnotesize Persisting Data Products } \\ \cdashline{1-4}
{\footnotesize DMS-REQ-0338 } &
\multicolumn{3}{p{11.1cm}}{\footnotesize Targeted Coadds } \\ \cdashline{1-4}
{\footnotesize DMS-REQ-0344 } &
\multicolumn{3}{p{11.1cm}}{\footnotesize Constraints on Level 1 Special Program Products Generation } \\ \cdashline{1-4}
{\footnotesize DMS-REQ-0363 } &
\multicolumn{3}{p{11.1cm}}{\footnotesize Access to Previous Data Releases } \\ \cdashline{1-4}
{\footnotesize DMS-REQ-0364 } &
\multicolumn{3}{p{11.1cm}}{\footnotesize Data Access Services } \\ \cdashline{1-4}
{\footnotesize DMS-REQ-0365 } &
\multicolumn{3}{p{11.1cm}}{\footnotesize Operations Subsets } \\ \cdashline{1-4}
{\footnotesize DMS-REQ-0366 } &
\multicolumn{3}{p{11.1cm}}{\footnotesize Subsets Support } \\ \cdashline{1-4}
{\footnotesize DMS-REQ-0367 } &
\multicolumn{3}{p{11.1cm}}{\footnotesize Access Services Performance } \\ \cdashline{1-4}
{\footnotesize DMS-REQ-0368 } &
\multicolumn{3}{p{11.1cm}}{\footnotesize Implementation Provisions } \\ \cdashline{1-4}
{\footnotesize DMS-REQ-0369 } &
\multicolumn{3}{p{11.1cm}}{\footnotesize Evolution } \\ \cdashline{1-4}
{\footnotesize DMS-REQ-0370 } &
\multicolumn{3}{p{11.1cm}}{\footnotesize Older Release Behavior } \\ \cdashline{1-4}
{\footnotesize OCS-DM-COM-ICD-0027 } &
\multicolumn{3}{p{11.1cm}}{\footnotesize Multiple Physically Separated Copies } \\ \cdashline{1-4}
{\footnotesize OCS-DM-COM-ICD-0028 } &
\multicolumn{3}{p{11.1cm}}{\footnotesize Expected Data Volume } \\ \cdashline{1-4}
{\footnotesize OCS-DM-COM-ICD-0029 } &
\multicolumn{3}{p{11.1cm}}{\footnotesize Archive Latency } \\ \cdashline{1-4}
{\footnotesize OCS-DM-COM-ICD-0030 } &
\multicolumn{3}{p{11.1cm}}{\footnotesize EFD Transformation Service Interface } \\ \cdashline{1-4}
\end{longtable}

     \newpage
\begin{longtable}{p{3.7cm}p{3.7cm}p{3.7cm}p{3.7cm}}\toprule
\multicolumn{2}{l}{\large \textbf{ \hypertarget{enccomm}{Commissioning Cluster Enclave} } }
& \multicolumn{2}{l}{(product in: Enclaves
)}
\\ \hline
\textbf{\footnotesize Manager} & \textbf{\footnotesize Owner} &
\textbf{\footnotesize WBS} & \textbf{\footnotesize Team} \\ \hline
\parbox{3.5cm}{
Michelle Butler
\vspace{2mm}%
} &
\begin{tabular}{@{}l@{}}
\parbox{3.5cm}{
Simon Krughoff
\vspace{2mm}%
} \\
\end{tabular} &
\begin{tabular}{@{}l@{}}
1.02C.08.01 \\
\end{tabular} & \begin{tabular}{@{}l@{}}
LDF \\
\end{tabular} \\ \hline
\multicolumn{4}{c}{
{\footnotesize ( Comm Clust Encl
 - ENCCOMM ) }
}\\ \hline
\end{longtable}

This product groups all DAC services required for Commissioning, at the
Base facility.


\begin{longtable}{p{3.7cm}p{3.7cm}p{3.7cm}p{3.7cm}}\hline
\textbf{\footnotesize Uses:}  & & \textbf{\footnotesize Used in:} & \\ \hline
\multicolumn{2}{c}{
\begin{tabular}{c}
\hyperlink{lspprtlsrv}{LSP Portal} \\ \hline
\hyperlink{lspnbl}{LSP Nublado} \\ \hline
\hyperlink{wdav}{WebDAV API} \\ \hline
\hyperlink{siasrv}{SIA API} \\ \hline
\hyperlink{soda}{SODA API} \\ \hline
\hyperlink{tapsev}{TAP API} \\ \hline
\hyperlink{lspdb}{LSP Database} \\ \hline
\end{tabular}
} &
\multicolumn{2}{c}{
\begin{tabular}{c}
\hyperlink{facbase}{Base Facility} \\ \hline
\end{tabular}
} \\ \bottomrule
\multicolumn{4}{c}{\textbf{Related Requirements} } \\ \hline
{\footnotesize DMS-REQ-0161 } &
\multicolumn{3}{p{11.1cm}}{\footnotesize Optimization of Cost, Reliability and Availability in Order } \\ \cdashline{1-4}
{\footnotesize DMS-REQ-0166 } &
\multicolumn{3}{p{11.1cm}}{\footnotesize Incorporate Fault-Tolerance } \\ \cdashline{1-4}
{\footnotesize DMS-REQ-0167 } &
\multicolumn{3}{p{11.1cm}}{\footnotesize Incorporate Autonomics } \\ \cdashline{1-4}
{\footnotesize DMS-REQ-0176 } &
\multicolumn{3}{p{11.1cm}}{\footnotesize Base Facility Infrastructure } \\ \cdashline{1-4}
{\footnotesize DMS-REQ-0314 } &
\multicolumn{3}{p{11.1cm}}{\footnotesize Compute Platform Heterogeneity } \\ \cdashline{1-4}
{\footnotesize DMS-REQ-0316 } &
\multicolumn{3}{p{11.1cm}}{\footnotesize Commissioning Cluster } \\ \cdashline{1-4}
{\footnotesize DMS-REQ-0318 } &
\multicolumn{3}{p{11.1cm}}{\footnotesize Data Management Unscheduled Downtime } \\ \cdashline{1-4}
\end{longtable}

     \newpage
\begin{longtable}{p{3.7cm}p{3.7cm}p{3.7cm}p{3.7cm}}\toprule
\multicolumn{2}{l}{\large \textbf{ \hypertarget{encdacc}{DAC Chile Enclave} } }
& \multicolumn{2}{l}{(product in: Enclaves
)}
\\ \hline
\textbf{\footnotesize Manager} & \textbf{\footnotesize Owner} &
\textbf{\footnotesize WBS} & \textbf{\footnotesize Team} \\ \hline
\parbox{3.5cm}{
Michelle Butler
\vspace{2mm}%
} &
\begin{tabular}{@{}l@{}}
\parbox{3.5cm}{
Michelle Butler
\vspace{2mm}%
} \\
\end{tabular} &
\begin{tabular}{@{}l@{}}
1.02C.08.02 \\
\end{tabular} & \begin{tabular}{@{}l@{}}
LDF \\
\end{tabular} \\ \hline
\multicolumn{4}{c}{
{\footnotesize ( DAC Chile Encl
 - ENCDACC ) }
}\\ \hline
\end{longtable}

This product groups all DAC services required for nominal operations,
located at the Base facility.


\begin{longtable}{p{3.7cm}p{3.7cm}p{3.7cm}p{3.7cm}}\hline
\textbf{\footnotesize Uses:}  & & \textbf{\footnotesize Used in:} & \\ \hline
\multicolumn{2}{c}{
\begin{tabular}{c}
\hyperlink{lspnbl}{LSP Nublado} \\ \hline
\hyperlink{lspprtlsrv}{LSP Portal} \\ \hline
\hyperlink{wdav}{WebDAV API} \\ \hline
\hyperlink{siasrv}{SIA API} \\ \hline
\hyperlink{soda}{SODA API} \\ \hline
\hyperlink{tapsev}{TAP API} \\ \hline
\hyperlink{lspdb}{LSP Database} \\ \hline
\end{tabular}
} &
\multicolumn{2}{c}{
\begin{tabular}{c}
\hyperlink{facbase}{Base Facility} \\ \hline
\end{tabular}
} \\ \bottomrule
\multicolumn{4}{c}{\textbf{Related Requirements} } \\ \hline
{\footnotesize DMS-REQ-0004 } &
\multicolumn{3}{p{11.1cm}}{\footnotesize Nightly Data Accessible Within 24 hrs } \\ \cdashline{1-4}
{\footnotesize DMS-REQ-0075 } &
\multicolumn{3}{p{11.1cm}}{\footnotesize Catalog Queries } \\ \cdashline{1-4}
{\footnotesize DMS-REQ-0077 } &
\multicolumn{3}{p{11.1cm}}{\footnotesize Maintain Archive Publicly Accessible } \\ \cdashline{1-4}
{\footnotesize DMS-REQ-0078 } &
\multicolumn{3}{p{11.1cm}}{\footnotesize Catalog Export Formats } \\ \cdashline{1-4}
{\footnotesize DMS-REQ-0089 } &
\multicolumn{3}{p{11.1cm}}{\footnotesize Solar System Objects Available Within Specified Time } \\ \cdashline{1-4}
{\footnotesize DMS-REQ-0094 } &
\multicolumn{3}{p{11.1cm}}{\footnotesize Keep Historical Alert Archive } \\ \cdashline{1-4}
{\footnotesize DMS-REQ-0102 } &
\multicolumn{3}{p{11.1cm}}{\footnotesize Provide Engineering \&  Facility Database Archive } \\ \cdashline{1-4}
{\footnotesize DMS-REQ-0119 } &
\multicolumn{3}{p{11.1cm}}{\footnotesize DAC resource allocation for Level 3 processing } \\ \cdashline{1-4}
{\footnotesize DMS-REQ-0120 } &
\multicolumn{3}{p{11.1cm}}{\footnotesize Level 3 Data Product Self Consistency } \\ \cdashline{1-4}
{\footnotesize DMS-REQ-0121 } &
\multicolumn{3}{p{11.1cm}}{\footnotesize Provenance for Level 3 processing at DACs } \\ \cdashline{1-4}
{\footnotesize DMS-REQ-0123 } &
\multicolumn{3}{p{11.1cm}}{\footnotesize Access to input catalogs for DAC-based Level 3 processing } \\ \cdashline{1-4}
{\footnotesize DMS-REQ-0127 } &
\multicolumn{3}{p{11.1cm}}{\footnotesize Access to input images for DAC-based Level 3 processing } \\ \cdashline{1-4}
{\footnotesize DMS-REQ-0131 } &
\multicolumn{3}{p{11.1cm}}{\footnotesize Calibration Images Available Within Specified Time } \\ \cdashline{1-4}
{\footnotesize DMS-REQ-0161 } &
\multicolumn{3}{p{11.1cm}}{\footnotesize Optimization of Cost, Reliability and Availability in Order } \\ \cdashline{1-4}
{\footnotesize DMS-REQ-0162 } &
\multicolumn{3}{p{11.1cm}}{\footnotesize Pipeline Throughput } \\ \cdashline{1-4}
{\footnotesize DMS-REQ-0166 } &
\multicolumn{3}{p{11.1cm}}{\footnotesize Incorporate Fault-Tolerance } \\ \cdashline{1-4}
{\footnotesize DMS-REQ-0167 } &
\multicolumn{3}{p{11.1cm}}{\footnotesize Incorporate Autonomics } \\ \cdashline{1-4}
{\footnotesize DMS-REQ-0176 } &
\multicolumn{3}{p{11.1cm}}{\footnotesize Base Facility Infrastructure } \\ \cdashline{1-4}
{\footnotesize DMS-REQ-0193 } &
\multicolumn{3}{p{11.1cm}}{\footnotesize Data Access Centers } \\ \cdashline{1-4}
{\footnotesize DMS-REQ-0194 } &
\multicolumn{3}{p{11.1cm}}{\footnotesize Data Access Center Simultaneous Connections } \\ \cdashline{1-4}
{\footnotesize DMS-REQ-0196 } &
\multicolumn{3}{p{11.1cm}}{\footnotesize Data Access Center Geographical Distribution } \\ \cdashline{1-4}
{\footnotesize DMS-REQ-0284 } &
\multicolumn{3}{p{11.1cm}}{\footnotesize Level-1 Production Completeness } \\ \cdashline{1-4}
{\footnotesize DMS-REQ-0287 } &
\multicolumn{3}{p{11.1cm}}{\footnotesize DIASource Precovery } \\ \cdashline{1-4}
{\footnotesize DMS-REQ-0291 } &
\multicolumn{3}{p{11.1cm}}{\footnotesize Query Repeatability } \\ \cdashline{1-4}
{\footnotesize DMS-REQ-0309 } &
\multicolumn{3}{p{11.1cm}}{\footnotesize Raw Data Archiving Reliability } \\ \cdashline{1-4}
{\footnotesize DMS-REQ-0310 } &
\multicolumn{3}{p{11.1cm}}{\footnotesize Un-Archived Data Product Cache } \\ \cdashline{1-4}
{\footnotesize DMS-REQ-0311 } &
\multicolumn{3}{p{11.1cm}}{\footnotesize Regenerate Un-archived Data Products } \\ \cdashline{1-4}
{\footnotesize DMS-REQ-0312 } &
\multicolumn{3}{p{11.1cm}}{\footnotesize Level 1 Data Product Access } \\ \cdashline{1-4}
{\footnotesize DMS-REQ-0313 } &
\multicolumn{3}{p{11.1cm}}{\footnotesize Level 1 \&  2 Catalog Access } \\ \cdashline{1-4}
{\footnotesize DMS-REQ-0314 } &
\multicolumn{3}{p{11.1cm}}{\footnotesize Compute Platform Heterogeneity } \\ \cdashline{1-4}
{\footnotesize DMS-REQ-0318 } &
\multicolumn{3}{p{11.1cm}}{\footnotesize Data Management Unscheduled Downtime } \\ \cdashline{1-4}
{\footnotesize DMS-REQ-0322 } &
\multicolumn{3}{p{11.1cm}}{\footnotesize Special Programs Database } \\ \cdashline{1-4}
{\footnotesize DMS-REQ-0323 } &
\multicolumn{3}{p{11.1cm}}{\footnotesize Calculating SSObject Parameters } \\ \cdashline{1-4}
{\footnotesize DMS-REQ-0324 } &
\multicolumn{3}{p{11.1cm}}{\footnotesize Matching DIASources to Objects } \\ \cdashline{1-4}
{\footnotesize DMS-REQ-0334 } &
\multicolumn{3}{p{11.1cm}}{\footnotesize Persisting Data Products } \\ \cdashline{1-4}
{\footnotesize DMS-REQ-0336 } &
\multicolumn{3}{p{11.1cm}}{\footnotesize b Regenerating Data Products from Previous Data Releases } \\ \cdashline{1-4}
{\footnotesize DMS-REQ-0338 } &
\multicolumn{3}{p{11.1cm}}{\footnotesize Targeted Coadds } \\ \cdashline{1-4}
{\footnotesize DMS-REQ-0339 } &
\multicolumn{3}{p{11.1cm}}{\footnotesize Tracking Characterization Changes Between Data Releases } \\ \cdashline{1-4}
{\footnotesize DMS-REQ-0340 } &
\multicolumn{3}{p{11.1cm}}{\footnotesize Access Controls of Level 3 Data Products } \\ \cdashline{1-4}
{\footnotesize DMS-REQ-0341 } &
\multicolumn{3}{p{11.1cm}}{\footnotesize Providing a Precovery Service } \\ \cdashline{1-4}
{\footnotesize DMS-REQ-0344 } &
\multicolumn{3}{p{11.1cm}}{\footnotesize Constraints on Level 1 Special Program Products Generation } \\ \cdashline{1-4}
{\footnotesize DMS-REQ-0345 } &
\multicolumn{3}{p{11.1cm}}{\footnotesize Logging of catalog queries } \\ \cdashline{1-4}
{\footnotesize DMS-REQ-0363 } &
\multicolumn{3}{p{11.1cm}}{\footnotesize Access to Previous Data Releases } \\ \cdashline{1-4}
{\footnotesize DMS-REQ-0364 } &
\multicolumn{3}{p{11.1cm}}{\footnotesize Data Access Services } \\ \cdashline{1-4}
{\footnotesize DMS-REQ-0365 } &
\multicolumn{3}{p{11.1cm}}{\footnotesize Operations Subsets } \\ \cdashline{1-4}
{\footnotesize DMS-REQ-0366 } &
\multicolumn{3}{p{11.1cm}}{\footnotesize Subsets Support } \\ \cdashline{1-4}
{\footnotesize DMS-REQ-0367 } &
\multicolumn{3}{p{11.1cm}}{\footnotesize Access Services Performance } \\ \cdashline{1-4}
{\footnotesize DMS-REQ-0368 } &
\multicolumn{3}{p{11.1cm}}{\footnotesize Implementation Provisions } \\ \cdashline{1-4}
{\footnotesize DMS-REQ-0369 } &
\multicolumn{3}{p{11.1cm}}{\footnotesize Evolution } \\ \cdashline{1-4}
{\footnotesize DMS-REQ-0370 } &
\multicolumn{3}{p{11.1cm}}{\footnotesize Older Release Behavior } \\ \cdashline{1-4}
{\footnotesize OCS-DM-COM-ICD-0029 } &
\multicolumn{3}{p{11.1cm}}{\footnotesize Archive Latency } \\ \cdashline{1-4}
\end{longtable}

     \newpage
\begin{longtable}{p{3.7cm}p{3.7cm}p{3.7cm}p{3.7cm}}\toprule
\multicolumn{2}{l}{\large \textbf{ \hypertarget{encdacu}{DAC US Enclave} } }
& \multicolumn{2}{l}{(product in: Enclaves
)}
\\ \hline
\textbf{\footnotesize Manager} & \textbf{\footnotesize Owner} &
\textbf{\footnotesize WBS} & \textbf{\footnotesize Team} \\ \hline
\parbox{3.5cm}{
Michelle Butler
\vspace{2mm}%
} &
\begin{tabular}{@{}l@{}}
\parbox{3.5cm}{
Michelle Butler
\vspace{2mm}%
} \\
\end{tabular} &
\begin{tabular}{@{}l@{}}
1.02C.07.09 \\
\end{tabular} & \begin{tabular}{@{}l@{}}
LDF \\
\end{tabular} \\ \hline
\multicolumn{4}{c}{
{\footnotesize ( DAC US Encl
 - ENCDACU ) }
}\\ \hline
\end{longtable}

This product groups all DAC services required for operations, located at
the NCSA facility.

~

~\url{https://github.com/lsst-sqre/lsp-deploy}


\begin{longtable}{p{3.7cm}p{3.7cm}p{3.7cm}p{3.7cm}}\hline
\textbf{\footnotesize Uses:}  & & \textbf{\footnotesize Used in:} & \\ \hline
\multicolumn{2}{c}{
\begin{tabular}{c}
\hyperlink{lspnbl}{LSP Nublado} \\ \hline
\hyperlink{lspprtlsrv}{LSP Portal} \\ \hline
\hyperlink{wdav}{WebDAV API} \\ \hline
\hyperlink{siasrv}{SIA API} \\ \hline
\hyperlink{soda}{SODA API} \\ \hline
\hyperlink{tapsev}{TAP API} \\ \hline
\hyperlink{lspdb}{LSP Database} \\ \hline
\end{tabular}
} &
\multicolumn{2}{c}{
\begin{tabular}{c}
\hyperlink{facncsa}{NCSA Facility} \\ \hline
\end{tabular}
} \\ \bottomrule
\multicolumn{4}{c}{\textbf{Related Requirements} } \\ \hline
{\footnotesize DMS-REQ-0004 } &
\multicolumn{3}{p{11.1cm}}{\footnotesize Nightly Data Accessible Within 24 hrs } \\ \cdashline{1-4}
{\footnotesize DMS-REQ-0075 } &
\multicolumn{3}{p{11.1cm}}{\footnotesize Catalog Queries } \\ \cdashline{1-4}
{\footnotesize DMS-REQ-0077 } &
\multicolumn{3}{p{11.1cm}}{\footnotesize Maintain Archive Publicly Accessible } \\ \cdashline{1-4}
{\footnotesize DMS-REQ-0078 } &
\multicolumn{3}{p{11.1cm}}{\footnotesize Catalog Export Formats } \\ \cdashline{1-4}
{\footnotesize DMS-REQ-0089 } &
\multicolumn{3}{p{11.1cm}}{\footnotesize Solar System Objects Available Within Specified Time } \\ \cdashline{1-4}
{\footnotesize DMS-REQ-0094 } &
\multicolumn{3}{p{11.1cm}}{\footnotesize Keep Historical Alert Archive } \\ \cdashline{1-4}
{\footnotesize DMS-REQ-0102 } &
\multicolumn{3}{p{11.1cm}}{\footnotesize Provide Engineering \&  Facility Database Archive } \\ \cdashline{1-4}
{\footnotesize DMS-REQ-0119 } &
\multicolumn{3}{p{11.1cm}}{\footnotesize DAC resource allocation for Level 3 processing } \\ \cdashline{1-4}
{\footnotesize DMS-REQ-0120 } &
\multicolumn{3}{p{11.1cm}}{\footnotesize Level 3 Data Product Self Consistency } \\ \cdashline{1-4}
{\footnotesize DMS-REQ-0121 } &
\multicolumn{3}{p{11.1cm}}{\footnotesize Provenance for Level 3 processing at DACs } \\ \cdashline{1-4}
{\footnotesize DMS-REQ-0123 } &
\multicolumn{3}{p{11.1cm}}{\footnotesize Access to input catalogs for DAC-based Level 3 processing } \\ \cdashline{1-4}
{\footnotesize DMS-REQ-0127 } &
\multicolumn{3}{p{11.1cm}}{\footnotesize Access to input images for DAC-based Level 3 processing } \\ \cdashline{1-4}
{\footnotesize DMS-REQ-0131 } &
\multicolumn{3}{p{11.1cm}}{\footnotesize Calibration Images Available Within Specified Time } \\ \cdashline{1-4}
{\footnotesize DMS-REQ-0161 } &
\multicolumn{3}{p{11.1cm}}{\footnotesize Optimization of Cost, Reliability and Availability in Order } \\ \cdashline{1-4}
{\footnotesize DMS-REQ-0162 } &
\multicolumn{3}{p{11.1cm}}{\footnotesize Pipeline Throughput } \\ \cdashline{1-4}
{\footnotesize DMS-REQ-0166 } &
\multicolumn{3}{p{11.1cm}}{\footnotesize Incorporate Fault-Tolerance } \\ \cdashline{1-4}
{\footnotesize DMS-REQ-0167 } &
\multicolumn{3}{p{11.1cm}}{\footnotesize Incorporate Autonomics } \\ \cdashline{1-4}
{\footnotesize DMS-REQ-0185 } &
\multicolumn{3}{p{11.1cm}}{\footnotesize Archive Center } \\ \cdashline{1-4}
{\footnotesize DMS-REQ-0186 } &
\multicolumn{3}{p{11.1cm}}{\footnotesize Archive Center Disaster Recovery } \\ \cdashline{1-4}
{\footnotesize DMS-REQ-0193 } &
\multicolumn{3}{p{11.1cm}}{\footnotesize Data Access Centers } \\ \cdashline{1-4}
{\footnotesize DMS-REQ-0194 } &
\multicolumn{3}{p{11.1cm}}{\footnotesize Data Access Center Simultaneous Connections } \\ \cdashline{1-4}
{\footnotesize DMS-REQ-0196 } &
\multicolumn{3}{p{11.1cm}}{\footnotesize Data Access Center Geographical Distribution } \\ \cdashline{1-4}
{\footnotesize DMS-REQ-0284 } &
\multicolumn{3}{p{11.1cm}}{\footnotesize Level-1 Production Completeness } \\ \cdashline{1-4}
{\footnotesize DMS-REQ-0287 } &
\multicolumn{3}{p{11.1cm}}{\footnotesize DIASource Precovery } \\ \cdashline{1-4}
{\footnotesize DMS-REQ-0291 } &
\multicolumn{3}{p{11.1cm}}{\footnotesize Query Repeatability } \\ \cdashline{1-4}
{\footnotesize DMS-REQ-0309 } &
\multicolumn{3}{p{11.1cm}}{\footnotesize Raw Data Archiving Reliability } \\ \cdashline{1-4}
{\footnotesize DMS-REQ-0310 } &
\multicolumn{3}{p{11.1cm}}{\footnotesize Un-Archived Data Product Cache } \\ \cdashline{1-4}
{\footnotesize DMS-REQ-0311 } &
\multicolumn{3}{p{11.1cm}}{\footnotesize Regenerate Un-archived Data Products } \\ \cdashline{1-4}
{\footnotesize DMS-REQ-0312 } &
\multicolumn{3}{p{11.1cm}}{\footnotesize Level 1 Data Product Access } \\ \cdashline{1-4}
{\footnotesize DMS-REQ-0313 } &
\multicolumn{3}{p{11.1cm}}{\footnotesize Level 1 \&  2 Catalog Access } \\ \cdashline{1-4}
{\footnotesize DMS-REQ-0314 } &
\multicolumn{3}{p{11.1cm}}{\footnotesize Compute Platform Heterogeneity } \\ \cdashline{1-4}
{\footnotesize DMS-REQ-0318 } &
\multicolumn{3}{p{11.1cm}}{\footnotesize Data Management Unscheduled Downtime } \\ \cdashline{1-4}
{\footnotesize DMS-REQ-0322 } &
\multicolumn{3}{p{11.1cm}}{\footnotesize Special Programs Database } \\ \cdashline{1-4}
{\footnotesize DMS-REQ-0323 } &
\multicolumn{3}{p{11.1cm}}{\footnotesize Calculating SSObject Parameters } \\ \cdashline{1-4}
{\footnotesize DMS-REQ-0324 } &
\multicolumn{3}{p{11.1cm}}{\footnotesize Matching DIASources to Objects } \\ \cdashline{1-4}
{\footnotesize DMS-REQ-0334 } &
\multicolumn{3}{p{11.1cm}}{\footnotesize Persisting Data Products } \\ \cdashline{1-4}
{\footnotesize DMS-REQ-0336 } &
\multicolumn{3}{p{11.1cm}}{\footnotesize b Regenerating Data Products from Previous Data Releases } \\ \cdashline{1-4}
{\footnotesize DMS-REQ-0338 } &
\multicolumn{3}{p{11.1cm}}{\footnotesize Targeted Coadds } \\ \cdashline{1-4}
{\footnotesize DMS-REQ-0339 } &
\multicolumn{3}{p{11.1cm}}{\footnotesize Tracking Characterization Changes Between Data Releases } \\ \cdashline{1-4}
{\footnotesize DMS-REQ-0340 } &
\multicolumn{3}{p{11.1cm}}{\footnotesize Access Controls of Level 3 Data Products } \\ \cdashline{1-4}
{\footnotesize DMS-REQ-0341 } &
\multicolumn{3}{p{11.1cm}}{\footnotesize Providing a Precovery Service } \\ \cdashline{1-4}
{\footnotesize DMS-REQ-0344 } &
\multicolumn{3}{p{11.1cm}}{\footnotesize Constraints on Level 1 Special Program Products Generation } \\ \cdashline{1-4}
{\footnotesize DMS-REQ-0345 } &
\multicolumn{3}{p{11.1cm}}{\footnotesize Logging of catalog queries } \\ \cdashline{1-4}
{\footnotesize DMS-REQ-0363 } &
\multicolumn{3}{p{11.1cm}}{\footnotesize Access to Previous Data Releases } \\ \cdashline{1-4}
{\footnotesize DMS-REQ-0364 } &
\multicolumn{3}{p{11.1cm}}{\footnotesize Data Access Services } \\ \cdashline{1-4}
{\footnotesize DMS-REQ-0365 } &
\multicolumn{3}{p{11.1cm}}{\footnotesize Operations Subsets } \\ \cdashline{1-4}
{\footnotesize DMS-REQ-0366 } &
\multicolumn{3}{p{11.1cm}}{\footnotesize Subsets Support } \\ \cdashline{1-4}
{\footnotesize DMS-REQ-0367 } &
\multicolumn{3}{p{11.1cm}}{\footnotesize Access Services Performance } \\ \cdashline{1-4}
{\footnotesize DMS-REQ-0368 } &
\multicolumn{3}{p{11.1cm}}{\footnotesize Implementation Provisions } \\ \cdashline{1-4}
{\footnotesize DMS-REQ-0369 } &
\multicolumn{3}{p{11.1cm}}{\footnotesize Evolution } \\ \cdashline{1-4}
{\footnotesize DMS-REQ-0370 } &
\multicolumn{3}{p{11.1cm}}{\footnotesize Older Release Behavior } \\ \cdashline{1-4}
{\footnotesize EP-DM-CON-ICD-0001 } &
\multicolumn{3}{p{11.1cm}}{\footnotesize US DAC Provides EPO Interface } \\ \cdashline{1-4}
{\footnotesize EP-DM-CON-ICD-0002 } &
\multicolumn{3}{p{11.1cm}}{\footnotesize EPO is an Authorized Science User } \\ \cdashline{1-4}
{\footnotesize EP-DM-CON-ICD-0034 } &
\multicolumn{3}{p{11.1cm}}{\footnotesize Citizen Science Data } \\ \cdashline{1-4}
{\footnotesize OCS-DM-COM-ICD-0029 } &
\multicolumn{3}{p{11.1cm}}{\footnotesize Archive Latency } \\ \cdashline{1-4}
\end{longtable}

     \newpage
\begin{longtable}{p{3.7cm}p{3.7cm}p{3.7cm}p{3.7cm}}\toprule
\multicolumn{2}{l}{\large \textbf{ \hypertarget{encoffl}{Offline Production Enclave} } }
& \multicolumn{2}{l}{(product in: Enclaves
)}
\\ \hline
\textbf{\footnotesize Manager} & \textbf{\footnotesize Owner} &
\textbf{\footnotesize WBS} & \textbf{\footnotesize Team} \\ \hline
\parbox{3.5cm}{
Michelle Butler
\vspace{2mm}%
} &
\begin{tabular}{@{}l@{}}
\parbox{3.5cm}{
Michelle Butler
\vspace{2mm}%
} \\
\end{tabular} &
\begin{tabular}{@{}l@{}}
1.02C.07.09 \\
\end{tabular} & \begin{tabular}{@{}l@{}}
LDF \\
\end{tabular} \\ \hline
\multicolumn{4}{c}{
{\footnotesize ( Offline Prod Encl
 - ENCOFFL ) }
}\\ \hline
\end{longtable}

This product groups all processing services to be executed offline, at
the NCSA facility.


\begin{longtable}{p{3.7cm}p{3.7cm}p{3.7cm}p{3.7cm}}\hline
\textbf{\footnotesize Uses:}  & & \textbf{\footnotesize Used in:} & \\ \hline
\multicolumn{2}{c}{
\begin{tabular}{c}
\hyperlink{prodsrv}{Batch Production} \\ \hline
\hyperlink{offlqcsrv}{Offline Quality Control} \\ \hline
\hyperlink{bulkdsrv}{Bulk Distribution} \\ \hline
\end{tabular}
} &
\multicolumn{2}{c}{
\begin{tabular}{c}
\hyperlink{facncsa}{NCSA Facility} \\ \hline
\end{tabular}
} \\ \bottomrule
\multicolumn{4}{c}{\textbf{Related Requirements} } \\ \hline
{\footnotesize DM-TS-CON-ICD-0003 } &
\multicolumn{3}{p{11.1cm}}{\footnotesize Wavefront image archive access } \\ \cdashline{1-4}
{\footnotesize DMS-REQ-0004 } &
\multicolumn{3}{p{11.1cm}}{\footnotesize Nightly Data Accessible Within 24 hrs } \\ \cdashline{1-4}
{\footnotesize DMS-REQ-0008 } &
\multicolumn{3}{p{11.1cm}}{\footnotesize Pipeline Availability } \\ \cdashline{1-4}
{\footnotesize DMS-REQ-0034 } &
\multicolumn{3}{p{11.1cm}}{\footnotesize Associate Sources to Objects } \\ \cdashline{1-4}
{\footnotesize DMS-REQ-0046 } &
\multicolumn{3}{p{11.1cm}}{\footnotesize Provide Photometric Redshifts of Galaxies } \\ \cdashline{1-4}
{\footnotesize DMS-REQ-0047 } &
\multicolumn{3}{p{11.1cm}}{\footnotesize Provide PSF for Coadded Images } \\ \cdashline{1-4}
{\footnotesize DMS-REQ-0059 } &
\multicolumn{3}{p{11.1cm}}{\footnotesize Bad Pixel Map } \\ \cdashline{1-4}
{\footnotesize DMS-REQ-0060 } &
\multicolumn{3}{p{11.1cm}}{\footnotesize Bias Residual Image } \\ \cdashline{1-4}
{\footnotesize DMS-REQ-0061 } &
\multicolumn{3}{p{11.1cm}}{\footnotesize Crosstalk Correction Matrix } \\ \cdashline{1-4}
{\footnotesize DMS-REQ-0062 } &
\multicolumn{3}{p{11.1cm}}{\footnotesize Illumination Correction Frame } \\ \cdashline{1-4}
{\footnotesize DMS-REQ-0063 } &
\multicolumn{3}{p{11.1cm}}{\footnotesize Monochromatic Flatfield Data Cube } \\ \cdashline{1-4}
{\footnotesize DMS-REQ-0103 } &
\multicolumn{3}{p{11.1cm}}{\footnotesize Produce Images for EPO } \\ \cdashline{1-4}
{\footnotesize DMS-REQ-0106 } &
\multicolumn{3}{p{11.1cm}}{\footnotesize Coadded Image Provenance } \\ \cdashline{1-4}
{\footnotesize DMS-REQ-0130 } &
\multicolumn{3}{p{11.1cm}}{\footnotesize Calibration Data Products } \\ \cdashline{1-4}
{\footnotesize DMS-REQ-0131 } &
\multicolumn{3}{p{11.1cm}}{\footnotesize Calibration Images Available Within Specified Time } \\ \cdashline{1-4}
{\footnotesize DMS-REQ-0132 } &
\multicolumn{3}{p{11.1cm}}{\footnotesize Calibration Image Provenance } \\ \cdashline{1-4}
{\footnotesize DMS-REQ-0161 } &
\multicolumn{3}{p{11.1cm}}{\footnotesize Optimization of Cost, Reliability and Availability in Order } \\ \cdashline{1-4}
{\footnotesize DMS-REQ-0162 } &
\multicolumn{3}{p{11.1cm}}{\footnotesize Pipeline Throughput } \\ \cdashline{1-4}
{\footnotesize DMS-REQ-0163 } &
\multicolumn{3}{p{11.1cm}}{\footnotesize Re-processing Capacity } \\ \cdashline{1-4}
{\footnotesize DMS-REQ-0166 } &
\multicolumn{3}{p{11.1cm}}{\footnotesize Incorporate Fault-Tolerance } \\ \cdashline{1-4}
{\footnotesize DMS-REQ-0167 } &
\multicolumn{3}{p{11.1cm}}{\footnotesize Incorporate Autonomics } \\ \cdashline{1-4}
{\footnotesize DMS-REQ-0185 } &
\multicolumn{3}{p{11.1cm}}{\footnotesize Archive Center } \\ \cdashline{1-4}
{\footnotesize DMS-REQ-0186 } &
\multicolumn{3}{p{11.1cm}}{\footnotesize Archive Center Disaster Recovery } \\ \cdashline{1-4}
{\footnotesize DMS-REQ-0267 } &
\multicolumn{3}{p{11.1cm}}{\footnotesize Source Catalog } \\ \cdashline{1-4}
{\footnotesize DMS-REQ-0268 } &
\multicolumn{3}{p{11.1cm}}{\footnotesize Forced-Source Catalog } \\ \cdashline{1-4}
{\footnotesize DMS-REQ-0275 } &
\multicolumn{3}{p{11.1cm}}{\footnotesize Object Catalog } \\ \cdashline{1-4}
{\footnotesize DMS-REQ-0277 } &
\multicolumn{3}{p{11.1cm}}{\footnotesize Coadd Source Catalog } \\ \cdashline{1-4}
{\footnotesize DMS-REQ-0278 } &
\multicolumn{3}{p{11.1cm}}{\footnotesize Coadd Image Method Constraints } \\ \cdashline{1-4}
{\footnotesize DMS-REQ-0279 } &
\multicolumn{3}{p{11.1cm}}{\footnotesize Deep Detection Coadds } \\ \cdashline{1-4}
{\footnotesize DMS-REQ-0280 } &
\multicolumn{3}{p{11.1cm}}{\footnotesize Template Coadds } \\ \cdashline{1-4}
{\footnotesize DMS-REQ-0281 } &
\multicolumn{3}{p{11.1cm}}{\footnotesize Multi-band Coadds } \\ \cdashline{1-4}
{\footnotesize DMS-REQ-0282 } &
\multicolumn{3}{p{11.1cm}}{\footnotesize Dark Current Correction Frame } \\ \cdashline{1-4}
{\footnotesize DMS-REQ-0283 } &
\multicolumn{3}{p{11.1cm}}{\footnotesize Fringe Correction Frame } \\ \cdashline{1-4}
{\footnotesize DMS-REQ-0284 } &
\multicolumn{3}{p{11.1cm}}{\footnotesize Level-1 Production Completeness } \\ \cdashline{1-4}
{\footnotesize DMS-REQ-0286 } &
\multicolumn{3}{p{11.1cm}}{\footnotesize SSObject Precovery } \\ \cdashline{1-4}
{\footnotesize DMS-REQ-0287 } &
\multicolumn{3}{p{11.1cm}}{\footnotesize DIASource Precovery } \\ \cdashline{1-4}
{\footnotesize DMS-REQ-0289 } &
\multicolumn{3}{p{11.1cm}}{\footnotesize Calibration Production Processing } \\ \cdashline{1-4}
{\footnotesize DMS-REQ-0314 } &
\multicolumn{3}{p{11.1cm}}{\footnotesize Compute Platform Heterogeneity } \\ \cdashline{1-4}
{\footnotesize DMS-REQ-0318 } &
\multicolumn{3}{p{11.1cm}}{\footnotesize Data Management Unscheduled Downtime } \\ \cdashline{1-4}
{\footnotesize DMS-REQ-0320 } &
\multicolumn{3}{p{11.1cm}}{\footnotesize Processing of Data From Special Programs } \\ \cdashline{1-4}
{\footnotesize DMS-REQ-0325 } &
\multicolumn{3}{p{11.1cm}}{\footnotesize Regenerating L1 Data Products During Data Release Processing } \\ \cdashline{1-4}
{\footnotesize DMS-REQ-0329 } &
\multicolumn{3}{p{11.1cm}}{\footnotesize All-Sky Visualization of Data Releases } \\ \cdashline{1-4}
{\footnotesize DMS-REQ-0330 } &
\multicolumn{3}{p{11.1cm}}{\footnotesize Best Seeing Coadds } \\ \cdashline{1-4}
{\footnotesize DMS-REQ-0334 } &
\multicolumn{3}{p{11.1cm}}{\footnotesize Persisting Data Products } \\ \cdashline{1-4}
{\footnotesize DMS-REQ-0335 } &
\multicolumn{3}{p{11.1cm}}{\footnotesize PSF-Matched Coadds } \\ \cdashline{1-4}
{\footnotesize DMS-REQ-0341 } &
\multicolumn{3}{p{11.1cm}}{\footnotesize Providing a Precovery Service } \\ \cdashline{1-4}
{\footnotesize EP-DM-CON-ICD-0037 } &
\multicolumn{3}{p{11.1cm}}{\footnotesize EPO Compute Cluster } \\ \cdashline{1-4}
\end{longtable}

     \newpage
\begin{longtable}{p{3.7cm}p{3.7cm}p{3.7cm}p{3.7cm}}\toprule
\multicolumn{2}{l}{\large \textbf{ \hypertarget{encprb}{Prompt Base Enclave} } }
& \multicolumn{2}{l}{(product in: Enclaves
)}
\\ \hline
\textbf{\footnotesize Manager} & \textbf{\footnotesize Owner} &
\textbf{\footnotesize WBS} & \textbf{\footnotesize Team} \\ \hline
\parbox{3.5cm}{
Michelle Butler
\vspace{2mm}%
} &
\begin{tabular}{@{}l@{}}
\parbox{3.5cm}{
Michelle Butler
\vspace{2mm}%
} \\
\end{tabular} &
\begin{tabular}{@{}l@{}}
1.02C.08.01 \\
\end{tabular} & \begin{tabular}{@{}l@{}}
LDF \\
\end{tabular} \\ \hline
\multicolumn{4}{c}{
{\footnotesize ( Prmpt Base Encl
 - ENCPRB ) }
}\\ \hline
\end{longtable}

This product groups all services required to be executed in a nearly
real time manner, at Base facility.


\begin{longtable}{p{3.7cm}p{3.7cm}p{3.7cm}p{3.7cm}}\hline
\textbf{\footnotesize Uses:}  & & \textbf{\footnotesize Used in:} & \\ \hline
\multicolumn{2}{c}{
\begin{tabular}{c}
\hyperlink{prpingsrv}{Prompt Processing Ingest} \\ \hline
\hyperlink{tmgsrv}{Telemetry Gateway} \\ \hline
\hyperlink{popsrv}{Planned Observation Publication} \\ \hline
\hyperlink{ocsbatsrv}{OCS-Driven Batch} \\ \hline
\hyperlink{oodssrv}{Observatory Operations Data} \\ \hline
\hyperlink{arcsrvo}{Archiving [Obsolete]} \\ \hline
\hyperlink{efdb}{EFD Cache} \\ \hline
\hyperlink{imas}{Image Archiver} \\ \hline
\hyperlink{heads}{Header Generator} \\ \hline
\hyperlink{efdts}{EFD Transformation} \\ \hline
\end{tabular}
} &
\multicolumn{2}{c}{
\begin{tabular}{c}
\hyperlink{facbase}{Base Facility} \\ \hline
\end{tabular}
} \\ \bottomrule
\multicolumn{4}{c}{\textbf{Related Requirements} } \\ \hline
{\footnotesize CA-DM-CON-ICD-0007 } &
\multicolumn{3}{p{11.1cm}}{\footnotesize Provide Data Management Conditions data } \\ \cdashline{1-4}
{\footnotesize CA-DM-CON-ICD-0008 } &
\multicolumn{3}{p{11.1cm}}{\footnotesize Data Management Conditions data latency } \\ \cdashline{1-4}
{\footnotesize CA-DM-CON-ICD-0014 } &
\multicolumn{3}{p{11.1cm}}{\footnotesize Provide science sensor data } \\ \cdashline{1-4}
{\footnotesize CA-DM-CON-ICD-0015 } &
\multicolumn{3}{p{11.1cm}}{\footnotesize Provide wavefront sensor data } \\ \cdashline{1-4}
{\footnotesize CA-DM-CON-ICD-0016 } &
\multicolumn{3}{p{11.1cm}}{\footnotesize Provide guide sensor data } \\ \cdashline{1-4}
{\footnotesize CA-DM-CON-ICD-0017 } &
\multicolumn{3}{p{11.1cm}}{\footnotesize Data Management load on image data interfaces } \\ \cdashline{1-4}
{\footnotesize CA-DM-CON-ICD-0019 } &
\multicolumn{3}{p{11.1cm}}{\footnotesize Camera engineering image data archiving } \\ \cdashline{1-4}
{\footnotesize DM-TS-CON-ICD-0002 } &
\multicolumn{3}{p{11.1cm}}{\footnotesize Timing } \\ \cdashline{1-4}
{\footnotesize DM-TS-CON-ICD-0003 } &
\multicolumn{3}{p{11.1cm}}{\footnotesize Wavefront image archive access } \\ \cdashline{1-4}
{\footnotesize DM-TS-CON-ICD-0004 } &
\multicolumn{3}{p{11.1cm}}{\footnotesize Use OCS for data transport } \\ \cdashline{1-4}
{\footnotesize DM-TS-CON-ICD-0006 } &
\multicolumn{3}{p{11.1cm}}{\footnotesize Data } \\ \cdashline{1-4}
{\footnotesize DM-TS-CON-ICD-0007 } &
\multicolumn{3}{p{11.1cm}}{\footnotesize Timing } \\ \cdashline{1-4}
{\footnotesize DM-TS-CON-ICD-0009 } &
\multicolumn{3}{p{11.1cm}}{\footnotesize Calibration Data Products } \\ \cdashline{1-4}
{\footnotesize DM-TS-CON-ICD-0011 } &
\multicolumn{3}{p{11.1cm}}{\footnotesize Data Format } \\ \cdashline{1-4}
{\footnotesize DMS-REQ-0004 } &
\multicolumn{3}{p{11.1cm}}{\footnotesize Nightly Data Accessible Within 24 hrs } \\ \cdashline{1-4}
{\footnotesize DMS-REQ-0008 } &
\multicolumn{3}{p{11.1cm}}{\footnotesize Pipeline Availability } \\ \cdashline{1-4}
{\footnotesize DMS-REQ-0018 } &
\multicolumn{3}{p{11.1cm}}{\footnotesize Raw Science Image Data Acquisition } \\ \cdashline{1-4}
{\footnotesize DMS-REQ-0020 } &
\multicolumn{3}{p{11.1cm}}{\footnotesize Wavefront Sensor Data Acquisition } \\ \cdashline{1-4}
{\footnotesize DMS-REQ-0022 } &
\multicolumn{3}{p{11.1cm}}{\footnotesize Crosstalk Corrected Science Image Data Acquisition } \\ \cdashline{1-4}
{\footnotesize DMS-REQ-0024 } &
\multicolumn{3}{p{11.1cm}}{\footnotesize Raw Image Assembly } \\ \cdashline{1-4}
{\footnotesize DMS-REQ-0068 } &
\multicolumn{3}{p{11.1cm}}{\footnotesize Raw Science Image Metadata } \\ \cdashline{1-4}
{\footnotesize DMS-REQ-0096 } &
\multicolumn{3}{p{11.1cm}}{\footnotesize Generate Data Quality Report Within Specified Time } \\ \cdashline{1-4}
{\footnotesize DMS-REQ-0097 } &
\multicolumn{3}{p{11.1cm}}{\footnotesize Level 1 Data Quality Report Definition } \\ \cdashline{1-4}
{\footnotesize DMS-REQ-0098 } &
\multicolumn{3}{p{11.1cm}}{\footnotesize Generate DMS Performance Report Within Specified Time } \\ \cdashline{1-4}
{\footnotesize DMS-REQ-0099 } &
\multicolumn{3}{p{11.1cm}}{\footnotesize Level 1 Performance Report Definition } \\ \cdashline{1-4}
{\footnotesize DMS-REQ-0100 } &
\multicolumn{3}{p{11.1cm}}{\footnotesize Generate Calibration Report Within Specified Time } \\ \cdashline{1-4}
{\footnotesize DMS-REQ-0101 } &
\multicolumn{3}{p{11.1cm}}{\footnotesize Level 1 Calibration Report Definition } \\ \cdashline{1-4}
{\footnotesize DMS-REQ-0102 } &
\multicolumn{3}{p{11.1cm}}{\footnotesize Provide Engineering \&  Facility Database Archive } \\ \cdashline{1-4}
{\footnotesize DMS-REQ-0161 } &
\multicolumn{3}{p{11.1cm}}{\footnotesize Optimization of Cost, Reliability and Availability in Order } \\ \cdashline{1-4}
{\footnotesize DMS-REQ-0162 } &
\multicolumn{3}{p{11.1cm}}{\footnotesize Pipeline Throughput } \\ \cdashline{1-4}
{\footnotesize DMS-REQ-0164 } &
\multicolumn{3}{p{11.1cm}}{\footnotesize Temporary Storage for Communications Links } \\ \cdashline{1-4}
{\footnotesize DMS-REQ-0165 } &
\multicolumn{3}{p{11.1cm}}{\footnotesize Infrastructure Sizing for "catching up" } \\ \cdashline{1-4}
{\footnotesize DMS-REQ-0166 } &
\multicolumn{3}{p{11.1cm}}{\footnotesize Incorporate Fault-Tolerance } \\ \cdashline{1-4}
{\footnotesize DMS-REQ-0167 } &
\multicolumn{3}{p{11.1cm}}{\footnotesize Incorporate Autonomics } \\ \cdashline{1-4}
{\footnotesize DMS-REQ-0176 } &
\multicolumn{3}{p{11.1cm}}{\footnotesize Base Facility Infrastructure } \\ \cdashline{1-4}
{\footnotesize DMS-REQ-0265 } &
\multicolumn{3}{p{11.1cm}}{\footnotesize Guider Calibration Data Acquisition } \\ \cdashline{1-4}
{\footnotesize DMS-REQ-0284 } &
\multicolumn{3}{p{11.1cm}}{\footnotesize Level-1 Production Completeness } \\ \cdashline{1-4}
{\footnotesize DMS-REQ-0309 } &
\multicolumn{3}{p{11.1cm}}{\footnotesize Raw Data Archiving Reliability } \\ \cdashline{1-4}
{\footnotesize DMS-REQ-0314 } &
\multicolumn{3}{p{11.1cm}}{\footnotesize Compute Platform Heterogeneity } \\ \cdashline{1-4}
{\footnotesize DMS-REQ-0315 } &
\multicolumn{3}{p{11.1cm}}{\footnotesize DMS Communication with OCS } \\ \cdashline{1-4}
{\footnotesize DMS-REQ-0318 } &
\multicolumn{3}{p{11.1cm}}{\footnotesize Data Management Unscheduled Downtime } \\ \cdashline{1-4}
{\footnotesize DMS-REQ-0353 } &
\multicolumn{3}{p{11.1cm}}{\footnotesize Publishing predicted visit schedule } \\ \cdashline{1-4}
{\footnotesize OCS-DM-COM-ICD-0003 } &
\multicolumn{3}{p{11.1cm}}{\footnotesize Data Management CSC Command Response Model } \\ \cdashline{1-4}
{\footnotesize OCS-DM-COM-ICD-0004 } &
\multicolumn{3}{p{11.1cm}}{\footnotesize Data Management Exposed CSCs } \\ \cdashline{1-4}
{\footnotesize OCS-DM-COM-ICD-0005 } &
\multicolumn{3}{p{11.1cm}}{\footnotesize Main Camera Archiver } \\ \cdashline{1-4}
{\footnotesize OCS-DM-COM-ICD-0006 } &
\multicolumn{3}{p{11.1cm}}{\footnotesize Catch-up Archiver } \\ \cdashline{1-4}
{\footnotesize OCS-DM-COM-ICD-0007 } &
\multicolumn{3}{p{11.1cm}}{\footnotesize Prompt Processing CSC } \\ \cdashline{1-4}
{\footnotesize OCS-DM-COM-ICD-0008 } &
\multicolumn{3}{p{11.1cm}}{\footnotesize EFD Transformation Service CSC } \\ \cdashline{1-4}
{\footnotesize OCS-DM-COM-ICD-0009 } &
\multicolumn{3}{p{11.1cm}}{\footnotesize Command Set Implementation by Data Management } \\ \cdashline{1-4}
{\footnotesize OCS-DM-COM-ICD-0017 } &
\multicolumn{3}{p{11.1cm}}{\footnotesize Data Management Telemetry Interface Model } \\ \cdashline{1-4}
{\footnotesize OCS-DM-COM-ICD-0018 } &
\multicolumn{3}{p{11.1cm}}{\footnotesize Data Management Telemetry Time Stamp } \\ \cdashline{1-4}
{\footnotesize OCS-DM-COM-ICD-0019 } &
\multicolumn{3}{p{11.1cm}}{\footnotesize Data Management Events and Telemetry Required by the OCS } \\ \cdashline{1-4}
{\footnotesize OCS-DM-COM-ICD-0020 } &
\multicolumn{3}{p{11.1cm}}{\footnotesize Image and Visit Processing and Archiving Status } \\ \cdashline{1-4}
{\footnotesize OCS-DM-COM-ICD-0021 } &
\multicolumn{3}{p{11.1cm}}{\footnotesize Data Quality Metrics } \\ \cdashline{1-4}
{\footnotesize OCS-DM-COM-ICD-0022 } &
\multicolumn{3}{p{11.1cm}}{\footnotesize System Health Metrics } \\ \cdashline{1-4}
{\footnotesize OCS-DM-COM-ICD-0025 } &
\multicolumn{3}{p{11.1cm}}{\footnotesize Expected Load of Queries from DM } \\ \cdashline{1-4}
{\footnotesize OCS-DM-COM-ICD-0026 } &
\multicolumn{3}{p{11.1cm}}{\footnotesize Engineering and Facilities Database Archiving by Data Management } \\ \cdashline{1-4}
{\footnotesize OCS-DM-COM-ICD-0027 } &
\multicolumn{3}{p{11.1cm}}{\footnotesize Multiple Physically Separated Copies } \\ \cdashline{1-4}
{\footnotesize OCS-DM-COM-ICD-0028 } &
\multicolumn{3}{p{11.1cm}}{\footnotesize Expected Data Volume } \\ \cdashline{1-4}
{\footnotesize OCS-DM-COM-ICD-0030 } &
\multicolumn{3}{p{11.1cm}}{\footnotesize EFD Transformation Service Interface } \\ \cdashline{1-4}
\end{longtable}

     \newpage
\begin{longtable}{p{3.7cm}p{3.7cm}p{3.7cm}p{3.7cm}}\toprule
\multicolumn{2}{l}{\large \textbf{ \hypertarget{encprn}{Prompt NCSA Enclave} } }
& \multicolumn{2}{l}{(product in: Enclaves
)}
\\ \hline
\textbf{\footnotesize Manager} & \textbf{\footnotesize Owner} &
\textbf{\footnotesize WBS} & \textbf{\footnotesize Team} \\ \hline
\parbox{3.5cm}{
Michelle Butler
\vspace{2mm}%
} &
\begin{tabular}{@{}l@{}}
\parbox{3.5cm}{
Michelle Butler
\vspace{2mm}%
} \\
\end{tabular} &
\begin{tabular}{@{}l@{}}
1.02C.07.09 \\
\end{tabular} & \begin{tabular}{@{}l@{}}
LDF \\
\end{tabular} \\ \hline
\multicolumn{4}{c}{
{\footnotesize ( Prmpt NCSA Encl
 - ENCPRN ) }
}\\ \hline
\end{longtable}

This product groups all services required to be executed in a nearly
real time manner, at NCSA facility.


\begin{longtable}{p{3.7cm}p{3.7cm}p{3.7cm}p{3.7cm}}\hline
\textbf{\footnotesize Uses:}  & & \textbf{\footnotesize Used in:} & \\ \hline
\multicolumn{2}{c}{
\begin{tabular}{c}
\hyperlink{alrtdstsrv}{Alert Distribution} \\ \hline
\hyperlink{prpingsrv}{Prompt Processing Ingest} \\ \hline
\hyperlink{offlqcsrv}{Offline Quality Control} \\ \hline
\hyperlink{prqcsrv}{Prompt Quality Control} \\ \hline
\hyperlink{prprsrv}{Prompt Processing} \\ \hline
\hyperlink{apdb}{APDB} \\ \hline
\end{tabular}
} &
\multicolumn{2}{c}{
\begin{tabular}{c}
\hyperlink{facncsa}{NCSA Facility} \\ \hline
\end{tabular}
} \\ \bottomrule
\multicolumn{4}{c}{\textbf{Related Requirements} } \\ \hline
{\footnotesize CA-DM-CON-ICD-0019 } &
\multicolumn{3}{p{11.1cm}}{\footnotesize Camera engineering image data archiving } \\ \cdashline{1-4}
{\footnotesize DMS-REQ-0002 } &
\multicolumn{3}{p{11.1cm}}{\footnotesize Transient Alert Distribution } \\ \cdashline{1-4}
{\footnotesize DMS-REQ-0004 } &
\multicolumn{3}{p{11.1cm}}{\footnotesize Nightly Data Accessible Within 24 hrs } \\ \cdashline{1-4}
{\footnotesize DMS-REQ-0008 } &
\multicolumn{3}{p{11.1cm}}{\footnotesize Pipeline Availability } \\ \cdashline{1-4}
{\footnotesize DMS-REQ-0010 } &
\multicolumn{3}{p{11.1cm}}{\footnotesize Difference Exposures } \\ \cdashline{1-4}
{\footnotesize DMS-REQ-0029 } &
\multicolumn{3}{p{11.1cm}}{\footnotesize Generate Photometric Zeropoint for Visit Image } \\ \cdashline{1-4}
{\footnotesize DMS-REQ-0030 } &
\multicolumn{3}{p{11.1cm}}{\footnotesize Generate WCS for Visit Images } \\ \cdashline{1-4}
{\footnotesize DMS-REQ-0069 } &
\multicolumn{3}{p{11.1cm}}{\footnotesize Processed Visit Images } \\ \cdashline{1-4}
{\footnotesize DMS-REQ-0070 } &
\multicolumn{3}{p{11.1cm}}{\footnotesize Generate PSF for Visit Images } \\ \cdashline{1-4}
{\footnotesize DMS-REQ-0072 } &
\multicolumn{3}{p{11.1cm}}{\footnotesize Processed Visit Image Content } \\ \cdashline{1-4}
{\footnotesize DMS-REQ-0074 } &
\multicolumn{3}{p{11.1cm}}{\footnotesize Difference Exposure Attributes } \\ \cdashline{1-4}
{\footnotesize DMS-REQ-0096 } &
\multicolumn{3}{p{11.1cm}}{\footnotesize Generate Data Quality Report Within Specified Time } \\ \cdashline{1-4}
{\footnotesize DMS-REQ-0097 } &
\multicolumn{3}{p{11.1cm}}{\footnotesize Level 1 Data Quality Report Definition } \\ \cdashline{1-4}
{\footnotesize DMS-REQ-0098 } &
\multicolumn{3}{p{11.1cm}}{\footnotesize Generate DMS Performance Report Within Specified Time } \\ \cdashline{1-4}
{\footnotesize DMS-REQ-0099 } &
\multicolumn{3}{p{11.1cm}}{\footnotesize Level 1 Performance Report Definition } \\ \cdashline{1-4}
{\footnotesize DMS-REQ-0100 } &
\multicolumn{3}{p{11.1cm}}{\footnotesize Generate Calibration Report Within Specified Time } \\ \cdashline{1-4}
{\footnotesize DMS-REQ-0101 } &
\multicolumn{3}{p{11.1cm}}{\footnotesize Level 1 Calibration Report Definition } \\ \cdashline{1-4}
{\footnotesize DMS-REQ-0102 } &
\multicolumn{3}{p{11.1cm}}{\footnotesize Provide Engineering \&  Facility Database Archive } \\ \cdashline{1-4}
{\footnotesize DMS-REQ-0131 } &
\multicolumn{3}{p{11.1cm}}{\footnotesize Calibration Images Available Within Specified Time } \\ \cdashline{1-4}
{\footnotesize DMS-REQ-0161 } &
\multicolumn{3}{p{11.1cm}}{\footnotesize Optimization of Cost, Reliability and Availability in Order } \\ \cdashline{1-4}
{\footnotesize DMS-REQ-0162 } &
\multicolumn{3}{p{11.1cm}}{\footnotesize Pipeline Throughput } \\ \cdashline{1-4}
{\footnotesize DMS-REQ-0165 } &
\multicolumn{3}{p{11.1cm}}{\footnotesize Infrastructure Sizing for "catching up" } \\ \cdashline{1-4}
{\footnotesize DMS-REQ-0166 } &
\multicolumn{3}{p{11.1cm}}{\footnotesize Incorporate Fault-Tolerance } \\ \cdashline{1-4}
{\footnotesize DMS-REQ-0167 } &
\multicolumn{3}{p{11.1cm}}{\footnotesize Incorporate Autonomics } \\ \cdashline{1-4}
{\footnotesize DMS-REQ-0185 } &
\multicolumn{3}{p{11.1cm}}{\footnotesize Archive Center } \\ \cdashline{1-4}
{\footnotesize DMS-REQ-0266 } &
\multicolumn{3}{p{11.1cm}}{\footnotesize Exposure Catalog } \\ \cdashline{1-4}
{\footnotesize DMS-REQ-0269 } &
\multicolumn{3}{p{11.1cm}}{\footnotesize DIASource Catalog } \\ \cdashline{1-4}
{\footnotesize DMS-REQ-0270 } &
\multicolumn{3}{p{11.1cm}}{\footnotesize Faint DIASource Measurements } \\ \cdashline{1-4}
{\footnotesize DMS-REQ-0271 } &
\multicolumn{3}{p{11.1cm}}{\footnotesize DIAObject Catalog } \\ \cdashline{1-4}
{\footnotesize DMS-REQ-0272 } &
\multicolumn{3}{p{11.1cm}}{\footnotesize DIAObject Attributes } \\ \cdashline{1-4}
{\footnotesize DMS-REQ-0273 } &
\multicolumn{3}{p{11.1cm}}{\footnotesize SSObject Catalog } \\ \cdashline{1-4}
{\footnotesize DMS-REQ-0274 } &
\multicolumn{3}{p{11.1cm}}{\footnotesize Alert Content } \\ \cdashline{1-4}
{\footnotesize DMS-REQ-0284 } &
\multicolumn{3}{p{11.1cm}}{\footnotesize Level-1 Production Completeness } \\ \cdashline{1-4}
{\footnotesize DMS-REQ-0309 } &
\multicolumn{3}{p{11.1cm}}{\footnotesize Raw Data Archiving Reliability } \\ \cdashline{1-4}
{\footnotesize DMS-REQ-0314 } &
\multicolumn{3}{p{11.1cm}}{\footnotesize Compute Platform Heterogeneity } \\ \cdashline{1-4}
{\footnotesize DMS-REQ-0317 } &
\multicolumn{3}{p{11.1cm}}{\footnotesize DIAForcedSource Catalog } \\ \cdashline{1-4}
{\footnotesize DMS-REQ-0318 } &
\multicolumn{3}{p{11.1cm}}{\footnotesize Data Management Unscheduled Downtime } \\ \cdashline{1-4}
{\footnotesize DMS-REQ-0319 } &
\multicolumn{3}{p{11.1cm}}{\footnotesize Characterizing Variability } \\ \cdashline{1-4}
{\footnotesize DMS-REQ-0320 } &
\multicolumn{3}{p{11.1cm}}{\footnotesize Processing of Data From Special Programs } \\ \cdashline{1-4}
{\footnotesize DMS-REQ-0321 } &
\multicolumn{3}{p{11.1cm}}{\footnotesize Level 1 Processing of Special Programs Data } \\ \cdashline{1-4}
{\footnotesize DMS-REQ-0327 } &
\multicolumn{3}{p{11.1cm}}{\footnotesize Background Model Calculation } \\ \cdashline{1-4}
{\footnotesize DMS-REQ-0328 } &
\multicolumn{3}{p{11.1cm}}{\footnotesize Documenting Image Characterization } \\ \cdashline{1-4}
{\footnotesize DMS-REQ-0343 } &
\multicolumn{3}{p{11.1cm}}{\footnotesize Performance Requirements for LSST Alert Filtering Service } \\ \cdashline{1-4}
{\footnotesize DMS-REQ-0344 } &
\multicolumn{3}{p{11.1cm}}{\footnotesize Constraints on Level 1 Special Program Products Generation } \\ \cdashline{1-4}
{\footnotesize EP-DM-CON-ICD-0023 } &
\multicolumn{3}{p{11.1cm}}{\footnotesize Nightly DM Transfer of Processed Visit Images (PVI)-Based Images to EPO } \\ \cdashline{1-4}
\end{longtable}

    
     \newpage
\subsection{Hardware and COTS Products}\label{hwcots}
\begin{longtable}{p{3.7cm}p{3.7cm}p{3.7cm}p{3.7cm}}\hline
\textbf{Manager} & \textbf{Owner} & \textbf{WBS} & \textbf{Team} \\ \hline
\parbox{3.5cm}{
Wil O'Mullane
\vspace{2mm}%
} &
\begin{tabular}{@{}l@{}}
\parbox{3.5cm}{

\vspace{2mm}%
} \\
\end{tabular}
 &
\begin{tabular}{@{}l@{}}
 \\
\end{tabular} &
\begin{tabular}{@{}l@{}}
 \\
\end{tabular} \\ \hline
\multicolumn{4}{c}{
{\footnotesize ( Resources
 - HWCOTS ) }
}\\ \hline
\end{longtable}

  External resources needed to implement the DM services:

\begin{itemize}
\tightlist
\item
  Hardware (computing hardware, network hardware, etc)
\item
  COTS
\item
  External software (not developed in DM)
\end{itemize}


\includegraphics[max width=\linewidth]{subtrees/Main_HWCOTS.pdf}

    \newpage
\begin{longtable}{p{3.7cm}p{3.7cm}p{3.7cm}p{3.7cm}}\toprule
\multicolumn{2}{l}{\large \textbf{ \hypertarget{hwcomp}{Compute Nodes} } }
& \multicolumn{2}{l}{(product in: Resources
)}
\\ \hline
\textbf{\footnotesize Manager} & \textbf{\footnotesize Owner} &
\textbf{\footnotesize WBS} & \textbf{\footnotesize Team} \\ \hline
\parbox{3.5cm}{
Michelle Butler
\vspace{2mm}%
} &
\begin{tabular}{@{}l@{}}
\parbox{3.5cm}{

\vspace{2mm}%
} \\
\end{tabular} &
\begin{tabular}{@{}l@{}}
 \\
\end{tabular} & \begin{tabular}{@{}l@{}}
 \\
\end{tabular} \\ \hline
\multicolumn{4}{c}{
{\footnotesize ( Compute Nodes
 - HWCOMP ) }
}\\ \hline
\end{longtable}

Computational elements: nodes that provide COP and memory to process the
data, run services. Secondary hardware parts, like for example racks, re
also included in this section. Details will be added here on a per need
base.


\begin{longtable}{p{3.7cm}p{3.7cm}p{3.7cm}p{3.7cm}}\hline
\textbf{\footnotesize Uses:}  & & \textbf{\footnotesize Used in:} & \\ \hline
\multicolumn{2}{c}{
} &
\multicolumn{2}{c}{
} \\ \bottomrule
\multicolumn{4}{c}{\textbf{Related Requirements} } \\ \hline
\end{longtable}

       \newpage
\begin{longtable}{p{3.7cm}p{3.7cm}p{3.7cm}p{3.7cm}}\toprule
\multicolumn{2}{l}{\large \textbf{ \hypertarget{hwstor}{Storage Nodes} } }
& \multicolumn{2}{l}{(product in: Resources
)}
\\ \hline
\textbf{\footnotesize Manager} & \textbf{\footnotesize Owner} &
\textbf{\footnotesize WBS} & \textbf{\footnotesize Team} \\ \hline
\parbox{3.5cm}{
Michelle Butler
\vspace{2mm}%
} &
\begin{tabular}{@{}l@{}}
\parbox{3.5cm}{

\vspace{2mm}%
} \\
\end{tabular} &
\begin{tabular}{@{}l@{}}
 \\
\end{tabular} & \begin{tabular}{@{}l@{}}
 \\
\end{tabular} \\ \hline
\multicolumn{4}{c}{
{\footnotesize ( Storage Nodes
 - HWSTOR ) }
}\\ \hline
\end{longtable}

Storage components to host data. In general are disks, but tape storage
for long term archive shall also be included in this area. Details will
be added here on a per need base.


\begin{longtable}{p{3.7cm}p{3.7cm}p{3.7cm}p{3.7cm}}\hline
\textbf{\footnotesize Uses:}  & & \textbf{\footnotesize Used in:} & \\ \hline
\multicolumn{2}{c}{
} &
\multicolumn{2}{c}{
} \\ \bottomrule
\multicolumn{4}{c}{\textbf{Related Requirements} } \\ \hline
\end{longtable}

       \newpage
\begin{longtable}{p{3.7cm}p{3.7cm}p{3.7cm}p{3.7cm}}\toprule
\multicolumn{2}{l}{\large \textbf{ \hypertarget{hwnet}{Network Nodes} } }
& \multicolumn{2}{l}{(product in: Resources
)}
\\ \hline
\textbf{\footnotesize Manager} & \textbf{\footnotesize Owner} &
\textbf{\footnotesize WBS} & \textbf{\footnotesize Team} \\ \hline
\parbox{3.5cm}{
Jeff Kantor
\vspace{2mm}%
} &
\begin{tabular}{@{}l@{}}
\parbox{3.5cm}{

\vspace{2mm}%
} \\
\end{tabular} &
\begin{tabular}{@{}l@{}}
 \\
\end{tabular} & \begin{tabular}{@{}l@{}}
 \\
\end{tabular} \\ \hline
\multicolumn{4}{c}{
{\footnotesize ( Network Nodes
 - HWNET ) }
}\\ \hline
\end{longtable}

The network components required in order to implement local and long
distance networks. Details will be added here on a per need base.


\begin{longtable}{p{3.7cm}p{3.7cm}p{3.7cm}p{3.7cm}}\hline
\textbf{\footnotesize Uses:}  & & \textbf{\footnotesize Used in:} & \\ \hline
\multicolumn{2}{c}{
} &
\multicolumn{2}{c}{
} \\ \bottomrule
\multicolumn{4}{c}{\textbf{Related Requirements} } \\ \hline
\end{longtable}

   \newpage
\subsubsection{COTS}\label{cots}
\begin{longtable}{p{3.7cm}p{3.7cm}p{3.7cm}p{3.7cm}}\hline
\textbf{Manager} & \textbf{Owner} & \textbf{WBS} & \textbf{Team} \\ \hline
\parbox{3.5cm}{
Wil O'Mullane
\vspace{2mm}%
} &
\begin{tabular}{@{}l@{}}
\parbox{3.5cm}{

\vspace{2mm}%
} \\
\end{tabular}
 &
\begin{tabular}{@{}l@{}}
 \\
\end{tabular} &
\begin{tabular}{@{}l@{}}
 \\
\end{tabular} \\ \hline
\multicolumn{4}{c}{
{\footnotesize ( COTS SW
 - COTS ) }
}\\ \hline
\end{longtable}

  This product includes all COTS used to implement the DM services.


   \newpage
\begin{longtable}{p{3.7cm}p{3.7cm}p{3.7cm}p{3.7cm}}\toprule
\multicolumn{2}{l}{\large \textbf{ \hypertarget{cilogon}{CILogon} } }
& \multicolumn{2}{l}{(product in: COTS SW
)}
\\ \hline
\textbf{\footnotesize Manager} & \textbf{\footnotesize Owner} &
\textbf{\footnotesize WBS} & \textbf{\footnotesize Team} \\ \hline
\parbox{3.5cm}{

\vspace{2mm}%
} &
\begin{tabular}{@{}l@{}}
\parbox{3.5cm}{
Michelle Gower
\vspace{2mm}%
} \\
\end{tabular} &
\begin{tabular}{@{}l@{}}
 \\
\end{tabular} & \begin{tabular}{@{}l@{}}
 \\
\end{tabular} \\ \hline
\multicolumn{4}{c}{
{\footnotesize ( CILogon
 - CILOGON ) }
}\\ \hline
\end{longtable}

CILogon provides an integrated open source identity and access
management platform for research collaborations.


\begin{longtable}{p{3.7cm}p{3.7cm}p{3.7cm}p{3.7cm}}\hline
\textbf{References} &
\multicolumn{3}{l}{\href{http://www.cilogon.org/}{http://www.cilogon.org/} }
\\ \hline \\ \hline
\textbf{\footnotesize Uses:}  & & \textbf{\footnotesize Used in:} & \\ \hline
\multicolumn{2}{c}{
} &
\multicolumn{2}{c}{
\begin{tabular}{c}
\hyperlink{}{Identity Management} \\ \hline
\end{tabular}
} \\ \bottomrule
\multicolumn{4}{c}{\textbf{Related Requirements} } \\ \hline
\end{longtable}

     \newpage
\begin{longtable}{p{3.7cm}p{3.7cm}p{3.7cm}p{3.7cm}}\toprule
\multicolumn{2}{l}{\large \textbf{ \hypertarget{docker}{Docker} } }
& \multicolumn{2}{l}{(product in: COTS SW
)}
\\ \hline
\textbf{\footnotesize Manager} & \textbf{\footnotesize Owner} &
\textbf{\footnotesize WBS} & \textbf{\footnotesize Team} \\ \hline
\parbox{3.5cm}{

\vspace{2mm}%
} &
\begin{tabular}{@{}l@{}}
\parbox{3.5cm}{
Frossie Economou
\vspace{2mm}%
} \\
\end{tabular} &
\begin{tabular}{@{}l@{}}
 \\
\end{tabular} & \begin{tabular}{@{}l@{}}
 \\
\end{tabular} \\ \hline
\multicolumn{4}{c}{
{\footnotesize ( Docker
 - DOCKER ) }
}\\ \hline
\end{longtable}

3rd party software product used to perform operating system level
virtualizations.


\begin{longtable}{p{3.7cm}p{3.7cm}p{3.7cm}p{3.7cm}}\hline
\textbf{References} &
\multicolumn{3}{l}{\href{https://www.docker.com/}{https://www.docker.com/} }
\\ \hline \\ \hline
\textbf{\footnotesize Uses:}  & & \textbf{\footnotesize Used in:} & \\ \hline
\multicolumn{2}{c}{
} &
\multicolumn{2}{c}{
\begin{tabular}{c}
\hyperlink{cam}{Containerized Application Management} \\ \hline
\end{tabular}
} \\ \bottomrule
\multicolumn{4}{c}{\textbf{Related Requirements} } \\ \hline
\end{longtable}

     \newpage
\begin{longtable}{p{3.7cm}p{3.7cm}p{3.7cm}p{3.7cm}}\toprule
\multicolumn{2}{l}{\large \textbf{ \hypertarget{gpfs}{GPFS} } }
& \multicolumn{2}{l}{(product in: COTS SW
)}
\\ \hline
\textbf{\footnotesize Manager} & \textbf{\footnotesize Owner} &
\textbf{\footnotesize WBS} & \textbf{\footnotesize Team} \\ \hline
\parbox{3.5cm}{

\vspace{2mm}%
} &
\begin{tabular}{@{}l@{}}
\parbox{3.5cm}{
Michelle Gower
\vspace{2mm}%
} \\
\end{tabular} &
\begin{tabular}{@{}l@{}}
 \\
\end{tabular} & \begin{tabular}{@{}l@{}}
 \\
\end{tabular} \\ \hline
\multicolumn{4}{c}{
{\footnotesize ( GPFS
 - GPFS ) }
}\\ \hline
\end{longtable}

The General Parallel File System (GPFS) is a high-performance clustered
file system developed by IBM.


\begin{longtable}{p{3.7cm}p{3.7cm}p{3.7cm}p{3.7cm}}\hline
\textbf{References} &
\multicolumn{3}{l}{\href{https://en.wikipedia.org/wiki/IBM_General_Parallel_File_System}{https://en.wikipedia.org/wiki/IBM\_General\_Parallel\_File\_System} }
\\ \hline \\ \hline
\textbf{\footnotesize Uses:}  & & \textbf{\footnotesize Used in:} & \\ \hline
\multicolumn{2}{c}{
} &
\multicolumn{2}{c}{
\begin{tabular}{c}
\hyperlink{}{ICT Provisioning and Management} \\ \hline
\end{tabular}
} \\ \bottomrule
\multicolumn{4}{c}{\textbf{Related Requirements} } \\ \hline
\end{longtable}

     \newpage
\begin{longtable}{p{3.7cm}p{3.7cm}p{3.7cm}p{3.7cm}}\toprule
\multicolumn{2}{l}{\large \textbf{ \hypertarget{grafana}{Grafana} } }
& \multicolumn{2}{l}{(product in: COTS SW
)}
\\ \hline
\textbf{\footnotesize Manager} & \textbf{\footnotesize Owner} &
\textbf{\footnotesize WBS} & \textbf{\footnotesize Team} \\ \hline
\parbox{3.5cm}{

\vspace{2mm}%
} &
\begin{tabular}{@{}l@{}}
\parbox{3.5cm}{
Michelle Gower
\vspace{2mm}%
} \\
\end{tabular} &
\begin{tabular}{@{}l@{}}
 \\
\end{tabular} & \begin{tabular}{@{}l@{}}
 \\
\end{tabular} \\ \hline
\multicolumn{4}{c}{
{\footnotesize ( Grafana
 - GRAFANA ) }
}\\ \hline
\end{longtable}

3rd party software product for analytics and monitoring


\begin{longtable}{p{3.7cm}p{3.7cm}p{3.7cm}p{3.7cm}}\hline
\textbf{References} &
\multicolumn{3}{l}{\href{https://grafana.com/}{https://grafana.com/} }
\\ \hline \\ \hline
\textbf{\footnotesize Uses:}  & & \textbf{\footnotesize Used in:} & \\ \hline
\multicolumn{2}{c}{
} &
\multicolumn{2}{c}{
\begin{tabular}{c}
\hyperlink{}{Monitoring} \\ \hline
\end{tabular}
} \\ \bottomrule
\multicolumn{4}{c}{\textbf{Related Requirements} } \\ \hline
\end{longtable}

     \newpage
\begin{longtable}{p{3.7cm}p{3.7cm}p{3.7cm}p{3.7cm}}\toprule
\multicolumn{2}{l}{\large \textbf{ \hypertarget{htcondor}{HTCondor} } }
& \multicolumn{2}{l}{(product in: COTS SW
)}
\\ \hline
\textbf{\footnotesize Manager} & \textbf{\footnotesize Owner} &
\textbf{\footnotesize WBS} & \textbf{\footnotesize Team} \\ \hline
\parbox{3.5cm}{

\vspace{2mm}%
} &
\begin{tabular}{@{}l@{}}
\parbox{3.5cm}{
Michelle Gower
\vspace{2mm}%
} \\
\end{tabular} &
\begin{tabular}{@{}l@{}}
 \\
\end{tabular} & \begin{tabular}{@{}l@{}}
 \\
\end{tabular} \\ \hline
\multicolumn{4}{c}{
{\footnotesize ( HTCondor
 - HTCONDOR ) }
}\\ \hline
\end{longtable}

HTCondor is an open-source high-throughput computing software framework
for coarse-grained distributed parallelization of computationally
intensive tasks.


\begin{longtable}{p{3.7cm}p{3.7cm}p{3.7cm}p{3.7cm}}\hline
\textbf{References} &
\multicolumn{3}{l}{\href{https://research.cs.wisc.edu/htcondor/}{https://research.cs.wisc.edu/htcondor/} }
\\ \cline{2-4}
& \multicolumn{3}{l}{\href{https://github.com/htcondor}{https://github.com/htcondor} }
\\ \hline \\ \hline
\textbf{\footnotesize Uses:}  & & \textbf{\footnotesize Used in:} & \\ \hline
\multicolumn{2}{c}{
} &
\multicolumn{2}{c}{
\begin{tabular}{c}
\hyperlink{prodsrv}{Batch Production} \\ \hline
\end{tabular}
} \\ \bottomrule
\multicolumn{4}{c}{\textbf{Related Requirements} } \\ \hline
\end{longtable}

     \newpage
\begin{longtable}{p{3.7cm}p{3.7cm}p{3.7cm}p{3.7cm}}\toprule
\multicolumn{2}{l}{\large \textbf{ \hypertarget{k8s}{Kubernetes} } }
& \multicolumn{2}{l}{(product in: COTS SW
)}
\\ \hline
\textbf{\footnotesize Manager} & \textbf{\footnotesize Owner} &
\textbf{\footnotesize WBS} & \textbf{\footnotesize Team} \\ \hline
\parbox{3.5cm}{

\vspace{2mm}%
} &
\begin{tabular}{@{}l@{}}
\parbox{3.5cm}{

\vspace{2mm}%
} \\
\end{tabular} &
\begin{tabular}{@{}l@{}}
 \\
\end{tabular} & \begin{tabular}{@{}l@{}}
 \\
\end{tabular} \\ \hline
\multicolumn{4}{c}{
{\footnotesize ( Kubernetes
 - K8S ) }
}\\ \hline
\end{longtable}

Kubernetes is an open-source system for automating deployment, scaling,
and management of containerized applications.


\begin{longtable}{p{3.7cm}p{3.7cm}p{3.7cm}p{3.7cm}}\hline
\textbf{References} &
\multicolumn{3}{l}{\href{https://kubernetes.io/}{https://kubernetes.io/} }
\\ \hline \\ \hline
\textbf{\footnotesize Uses:}  & & \textbf{\footnotesize Used in:} & \\ \hline
\multicolumn{2}{c}{
} &
\multicolumn{2}{c}{
\begin{tabular}{c}
\hyperlink{cam}{Containerized Application Management} \\ \hline
\end{tabular}
} \\ \bottomrule
\multicolumn{4}{c}{\textbf{Related Requirements} } \\ \hline
\end{longtable}

     \newpage
\begin{longtable}{p{3.7cm}p{3.7cm}p{3.7cm}p{3.7cm}}\toprule
\multicolumn{2}{l}{\large \textbf{ \hypertarget{oracle}{Oracle} } }
& \multicolumn{2}{l}{(product in: COTS SW
)}
\\ \hline
\textbf{\footnotesize Manager} & \textbf{\footnotesize Owner} &
\textbf{\footnotesize WBS} & \textbf{\footnotesize Team} \\ \hline
\parbox{3.5cm}{

\vspace{2mm}%
} &
\begin{tabular}{@{}l@{}}
\parbox{3.5cm}{

\vspace{2mm}%
} \\
\end{tabular} &
\begin{tabular}{@{}l@{}}
 \\
\end{tabular} & \begin{tabular}{@{}l@{}}
 \\
\end{tabular} \\ \hline
\multicolumn{4}{c}{
{\footnotesize ( Oracle
 - ORACLE ) }
}\\ \hline
\end{longtable}

Oracle Relational Database 3rd party software product.


\begin{longtable}{p{3.7cm}p{3.7cm}p{3.7cm}p{3.7cm}}\hline
\textbf{References} &
\multicolumn{3}{l}{\href{https://www.oracle.com/index.html}{https://www.oracle.com/index.html} }
\\ \hline \\ \hline
\textbf{\footnotesize Uses:}  & & \textbf{\footnotesize Used in:} & \\ \hline
\multicolumn{2}{c}{
} &
\multicolumn{2}{c}{
\begin{tabular}{c}
\hyperlink{}{Database Management} \\ \hline
\end{tabular}
} \\ \bottomrule
\multicolumn{4}{c}{\textbf{Related Requirements} } \\ \hline
\end{longtable}

     \newpage
\begin{longtable}{p{3.7cm}p{3.7cm}p{3.7cm}p{3.7cm}}\toprule
\multicolumn{2}{l}{\large \textbf{ \hypertarget{pth}{Python} } }
& \multicolumn{2}{l}{(product in: COTS SW
)}
\\ \hline
\textbf{\footnotesize Manager} & \textbf{\footnotesize Owner} &
\textbf{\footnotesize WBS} & \textbf{\footnotesize Team} \\ \hline
\parbox{3.5cm}{

\vspace{2mm}%
} &
\begin{tabular}{@{}l@{}}
\parbox{3.5cm}{

\vspace{2mm}%
} \\
\end{tabular} &
\begin{tabular}{@{}l@{}}
 \\
\end{tabular} & \begin{tabular}{@{}l@{}}
 \\
\end{tabular} \\ \hline
\multicolumn{4}{c}{
{\footnotesize ( Python
 - PTH ) }
}\\ \hline
\end{longtable}




\begin{longtable}{p{3.7cm}p{3.7cm}p{3.7cm}p{3.7cm}}\hline
\textbf{\footnotesize Uses:}  & & \textbf{\footnotesize Used in:} & \\ \hline
\multicolumn{2}{c}{
} &
\multicolumn{2}{c}{
} \\ \bottomrule
\multicolumn{4}{c}{\textbf{Related Requirements} } \\ \hline
\end{longtable}

     \newpage
\begin{longtable}{p{3.7cm}p{3.7cm}p{3.7cm}p{3.7cm}}\toprule
\multicolumn{2}{l}{\large \textbf{ \hypertarget{puppet}{Puppet} } }
& \multicolumn{2}{l}{(product in: COTS SW
)}
\\ \hline
\textbf{\footnotesize Manager} & \textbf{\footnotesize Owner} &
\textbf{\footnotesize WBS} & \textbf{\footnotesize Team} \\ \hline
\parbox{3.5cm}{

\vspace{2mm}%
} &
\begin{tabular}{@{}l@{}}
\parbox{3.5cm}{

\vspace{2mm}%
} \\
\end{tabular} &
\begin{tabular}{@{}l@{}}
 \\
\end{tabular} & \begin{tabular}{@{}l@{}}
 \\
\end{tabular} \\ \hline
\multicolumn{4}{c}{
{\footnotesize ( Puppet
 - PUPPET ) }
}\\ \hline
\end{longtable}

Puppet is an open-source software configuration management tool.


\begin{longtable}{p{3.7cm}p{3.7cm}p{3.7cm}p{3.7cm}}\hline
\textbf{References} &
\multicolumn{3}{l}{\href{https://en.wikipedia.org/wiki/Puppet_(software)}{https://en.wikipedia.org/wiki/Puppet\_(software)} }
\\ \cline{2-4}
& \multicolumn{3}{l}{\href{https://puppet.com/}{https://puppet.com/} }
\\ \hline \\ \hline
\textbf{\footnotesize Uses:}  & & \textbf{\footnotesize Used in:} & \\ \hline
\multicolumn{2}{c}{
} &
\multicolumn{2}{c}{
\begin{tabular}{c}
\hyperlink{}{ICT Provisioning and Management} \\ \hline
\end{tabular}
} \\ \bottomrule
\multicolumn{4}{c}{\textbf{Related Requirements} } \\ \hline
\end{longtable}

     \newpage
\begin{longtable}{p{3.7cm}p{3.7cm}p{3.7cm}p{3.7cm}}\toprule
\multicolumn{2}{l}{\large \textbf{ \hypertarget{rucio}{Rucio} } }
& \multicolumn{2}{l}{(product in: COTS SW
)}
\\ \hline
\textbf{\footnotesize Manager} & \textbf{\footnotesize Owner} &
\textbf{\footnotesize WBS} & \textbf{\footnotesize Team} \\ \hline
\parbox{3.5cm}{

\vspace{2mm}%
} &
\begin{tabular}{@{}l@{}}
\parbox{3.5cm}{
Michelle Butler
\vspace{2mm}%
} \\
\end{tabular} &
\begin{tabular}{@{}l@{}}
 \\
\end{tabular} & \begin{tabular}{@{}l@{}}
 \\
\end{tabular} \\ \hline
\multicolumn{4}{c}{
{\footnotesize ( Rucio
 - RUCIO ) }
}\\ \hline
\end{longtable}




\begin{longtable}{p{3.7cm}p{3.7cm}p{3.7cm}p{3.7cm}}\hline
\textbf{References} &
\multicolumn{3}{l}{\href{http://rucio.cern.ch/}{http://rucio.cern.ch/} }
\\ \cline{2-4}
& \multicolumn{3}{l}{\href{http://rucio.readthedocs.io/}{http://rucio.readthedocs.io/} }
\\ \hline \\ \hline
\textbf{\footnotesize Uses:}  & & \textbf{\footnotesize Used in:} & \\ \hline
\multicolumn{2}{c}{
} &
\multicolumn{2}{c}{
\begin{tabular}{c}
\hyperlink{bulkdsrv}{Bulk Distribution} \\ \hline
\end{tabular}
} \\ \bottomrule
\multicolumn{4}{c}{\textbf{Related Requirements} } \\ \hline
\end{longtable}

     \newpage
\begin{longtable}{p{3.7cm}p{3.7cm}p{3.7cm}p{3.7cm}}\toprule
\multicolumn{2}{l}{\large \textbf{ \hypertarget{security}{IT Security SW} } }
& \multicolumn{2}{l}{(product in: COTS SW
)}
\\ \hline
\textbf{\footnotesize Manager} & \textbf{\footnotesize Owner} &
\textbf{\footnotesize WBS} & \textbf{\footnotesize Team} \\ \hline
\parbox{3.5cm}{

\vspace{2mm}%
} &
\begin{tabular}{@{}l@{}}
\parbox{3.5cm}{

\vspace{2mm}%
} \\
\end{tabular} &
\begin{tabular}{@{}l@{}}
 \\
\end{tabular} & \begin{tabular}{@{}l@{}}
 \\
\end{tabular} \\ \hline
\multicolumn{4}{c}{
{\footnotesize ( IT Security
 - SECURITY ) }
}\\ \hline
\end{longtable}




\begin{longtable}{p{3.7cm}p{3.7cm}p{3.7cm}p{3.7cm}}\hline
\textbf{\footnotesize Uses:}  & & \textbf{\footnotesize Used in:} & \\ \hline
\multicolumn{2}{c}{
} &
\multicolumn{2}{c}{
\begin{tabular}{c}
\hyperlink{}{IT Security} \\ \hline
\end{tabular}
} \\ \bottomrule
\multicolumn{4}{c}{\textbf{Related Requirements} } \\ \hline
\end{longtable}

     \newpage
\begin{longtable}{p{3.7cm}p{3.7cm}p{3.7cm}p{3.7cm}}\toprule
\multicolumn{2}{l}{\large \textbf{ \hypertarget{vsphere}{vSphere} } }
& \multicolumn{2}{l}{(product in: COTS SW
)}
\\ \hline
\textbf{\footnotesize Manager} & \textbf{\footnotesize Owner} &
\textbf{\footnotesize WBS} & \textbf{\footnotesize Team} \\ \hline
\parbox{3.5cm}{

\vspace{2mm}%
} &
\begin{tabular}{@{}l@{}}
\parbox{3.5cm}{

\vspace{2mm}%
} \\
\end{tabular} &
\begin{tabular}{@{}l@{}}
 \\
\end{tabular} & \begin{tabular}{@{}l@{}}
 \\
\end{tabular} \\ \hline
\multicolumn{4}{c}{
{\footnotesize ( vSphere
 - VSPHERE ) }
}\\ \hline
\end{longtable}

Third party software product for virtualization.


\begin{longtable}{p{3.7cm}p{3.7cm}p{3.7cm}p{3.7cm}}\hline
\textbf{References} &
\multicolumn{3}{l}{\href{https://www.vmware.com/products/vsphere.html}{https://www.vmware.com/products/vsphere.html} }
\\ \hline \\ \hline
\textbf{\footnotesize Uses:}  & & \textbf{\footnotesize Used in:} & \\ \hline
\multicolumn{2}{c}{
} &
\multicolumn{2}{c}{
\begin{tabular}{c}
\hyperlink{}{ICT Provisioning and Management} \\ \hline
\end{tabular}
} \\ \bottomrule
\multicolumn{4}{c}{\textbf{Related Requirements} } \\ \hline
\end{longtable}

    
   \newpage
\subsubsection{Low Level SW}\label{llsw}
\begin{longtable}{p{3.7cm}p{3.7cm}p{3.7cm}p{3.7cm}}\hline
\textbf{Manager} & \textbf{Owner} & \textbf{WBS} & \textbf{Team} \\ \hline
\parbox{3.5cm}{

\vspace{2mm}%
} &
\begin{tabular}{@{}l@{}}
\parbox{3.5cm}{
Multiple
\vspace{2mm}%
} \\
\end{tabular}
 &
\begin{tabular}{@{}l@{}}
 \\
\end{tabular} &
\begin{tabular}{@{}l@{}}
 \\
\end{tabular} \\ \hline
\multicolumn{4}{c}{
{\footnotesize ( Low Level SW
 - LLSW ) }
}\\ \hline
\end{longtable}

  Low level software required by the hardware components to work. In
general these components are provided by the vendor, but in some cases
it may be required to customize specifically for the project purposes.
Some examples are: - Firmwares - Operating Systems Details will be added
here on a per need base.


   \newpage
\begin{longtable}{p{3.7cm}p{3.7cm}p{3.7cm}p{3.7cm}}\toprule
\multicolumn{2}{l}{\large \textbf{ \hypertarget{centos}{RedHat CentOS} } }
& \multicolumn{2}{l}{(product in: Low Level SW
)}
\\ \hline
\textbf{\footnotesize Manager} & \textbf{\footnotesize Owner} &
\textbf{\footnotesize WBS} & \textbf{\footnotesize Team} \\ \hline
\parbox{3.5cm}{

\vspace{2mm}%
} &
\begin{tabular}{@{}l@{}}
\parbox{3.5cm}{

\vspace{2mm}%
} \\
\end{tabular} &
\begin{tabular}{@{}l@{}}
 \\
\end{tabular} & \begin{tabular}{@{}l@{}}
 \\
\end{tabular} \\ \hline
\multicolumn{4}{c}{
{\footnotesize ( OS (e.g.)
 - CENTOS ) }
}\\ \hline
\end{longtable}




\begin{longtable}{p{3.7cm}p{3.7cm}p{3.7cm}p{3.7cm}}\hline
\textbf{\footnotesize Uses:}  & & \textbf{\footnotesize Used in:} & \\ \hline
\multicolumn{2}{c}{
} &
\multicolumn{2}{c}{
} \\ \bottomrule
\multicolumn{4}{c}{\textbf{Related Requirements} } \\ \hline
\end{longtable}

    
   \newpage
\subsubsection{Third Party Libraries}\label{thpl}
\begin{longtable}{p{3.7cm}p{3.7cm}p{3.7cm}p{3.7cm}}\hline
\textbf{Manager} & \textbf{Owner} & \textbf{WBS} & \textbf{Team} \\ \hline
\parbox{3.5cm}{

\vspace{2mm}%
} &
\begin{tabular}{@{}l@{}}
\parbox{3.5cm}{

\vspace{2mm}%
} \\
\end{tabular}
 &
\begin{tabular}{@{}l@{}}
 \\
\end{tabular} &
\begin{tabular}{@{}l@{}}
 \\
\end{tabular} \\ \hline
\multicolumn{4}{c}{
{\footnotesize ( Third Party Libs
 - THPL ) }
}\\ \hline
\end{longtable}

  External libraries required by the DM SW products in order to compile
and to run.


   \newpage
\begin{longtable}{p{3.7cm}p{3.7cm}p{3.7cm}p{3.7cm}}\toprule
\multicolumn{2}{l}{\large \textbf{ \hypertarget{boost}{Boost} } }
& \multicolumn{2}{l}{(product in: Third Party Libs
)}
\\ \hline
\textbf{\footnotesize Manager} & \textbf{\footnotesize Owner} &
\textbf{\footnotesize WBS} & \textbf{\footnotesize Team} \\ \hline
\parbox{3.5cm}{

\vspace{2mm}%
} &
\begin{tabular}{@{}l@{}}
\parbox{3.5cm}{

\vspace{2mm}%
} \\
\end{tabular} &
\begin{tabular}{@{}l@{}}
 \\
\end{tabular} & \begin{tabular}{@{}l@{}}
 \\
\end{tabular} \\ \hline
\multicolumn{4}{c}{
{\footnotesize ( Boost
 - BOOST ) }
}\\ \hline
\end{longtable}




\begin{longtable}{p{3.7cm}p{3.7cm}p{3.7cm}p{3.7cm}}\hline
\textbf{\footnotesize Uses:}  & & \textbf{\footnotesize Used in:} & \\ \hline
\multicolumn{2}{c}{
} &
\multicolumn{2}{c}{
} \\ \bottomrule
\multicolumn{4}{c}{\textbf{Related Requirements} } \\ \hline
\end{longtable}

     \newpage
\begin{longtable}{p{3.7cm}p{3.7cm}p{3.7cm}p{3.7cm}}\toprule
\multicolumn{2}{l}{\large \textbf{ \hypertarget{ffaa}{Firefly API Access} } }
& \multicolumn{2}{l}{(product in: Third Party Libs
)}
\\ \hline
\textbf{\footnotesize Manager} & \textbf{\footnotesize Owner} &
\textbf{\footnotesize WBS} & \textbf{\footnotesize Team} \\ \hline
\parbox{3.5cm}{
Xiuqin Wu
\vspace{2mm}%
} &
\begin{tabular}{@{}l@{}}
\parbox{3.5cm}{
Gregory Dubois-Felsmann
\vspace{2mm}%
} \\
\end{tabular} &
\begin{tabular}{@{}l@{}}
 \\
\end{tabular} & \begin{tabular}{@{}l@{}}
SUIT \\
\end{tabular} \\ \hline
\multicolumn{4}{c}{
{\footnotesize ( Firefly API A.
 - FFAA ) }
}\\ \hline
\end{longtable}

Provides hooks to allow Firefly's Javascript code to be loaded starting
from npm-style package loading


\begin{longtable}{p{3.7cm}p{3.7cm}p{3.7cm}p{3.7cm}}\hline
\textbf{References} &
\multicolumn{3}{l}{\href{https://github.com/Caltech-IPAC/firefly-api-access}{https://github.com/Caltech-IPAC/firefly-api-access} }
\\ \hline \\ \hline
\textbf{\footnotesize Uses:}  & & \textbf{\footnotesize Used in:} & \\ \hline
\multicolumn{2}{c}{
} &
\multicolumn{2}{c}{
} \\ \bottomrule
\multicolumn{4}{c}{\textbf{Related Requirements} } \\ \hline
\end{longtable}

     \newpage
\begin{longtable}{p{3.7cm}p{3.7cm}p{3.7cm}p{3.7cm}}\toprule
\multicolumn{2}{l}{\large \textbf{ \hypertarget{firefly}{Firefly} } }
& \multicolumn{2}{l}{(product in: Third Party Libs
)}
\\ \hline
\textbf{\footnotesize Manager} & \textbf{\footnotesize Owner} &
\textbf{\footnotesize WBS} & \textbf{\footnotesize Team} \\ \hline
\parbox{3.5cm}{
Xiuqin Wu
\vspace{2mm}%
} &
\begin{tabular}{@{}l@{}}
\parbox{3.5cm}{
Gregory Dubois-Felsmann
\vspace{2mm}%
} \\
\end{tabular} &
\begin{tabular}{@{}l@{}}
1.02C.05.06 \\
\end{tabular} & \begin{tabular}{@{}l@{}}
SUIT \\
\end{tabular} \\ \hline
\multicolumn{4}{c}{
{\footnotesize ( Firefly
 - FIREFLY ) }
}\\ \hline
\end{longtable}

IPAC's Advanced Astronomy WEB UI Framework.


\begin{longtable}{p{3.7cm}p{3.7cm}p{3.7cm}p{3.7cm}}\hline
\textbf{References} &
\multicolumn{3}{l}{\href{https://github.com/Caltech-IPAC/firefly}{https://github.com/Caltech-IPAC/firefly} }
\\ \cline{2-4}
& \multicolumn{3}{l}{\href{http://web.ipac.caltech.edu/staff/roby/camera-team-ff-overview.pdf}{http://web.ipac.caltech.edu/staff/roby/camera-team-ff-overview.pdf} }
\\ \hline \\ \hline
\textbf{\footnotesize Uses:}  & & \textbf{\footnotesize Used in:} & \\ \hline
\multicolumn{2}{c}{
} &
\multicolumn{2}{c}{
\begin{tabular}{c}
\hyperlink{suit}{SUIT} \\ \hline
\end{tabular}
} \\ \bottomrule
\multicolumn{4}{c}{\textbf{Related Requirements} } \\ \hline
{\footnotesize DMS-REQ-0160 } &
\multicolumn{3}{p{11.1cm}}{\footnotesize Provide User Interface Services } \\ \cdashline{1-4}
\end{longtable}

     \newpage
\begin{longtable}{p{3.7cm}p{3.7cm}p{3.7cm}p{3.7cm}}\toprule
\multicolumn{2}{l}{\large \textbf{ \hypertarget{jh3}{Jupyterhub} } }
& \multicolumn{2}{l}{(product in: Third Party Libs
)}
\\ \hline
\textbf{\footnotesize Manager} & \textbf{\footnotesize Owner} &
\textbf{\footnotesize WBS} & \textbf{\footnotesize Team} \\ \hline
\parbox{3.5cm}{

\vspace{2mm}%
} &
\begin{tabular}{@{}l@{}}
\parbox{3.5cm}{

\vspace{2mm}%
} \\
\end{tabular} &
\begin{tabular}{@{}l@{}}
 \\
\end{tabular} & \begin{tabular}{@{}l@{}}
 \\
\end{tabular} \\ \hline
\multicolumn{4}{c}{
{\footnotesize ( Jupyterhub
 - JH3 ) }
}\\ \hline
\end{longtable}

Provides multi-user and multi-instance support to Jupyterlab.


\begin{longtable}{p{3.7cm}p{3.7cm}p{3.7cm}p{3.7cm}}\hline
\textbf{References} &
\multicolumn{3}{l}{\href{https://github.com/jupyterhub/jupyterhub/}{https://github.com/jupyterhub/jupyterhub/} }
\\ \hline \\ \hline
\textbf{\footnotesize Uses:}  & & \textbf{\footnotesize Used in:} & \\ \hline
\multicolumn{2}{c}{
} &
\multicolumn{2}{c}{
\begin{tabular}{c}
\hyperlink{lspjl}{LSP JupyterLab SW} \\ \hline
\end{tabular}
} \\ \bottomrule
\multicolumn{4}{c}{\textbf{Related Requirements} } \\ \hline
\end{longtable}

     \newpage
\begin{longtable}{p{3.7cm}p{3.7cm}p{3.7cm}p{3.7cm}}\toprule
\multicolumn{2}{l}{\large \textbf{ \hypertarget{jl3}{Jupyterlab} } }
& \multicolumn{2}{l}{(product in: Third Party Libs
)}
\\ \hline
\textbf{\footnotesize Manager} & \textbf{\footnotesize Owner} &
\textbf{\footnotesize WBS} & \textbf{\footnotesize Team} \\ \hline
\parbox{3.5cm}{

\vspace{2mm}%
} &
\begin{tabular}{@{}l@{}}
\parbox{3.5cm}{

\vspace{2mm}%
} \\
\end{tabular} &
\begin{tabular}{@{}l@{}}
 \\
\end{tabular} & \begin{tabular}{@{}l@{}}
 \\
\end{tabular} \\ \hline
\multicolumn{4}{c}{
{\footnotesize ( Jupyterlab
 - JL3 ) }
}\\ \hline
\end{longtable}

An extensible environment for interactive and reproducible computing,
based on the Jupyter Notebook and Architecture.


\begin{longtable}{p{3.7cm}p{3.7cm}p{3.7cm}p{3.7cm}}\hline
\textbf{References} &
\multicolumn{3}{l}{\href{https://github.com/jupyterlab/jupyterlab/}{https://github.com/jupyterlab/jupyterlab/} }
\\ \hline \\ \hline
\textbf{\footnotesize Uses:}  & & \textbf{\footnotesize Used in:} & \\ \hline
\multicolumn{2}{c}{
} &
\multicolumn{2}{c}{
\begin{tabular}{c}
\hyperlink{lspjl}{LSP JupyterLab SW} \\ \hline
\end{tabular}
} \\ \bottomrule
\multicolumn{4}{c}{\textbf{Related Requirements} } \\ \hline
\end{longtable}

    
     \newpage
\subsection{Reference Data Products}\label{refd}
\begin{longtable}{p{3.7cm}p{3.7cm}p{3.7cm}p{3.7cm}}\hline
\textbf{Manager} & \textbf{Owner} & \textbf{WBS} & \textbf{Team} \\ \hline
\parbox{3.5cm}{

\vspace{2mm}%
} &
\begin{tabular}{@{}l@{}}
\parbox{3.5cm}{

\vspace{2mm}%
} \\
\end{tabular}
 &
\begin{tabular}{@{}l@{}}
 \\
\end{tabular} &
\begin{tabular}{@{}l@{}}
 \\
\end{tabular} \\ \hline
\multicolumn{4}{c}{
{\footnotesize ( Ref Data
 - REFD ) }
}\\ \hline
\end{longtable}

  DM reference datasets to be used during operations, and in preparation,
during constructions..


\includegraphics[max width=\linewidth]{subtrees/Main_REFD.pdf}

    \newpage
\begin{longtable}{p{3.7cm}p{3.7cm}p{3.7cm}p{3.7cm}}\toprule
\multicolumn{2}{l}{\large \textbf{ \hypertarget{and}{Astrometry.net Data} } }
& \multicolumn{2}{l}{(product in: Ref Data
)}
\\ \hline
\textbf{\footnotesize Manager} & \textbf{\footnotesize Owner} &
\textbf{\footnotesize WBS} & \textbf{\footnotesize Team} \\ \hline
\parbox{3.5cm}{

\vspace{2mm}%
} &
\begin{tabular}{@{}l@{}}
\parbox{3.5cm}{

\vspace{2mm}%
} \\
\end{tabular} &
\begin{tabular}{@{}l@{}}
 \\
\end{tabular} & \begin{tabular}{@{}l@{}}
 \\
\end{tabular} \\ \hline
\multicolumn{4}{c}{
{\footnotesize ( Astronomy.net Data
 - AND ) }
}\\ \hline
\end{longtable}




\begin{longtable}{p{3.7cm}p{3.7cm}p{3.7cm}p{3.7cm}}\hline
\multicolumn{2}{r}{\textbf{GutHub Packages:}} &
\multicolumn{2}{l}{\href{https://github.com/lsst/astrometry_net_data}{astrometry\_net\_data} }\ref{astrometry_net_data}
\\ \hline \\ \hline
\textbf{\footnotesize Uses:}  & & \textbf{\footnotesize Used in:} & \\ \hline
\multicolumn{2}{c}{
} &
\multicolumn{2}{c}{
\begin{tabular}{c}
\hyperlink{spdist}{Science Pipelines Distribution} \\ \hline
\end{tabular}
} \\ \bottomrule
\multicolumn{4}{c}{\textbf{Related Requirements} } \\ \hline
\end{longtable}

       \newpage
\begin{longtable}{p{3.7cm}p{3.7cm}p{3.7cm}p{3.7cm}}\toprule
\multicolumn{2}{l}{\large \textbf{ \hypertarget{gdata}{Gaia Data} } }
& \multicolumn{2}{l}{(product in: Ref Data
)}
\\ \hline
\textbf{\footnotesize Manager} & \textbf{\footnotesize Owner} &
\textbf{\footnotesize WBS} & \textbf{\footnotesize Team} \\ \hline
\parbox{3.5cm}{

\vspace{2mm}%
} &
\begin{tabular}{@{}l@{}}
\parbox{3.5cm}{

\vspace{2mm}%
} \\
\end{tabular} &
\begin{tabular}{@{}l@{}}
 \\
\end{tabular} & \begin{tabular}{@{}l@{}}
 \\
\end{tabular} \\ \hline
\multicolumn{4}{c}{
{\footnotesize ( Gaia Data
 - GDATA ) }
}\\ \hline
\end{longtable}




\begin{longtable}{p{3.7cm}p{3.7cm}p{3.7cm}p{3.7cm}}\hline
\textbf{\footnotesize Uses:}  & & \textbf{\footnotesize Used in:} & \\ \hline
\multicolumn{2}{c}{
} &
\multicolumn{2}{c}{
} \\ \bottomrule
\multicolumn{4}{c}{\textbf{Related Requirements} } \\ \hline
\end{longtable}

         



\newpage
\section{Supporting products}\label{sec:sups}

This section will list all DM products that are not involved in the operational data processing activities. 
However these products have an important role for construction and maintenance.

These products are organized and maintained in MagicDraw by the DM System Engineering group.
\newpage
\subsection{DM Development and Maintenance products}
% auto generated from MagicDraw (revision) 294 - DO NOT EDIT!
% using template at <template>.
% Collecting data for component: ""
% using docsteady version 
%
% This file is meant to be included in LaTeX document in order to provide:
%   -  MagicDraw Top Level Product Tree (section 2)

\begin{longtable}{p{3.7cm}p{3.7cm}p{3.7cm}p{3.7cm}}\hline
\textbf{Manager} & \textbf{Owner} & \textbf{WBS} & \textbf{Team} \\ \hline
\parbox{3.5cm}{

\vspace{2mm}%
} &
\begin{tabular}{@{}l@{}}
\parbox{3.5cm}{

\vspace{2mm}%
} \\
\end{tabular}
 &
\begin{tabular}{@{}l@{}}
 \\
\end{tabular} &
\begin{tabular}{@{}l@{}}
 \\
\end{tabular} \\ \hline
\multicolumn{4}{c}{
{\footnotesize ( DM Support Products
 - DMDEV ) }
}\\ \hline
\end{longtable}

  

The product tree graph for DMDEV is available
\href{https://DMTN-104.lsst.io/Development_mixedLand.pdf}{ here }.

\newpage
\subsection{Development Service Products}\label{dmdsrv}
\begin{longtable}{p{3.7cm}p{3.7cm}p{3.7cm}p{3.7cm}}\hline
\textbf{Manager} & \textbf{Owner} & \textbf{WBS} & \textbf{Team} \\ \hline
\parbox{3.5cm}{

\vspace{2mm}%
} &
\begin{tabular}{@{}l@{}}
\parbox{3.5cm}{

\vspace{2mm}%
} \\
\end{tabular}
 &
\begin{tabular}{@{}l@{}}
 \\
\end{tabular} &
\begin{tabular}{@{}l@{}}
 \\
\end{tabular} \\ \hline
\multicolumn{4}{c}{
{\footnotesize ( Support Services
 - DMDSRV ) }
}\\ \hline
\end{longtable}

  


\includegraphics[max width=\linewidth]{subtrees/Development_DMDSRV.pdf}

    \newpage
\begin{longtable}{p{3.7cm}p{3.7cm}p{3.7cm}p{3.7cm}}\toprule
\multicolumn{2}{l}{\large \textbf{ \hypertarget{bci}{Build/ CI } } }
& \multicolumn{2}{l}{(product in: Support Services
)}
\\ \hline
\textbf{\footnotesize Manager} & \textbf{\footnotesize Owner} &
\textbf{\footnotesize WBS} & \textbf{\footnotesize Team} \\ \hline
\parbox{3.5cm}{

\vspace{2mm}%
} &
\begin{tabular}{@{}l@{}}
\parbox{3.5cm}{

\vspace{2mm}%
} \\
\end{tabular} &
\begin{tabular}{@{}l@{}}
 \\
\end{tabular} & \begin{tabular}{@{}l@{}}
 \\
\end{tabular} \\ \hline
\multicolumn{4}{c}{
{\footnotesize ( Build and CI
 - BCI ) }
}\\ \hline
\end{longtable}

Build and Continuous Integration services, This service is running the
compilation and the unit test continuously, in order to ensure that
latest changes introduced in the repository do not affect the
compilation nor the functionality. Continuous Integration is not only
exercise in a single software package, but ensure that all packages
together are still building and provides nightly and weekly a working
build as tags in the git repository and as EUPS distribution packages.
Documents are build continuously using the same approach.


\begin{longtable}{p{3.7cm}p{3.7cm}p{3.7cm}p{3.7cm}}\hline
\textbf{\footnotesize Uses:}  & & \textbf{\footnotesize Used in:} & \\ \hline
\multicolumn{2}{c}{
\begin{tabular}{c}
\hyperlink{jnkns}{Jenkins} \\ \hline
\hyperlink{jscr}{jenkins scripting} \\ \hline
\end{tabular}
} &
\multicolumn{2}{c}{
} \\ \bottomrule
\multicolumn{4}{c}{\textbf{Related Requirements} } \\ \hline
\end{longtable}

       \newpage
\begin{longtable}{p{3.7cm}p{3.7cm}p{3.7cm}p{3.7cm}}\toprule
\multicolumn{2}{l}{\large \textbf{ \hypertarget{cam}{Containerized Application Management} } }
& \multicolumn{2}{l}{(product in: Support Services
)}
\\ \hline
\textbf{\footnotesize Manager} & \textbf{\footnotesize Owner} &
\textbf{\footnotesize WBS} & \textbf{\footnotesize Team} \\ \hline
\parbox{3.5cm}{

\vspace{2mm}%
} &
\begin{tabular}{@{}l@{}}
\parbox{3.5cm}{

\vspace{2mm}%
} \\
\end{tabular} &
\begin{tabular}{@{}l@{}}
 \\
\end{tabular} & \begin{tabular}{@{}l@{}}
 \\
\end{tabular} \\ \hline
\multicolumn{4}{c}{
{\footnotesize ( Cont. Application Mng
 - CAM ) }
}\\ \hline
\end{longtable}

To be described.


\begin{longtable}{p{3.7cm}p{3.7cm}p{3.7cm}p{3.7cm}}\hline
\textbf{\footnotesize Uses:}  & & \textbf{\footnotesize Used in:} & \\ \hline
\multicolumn{2}{c}{
\begin{tabular}{c}
\hyperlink{k8s}{Kubernetes} \\ \hline
\hyperlink{docker}{Docker} \\ \hline
\end{tabular}
} &
\multicolumn{2}{c}{
} \\ \bottomrule
\multicolumn{4}{c}{\textbf{Related Requirements} } \\ \hline
\end{longtable}

       \newpage
\begin{longtable}{p{3.7cm}p{3.7cm}p{3.7cm}p{3.7cm}}\toprule
\multicolumn{2}{l}{\large \textbf{ \hypertarget{dmdcom}{Developer Communication Tools} } }
& \multicolumn{2}{l}{(product in: Support Services
)}
\\ \hline
\textbf{\footnotesize Manager} & \textbf{\footnotesize Owner} &
\textbf{\footnotesize WBS} & \textbf{\footnotesize Team} \\ \hline
\parbox{3.5cm}{

\vspace{2mm}%
} &
\begin{tabular}{@{}l@{}}
\parbox{3.5cm}{

\vspace{2mm}%
} \\
\end{tabular} &
\begin{tabular}{@{}l@{}}
 \\
\end{tabular} & \begin{tabular}{@{}l@{}}
 \\
\end{tabular} \\ \hline
\multicolumn{4}{c}{
{\footnotesize ( Devlop. Comm
 - DMDCOM ) }
}\\ \hline
\end{longtable}

Developer Communication Service (SLACK?)


\begin{longtable}{p{3.7cm}p{3.7cm}p{3.7cm}p{3.7cm}}\hline
\textbf{\footnotesize Uses:}  & & \textbf{\footnotesize Used in:} & \\ \hline
\multicolumn{2}{c}{
} &
\multicolumn{2}{c}{
} \\ \bottomrule
\multicolumn{4}{c}{\textbf{Related Requirements} } \\ \hline
\end{longtable}

       \newpage
\begin{longtable}{p{3.7cm}p{3.7cm}p{3.7cm}p{3.7cm}}\toprule
\multicolumn{2}{l}{\large \textbf{ \hypertarget{ddvsrv}{Developer Services} } }
& \multicolumn{2}{l}{(product in: Support Services
)}
\\ \hline
\textbf{\footnotesize Manager} & \textbf{\footnotesize Owner} &
\textbf{\footnotesize WBS} & \textbf{\footnotesize Team} \\ \hline
\parbox{3.5cm}{

\vspace{2mm}%
} &
\begin{tabular}{@{}l@{}}
\parbox{3.5cm}{

\vspace{2mm}%
} \\
\end{tabular} &
\begin{tabular}{@{}l@{}}
 \\
\end{tabular} & \begin{tabular}{@{}l@{}}
 \\
\end{tabular} \\ \hline
\multicolumn{4}{c}{
{\footnotesize ( DM Dev Services
 - DDVSRV ) }
}\\ \hline
\end{longtable}

This service includes all required at NCSA in order to develop and
maintain the DMS components.


\begin{longtable}{p{3.7cm}p{3.7cm}p{3.7cm}p{3.7cm}}\hline
\textbf{\footnotesize Uses:}  & & \textbf{\footnotesize Used in:} & \\ \hline
\multicolumn{2}{c}{
} &
\multicolumn{2}{c}{
\begin{tabular}{c}
\hyperlink{}{Development and Integration E.} \\ \hline
\end{tabular}
} \\ \bottomrule
\multicolumn{4}{c}{\textbf{Related Requirements} } \\ \hline
\end{longtable}

       \newpage
\begin{longtable}{p{3.7cm}p{3.7cm}p{3.7cm}p{3.7cm}}\toprule
\multicolumn{2}{l}{\large \textbf{ \hypertarget{ddcpub}{Documentation Publication} } }
& \multicolumn{2}{l}{(product in: Support Services
)}
\\ \hline
\textbf{\footnotesize Manager} & \textbf{\footnotesize Owner} &
\textbf{\footnotesize WBS} & \textbf{\footnotesize Team} \\ \hline
\parbox{3.5cm}{

\vspace{2mm}%
} &
\begin{tabular}{@{}l@{}}
\parbox{3.5cm}{

\vspace{2mm}%
} \\
\end{tabular} &
\begin{tabular}{@{}l@{}}
 \\
\end{tabular} & \begin{tabular}{@{}l@{}}
 \\
\end{tabular} \\ \hline
\multicolumn{4}{c}{
{\footnotesize ( DM Doc Publication
 - DDCPUB ) }
}\\ \hline
\end{longtable}

Documentation Publication service


\begin{longtable}{p{3.7cm}p{3.7cm}p{3.7cm}p{3.7cm}}\hline
\textbf{\footnotesize Uses:}  & & \textbf{\footnotesize Used in:} & \\ \hline
\multicolumn{2}{c}{
} &
\multicolumn{2}{c}{
} \\ \bottomrule
\multicolumn{4}{c}{\textbf{Related Requirements} } \\ \hline
\end{longtable}

       \newpage
\begin{longtable}{p{3.7cm}p{3.7cm}p{3.7cm}p{3.7cm}}\toprule
\multicolumn{2}{l}{\large \textbf{ \hypertarget{pkgdst}{Packaging/ Distribution} } }
& \multicolumn{2}{l}{(product in: Support Services
)}
\\ \hline
\textbf{\footnotesize Manager} & \textbf{\footnotesize Owner} &
\textbf{\footnotesize WBS} & \textbf{\footnotesize Team} \\ \hline
\parbox{3.5cm}{

\vspace{2mm}%
} &
\begin{tabular}{@{}l@{}}
\parbox{3.5cm}{

\vspace{2mm}%
} \\
\end{tabular} &
\begin{tabular}{@{}l@{}}
 \\
\end{tabular} & \begin{tabular}{@{}l@{}}
 \\
\end{tabular} \\ \hline
\multicolumn{4}{c}{
{\footnotesize ( Packagin Distrib
 - PKGDST ) }
}\\ \hline
\end{longtable}

Packaging and Distribution Service Is not this service duplicated
(partially at least) with SW Deployment Srv.?


\begin{longtable}{p{3.7cm}p{3.7cm}p{3.7cm}p{3.7cm}}\hline
\textbf{\footnotesize Uses:}  & & \textbf{\footnotesize Used in:} & \\ \hline
\multicolumn{2}{c}{
} &
\multicolumn{2}{c}{
} \\ \bottomrule
\multicolumn{4}{c}{\textbf{Related Requirements} } \\ \hline
\end{longtable}

       \newpage
\begin{longtable}{p{3.7cm}p{3.7cm}p{3.7cm}p{3.7cm}}\toprule
\multicolumn{2}{l}{\large \textbf{ \hypertarget{deploy}{SW Deployment} } }
& \multicolumn{2}{l}{(product in: Support Services
)}
\\ \hline
\textbf{\footnotesize Manager} & \textbf{\footnotesize Owner} &
\textbf{\footnotesize WBS} & \textbf{\footnotesize Team} \\ \hline
\parbox{3.5cm}{

\vspace{2mm}%
} &
\begin{tabular}{@{}l@{}}
\parbox{3.5cm}{

\vspace{2mm}%
} \\
\end{tabular} &
\begin{tabular}{@{}l@{}}
 \\
\end{tabular} & \begin{tabular}{@{}l@{}}
 \\
\end{tabular} \\ \hline
\multicolumn{4}{c}{
{\footnotesize ( SW Deployment
 - DEPLOY ) }
}\\ \hline
\end{longtable}

This service provide the ability of DMS to deploy the different software
packages in instantiated services in order to fulfill DMS objectives and
requirements. This imply for example the capability to deploy for the
general production DRP payload the corresponding science pipeline, with
the proper version and configuration in order to process the data and
provide the periodic data release. Different deployment strategies can
be identified depending of the type of software and service. 3rd party
software will be usually deployed manually.


\begin{longtable}{p{3.7cm}p{3.7cm}p{3.7cm}p{3.7cm}}\hline
\textbf{\footnotesize Uses:}  & & \textbf{\footnotesize Used in:} & \\ \hline
\multicolumn{2}{c}{
} &
\multicolumn{2}{c}{
} \\ \bottomrule
\multicolumn{4}{c}{\textbf{Related Requirements} } \\ \hline
\end{longtable}

       \newpage
\begin{longtable}{p{3.7cm}p{3.7cm}p{3.7cm}p{3.7cm}}\toprule
\multicolumn{2}{l}{\large \textbf{ \hypertarget{swver}{SW Version Control} } }
& \multicolumn{2}{l}{(product in: Support Services
)}
\\ \hline
\textbf{\footnotesize Manager} & \textbf{\footnotesize Owner} &
\textbf{\footnotesize WBS} & \textbf{\footnotesize Team} \\ \hline
\parbox{3.5cm}{

\vspace{2mm}%
} &
\begin{tabular}{@{}l@{}}
\parbox{3.5cm}{

\vspace{2mm}%
} \\
\end{tabular} &
\begin{tabular}{@{}l@{}}
 \\
\end{tabular} & \begin{tabular}{@{}l@{}}
 \\
\end{tabular} \\ \hline
\multicolumn{4}{c}{
{\footnotesize ( SW Version Control
 - SWVER ) }
}\\ \hline
\end{longtable}

This service provides software release management in order to obtain
consistent releases for the execution of the main DMS services that will
provide the final data products required.


\begin{longtable}{p{3.7cm}p{3.7cm}p{3.7cm}p{3.7cm}}\hline
\textbf{\footnotesize Uses:}  & & \textbf{\footnotesize Used in:} & \\ \hline
\multicolumn{2}{c}{
\begin{tabular}{c}
\hyperlink{github}{Github} \\ \hline
\end{tabular}
} &
\multicolumn{2}{c}{
} \\ \bottomrule
\multicolumn{4}{c}{\textbf{Related Requirements} } \\ \hline
\end{longtable}

     \newpage
\subsection{Development Software Products}\label{dmdsw}
\begin{longtable}{p{3.7cm}p{3.7cm}p{3.7cm}p{3.7cm}}\hline
\textbf{Manager} & \textbf{Owner} & \textbf{WBS} & \textbf{Team} \\ \hline
\parbox{3.5cm}{

\vspace{2mm}%
} &
\begin{tabular}{@{}l@{}}
\parbox{3.5cm}{

\vspace{2mm}%
} \\
\end{tabular}
 &
\begin{tabular}{@{}l@{}}
 \\
\end{tabular} &
\begin{tabular}{@{}l@{}}
 \\
\end{tabular} \\ \hline
\multicolumn{4}{c}{
{\footnotesize ( Dm Dev Software
 - DMDSW ) }
}\\ \hline
\end{longtable}

  


\includegraphics[max width=\linewidth]{subtrees/Development_DMDSW.pdf}

\newpage
\subsubsection{DevM Supporting SW}\label{dmdss}
\begin{longtable}{p{3.7cm}p{3.7cm}p{3.7cm}p{3.7cm}}\hline
\textbf{Manager} & \textbf{Owner} & \textbf{WBS} & \textbf{Team} \\ \hline
\parbox{3.5cm}{

\vspace{2mm}%
} &
\begin{tabular}{@{}l@{}}
\parbox{3.5cm}{

\vspace{2mm}%
} \\
\end{tabular}
 &
\begin{tabular}{@{}l@{}}
 \\
\end{tabular} &
\begin{tabular}{@{}l@{}}
 \\
\end{tabular} \\ \hline
\multicolumn{4}{c}{
{\footnotesize ( DMDev Supp SW
 - DMDSS ) }
}\\ \hline
\end{longtable}

  


   \newpage
\begin{longtable}{p{3.7cm}p{3.7cm}p{3.7cm}p{3.7cm}}\toprule
\multicolumn{2}{l}{\large \textbf{ \hypertarget{cdkt}{codekit} } }
& \multicolumn{2}{l}{(product in: DMDev Supp SW
)}
\\ \hline
\textbf{\footnotesize Manager} & \textbf{\footnotesize Owner} &
\textbf{\footnotesize WBS} & \textbf{\footnotesize Team} \\ \hline
\parbox{3.5cm}{

\vspace{2mm}%
} &
\begin{tabular}{@{}l@{}}
\parbox{3.5cm}{

\vspace{2mm}%
} \\
\end{tabular} &
\begin{tabular}{@{}l@{}}
 \\
\end{tabular} & \begin{tabular}{@{}l@{}}
 \\
\end{tabular} \\ \hline
\multicolumn{4}{c}{
{\footnotesize ( Codekit
 - CDKT ) }
}\\ \hline
\end{longtable}




\begin{longtable}{p{3.7cm}p{3.7cm}p{3.7cm}p{3.7cm}}\hline
\multicolumn{2}{r}{\textbf{GutHub Packages:}} &
\multicolumn{2}{l}{\href{https://github.com/lsst-sqre/sqre-codekit}{lsst-sqre/sqre-codekit} }
\\ \hline \\ \hline
\textbf{\footnotesize Uses:}  & & \textbf{\footnotesize Used in:} & \\ \hline
\multicolumn{2}{c}{
} &
\multicolumn{2}{c}{
} \\ \bottomrule
\multicolumn{4}{c}{\textbf{Related Requirements} } \\ \hline
\end{longtable}

     \newpage
\begin{longtable}{p{3.7cm}p{3.7cm}p{3.7cm}p{3.7cm}}\toprule
\multicolumn{2}{l}{\large \textbf{ \hypertarget{jscr}{jenkins scripting} } }
& \multicolumn{2}{l}{(product in: DMDev Supp SW
)}
\\ \hline
\textbf{\footnotesize Manager} & \textbf{\footnotesize Owner} &
\textbf{\footnotesize WBS} & \textbf{\footnotesize Team} \\ \hline
\parbox{3.5cm}{

\vspace{2mm}%
} &
\begin{tabular}{@{}l@{}}
\parbox{3.5cm}{

\vspace{2mm}%
} \\
\end{tabular} &
\begin{tabular}{@{}l@{}}
 \\
\end{tabular} & \begin{tabular}{@{}l@{}}
 \\
\end{tabular} \\ \hline
\multicolumn{4}{c}{
{\footnotesize ( JenkinsScripting
 - JSCR ) }
}\\ \hline
\end{longtable}

Scripting implementing the Jenkins jobs


\begin{longtable}{p{3.7cm}p{3.7cm}p{3.7cm}p{3.7cm}}\hline
\multicolumn{2}{r}{\textbf{GutHub Packages:}} &
\multicolumn{2}{l}{\href{https://github.com/lsst-dm/jenkins-dm-jobs}{lsst-dm/jenkins-dm-jobs} }
\\ \hline \\ \hline
\textbf{\footnotesize Uses:}  & & \textbf{\footnotesize Used in:} & \\ \hline
\multicolumn{2}{c}{
} &
\multicolumn{2}{c}{
\begin{tabular}{c}
\hyperlink{bci}{Build/ CI } \\ \hline
\end{tabular}
} \\ \bottomrule
\multicolumn{4}{c}{\textbf{Related Requirements} } \\ \hline
\end{longtable}

     \newpage
\begin{longtable}{p{3.7cm}p{3.7cm}p{3.7cm}p{3.7cm}}\toprule
\multicolumn{2}{l}{\large \textbf{ \hypertarget{lbld}{lsst\_build} } }
& \multicolumn{2}{l}{(product in: DMDev Supp SW
)}
\\ \hline
\textbf{\footnotesize Manager} & \textbf{\footnotesize Owner} &
\textbf{\footnotesize WBS} & \textbf{\footnotesize Team} \\ \hline
\parbox{3.5cm}{

\vspace{2mm}%
} &
\begin{tabular}{@{}l@{}}
\parbox{3.5cm}{

\vspace{2mm}%
} \\
\end{tabular} &
\begin{tabular}{@{}l@{}}
 \\
\end{tabular} & \begin{tabular}{@{}l@{}}
 \\
\end{tabular} \\ \hline
\multicolumn{4}{c}{
{\footnotesize ( lsst\_build
 - LBLD ) }
}\\ \hline
\end{longtable}

Builder and Continuous Integration Tools for LSST.


\begin{longtable}{p{3.7cm}p{3.7cm}p{3.7cm}p{3.7cm}}\hline
\multicolumn{2}{r}{\textbf{GutHub Packages:}} &
\multicolumn{2}{l}{\href{https://github.com/lsst/lsst_build}{lsst\_build} }
\\ \hline \\ \hline
\textbf{\footnotesize Uses:}  & & \textbf{\footnotesize Used in:} & \\ \hline
\multicolumn{2}{c}{
} &
\multicolumn{2}{c}{
} \\ \bottomrule
\multicolumn{4}{c}{\textbf{Related Requirements} } \\ \hline
\end{longtable}

     \newpage
\begin{longtable}{p{3.7cm}p{3.7cm}p{3.7cm}p{3.7cm}}\toprule
\multicolumn{2}{l}{\large \textbf{ \hypertarget{lsstsw}{lsstsw} } }
& \multicolumn{2}{l}{(product in: DMDev Supp SW
)}
\\ \hline
\textbf{\footnotesize Manager} & \textbf{\footnotesize Owner} &
\textbf{\footnotesize WBS} & \textbf{\footnotesize Team} \\ \hline
\parbox{3.5cm}{

\vspace{2mm}%
} &
\begin{tabular}{@{}l@{}}
\parbox{3.5cm}{

\vspace{2mm}%
} \\
\end{tabular} &
\begin{tabular}{@{}l@{}}
 \\
\end{tabular} & \begin{tabular}{@{}l@{}}
 \\
\end{tabular} \\ \hline
\multicolumn{4}{c}{
{\footnotesize ( lsstsw
 - LSSTSW ) }
}\\ \hline
\end{longtable}




\begin{longtable}{p{3.7cm}p{3.7cm}p{3.7cm}p{3.7cm}}\hline
\multicolumn{2}{r}{\textbf{GutHub Packages:}} &
\multicolumn{2}{l}{\href{https://github.com/lsst/lsstsw}{lsstsw} }
\\ \hline \\ \hline
\textbf{\footnotesize Uses:}  & & \textbf{\footnotesize Used in:} & \\ \hline
\multicolumn{2}{c}{
} &
\multicolumn{2}{c}{
} \\ \bottomrule
\multicolumn{4}{c}{\textbf{Related Requirements} } \\ \hline
\end{longtable}

    
   \newpage
\subsubsection{SciencePipelines SW}\label{dsplsw}
\begin{longtable}{p{3.7cm}p{3.7cm}p{3.7cm}p{3.7cm}}\hline
\textbf{Manager} & \textbf{Owner} & \textbf{WBS} & \textbf{Team} \\ \hline
\parbox{3.5cm}{

\vspace{2mm}%
} &
\begin{tabular}{@{}l@{}}
\parbox{3.5cm}{

\vspace{2mm}%
} \\
\end{tabular}
 &
\begin{tabular}{@{}l@{}}
 \\
\end{tabular} &
\begin{tabular}{@{}l@{}}
 \\
\end{tabular} \\ \hline
\multicolumn{4}{c}{
{\footnotesize ( Dev Science Pipelines
 - DSPLSW ) }
}\\ \hline
\end{longtable}

  


   \newpage
\begin{longtable}{p{3.7cm}p{3.7cm}p{3.7cm}p{3.7cm}}\toprule
\multicolumn{2}{l}{\large \textbf{ \hypertarget{splwf}{SPL Workflow} } }
& \multicolumn{2}{l}{(product in: Dev Science Pipelines
)}
\\ \hline
\textbf{\footnotesize Manager} & \textbf{\footnotesize Owner} &
\textbf{\footnotesize WBS} & \textbf{\footnotesize Team} \\ \hline
\parbox{3.5cm}{

\vspace{2mm}%
} &
\begin{tabular}{@{}l@{}}
\parbox{3.5cm}{

\vspace{2mm}%
} \\
\end{tabular} &
\begin{tabular}{@{}l@{}}
 \\
\end{tabular} & \begin{tabular}{@{}l@{}}
 \\
\end{tabular} \\ \hline
\multicolumn{4}{c}{
{\footnotesize ( SPL Workflow
 - SPLWF ) }
}\\ \hline
\end{longtable}




\begin{longtable}{p{3.7cm}p{3.7cm}p{3.7cm}p{3.7cm}}\hline
\textbf{\footnotesize Uses:}  & & \textbf{\footnotesize Used in:} & \\ \hline
\multicolumn{2}{c}{
} &
\multicolumn{2}{c}{
} \\ \bottomrule
\multicolumn{4}{c}{\textbf{Related Requirements} } \\ \hline
\end{longtable}

    
     \newpage
\subsection{Test Data Products<DM Product>>}\label{dmdtdp}
\begin{longtable}{p{3.7cm}p{3.7cm}p{3.7cm}p{3.7cm}}\hline
\textbf{Manager} & \textbf{Owner} & \textbf{WBS} & \textbf{Team} \\ \hline
\parbox{3.5cm}{

\vspace{2mm}%
} &
\begin{tabular}{@{}l@{}}
\parbox{3.5cm}{

\vspace{2mm}%
} \\
\end{tabular}
 &
\begin{tabular}{@{}l@{}}
 \\
\end{tabular} &
\begin{tabular}{@{}l@{}}
 \\
\end{tabular} \\ \hline
\multicolumn{4}{c}{
{\footnotesize ( Test Data
 - DMDTDP ) }
}\\ \hline
\end{longtable}

  


\includegraphics[max width=\linewidth]{subtrees/Development_DMDTDP.pdf}

    \newpage
\begin{longtable}{p{3.7cm}p{3.7cm}p{3.7cm}p{3.7cm}}\toprule
\multicolumn{2}{l}{\large \textbf{ \hypertarget{hscrc1}{HSC-RC1} } }
& \multicolumn{2}{l}{(product in: Test Data
)}
\\ \hline
\textbf{\footnotesize Manager} & \textbf{\footnotesize Owner} &
\textbf{\footnotesize WBS} & \textbf{\footnotesize Team} \\ \hline
\parbox{3.5cm}{

\vspace{2mm}%
} &
\begin{tabular}{@{}l@{}}
\parbox{3.5cm}{

\vspace{2mm}%
} \\
\end{tabular} &
\begin{tabular}{@{}l@{}}
 \\
\end{tabular} & \begin{tabular}{@{}l@{}}
 \\
\end{tabular} \\ \hline
\multicolumn{4}{c}{
{\footnotesize ( HSC-RC1
 - HSCRC1 ) }
}\\ \hline
\end{longtable}

Hyper Suprime-Cam "RC1" This is an example of Dataset that can be
included in this package. To be better characterized.


\begin{longtable}{p{3.7cm}p{3.7cm}p{3.7cm}p{3.7cm}}\hline
\textbf{\footnotesize Uses:}  & & \textbf{\footnotesize Used in:} & \\ \hline
\multicolumn{2}{c}{
} &
\multicolumn{2}{c}{
} \\ \bottomrule
\multicolumn{4}{c}{\textbf{Related Requirements} } \\ \hline
\end{longtable}

       \newpage
\begin{longtable}{p{3.7cm}p{3.7cm}p{3.7cm}p{3.7cm}}\toprule
\multicolumn{2}{l}{\large \textbf{ \hypertarget{spltd}{Science Pipelines Test Data} } }
& \multicolumn{2}{l}{(product in: Test Data
)}
\\ \hline
\textbf{\footnotesize Manager} & \textbf{\footnotesize Owner} &
\textbf{\footnotesize WBS} & \textbf{\footnotesize Team} \\ \hline
\parbox{3.5cm}{

\vspace{2mm}%
} &
\begin{tabular}{@{}l@{}}
\parbox{3.5cm}{

\vspace{2mm}%
} \\
\end{tabular} &
\begin{tabular}{@{}l@{}}
 \\
\end{tabular} & \begin{tabular}{@{}l@{}}
 \\
\end{tabular} \\ \hline
\multicolumn{4}{c}{
{\footnotesize ( SPL Test Data
 - SPLTD ) }
}\\ \hline
\end{longtable}




\begin{longtable}{p{3.7cm}p{3.7cm}p{3.7cm}p{3.7cm}}\hline
\textbf{\footnotesize Uses:}  & & \textbf{\footnotesize Used in:} & \\ \hline
\multicolumn{2}{c}{
} &
\multicolumn{2}{c}{
} \\ \bottomrule
\multicolumn{4}{c}{\textbf{Related Requirements} } \\ \hline
\end{longtable}

     \newpage
\subsection{Development COTS and Environments}\label{dmdc3e}
\begin{longtable}{p{3.7cm}p{3.7cm}p{3.7cm}p{3.7cm}}\hline
\textbf{Manager} & \textbf{Owner} & \textbf{WBS} & \textbf{Team} \\ \hline
\parbox{3.5cm}{

\vspace{2mm}%
} &
\begin{tabular}{@{}l@{}}
\parbox{3.5cm}{

\vspace{2mm}%
} \\
\end{tabular}
 &
\begin{tabular}{@{}l@{}}
 \\
\end{tabular} &
\begin{tabular}{@{}l@{}}
 \\
\end{tabular} \\ \hline
\multicolumn{4}{c}{
{\footnotesize ( COTS 3rdPLs ENVs
 - DMDC3E ) }
}\\ \hline
\end{longtable}

  


\includegraphics[max width=\linewidth]{subtrees/Development_DMDC3E.pdf}

\newpage
\subsubsection{DevM COTS}\label{cotsdm}
\begin{longtable}{p{3.7cm}p{3.7cm}p{3.7cm}p{3.7cm}}\hline
\textbf{Manager} & \textbf{Owner} & \textbf{WBS} & \textbf{Team} \\ \hline
\parbox{3.5cm}{

\vspace{2mm}%
} &
\begin{tabular}{@{}l@{}}
\parbox{3.5cm}{

\vspace{2mm}%
} \\
\end{tabular}
 &
\begin{tabular}{@{}l@{}}
 \\
\end{tabular} &
\begin{tabular}{@{}l@{}}
 \\
\end{tabular} \\ \hline
\multicolumn{4}{c}{
{\footnotesize ( DM DevM COTS
 - COTSDM ) }
}\\ \hline
\end{longtable}

  


   \newpage
\begin{longtable}{p{3.7cm}p{3.7cm}p{3.7cm}p{3.7cm}}\toprule
\multicolumn{2}{l}{\large \textbf{ \hypertarget{eups}{EUPS} } }
& \multicolumn{2}{l}{(product in: DM DevM COTS
)}
\\ \hline
\textbf{\footnotesize Manager} & \textbf{\footnotesize Owner} &
\textbf{\footnotesize WBS} & \textbf{\footnotesize Team} \\ \hline
\parbox{3.5cm}{

\vspace{2mm}%
} &
\begin{tabular}{@{}l@{}}
\parbox{3.5cm}{

\vspace{2mm}%
} \\
\end{tabular} &
\begin{tabular}{@{}l@{}}
 \\
\end{tabular} & \begin{tabular}{@{}l@{}}
 \\
\end{tabular} \\ \hline
\multicolumn{4}{c}{
{\footnotesize ( EUPS
 - EUPS ) }
}\\ \hline
\end{longtable}




\begin{longtable}{p{3.7cm}p{3.7cm}p{3.7cm}p{3.7cm}}\hline
\textbf{\footnotesize Uses:}  & & \textbf{\footnotesize Used in:} & \\ \hline
\multicolumn{2}{c}{
} &
\multicolumn{2}{c}{
} \\ \bottomrule
\multicolumn{4}{c}{\textbf{Related Requirements} } \\ \hline
\end{longtable}

     \newpage
\begin{longtable}{p{3.7cm}p{3.7cm}p{3.7cm}p{3.7cm}}\toprule
\multicolumn{2}{l}{\large \textbf{ \hypertarget{gcc}{GCC} } }
& \multicolumn{2}{l}{(product in: DM DevM COTS
)}
\\ \hline
\textbf{\footnotesize Manager} & \textbf{\footnotesize Owner} &
\textbf{\footnotesize WBS} & \textbf{\footnotesize Team} \\ \hline
\parbox{3.5cm}{

\vspace{2mm}%
} &
\begin{tabular}{@{}l@{}}
\parbox{3.5cm}{

\vspace{2mm}%
} \\
\end{tabular} &
\begin{tabular}{@{}l@{}}
 \\
\end{tabular} & \begin{tabular}{@{}l@{}}
 \\
\end{tabular} \\ \hline
\multicolumn{4}{c}{
{\footnotesize ( GCC
 - GCC ) }
}\\ \hline
\end{longtable}




\begin{longtable}{p{3.7cm}p{3.7cm}p{3.7cm}p{3.7cm}}\hline
\textbf{\footnotesize Uses:}  & & \textbf{\footnotesize Used in:} & \\ \hline
\multicolumn{2}{c}{
} &
\multicolumn{2}{c}{
} \\ \bottomrule
\multicolumn{4}{c}{\textbf{Related Requirements} } \\ \hline
\end{longtable}

     \newpage
\begin{longtable}{p{3.7cm}p{3.7cm}p{3.7cm}p{3.7cm}}\toprule
\multicolumn{2}{l}{\large \textbf{ \hypertarget{github}{Github} } }
& \multicolumn{2}{l}{(product in: DM DevM COTS
)}
\\ \hline
\textbf{\footnotesize Manager} & \textbf{\footnotesize Owner} &
\textbf{\footnotesize WBS} & \textbf{\footnotesize Team} \\ \hline
\parbox{3.5cm}{

\vspace{2mm}%
} &
\begin{tabular}{@{}l@{}}
\parbox{3.5cm}{

\vspace{2mm}%
} \\
\end{tabular} &
\begin{tabular}{@{}l@{}}
 \\
\end{tabular} & \begin{tabular}{@{}l@{}}
 \\
\end{tabular} \\ \hline
\multicolumn{4}{c}{
{\footnotesize ( Github
 - GITHUB ) }
}\\ \hline
\end{longtable}




\begin{longtable}{p{3.7cm}p{3.7cm}p{3.7cm}p{3.7cm}}\hline
\textbf{\footnotesize Uses:}  & & \textbf{\footnotesize Used in:} & \\ \hline
\multicolumn{2}{c}{
} &
\multicolumn{2}{c}{
\begin{tabular}{c}
\hyperlink{swver}{SW Version Control} \\ \hline
\end{tabular}
} \\ \bottomrule
\multicolumn{4}{c}{\textbf{Related Requirements} } \\ \hline
\end{longtable}

     \newpage
\begin{longtable}{p{3.7cm}p{3.7cm}p{3.7cm}p{3.7cm}}\toprule
\multicolumn{2}{l}{\large \textbf{ \hypertarget{jnkns}{Jenkins} } }
& \multicolumn{2}{l}{(product in: DM DevM COTS
)}
\\ \hline
\textbf{\footnotesize Manager} & \textbf{\footnotesize Owner} &
\textbf{\footnotesize WBS} & \textbf{\footnotesize Team} \\ \hline
\parbox{3.5cm}{

\vspace{2mm}%
} &
\begin{tabular}{@{}l@{}}
\parbox{3.5cm}{

\vspace{2mm}%
} \\
\end{tabular} &
\begin{tabular}{@{}l@{}}
 \\
\end{tabular} & \begin{tabular}{@{}l@{}}
 \\
\end{tabular} \\ \hline
\multicolumn{4}{c}{
{\footnotesize ( Jenkins
 - JNKNS ) }
}\\ \hline
\end{longtable}




\begin{longtable}{p{3.7cm}p{3.7cm}p{3.7cm}p{3.7cm}}\hline
\textbf{\footnotesize Uses:}  & & \textbf{\footnotesize Used in:} & \\ \hline
\multicolumn{2}{c}{
} &
\multicolumn{2}{c}{
\begin{tabular}{c}
\hyperlink{bci}{Build/ CI } \\ \hline
\end{tabular}
} \\ \bottomrule
\multicolumn{4}{c}{\textbf{Related Requirements} } \\ \hline
{\footnotesize  } &
\multicolumn{3}{p{11.1cm}}{\footnotesize Build/ CI  } \\ \cdashline{1-4}
\end{longtable}

     \newpage
\begin{longtable}{p{3.7cm}p{3.7cm}p{3.7cm}p{3.7cm}}\toprule
\multicolumn{2}{l}{\large \textbf{ \hypertarget{jra}{JIRA} } }
& \multicolumn{2}{l}{(product in: DM DevM COTS
)}
\\ \hline
\textbf{\footnotesize Manager} & \textbf{\footnotesize Owner} &
\textbf{\footnotesize WBS} & \textbf{\footnotesize Team} \\ \hline
\parbox{3.5cm}{

\vspace{2mm}%
} &
\begin{tabular}{@{}l@{}}
\parbox{3.5cm}{

\vspace{2mm}%
} \\
\end{tabular} &
\begin{tabular}{@{}l@{}}
 \\
\end{tabular} & \begin{tabular}{@{}l@{}}
 \\
\end{tabular} \\ \hline
\multicolumn{4}{c}{
{\footnotesize ( Jira
 - JRA ) }
}\\ \hline
\end{longtable}




\begin{longtable}{p{3.7cm}p{3.7cm}p{3.7cm}p{3.7cm}}\hline
\textbf{\footnotesize Uses:}  & & \textbf{\footnotesize Used in:} & \\ \hline
\multicolumn{2}{c}{
} &
\multicolumn{2}{c}{
\begin{tabular}{c}
\hyperlink{}{Issue Tracking} \\ \hline
\end{tabular}
} \\ \bottomrule
\multicolumn{4}{c}{\textbf{Related Requirements} } \\ \hline
\end{longtable}

     \newpage
\begin{longtable}{p{3.7cm}p{3.7cm}p{3.7cm}p{3.7cm}}\toprule
\multicolumn{2}{l}{\large \textbf{ \hypertarget{trvs}{Travis} } }
& \multicolumn{2}{l}{(product in: DM DevM COTS
)}
\\ \hline
\textbf{\footnotesize Manager} & \textbf{\footnotesize Owner} &
\textbf{\footnotesize WBS} & \textbf{\footnotesize Team} \\ \hline
\parbox{3.5cm}{

\vspace{2mm}%
} &
\begin{tabular}{@{}l@{}}
\parbox{3.5cm}{

\vspace{2mm}%
} \\
\end{tabular} &
\begin{tabular}{@{}l@{}}
 \\
\end{tabular} & \begin{tabular}{@{}l@{}}
 \\
\end{tabular} \\ \hline
\multicolumn{4}{c}{
{\footnotesize ( Travis
 - TRVS ) }
}\\ \hline
\end{longtable}




\begin{longtable}{p{3.7cm}p{3.7cm}p{3.7cm}p{3.7cm}}\hline
\textbf{\footnotesize Uses:}  & & \textbf{\footnotesize Used in:} & \\ \hline
\multicolumn{2}{c}{
} &
\multicolumn{2}{c}{
} \\ \bottomrule
\multicolumn{4}{c}{\textbf{Related Requirements} } \\ \hline
\end{longtable}

    
   \newpage
\subsubsection{DevM Environments}\label{denv}
\begin{longtable}{p{3.7cm}p{3.7cm}p{3.7cm}p{3.7cm}}\hline
\textbf{Manager} & \textbf{Owner} & \textbf{WBS} & \textbf{Team} \\ \hline
\parbox{3.5cm}{

\vspace{2mm}%
} &
\begin{tabular}{@{}l@{}}
\parbox{3.5cm}{

\vspace{2mm}%
} \\
\end{tabular}
 &
\begin{tabular}{@{}l@{}}
 \\
\end{tabular} &
\begin{tabular}{@{}l@{}}
 \\
\end{tabular} \\ \hline
\multicolumn{4}{c}{
{\footnotesize ( DM DevM Envs
 - DENV ) }
}\\ \hline
\end{longtable}

  


   \newpage
\begin{longtable}{p{3.7cm}p{3.7cm}p{3.7cm}p{3.7cm}}\toprule
\multicolumn{2}{l}{\large \textbf{ \hypertarget{splce}{SciencePipelines Conda Env.} } }
& \multicolumn{2}{l}{(product in: DM DevM Envs
)}
\\ \hline
\textbf{\footnotesize Manager} & \textbf{\footnotesize Owner} &
\textbf{\footnotesize WBS} & \textbf{\footnotesize Team} \\ \hline
\parbox{3.5cm}{

\vspace{2mm}%
} &
\begin{tabular}{@{}l@{}}
\parbox{3.5cm}{

\vspace{2mm}%
} \\
\end{tabular} &
\begin{tabular}{@{}l@{}}
 \\
\end{tabular} & \begin{tabular}{@{}l@{}}
 \\
\end{tabular} \\ \hline
\multicolumn{4}{c}{
{\footnotesize ( SPL Conda Env
 - SPLCE ) }
}\\ \hline
\end{longtable}




\begin{longtable}{p{3.7cm}p{3.7cm}p{3.7cm}p{3.7cm}}\hline
\multicolumn{2}{r}{\textbf{GutHub Packages:}} &
\multicolumn{2}{l}{\href{https://github.com/lsst/scipipe_conda_env}{scipipe\_conda\_env} }
\\ \hline \\ \hline
\textbf{\footnotesize Uses:}  & & \textbf{\footnotesize Used in:} & \\ \hline
\multicolumn{2}{c}{
} &
\multicolumn{2}{c}{
} \\ \bottomrule
\multicolumn{4}{c}{\textbf{Related Requirements} } \\ \hline
\end{longtable}

    
         



\newpage
\section{GitHub Packages}\label{sec:low}

This section lists the GitHub packages related to the DM products listed in previous sections \ref{sec:top} and \ref{sec:sups}.
The detailed information is extracted from GitHub.

The information provided includes, when available, the list of dependencies tracted from the ups table file.
Note that the release procedure as described in \citeds{DMTN-106} 
can be applied to a software product only if all dependencies 
are not used in other software products. If this is not the case, only one of these software products can be released.

As it can be evinced by a quick inspection in the following subsections, 
all Science Pipelines software products share a large number of dependencies. 
Therefore, the only releasebale software product, at the time of writing (July 2020) 
is the Science Pipeline distribution.


% do not edit, generated automatically from Github

\newpage
\subsection{nublado}\label{lsst-sqre/nublado}

JupyterLab + JupyterHub + k8s deployment used by LSST for its Science
Platform


{\footnotesize
\begin{longtable}{rl}
\hline
Open it in GitHUb: & \href{https://github.com/lsst-sqre/nublado}{https://github.com/lsst-sqre/nublado} \\ \cdashline{1-2}
Top Level Component: & \hyperlink{nblsrv}{LSP Nublado} \\
\hline
\end{longtable} }


README.md (First 20 lines only)
{\scriptsize
\begin{lstlisting}[breaklines]
# LSST Science Platform Notebook Aspect

## Do Not Use This

If what you want to do is simply deploy a Jupyter setup under Kubernetes
you're much better off
using
[Zero to JupyterHub](https://zero-to-jupyterhub.readthedocs.io/en/latest/),
which is an excellent general tutorial for setting up
JupyterHub in a Kubernetes environment.

This cluster is much more specifically tailored to the needs
of [LSST](https://lsst.org).  If you want an example of how to set up
persistent storage for your users, how to ship logs to a remote ELK
stack, a worked example of how to subclass a spawner, or how to use an
image-spawner options menu, you may find it useful.

## Overview

The LSST Science Platform Notebook Aspect is a JupyterHub + JupyterLab
\end{lstlisting}
}


\newpage
\subsection{lsst-tap-service}\label{lsst-sqre/lsst-tap-service}

IVOA TAP service for LSST


{\footnotesize
\begin{longtable}{rl}
\hline
Open it in GitHUb: & \href{https://github.com/lsst-sqre/lsst-tap-service}{https://github.com/lsst-sqre/lsst-tap-service} \\ \cdashline{1-2}
Top Level Component: & \hyperlink{tapsrv}{TAP API} \\
\hline
\end{longtable} }


README.md (First 20 lines only)
{\scriptsize
\begin{lstlisting}[breaklines]
# LSST TAP Service

This repository contains the LSST TAP service.  It is based on the CADC TAP service
code and uses this as a dependency, and then adds special logic to work with QServ.

## Build

Run ./build.sh

## Deployment

### Docker
After the [Build](#build) step above, a set of containers with the `dev` tag will exist
on your local machine.  Then when you run:

`docker-compose up -d && ./waitForContainersReady.sh && ./checkAvailability.sh`

This should start a local group of containers, wait for them to be ready, and then
check that the availability endpoint returns a 200 and a simple sync query works.
This validates that your local TAP implementation is working.  You can now either
\end{lstlisting}
}


\newpage
\subsection{davt}\label{davt}

WebDAV with substitute user impersonation per-request


{\footnotesize
\begin{longtable}{rl}
\hline
Open it in GitHUb: & \href{https://github.com/lsst/davt}{https://github.com/lsst/davt} \\ \cdashline{1-2}
Top Level Component: & \hyperlink{wdavsrv}{WebDAV API} \\
\cdashline{1-2}
{GitHub Teams:} &
 Overlords \\
 & Data Management \\
 & Database \\
\hline
\end{longtable} }


README.md (First 20 lines only)
{\scriptsize
\begin{lstlisting}[breaklines]
# davt

`davt` is a lua module for nginx to aid with impersonation. Its target use case is for use with 
WebDAV, so that all operations are executed _as the user in the request_. 

For every incoming request, davt enables nginx to switch the OS user (with `setfsuid`) and/or 
group IDs/supplementary group IDs (via `setfsgid`, `setgroups`, `initgroups`) to match the 
authenticated user or specific groups before performing any file opertions.

As davt enables impersonation, a few properties follow:

* The files do NOT need to be owned by an nginx service account user, nor does an ACL need to be 
modified to allow for access to an service group (for filesystems supporting ACLs). This allows 
you to transperently operate the service over existing directories.

* Ownership when creating files is preserved for the files in question. This ensures that files 
created for the user via WebDAV are also readable when the user is in a shell, for example.

## Requirements
davt requires ljsyscall. It also used the ffi library from LuaJIT.
\end{lstlisting}
}


\newpage
\subsection{HeaderService}\label{lsst-dm/headerservice}

LSST Meta-data aggregator for FITS header service


{\footnotesize
\begin{longtable}{rl}
\hline
Open it in GitHUb: & \href{https://github.com/lsst-dm/HeaderService}{https://github.com/lsst-dm/HeaderService} \\ \cdashline{1-2}
Top Level Component: & \hyperlink{header}{Header Service SW} \\
\hline
\end{longtable} }

\begin{longtable}{rl}
\multicolumn{2}{c}{EUPS dependencies} \\ \hline
\textbf{name} & \textbf{description} \\ \hline
{\footnotesize fitsio } & \\ \hline
\end{longtable}

README.md (First 20 lines only)
{\scriptsize
\begin{lstlisting}[breaklines]
# HeaderService

Development for LSST Meta-data FITS header service

Description
-----------

This is the development for the LSST Meta-data FITS header client. It
uses a set of FITS header library templates and DDS/SAL Python-based
communication layer to populate meta-data and command the header
client to write header files.

Requirements
------------
+ numpy
+ astropy
+ fitsio (https://github.com/esheldon/fitsio)
+ salobj
+ OpenSplice compiled binaries for centOS7
+ A CentOS7 VM or docker container
\end{lstlisting}
}


\newpage
\subsection{ctrl\_oods}\label{lsst-dm/ctrl_oods}

Observatory Operations Data Service


{\footnotesize
\begin{longtable}{rl}
\hline
Open it in GitHUb: & \href{https://github.com/lsst-dm/ctrl_oods}{https://github.com/lsst-dm/ctrl\_oods} \\ \cdashline{1-2}
Top Level Component: & \hyperlink{oods}{Observatory Operations Data Service SW} \\
\hline
\end{longtable} }

\begin{longtable}{rl}
\multicolumn{2}{c}{EUPS dependencies} \\ \hline
\textbf{name} & \textbf{description} \\ \hline
{\footnotesize base } & \\ \hline
{\footnotesize utils } & \\ \hline
{\footnotesize sconsUtils } & \\ \hline
{\footnotesize obs\_lsst } & \\ \hline
\end{longtable}

README.rst (First 20 lines only)
{\scriptsize
\begin{lstlisting}[breaklines]
#########
ctrl_oods
#########

``ctrl_oods`` is an `LDF Prompt Enclave Software`_ package.

.. Add a brief (few sentence) description of what this package provides.

The Observatory Operations Data Service watches for files in one or more directories, and then ingests them into an LSST Butler repository.   
Files are expired from the repository at specified intervals.
\end{lstlisting}
}


\newpage
\subsection{ctrl\_iip}\label{ctrl_iip}

Image ingest and processing


{\footnotesize
\begin{longtable}{rl}
\hline
Open it in GitHUb: & \href{https://github.com/lsst/ctrl_iip}{https://github.com/lsst/ctrl\_iip} \\ \cdashline{1-2}
Top Level Component: & \hyperlink{iip}{Image Ingest and Processing} \\
\cdashline{1-2}
{GitHub Teams:} &
 Data Management \\
\hline
\end{longtable} }


README.md (First 20 lines only)
{\scriptsize
\begin{lstlisting}[breaklines]
# ctrl_iip
Image ingest and processing

environment variables:

Set CTRL_IIP_DIR to the root of this repository. (this will be set automatically
when this is integrated with the DM system)

Set PYTHONPATH to include $CTRL_IPP_DIR/python



Note about configuration files:

Configuration files are loaded from $CTRL_IIP_DIR/etc/config by default

If the environment variable IIP_CONFIG_DIR is set, it will look in 
this directory for configuration files.
\end{lstlisting}
}


\newpage
\subsection{alert\_stream}\label{lsst-dm/alert_stream}

Mock alert stream distribution system using Kafka producers and
consumers.


{\footnotesize
\begin{longtable}{rl}
\hline
Open it in GitHUb: & \href{https://github.com/lsst-dm/alert_stream}{https://github.com/lsst-dm/alert\_stream} \\ \cdashline{1-2}
Top Level Component: & \hyperlink{alrtdstr}{Alert Distribution SW} \\
\hline
\end{longtable} }


README.rst (First 20 lines only)
{\scriptsize
\begin{lstlisting}[breaklines]
############
alert_stream
############

This package provides a demonstration of the LSST Alert Distribution Service.
The Alert Distribution Service provides a mechanism for rapidly disseminating and filtering notifications of transient and variable sources observed by LSST.
The service is described in detail in `DMTN-093`_.
This repository provides instructions, code and sample data for creating, filtering, and consuming alert streams following the conventions we expect to adopt for LSST.
It should be used in conjunction with the `lsst-dm/sample-avro-alert`_ repository, which provides details of, and code for working with, LSST alert packets.

.. _DMTN-093: https://dmtn-093.lsst.io/
.. _lsst-dm/sample-avro-alert: https://github.com/lsst-dm/sample-avro-alert

Prerequisites
=============

- Cloning this repository requires `Git LFS`_ (Large File Storage) support.
  Refer to the `DM Developer Guide`_ for more information.
- `Docker`_ and `Docker Compose`_ are required to create and manage the services within this repository, including running all of the examples below.

\end{lstlisting}
}


\newpage
\subsection{squash}\label{lsst-sqre/squash}

SQuaSH web interface


{\footnotesize
\begin{longtable}{rl}
\hline
Open it in GitHUb: & \href{https://github.com/lsst-sqre/squash}{https://github.com/lsst-sqre/squash} \\ \cdashline{1-2}
Top Level Component: & \hyperlink{qcsw}{Quality Control SW} \\
\hline
\end{longtable} }


README.md (First 20 lines only)
{\scriptsize
\begin{lstlisting}[breaklines]
# squash

`squash` is the web frontend to embed the bokeh apps and navigate through them. You can learn more about SQuaSH at [SQR-009](https://sqr-009.lsst.io).

[![Build Status](https://travis-ci.org/lsst-sqre/squash.svg?branch=master)](https://travis-ci.org/lsst-sqre/squash)

## Requirements

The `squash` web frontend requires the [squash-restful-api](https://github.com/lsst-sqre/squash-restful-api) and [squash-bokeh](https://github.com/lsst-sqre/squash-bokeh) microservices, and the TLS certificats that are installed by the
[`squash-deployment`](https://github.com/lsst-sqre/squash-deployment).

## Kubernetes deployment

You can provision a Kubernetes cluster in GKE, clone this repo and deploy the `squash` microservice using:

```
cd squash
TAG=latest make service deployment
```

\end{lstlisting}
}


\newpage
\subsection{dbb\_gwclient}\label{lsst-dm/dbb_gwclient}

Prototype code to save raw files to Data Backbone Gateway


{\footnotesize
\begin{longtable}{rl}
\hline
Open it in GitHUb: & \href{https://github.com/lsst-dm/dbb_gwclient}{https://github.com/lsst-dm/dbb\_gwclient} \\ \cdashline{1-2}
Top Level Component: & \hyperlink{dbbmd}{DBB Ingest/ Metadata Management SW} \\
\hline
\end{longtable} }


README.md (First 20 lines only)
{\scriptsize
\begin{lstlisting}[breaklines]
# dbb_gwclient
Prototype code to save raw files to Data Backbone Gateway.

This is a Python 3 only package (we assume Python 3.6 or higher).
\end{lstlisting}
}


\newpage
\subsection{dbb\_gateway}\label{lsst-dm/dbb_gateway}

Prototype code that ingests into the Data Backbone raw files delivered
by the dbb\_gwclient to the DBB gateway


{\footnotesize
\begin{longtable}{rl}
\hline
Open it in GitHUb: & \href{https://github.com/lsst-dm/dbb_gateway}{https://github.com/lsst-dm/dbb\_gateway} \\ \cdashline{1-2}
Top Level Component: & \hyperlink{dbbmd}{DBB Ingest/ Metadata Management SW} \\
\hline
\end{longtable} }

\begin{longtable}{rl}
\multicolumn{2}{c}{EUPS dependencies} \\ \hline
\textbf{name} & \textbf{description} \\ \hline
{\footnotesize pyfits } & \\ \hline
\end{longtable}

README.md (First 20 lines only)
{\scriptsize
\begin{lstlisting}[breaklines]
# dbb_gateway
Prototype code that ingests into the Data Backbone raw files delivered by the dbb_gwclient to the DBB gateway
\end{lstlisting}
}


\newpage
\subsection{suit}\label{suit}



{\footnotesize
\begin{longtable}{rl}
\hline
Open it in GitHUb: & \href{https://github.com/lsst/suit}{https://github.com/lsst/suit} \\ \cdashline{1-2}
Top Level Component: & \hyperlink{prtlsw}{LSP Portal Software} \\
\cdashline{1-2}
{GitHub Teams:} &
 Data Management \\
\hline
\end{longtable} }


README.md (First 20 lines only)
{\scriptsize
\begin{lstlisting}[breaklines]
# SUIT 


## Description
The SUIT (Science User Interface and Tools) repository contains applications built on the Firefly Toolkit.
It is meant to be used with [Firefly](https://github.com/Caltech-IPAC/firefly).

The principal current application is "suit", otherwise known as the Portal Aspect application, which
contains both the Portal search screens and visualization capabilities, and the "slate" endpoint that
is used for Python-based image and table visualizations.


## Build Instuctions
 
 - Install [JDK 8](http://www.oracle.com/technetwork/java/javase/downloads/jdk8-downloads-2133151.html)   
   
 - Install [Gradle 4.x](https://gradle.org/install/)
 
 - Install [Node.js 8.x](https://nodejs.org/en/download/)
 
\end{lstlisting}
}


\newpage
\subsection{jupyterhubutils}\label{lsst-sqre/jupyterhubutils}

Utilities for LSST LSP notebook environment (Hub/spawner side)


{\footnotesize
\begin{longtable}{rl}
\hline
Open it in GitHUb: & \href{https://github.com/lsst-sqre/jupyterhubutils}{https://github.com/lsst-sqre/jupyterhubutils} \\ \cdashline{1-2}
Top Level Component: & \hyperlink{nbsw}{LSP Notebook Software} \\
\hline
\end{longtable} }


README.md (First 20 lines only)
{\scriptsize
\begin{lstlisting}[breaklines]
# Utilities for LSST LSP notebook environment (Hub/spawner side)
\end{lstlisting}
}


\newpage
\subsection{jupyterlabutils}\label{lsst-sqre/jupyterlabutils}

Utilities for JupyterLab containers in LSST Science Platform environment


{\footnotesize
\begin{longtable}{rl}
\hline
Open it in GitHUb: & \href{https://github.com/lsst-sqre/jupyterlabutils}{https://github.com/lsst-sqre/jupyterlabutils} \\ \cdashline{1-2}
Top Level Component: & \hyperlink{nbsw}{LSP Notebook Software} \\
\hline
\end{longtable} }


README.md (First 20 lines only)
{\scriptsize
\begin{lstlisting}[breaklines]
# Utilities for LSST LSP Science Platform notebook aspect (user pod side)
\end{lstlisting}
}


\newpage
\subsection{dax\_webserv}\label{dax_webserv}

Web Interface for LSST Data Access Services


{\footnotesize
\begin{longtable}{rl}
\hline
Open it in GitHUb: & \href{https://github.com/lsst/dax_webserv}{https://github.com/lsst/dax\_webserv} \\ \cdashline{1-2}
Top Level Component: & \hyperlink{lspweb}{LSP Web API SW (obsolete product)} \\
\cdashline{1-2}
{GitHub Teams:} &
 Overlords \\
 & Data Management \\
 & Database \\
\hline
\end{longtable} }

\begin{longtable}{rl}
\multicolumn{2}{c}{EUPS dependencies} \\ \hline
\textbf{name} & \textbf{description} \\ \hline
{\footnotesize db } & \\ \hline
{\footnotesize flask } & \\ \hline
{\footnotesize dax\_dbserv } & \\ \hline
{\footnotesize dax\_imgserv } & \\ \hline
{\footnotesize dax\_metaserv } & \\ \hline
{\footnotesize python\_mysqlclient } & \\ \hline
\end{longtable}

README.txt (First 20 lines only)
{\scriptsize
\begin{lstlisting}[breaklines]
# Useful link:
http://blog.miguelgrinberg.com/post/designing-a-restful-api-with-python-and-flask
http://pycoder.net/bospy/presentation.html

# To install flask:
sudo aptitude install python-flask


# An example Tap query to dbserv (if running locally)
  curl -d 'query=SELECT+ra,decl,filterName+FROM+DC_W13_Stripe82.Science_Ccd_Exposure+WHERE+scienceCcdExposureId=125230127' http://localhost:5000/db/v0/sync
\end{lstlisting}
}


\newpage
\subsection{suit-onlinehelp}\label{suit-onlinehelp}



{\footnotesize
\begin{longtable}{rl}
\hline
Open it in GitHUb: & \href{https://github.com/lsst/suit-onlinehelp}{https://github.com/lsst/suit-onlinehelp} \\ \cdashline{1-2}
Top Level Component: & \hyperlink{prtloh}{LSP Portal Online Help} \\
\hline
\end{longtable} }


README.md (First 20 lines only)
{\scriptsize
\begin{lstlisting}[breaklines]
# suit-onlinehelp

Prerequisites
-------------
    - gradle v2.2+
    - clone onlinehelp repository and its dependent repository
        - git clone https://github.com/lsst/suit-onlinehelp
        - git clone https://github.com/Caltech-IPAC/firefly


Build and Install Individually
------------------------------
- cd suit-onlinehelp
- gradle :<project_name>:build      // build only
    - creates an archive of html and supporting files to be install to a webserver
    - the file is placed in ./build/libs/

- gradle :<project_name>:install    // build and install.
    - crates and install online help files
    - HTML_DOC_ROOT environment variable is required to locate the path to the webserver's document root.
\end{lstlisting}
}


\newpage
\subsection{dax\_imgserv}\label{dax_imgserv}

Web Interface for LSST Image Services


{\footnotesize
\begin{longtable}{rl}
\hline
Open it in GitHUb: & \href{https://github.com/lsst/dax_imgserv}{https://github.com/lsst/dax\_imgserv} \\ \cdashline{1-2}
Top Level Component: & \hyperlink{daximg}{Image/ Cutout Server} \\
\cdashline{1-2}
{GitHub Teams:} &
 Overlords \\
 & Data Management \\
 & Database \\
\hline
\end{longtable} }

\begin{longtable}{rl}
\multicolumn{2}{c}{EUPS dependencies} \\ \hline
\textbf{name} & \textbf{description} \\ \hline
{\footnotesize afw } & \\ \hline
{\footnotesize base } & \\ \hline
{\footnotesize boost } & \\ \hline
{\footnotesize daf\_base } & \\ \hline
{\footnotesize daf\_persistence } & \\ \hline
{\footnotesize doxygen } & \\ \hline
{\footnotesize log } & \\ \hline
{\footnotesize obs\_sdss } & \\ \hline
{\footnotesize scons } & \\ \hline
{\footnotesize sconsUtils } & \\ \hline
{\footnotesize skymap } & \\ \hline
{\footnotesize sqlalchemy } & \\ \hline
\end{longtable}

README.txt (First 20 lines only)
{\scriptsize
\begin{lstlisting}[breaklines]
# Useful link:
http://blog.miguelgrinberg.com/post/designing-a-restful-api-with-python-and-flask

# To install flask:
sudo aptitude install python-flask

# To run some quick tests:

  # run the server
  python bin/imageServer.py

  # and fetch the urls:
  http://localhost:5000/api/image/soda/availability
  http://localhost:5000/api/image/soda/capabilities
  http://localhost:5000/api/image/soda/examples
  http://localhost:5000/api/image/soda/sync?ID=DC_W13_Stripe82.calexp.r&POS=CIRCLE+37.644598+0.104625+100
  http://localhost:5000/api/image/soda/sync?ID=DC_W13_Stripe82.calexp.r&POS=RANGE+37.616820222+37.67235778+0.07684722222+0.132402777
  http://localhost:5000/api/image/soda/sync?ID=DC_W13_Stripe82.calexp.r&POS=POLYGON+37.6580803+0.0897081+37.6580803+0.1217858+37.6186104+0.1006648
  http://localhost:5000/api/image/soda/sync?ID=DC_W13_Stripe82.calexp.r&POS=BRECT+37.644598+0.104625+100+100+pixel
\end{lstlisting}
}


\newpage
\subsection{ap\_pipe}\label{ap_pipe}

LSST Data Management Alert Production Pipeline


{\footnotesize
\begin{longtable}{rl}
\hline
Open it in GitHUb: & \href{https://github.com/lsst/ap_pipe}{https://github.com/lsst/ap\_pipe} \\ \cdashline{1-2}
Top Level Component: & \hyperlink{apprmpt}{Alert Production} \\
\cdashline{1-2}
{GitHub Teams:} &
 Overlords \\
 & Data Management \\
\hline
\end{longtable} }

\begin{longtable}{rl}
\multicolumn{2}{c}{EUPS dependencies} \\ \hline
\textbf{name} & \textbf{description} \\ \hline
{\footnotesize utils } & \\ \hline
{\footnotesize pex\_config } & \\ \hline
{\footnotesize pipe\_base } & \\ \hline
{\footnotesize pipe\_tasks } & \\ \hline
{\footnotesize ap\_association } & \\ \hline
\end{longtable}

README.md (First 20 lines only)
{\scriptsize
\begin{lstlisting}[breaklines]
# ap_pipe

This package contains the LSST Data Management Alert Production Pipeline.

For up-to-date documentation, including a tutorial, see the `doc` directory.

ap_pipe processes raw images that have been ingested into a Butler repository
with corresponding calibration products and templates. It produces calexps,
difference images and source catalogs, and an association database.

The user must specify the main repository with ingested images (and the
location of the calibration products and templates if they reside elsewhere),
the name of the association database (may be either created from scratch or
connected to for continued associating), and a Butler data ID.
\end{lstlisting}
}


\newpage
\subsection{cp\_pipe}\label{cp_pipe}

Calibration-products production pipeline


{\footnotesize
\begin{longtable}{rl}
\hline
Open it in GitHUb: & \href{https://github.com/lsst/cp_pipe}{https://github.com/lsst/cp\_pipe} \\ \cdashline{1-2}
Top Level Component: & \hyperlink{dmcal}{Calibration Software} \\
\cdashline{1-2}
{GitHub Teams:} &
 Overlords \\
 & Data Management \\
 & DMLT \\
\hline
\end{longtable} }

\begin{longtable}{rl}
\multicolumn{2}{c}{EUPS dependencies} \\ \hline
\textbf{name} & \textbf{description} \\ \hline
{\footnotesize pex\_config } & \\ \hline
{\footnotesize pipe\_base } & \\ \hline
{\footnotesize log } & \\ \hline
{\footnotesize ip\_isr } & \\ \hline
{\footnotesize afw } & \\ \hline
{\footnotesize meas\_algorithms } & \\ \hline
\end{longtable}

README.rst (First 20 lines only)
{\scriptsize
\begin{lstlisting}[breaklines]
#######################################
Calibration Products Production Package
#######################################

Code to produce calibration products, required to perform ISR and other calibration tasks.
\end{lstlisting}
}


\newpage
\subsection{mops\_daymops}\label{mops_daymops}



{\footnotesize
\begin{longtable}{rl}
\hline
Open it in GitHUb: & \href{https://github.com/lsst/mops_daymops}{https://github.com/lsst/mops\_daymops} \\ \cdashline{1-2}
Top Level Component: & \hyperlink{mops}{MOPS and Forced Photometry} \\
\cdashline{1-2}
{GitHub Teams:} &
 Overlords \\
 & Data Management \\
\hline
\end{longtable} }

\begin{longtable}{rl}
\multicolumn{2}{c}{EUPS dependencies} \\ \hline
\textbf{name} & \textbf{description} \\ \hline
{\footnotesize "scons" } & \\ \hline
{\footnotesize "swig" } & \\ \hline
{\footnotesize "gsl" } & \\ \hline
{\footnotesize "daf\_base" } & \\ \hline
{\footnotesize pex\_exceptions } & \\ \hline
{\footnotesize "slalib" } & \\ \hline
{\footnotesize "eigen" >=3.0.0 } & \\ \hline
{\footnotesize "mysqlpython" } & \\ \hline
{\footnotesize "numpy" } & \\ \hline
\end{longtable}

README.quickstart.txt (First 20 lines only)
{\scriptsize
\begin{lstlisting}[breaklines]
Here's a greatly simplified guide to running MOPS at the moment, which
will run findTracklets, collapseTracklets linkTracklets for you.

I was using Bash when I came up with these, you may need to change a
few things if you're using *csh.

# set up your environment
setlsst
setup mysqlpython
setup mops_daymops
export MOPS_HACKS=$MOPS_DAYMOPS_DIR/tests/experimentScripts/

# get data
mkdir myMopsRun
cd myMopsRun
wget --user=USER --password=PASSWORD dias_pt1_nodeep.short.astromErr


# populate the DB for later. I assume you have the OpSim DB already.
echo "CREATE DATABASE myMops; USE myMops; `cat fullerDiaSource.sql`;" | mysql
\end{lstlisting}
}
README.txt (First 20 lines only)
{\scriptsize
\begin{lstlisting}[breaklines]
Jmyers Oct 22

Updated thoroughly to describe how I'm currently doing things.

The following is a set of instructions for running
find/collapse/linkTracklets on some diaSources. 

In the future these scripts (or more likely, better versions of
all of this) will be modified so that pipelines can run each
stage of find/collapse/linkTracklets on particular sets of data.

All the scripts should be in the same directory as this readme file.



INSTALLING/BUILDING C++ FIND/LINKTRACKLETS (etc.)
---------------------------------------

Build the C++ tools using instructions online at http://dev.lsstcorp.org/trac/wiki/MOPS/Installing_MOPS

\end{lstlisting}
}


\newpage
\subsection{lsst\_distrib}\label{lsst_distrib}



{\footnotesize
\begin{longtable}{rl}
\hline
Open it in GitHUb: & \href{https://github.com/lsst/lsst_distrib}{https://github.com/lsst/lsst\_distrib} \\ \cdashline{1-2}
Top Level Component: & \hyperlink{spdist}{Science Pipelines Distribution} \\
\cdashline{1-2}
{GitHub Teams:} &
 Overlords \\
 & Data Management \\
\hline
\end{longtable} }

\begin{longtable}{rl}
\multicolumn{2}{c}{EUPS dependencies} \\ \hline
\textbf{name} & \textbf{description} \\ \hline
{\footnotesize lsst\_apps } & \\ \hline
{\footnotesize ctrl\_execute } & \\ \hline
{\footnotesize ctrl\_mpexec } & \\ \hline
{\footnotesize ctrl\_platform\_lsstvc } & \\ \hline
{\footnotesize jointcal } & \\ \hline
{\footnotesize verify } & \\ \hline
{\footnotesize ap\_verify } & \\ \hline
{\footnotesize display\_firefly } & \\ \hline
{\footnotesize display\_matplotlib } & \\ \hline
{\footnotesize cp\_pipe } & \\ \hline
{\footnotesize validate\_drp } & \\ \hline
{\footnotesize fgcmcal } & \\ \hline
\end{longtable}



\newpage
\subsection{lsst\_apps}\label{lsst_apps}



{\footnotesize
\begin{longtable}{rl}
\hline
Open it in GitHUb: & \href{https://github.com/lsst/lsst_apps}{https://github.com/lsst/lsst\_apps} \\ \cdashline{1-2}
Top Level Component: & \hyperlink{scipipe}{Science Pipelines Libraries} \\
\cdashline{1-2}
{GitHub Teams:} &
 Overlords \\
 & Data Management \\
\hline
\end{longtable} }

\begin{longtable}{rl}
\multicolumn{2}{c}{EUPS dependencies} \\ \hline
\textbf{name} & \textbf{description} \\ \hline
{\footnotesize meas\_deblender } & \\ \hline
{\footnotesize meas\_modelfit } & \\ \hline
{\footnotesize pipe\_tasks } & \\ \hline
{\footnotesize ap\_pipe } & \\ \hline
{\footnotesize obs\_sdss } & \\ \hline
{\footnotesize obs\_test } & \\ \hline
{\footnotesize meas\_extensions\_simpleShape } & \\ \hline
\end{longtable}

README.md (First 20 lines only)
{\scriptsize
\begin{lstlisting}[breaklines]
# lsst_apps

This is a metapackage providing a minimalist set of [LSST Science Pipelines](https://pipelines.lsst.io) packages. This is a subset of the full [lsst_distrib](https://github.com/lsst/lsst_distrib) package. This dependency list is tracked via [eups](https://github.com/RobertLuptonTheGood/eups) in the `ups/lsst_apps.table` file.
\end{lstlisting}
}


\newpage
\subsection{daf\_butler}\label{daf_butler}

Prototype for data access framework described in \citeds{DMTN-056}

{\footnotesize
\begin{longtable}{rl}
\hline
Open it in GitHUb: & \href{https://github.com/lsst/daf_butler}{https://github.com/lsst/daf\_butler} \\ \cdashline{1-2}
Top Level Component: & \hyperlink{butler}{Data Butler} \\
\cdashline{1-2}
{GitHub Teams:} &
 Overlords \\
 & Data Management \\
 & Database \\
\hline
\end{longtable} }

\begin{longtable}{rl}
\multicolumn{2}{c}{EUPS dependencies} \\ \hline
\textbf{name} & \textbf{description} \\ \hline
{\footnotesize sphgeom } & \\ \hline
{\footnotesize sconsUtils } & \\ \hline
{\footnotesize utils } & \\ \hline
{\footnotesize astro\_metadata\_translator } & \\ \hline
{\footnotesize geom } & \\ \hline
{\footnotesize afw } & \\ \hline
\end{longtable}

README.md (First 20 lines only)
{\scriptsize
\begin{lstlisting}[breaklines]
# daf_butler

LSST Data Access framework described in [DMTN-056](https://dmtn-056.lsst.io).

This is a **Python 3 only** package (we assume Python 3.6 or higher).
\end{lstlisting}
}


\newpage
\subsection{pipe\_supertask}\label{pipe_supertask}

Super Task implementation


{\footnotesize
\begin{longtable}{rl}
\hline
Open it in GitHUb: & \href{https://github.com/lsst/pipe_supertask}{https://github.com/lsst/pipe\_supertask} \\ \cdashline{1-2}
Top Level Component: & \hyperlink{txf}{Task Framework} \\
\hline
\end{longtable} }

\begin{longtable}{rl}
\multicolumn{2}{c}{EUPS dependencies} \\ \hline
\textbf{name} & \textbf{description} \\ \hline
{\footnotesize daf\_butler } & \\ \hline
{\footnotesize log } & \\ \hline
{\footnotesize pex\_config } & \\ \hline
{\footnotesize pipe\_base } & \\ \hline
\end{longtable}

README.md (First 20 lines only)
{\scriptsize
\begin{lstlisting}[breaklines]
# pipe_supertask
Super Task implementation
\end{lstlisting}
}


\newpage
\subsection{qserv}\label{qserv}

LSST Query Services


{\footnotesize
\begin{longtable}{rl}
\hline
Open it in GitHUb: & \href{https://github.com/lsst/qserv}{https://github.com/lsst/qserv} \\ \cdashline{1-2}
Top Level Component: & \hyperlink{qserv}{Distributed Database} \\
\cdashline{1-2}
{GitHub Teams:} &
 Overlords \\
 & Data Management \\
\hline
\end{longtable} }

\begin{longtable}{rl}
\multicolumn{2}{c}{EUPS dependencies} \\ \hline
\textbf{name} & \textbf{description} \\ \hline
{\footnotesize antlr4 } & \\ \hline
{\footnotesize boost } & \\ \hline
{\footnotesize db } & \\ \hline
{\footnotesize doxygen } & \\ \hline
{\footnotesize flask } & \\ \hline
{\footnotesize jemalloc } & \\ \hline
{\footnotesize libcurl } & \\ \hline
{\footnotesize log } & \\ \hline
{\footnotesize log4cxx } & \\ \hline
{\footnotesize lua } & \\ \hline
{\footnotesize mariadb } & \\ \hline
{\footnotesize mysqlproxy } & \\ \hline
{\footnotesize json\_nlohmann } & \\ \hline
{\footnotesize python\_mysqlclient } & \\ \hline
{\footnotesize partition } & \\ \hline
{\footnotesize protobuf } & \\ \hline
{\footnotesize pybind11 } & \\ \hline
{\footnotesize python } & \\ \hline
{\footnotesize redis\_plus\_plus } & \\ \hline
{\footnotesize requests } & \\ \hline
{\footnotesize scisql } & \\ \hline
{\footnotesize scons } & \\ \hline
{\footnotesize sphgeom } & \\ \hline
{\footnotesize sqlalchemy } & \\ \hline
{\footnotesize xrootd } & \\ \hline
\end{longtable}

README.md (First 20 lines only)
{\scriptsize
\begin{lstlisting}[breaklines]
# Qserv: petascale distributed database

## Master branch status

Continuous integration server launches Qserv build and also multi-node integration tests:

[![Build Status](https://travis-ci.org/lsst/qserv.svg?branch=master)](https://travis-ci.org/lsst/qserv)

[![Code Climate](https://codeclimate.com/github/lsst/qserv/badges/gpa.svg)](https://codeclimate.com/github/lsst/qserv)

[![Issue Count](https://codeclimate.com/github/lsst/qserv/badges/issue_count.svg)](https://codeclimate.com/github/lsst/qserv)

## Documentation

### Current release


See http://slac.stanford.edu/exp/lsst/qserv/

### Development version
\end{lstlisting}
}


\newpage
\subsection{albuquery}\label{albuquery}

DAX Query Services in Kotlin


{\footnotesize
\begin{longtable}{rl}
\hline
Open it in GitHUb: & \href{https://github.com/lsst/albuquery}{https://github.com/lsst/albuquery} \\ \cdashline{1-2}
Top Level Component: & \hyperlink{adql}{ADQL Translator} \\
\hline
\end{longtable} }


README.rst (First 20 lines only)
{\scriptsize
\begin{lstlisting}[breaklines]
#########
albuquery
#########

``albuquery`` will implement a TAP database query service for the Web APIs Aspect of the LSST Science Platform (a.k.a. Data Access Services/DAX).
\end{lstlisting}
}


\newpage
\subsection{scipipe\_conda\_env}\label{scipipe_conda_env}

Conda environment for LSST Science Pipelines


{\footnotesize
\begin{longtable}{rl}
\hline
Open it in GitHUb: & \href{https://github.com/lsst/scipipe_conda_env}{https://github.com/lsst/scipipe\_conda\_env} \\ \cdashline{1-2}
Top Level Component: & \hyperlink{splce}{SciencePipelines Conda Env.} \\
\cdashline{1-2}
{GitHub Teams:} &
 Data Management \\
 & DM Auxilliaries \\
\hline
\end{longtable} }


README.md (First 20 lines only)
{\scriptsize
\begin{lstlisting}[breaklines]
# Conda Environment for Science Pipelines

This repository contains the definition of the Conda environment used by the LSST Science Pipelines.

## Contents

Files in the `etc` directory are named following the pattern:

```
conda3_<filetype>-<platform>-64.yml
```

Where `<filetype>` is one of:

- `bleed`, indicating the names of packages on which the Science Pipelines directly depend, or
- `packages`, indicating a specific versioned set of those packages, and packages upon which they depend, which can be directly instantiated as a Conda environment.

And `<platform>` is one of:

- `linux`, indicating that this file has been tested on CentOS (our reference platform), and, by extension, is appropriate for use on a Linux systems;
\end{lstlisting}
}


\newpage
\subsection{astrometry\_net\_data}\label{astrometry_net_data}



{\footnotesize
\begin{longtable}{rl}
\hline
Open it in GitHUb: & \href{https://github.com/lsst/astrometry_net_data}{https://github.com/lsst/astrometry\_net\_data} \\ \cdashline{1-2}
Top Level Component: & \hyperlink{and}{Astrometry.net Data} \\
\cdashline{1-2}
{GitHub Teams:} &
 Overlords \\
 & Data Management \\
 & DM Externals \\
\hline
\end{longtable} }

\begin{longtable}{rl}
\multicolumn{2}{c}{EUPS dependencies} \\ \hline
\textbf{name} & \textbf{description} \\ \hline
{\footnotesize sconsUtils } & \\ \hline
\end{longtable}



\newpage
\subsection{lsst\_build}\label{lsst_build}



{\footnotesize
\begin{longtable}{rl}
\hline
Open it in GitHUb: & \href{https://github.com/lsst/lsst_build}{https://github.com/lsst/lsst\_build} \\ \cdashline{1-2}
Top Level Component: & \hyperlink{lbld}{lsst\_build} \\
\cdashline{1-2}
{GitHub Teams:} &
 Overlords \\
 & Data Management \\
 & DM Auxilliaries \\
\hline
\end{longtable} }


README.md (First 20 lines only)
{\scriptsize
\begin{lstlisting}[breaklines]
lsst-build, a builder and continuous integration tool for LSST
==============================================================

[![Build Status](https://travis-ci.org/lsst/lsst_build.svg?branch=master)](https://travis-ci.org/lsst/lsst_build)

Provides the following capabilities:

* Given one or more top-level packages, intelligently clone their git
  repositories and check out the requested branches into a build directory:

  ```bash
  lsst-build prepare
     [--repository-pattern=format_pattern_for_repo_URLs]
     [--exclusion-map=exclusions.txt]
     [--version-git-repo=versiondbdir]
     [--ref=branch1 [--ref=branch2 [...]]]
     <builddir> <product1> [product2 [product3 [...]]]
  ```

  Run `lsst-build prepare -h` to see the full list of options.
\end{lstlisting}
}


\newpage
\subsection{jenkins-dm-jobs}\label{lsst-dm/jenkins-dm-jobs}

Jenkins jobs and pipeline scripts for LSST DM


{\footnotesize
\begin{longtable}{rl}
\hline
Open it in GitHUb: & \href{https://github.com/lsst-dm/jenkins-dm-jobs}{https://github.com/lsst-dm/jenkins-dm-jobs} \\ \cdashline{1-2}
Top Level Component: & \hyperlink{jscr}{jenkins scripting} \\
\hline
\end{longtable} }


README.md (First 20 lines only)
{\scriptsize
\begin{lstlisting}[breaklines]
jenkins-dm-jobs
===

[![Build Status](https://travis-ci.org/lsst-dm/jenkins-dm-jobs.png)](https://travis-ci.org/lsst-dm/jenkins-dm-jobs)
\end{lstlisting}
}


\newpage
\subsection{sqre-codekit}\label{lsst-sqre/sqre-codekit}

LSST DM SQuaRE misc. code management tools


{\footnotesize
\begin{longtable}{rl}
\hline
Open it in GitHUb: & \href{https://github.com/lsst-sqre/sqre-codekit}{https://github.com/lsst-sqre/sqre-codekit} \\ \cdashline{1-2}
Top Level Component: & \hyperlink{cdkt}{codekit} \\
\hline
\end{longtable} }


README.md (First 20 lines only)
{\scriptsize
\begin{lstlisting}[breaklines]
[![Build Status](https://travis-ci.org/lsst-sqre/sqre-codekit.svg?branch=master)](https://travis-ci.org/lsst-sqre/sqre-codekit)

# sqre-codekit

LSST DM SQuaRE misc. code management tools

## Installation

sqre-codekit runs on Python 3.6 or newer. You can install it with

```bash
pip install sqre-codekit
```

## Available commands

- `github-auth`: Generate a GitHub authentication token.
- `github-decimate-org`: Delete repos and/or teams from a GitHub organization.
- `github-fork-org`: Fork repositories from one GitHub organization to another.
- `github-get-ratelimit`: Display the current github ReST API request ratelimit.
\end{lstlisting}
}


\newpage
\subsection{lsstsw}\label{lsstsw}

loadLSST


{\footnotesize
\begin{longtable}{rl}
\hline
Open it in GitHUb: & \href{https://github.com/lsst/lsstsw}{https://github.com/lsst/lsstsw} \\ \cdashline{1-2}
Top Level Component: & \hyperlink{lsstsw}{lsstsw} \\
\cdashline{1-2}
{GitHub Teams:} &
 Overlords \\
 & Data Management \\
\hline
\end{longtable} }


README.md (First 20 lines only)
{\scriptsize
\begin{lstlisting}[breaklines]
LSST Distribution Server Account
================================

[![Build Status](https://travis-ci.org/lsst/lsstsw.png)](https://travis-ci.org/lsst/lsstsw)

**`repos.yaml` has been migrated to [`lsst/repos`](https://github.com/lsst/repos).**

For a guide to using `lsstsw`, see:

http://developer.lsst.io/en/latest/build-ci/lsstsw.html

*Note: this directory is git managed.*

Structure
---------

| path       | description                                                    |
| :----------|:---------------------------------------------------------------|
| miniconda  | Anaconda Python distribution                                   |
| bin        | software distribution binaries (rebuild, publish)              |
\end{lstlisting}
}






\newpage
\section{Non DM Products}\label{sec:nondm}

This section will list non DM productsi provided by other Rubin/LSST subsystems,  that are relevant in order to fulfill DMS requirements.


\newpage
\section{External Products}\label{sec:externals}

% auto-generated from MagicDraw (revision) 346 - DO NOT EDIT!
% using template at <template>.
% Collecting data for component: ""
% using docsteady version 
%
% This file is meant to be included in a LaTeX document.

\begin{longtable}{p{3.7cm}p{3.7cm}p{3.7cm}p{3.7cm}}\hline
\textbf{Manager} & \textbf{Owner} & \textbf{WBS} & \textbf{Team} \\ \hline
\parbox{3.5cm}{

\vspace{2mm}%
} &
\begin{tabular}{@{}l@{}}
\parbox{3.5cm}{

\vspace{2mm}%
} \\
\end{tabular}
 &
\begin{tabular}{@{}l@{}}
 \\
\end{tabular} &
\begin{tabular}{@{}l@{}}
 \\
\end{tabular} \\ \hline
\multicolumn{4}{c}{
{\footnotesize Short name: \textbf{ Externals
 } - Product key: \textbf{ EXT } }
}\\ \hline
\end{longtable}

  In these area of the product tree are listed those components non
provided by LSST/Rubin but relevant for DM.

~

The products are organized in the following sections:

\begin{itemize}
\tightlist
\item
  resource products (hardware, COTS and external provided softwares)
\item
  data products (reference data for development and operations)
\end{itemize}

~


\newpage
\subsection{Hardware and COTS Products}\label{hwcots}
\begin{longtable}{p{3.7cm}p{3.7cm}p{3.7cm}p{3.7cm}}\hline
\textbf{Manager} & \textbf{Owner} & \textbf{WBS} & \textbf{Team} \\ \hline
\parbox{3.5cm}{

\vspace{2mm}%
} &
\begin{tabular}{@{}l@{}}
\parbox{3.5cm}{

\vspace{2mm}%
} \\
\end{tabular}
 &
\begin{tabular}{@{}l@{}}
 \\
\end{tabular} &
\begin{tabular}{@{}l@{}}
 \\
\end{tabular} \\ \hline
\multicolumn{4}{c}{
{\footnotesize Short name: \textbf{ Resources
 } - Product key: \textbf{ HWCOTS } }
}\\ \hline
\end{longtable}

  External resources needed to implement the DM services:

\begin{itemize}
\tightlist
\item
  Hardware (computing hardware, network hardware, etc)
\item
  COTS
\item
  External software (not developed in DM)
\end{itemize}


\newpage
\subsubsection{Hardware}\label{dmhw}
\begin{longtable}{p{3.7cm}p{3.7cm}p{3.7cm}p{3.7cm}}\hline
\textbf{Manager} & \textbf{Owner} & \textbf{WBS} & \textbf{Team} \\ \hline
\parbox{3.5cm}{

\vspace{2mm}%
} &
\begin{tabular}{@{}l@{}}
\parbox{3.5cm}{

\vspace{2mm}%
} \\
\end{tabular}
 &
\begin{tabular}{@{}l@{}}
 \\
\end{tabular} &
\begin{tabular}{@{}l@{}}
 \\
\end{tabular} \\ \hline
\multicolumn{4}{c}{
{\footnotesize Short name: \textbf{ DM Hardware
 } - Product key: \textbf{ DMHW } }
}\\ \hline
\end{longtable}

  This section of the product tree shall list the hardware components used
by DM.

Considering the actual cloud oriented approach in terms of
infrastructure, this section is kept here for reference, and will be
used only in case a specific hardware component will show up in the
future. Since subsections are empty, there is no need to be reviewed by
any manager or owner until specific components are listed in them.~

No requirements are traced here.


{\footnotesize
\textbf{Products included in this section}:
\begin{itemize}
\item Compute Nodes - \hyperlink{hwcomp}{HWCOMP}
\item Storage Nodes - \hyperlink{hwstor}{HWSTOR}
\item Network Nodes - \hyperlink{hwnet}{HWNET}
\end{itemize}
}

   \newpage
\begin{longtable}{p{3.7cm}p{3.7cm}p{3.7cm}p{3.7cm}}\toprule
\multicolumn{2}{l}{\large \textbf{ \hypertarget{hwcomp}{Compute Nodes} } }
& \multicolumn{2}{l}{(product in: DM Hardware
)}
\\ \hline
\textbf{\footnotesize Manager} & \textbf{\footnotesize Owner} &
\textbf{\footnotesize WBS} & \textbf{\footnotesize Team} \\ \hline
\parbox{3.5cm}{

\vspace{2mm}%
} &
\begin{tabular}{@{}l@{}}
\parbox{3.5cm}{

\vspace{2mm}%
} \\
\end{tabular} &
\begin{tabular}{@{}l@{}}
 \\
\end{tabular} & \begin{tabular}{@{}l@{}}
 \\
\end{tabular} \\ \hline
\multicolumn{4}{c}{
{\footnotesize Short name: \textbf{ Compute Nodes
 } - Product key: \textbf{ HWCOMP } }
}\\ \hline
\end{longtable}

Computational elements. Details will be added here on a per need base.


\begin{longtable}{p{3.7cm}p{3.7cm}p{3.7cm}p{3.7cm}}\hline
\textbf{\footnotesize Uses:}  & & \textbf{\footnotesize Used in:} & \\ \hline
\multicolumn{2}{c}{
} &
\multicolumn{2}{c}{
} \\ \bottomrule
\end{longtable}

     \newpage
\begin{longtable}{p{3.7cm}p{3.7cm}p{3.7cm}p{3.7cm}}\toprule
\multicolumn{2}{l}{\large \textbf{ \hypertarget{hwstor}{Storage Nodes} } }
& \multicolumn{2}{l}{(product in: DM Hardware
)}
\\ \hline
\textbf{\footnotesize Manager} & \textbf{\footnotesize Owner} &
\textbf{\footnotesize WBS} & \textbf{\footnotesize Team} \\ \hline
\parbox{3.5cm}{

\vspace{2mm}%
} &
\begin{tabular}{@{}l@{}}
\parbox{3.5cm}{

\vspace{2mm}%
} \\
\end{tabular} &
\begin{tabular}{@{}l@{}}
 \\
\end{tabular} & \begin{tabular}{@{}l@{}}
 \\
\end{tabular} \\ \hline
\multicolumn{4}{c}{
{\footnotesize Short name: \textbf{ Storage Nodes
 } - Product key: \textbf{ HWSTOR } }
}\\ \hline
\end{longtable}

Storage components to host data. Details will be added here on a per
need base.


\begin{longtable}{p{3.7cm}p{3.7cm}p{3.7cm}p{3.7cm}}\hline
\textbf{\footnotesize Uses:}  & & \textbf{\footnotesize Used in:} & \\ \hline
\multicolumn{2}{c}{
} &
\multicolumn{2}{c}{
} \\ \bottomrule
\end{longtable}

     \newpage
\begin{longtable}{p{3.7cm}p{3.7cm}p{3.7cm}p{3.7cm}}\toprule
\multicolumn{2}{l}{\large \textbf{ \hypertarget{hwnet}{Network Nodes} } }
& \multicolumn{2}{l}{(product in: DM Hardware
)}
\\ \hline
\textbf{\footnotesize Manager} & \textbf{\footnotesize Owner} &
\textbf{\footnotesize WBS} & \textbf{\footnotesize Team} \\ \hline
\parbox{3.5cm}{
Jeff Kantor
\vspace{2mm}%
} &
\begin{tabular}{@{}l@{}}
\parbox{3.5cm}{

\vspace{2mm}%
} \\
\end{tabular} &
\begin{tabular}{@{}l@{}}
 \\
\end{tabular} & \begin{tabular}{@{}l@{}}
 \\
\end{tabular} \\ \hline
\multicolumn{4}{c}{
{\footnotesize Short name: \textbf{ Network Nodes
 } - Product key: \textbf{ HWNET } }
}\\ \hline
\end{longtable}

The network components required in order to implement local and long
distance networks. Details will be added here on a per need base.


\begin{longtable}{p{3.7cm}p{3.7cm}p{3.7cm}p{3.7cm}}\hline
\textbf{\footnotesize Uses:}  & & \textbf{\footnotesize Used in:} & \\ \hline
\multicolumn{2}{c}{
} &
\multicolumn{2}{c}{
} \\ \bottomrule
\end{longtable}

    
   \newpage
\subsubsection{COTS}\label{cots}
\begin{longtable}{p{3.7cm}p{3.7cm}p{3.7cm}p{3.7cm}}\hline
\textbf{Manager} & \textbf{Owner} & \textbf{WBS} & \textbf{Team} \\ \hline
\parbox{3.5cm}{

\vspace{2mm}%
} &
\begin{tabular}{@{}l@{}}
\parbox{3.5cm}{

\vspace{2mm}%
} \\
\end{tabular}
 &
\begin{tabular}{@{}l@{}}
 \\
\end{tabular} &
\begin{tabular}{@{}l@{}}
 \\
\end{tabular} \\ \hline
\multicolumn{4}{c}{
{\footnotesize Short name: \textbf{ COTS SW
 } - Product key: \textbf{ COTS } }
}\\ \hline
\end{longtable}

  This section includes all COTS used to implement the DM services.\\
These are not DM~s directly, but are listed here for reference. In many
cases, no owner nor manager is specified.\\
~\\


{\footnotesize
\textbf{Products included in this section}:
\begin{itemize}
\item CILogon - \hyperlink{cilogon}{CILOGON}
\item Docker - \hyperlink{docker}{DOCKER}
\item Firefly - \hyperlink{firefly}{FIREFLY}
\item General Parallel File System - \hyperlink{gpfs}{GPFS}
\item Grafana - \hyperlink{grafana}{GRAFANA}
\item HTCondor - \hyperlink{htcondor}{HTCONDOR}
\item Jupyterhub - \hyperlink{jh3}{JH3}
\item Jupyterlab - \hyperlink{jl3}{JL3}
\item JIRA - \hyperlink{jra}{JRA}
\item Kubernetes - \hyperlink{k8s}{K8S}
\item PostgreSQL - \hyperlink{psql}{PSQL}
\item Python - \hyperlink{pth}{PTH}
\item Puppet - \hyperlink{puppet}{PUPPET}
\item Rucio - \hyperlink{rucio}{RUCIO}
\item IT Security SW - \hyperlink{security}{SECURITY}
\item vSphere - \hyperlink{vsphere}{VSPHERE}
\end{itemize}
}

   \newpage
\begin{longtable}{p{3.7cm}p{3.7cm}p{3.7cm}p{3.7cm}}\toprule
\multicolumn{2}{l}{\large \textbf{ \hypertarget{cilogon}{CILogon} } }
& \multicolumn{2}{l}{(product in: COTS SW
)}
\\ \hline
\textbf{\footnotesize Manager} & \textbf{\footnotesize Owner} &
\textbf{\footnotesize WBS} & \textbf{\footnotesize Team} \\ \hline
\parbox{3.5cm}{

\vspace{2mm}%
} &
\begin{tabular}{@{}l@{}}
\parbox{3.5cm}{

\vspace{2mm}%
} \\
\end{tabular} &
\begin{tabular}{@{}l@{}}
 \\
\end{tabular} & \begin{tabular}{@{}l@{}}
 \\
\end{tabular} \\ \hline
\multicolumn{4}{c}{
{\footnotesize Short name: \textbf{ CILogon
 } - Product key: \textbf{ CILOGON } }
}\\ \hline
\end{longtable}

CILogon provides an integrated open source identity and access
management platform for research collaborations.


\begin{longtable}{p{3.7cm}p{3.7cm}p{3.7cm}p{3.7cm}}\hline
\textbf{References} &
\multicolumn{3}{l}{\href{http://www.cilogon.org/}{http://www.cilogon.org/} }
\\ \hline \\ \hline
\textbf{\footnotesize Uses:}  & & \textbf{\footnotesize Used in:} & \\ \hline
\multicolumn{2}{c}{
} &
\multicolumn{2}{c}{
\begin{tabular}{c}
\hyperlink{idnmng}{Identity Management} \\ \hline
\hyperlink{ncsaidnmng}{NCSA Identity Management} \\ \hline
\end{tabular}
} \\ \bottomrule
\end{longtable}

     \newpage
\begin{longtable}{p{3.7cm}p{3.7cm}p{3.7cm}p{3.7cm}}\toprule
\multicolumn{2}{l}{\large \textbf{ \hypertarget{docker}{Docker} } }
& \multicolumn{2}{l}{(product in: COTS SW
)}
\\ \hline
\textbf{\footnotesize Manager} & \textbf{\footnotesize Owner} &
\textbf{\footnotesize WBS} & \textbf{\footnotesize Team} \\ \hline
\parbox{3.5cm}{

\vspace{2mm}%
} &
\begin{tabular}{@{}l@{}}
\parbox{3.5cm}{

\vspace{2mm}%
} \\
\end{tabular} &
\begin{tabular}{@{}l@{}}
 \\
\end{tabular} & \begin{tabular}{@{}l@{}}
 \\
\end{tabular} \\ \hline
\multicolumn{4}{c}{
{\footnotesize Short name: \textbf{ Docker
 } - Product key: \textbf{ DOCKER } }
}\\ \hline
\end{longtable}

3rd party software product used for creating distributions.


\begin{longtable}{p{3.7cm}p{3.7cm}p{3.7cm}p{3.7cm}}\hline
\textbf{References} &
\multicolumn{3}{l}{\href{https://www.docker.com/}{https://www.docker.com/} }
\\ \hline \\ \hline
\textbf{\footnotesize Uses:}  & & \textbf{\footnotesize Used in:} & \\ \hline
\multicolumn{2}{c}{
} &
\multicolumn{2}{c}{
\begin{tabular}{c}
\hyperlink{cam}{Containerized Application Management} \\ \hline
\end{tabular}
} \\ \bottomrule
\end{longtable}

     \newpage
\begin{longtable}{p{3.7cm}p{3.7cm}p{3.7cm}p{3.7cm}}\toprule
\multicolumn{2}{l}{\large \textbf{ \hypertarget{firefly}{Firefly} } }
& \multicolumn{2}{l}{(product in: COTS SW
)}
\\ \hline
\textbf{\footnotesize Manager} & \textbf{\footnotesize Owner} &
\textbf{\footnotesize WBS} & \textbf{\footnotesize Team} \\ \hline
\parbox{3.5cm}{

\vspace{2mm}%
} &
\begin{tabular}{@{}l@{}}
\parbox{3.5cm}{
Gregory Dubois-Felsmann
\vspace{2mm}%
} \\
\end{tabular} &
\begin{tabular}{@{}l@{}}
1.02C.05.06 \\
\end{tabular} & \begin{tabular}{@{}l@{}}
SUIT \\
\end{tabular} \\ \hline
\multicolumn{4}{c}{
{\footnotesize Short name: \textbf{ Firefly
 } - Product key: \textbf{ FIREFLY } }
}\\ \hline
\end{longtable}

IPAC's Advanced Astronomy WEB UI Framework.\\
~\\
\emph{{{[}last reviewed: ~G. Dubois-Felsmann - Feb 2020{]} }}\\


\begin{longtable}{p{3.7cm}p{3.7cm}p{3.7cm}p{3.7cm}}\hline
\textbf{References} &
\multicolumn{3}{l}{\href{https://github.com/Caltech-IPAC/firefly}{https://github.com/Caltech-IPAC/firefly} }
\\ \hline \\ \hline
\textbf{\footnotesize Uses:}  & & \textbf{\footnotesize Used in:} & \\ \hline
\multicolumn{2}{c}{
} &
\multicolumn{2}{c}{
\begin{tabular}{c}
\hyperlink{prtlsw}{LSP Portal Software } \\ \hline
\end{tabular}
} \\ \bottomrule
\end{longtable}

     \newpage
\begin{longtable}{p{3.7cm}p{3.7cm}p{3.7cm}p{3.7cm}}\toprule
\multicolumn{2}{l}{\large \textbf{ \hypertarget{gpfs}{General Parallel File System} } }
& \multicolumn{2}{l}{(product in: COTS SW
)}
\\ \hline
\textbf{\footnotesize Manager} & \textbf{\footnotesize Owner} &
\textbf{\footnotesize WBS} & \textbf{\footnotesize Team} \\ \hline
\parbox{3.5cm}{

\vspace{2mm}%
} &
\begin{tabular}{@{}l@{}}
\parbox{3.5cm}{

\vspace{2mm}%
} \\
\end{tabular} &
\begin{tabular}{@{}l@{}}
 \\
\end{tabular} & \begin{tabular}{@{}l@{}}
 \\
\end{tabular} \\ \hline
\multicolumn{4}{c}{
{\footnotesize Short name: \textbf{ Gen. P. File System
 } - Product key: \textbf{ GPFS } }
}\\ \hline
\end{longtable}

The General Parallel File System (GPFS) is a high-performance clustered
file system developed by IBM.


\begin{longtable}{p{3.7cm}p{3.7cm}p{3.7cm}p{3.7cm}}\hline
\textbf{References} &
\multicolumn{3}{l}{\href{https://en.wikipedia.org/wiki/IBM_General_Parallel_File_System}{https://en.wikipedia.org/wiki/IBM\_General\_Parallel\_File\_System} }
\\ \hline \\ \hline
\textbf{\footnotesize Uses:}  & & \textbf{\footnotesize Used in:} & \\ \hline
\multicolumn{2}{c}{
} &
\multicolumn{2}{c}{
\begin{tabular}{c}
\hyperlink{}{ICT Provisioning and Management} \\ \hline
\end{tabular}
} \\ \bottomrule
\end{longtable}

     \newpage
\begin{longtable}{p{3.7cm}p{3.7cm}p{3.7cm}p{3.7cm}}\toprule
\multicolumn{2}{l}{\large \textbf{ \hypertarget{grafana}{Grafana} } }
& \multicolumn{2}{l}{(product in: COTS SW
)}
\\ \hline
\textbf{\footnotesize Manager} & \textbf{\footnotesize Owner} &
\textbf{\footnotesize WBS} & \textbf{\footnotesize Team} \\ \hline
\parbox{3.5cm}{

\vspace{2mm}%
} &
\begin{tabular}{@{}l@{}}
\parbox{3.5cm}{

\vspace{2mm}%
} \\
\end{tabular} &
\begin{tabular}{@{}l@{}}
 \\
\end{tabular} & \begin{tabular}{@{}l@{}}
 \\
\end{tabular} \\ \hline
\multicolumn{4}{c}{
{\footnotesize Short name: \textbf{ Grafana
 } - Product key: \textbf{ GRAFANA } }
}\\ \hline
\end{longtable}

3rd party software product for analytics and monitoring


\begin{longtable}{p{3.7cm}p{3.7cm}p{3.7cm}p{3.7cm}}\hline
\textbf{References} &
\multicolumn{3}{l}{\href{https://grafana.com/}{https://grafana.com/} }
\\ \hline \\ \hline
\textbf{\footnotesize Uses:}  & & \textbf{\footnotesize Used in:} & \\ \hline
\multicolumn{2}{c}{
} &
\multicolumn{2}{c}{
\begin{tabular}{c}
\hyperlink{}{Monitoring} \\ \hline
\end{tabular}
} \\ \bottomrule
\end{longtable}

     \newpage
\begin{longtable}{p{3.7cm}p{3.7cm}p{3.7cm}p{3.7cm}}\toprule
\multicolumn{2}{l}{\large \textbf{ \hypertarget{htcondor}{HTCondor} } }
& \multicolumn{2}{l}{(product in: COTS SW
)}
\\ \hline
\textbf{\footnotesize Manager} & \textbf{\footnotesize Owner} &
\textbf{\footnotesize WBS} & \textbf{\footnotesize Team} \\ \hline
\parbox{3.5cm}{

\vspace{2mm}%
} &
\begin{tabular}{@{}l@{}}
\parbox{3.5cm}{

\vspace{2mm}%
} \\
\end{tabular} &
\begin{tabular}{@{}l@{}}
 \\
\end{tabular} & \begin{tabular}{@{}l@{}}
 \\
\end{tabular} \\ \hline
\multicolumn{4}{c}{
{\footnotesize Short name: \textbf{ HTCondor
 } - Product key: \textbf{ HTCONDOR } }
}\\ \hline
\end{longtable}

HTCondor is an open-source high-throughput computing software framework
for coarse-grained distributed parallelization of computationally
intensive tasks.


\begin{longtable}{p{3.7cm}p{3.7cm}p{3.7cm}p{3.7cm}}\hline
\textbf{References} &
\multicolumn{3}{l}{\href{https://research.cs.wisc.edu/htcondor/}{https://research.cs.wisc.edu/htcondor/} }
\\ \cline{2-4}
& \multicolumn{3}{l}{\href{https://github.com/htcondor}{https://github.com/htcondor} }
\\ \hline \\ \hline
\textbf{\footnotesize Uses:}  & & \textbf{\footnotesize Used in:} & \\ \hline
\multicolumn{2}{c}{
} &
\multicolumn{2}{c}{
\begin{tabular}{c}
\hyperlink{prodsrv}{Batch Production} \\ \hline
\end{tabular}
} \\ \bottomrule
\end{longtable}

     \newpage
\begin{longtable}{p{3.7cm}p{3.7cm}p{3.7cm}p{3.7cm}}\toprule
\multicolumn{2}{l}{\large \textbf{ \hypertarget{jh3}{Jupyterhub} } }
& \multicolumn{2}{l}{(product in: COTS SW
)}
\\ \hline
\textbf{\footnotesize Manager} & \textbf{\footnotesize Owner} &
\textbf{\footnotesize WBS} & \textbf{\footnotesize Team} \\ \hline
\parbox{3.5cm}{

\vspace{2mm}%
} &
\begin{tabular}{@{}l@{}}
\parbox{3.5cm}{

\vspace{2mm}%
} \\
\end{tabular} &
\begin{tabular}{@{}l@{}}
 \\
\end{tabular} & \begin{tabular}{@{}l@{}}
 \\
\end{tabular} \\ \hline
\multicolumn{4}{c}{
{\footnotesize Short name: \textbf{ Jupyterhub
 } - Product key: \textbf{ JH3 } }
}\\ \hline
\end{longtable}

Provides multi-user and multi-instance support to Jupyterlab.


\begin{longtable}{p{3.7cm}p{3.7cm}p{3.7cm}p{3.7cm}}\hline
\textbf{References} &
\multicolumn{3}{l}{\href{https://github.com/jupyterhub/jupyterhub/}{https://github.com/jupyterhub/jupyterhub/} }
\\ \hline \\ \hline
\textbf{\footnotesize Uses:}  & & \textbf{\footnotesize Used in:} & \\ \hline
\multicolumn{2}{c}{
} &
\multicolumn{2}{c}{
\begin{tabular}{c}
\hyperlink{nbsw}{LSP Notebook Software} \\ \hline
\end{tabular}
} \\ \bottomrule
\end{longtable}

     \newpage
\begin{longtable}{p{3.7cm}p{3.7cm}p{3.7cm}p{3.7cm}}\toprule
\multicolumn{2}{l}{\large \textbf{ \hypertarget{jl3}{Jupyterlab} } }
& \multicolumn{2}{l}{(product in: COTS SW
)}
\\ \hline
\textbf{\footnotesize Manager} & \textbf{\footnotesize Owner} &
\textbf{\footnotesize WBS} & \textbf{\footnotesize Team} \\ \hline
\parbox{3.5cm}{

\vspace{2mm}%
} &
\begin{tabular}{@{}l@{}}
\parbox{3.5cm}{

\vspace{2mm}%
} \\
\end{tabular} &
\begin{tabular}{@{}l@{}}
 \\
\end{tabular} & \begin{tabular}{@{}l@{}}
 \\
\end{tabular} \\ \hline
\multicolumn{4}{c}{
{\footnotesize Short name: \textbf{ Jupyterlab
 } - Product key: \textbf{ JL3 } }
}\\ \hline
\end{longtable}

An extensible environment for interactive and reproducible computing,
based on the Jupyter Notebook and Architecture.


\begin{longtable}{p{3.7cm}p{3.7cm}p{3.7cm}p{3.7cm}}\hline
\textbf{References} &
\multicolumn{3}{l}{\href{https://github.com/jupyterlab/jupyterlab/}{https://github.com/jupyterlab/jupyterlab/} }
\\ \hline \\ \hline
\textbf{\footnotesize Uses:}  & & \textbf{\footnotesize Used in:} & \\ \hline
\multicolumn{2}{c}{
} &
\multicolumn{2}{c}{
\begin{tabular}{c}
\hyperlink{nbsw}{LSP Notebook Software} \\ \hline
\end{tabular}
} \\ \bottomrule
\end{longtable}

     \newpage
\begin{longtable}{p{3.7cm}p{3.7cm}p{3.7cm}p{3.7cm}}\toprule
\multicolumn{2}{l}{\large \textbf{ \hypertarget{jra}{JIRA} } }
& \multicolumn{2}{l}{(product in: COTS SW
)}
\\ \hline
\textbf{\footnotesize Manager} & \textbf{\footnotesize Owner} &
\textbf{\footnotesize WBS} & \textbf{\footnotesize Team} \\ \hline
\parbox{3.5cm}{

\vspace{2mm}%
} &
\begin{tabular}{@{}l@{}}
\parbox{3.5cm}{

\vspace{2mm}%
} \\
\end{tabular} &
\begin{tabular}{@{}l@{}}
 \\
\end{tabular} & \begin{tabular}{@{}l@{}}
 \\
\end{tabular} \\ \hline
\multicolumn{4}{c}{
{\footnotesize Short name: \textbf{ Jira
 } - Product key: \textbf{ JRA } }
}\\ \hline
\end{longtable}

Issue Tracking and Project Management tool.


\begin{longtable}{p{3.7cm}p{3.7cm}p{3.7cm}p{3.7cm}}\hline
\textbf{\footnotesize Uses:}  & & \textbf{\footnotesize Used in:} & \\ \hline
\multicolumn{2}{c}{
} &
\multicolumn{2}{c}{
\begin{tabular}{c}
\hyperlink{itrck}{Issue Tracking} \\ \hline
\end{tabular}
} \\ \bottomrule
\end{longtable}

     \newpage
\begin{longtable}{p{3.7cm}p{3.7cm}p{3.7cm}p{3.7cm}}\toprule
\multicolumn{2}{l}{\large \textbf{ \hypertarget{k8s}{Kubernetes} } }
& \multicolumn{2}{l}{(product in: COTS SW
)}
\\ \hline
\textbf{\footnotesize Manager} & \textbf{\footnotesize Owner} &
\textbf{\footnotesize WBS} & \textbf{\footnotesize Team} \\ \hline
\parbox{3.5cm}{

\vspace{2mm}%
} &
\begin{tabular}{@{}l@{}}
\parbox{3.5cm}{

\vspace{2mm}%
} \\
\end{tabular} &
\begin{tabular}{@{}l@{}}
 \\
\end{tabular} & \begin{tabular}{@{}l@{}}
 \\
\end{tabular} \\ \hline
\multicolumn{4}{c}{
{\footnotesize Short name: \textbf{ Kubernetes
 } - Product key: \textbf{ K8S } }
}\\ \hline
\end{longtable}

Kubernetes is an open-source system for automating deployment, scaling,
and management of containerized applications.


\begin{longtable}{p{3.7cm}p{3.7cm}p{3.7cm}p{3.7cm}}\hline
\textbf{References} &
\multicolumn{3}{l}{\href{https://kubernetes.io/}{https://kubernetes.io/} }
\\ \hline \\ \hline
\textbf{\footnotesize Uses:}  & & \textbf{\footnotesize Used in:} & \\ \hline
\multicolumn{2}{c}{
} &
\multicolumn{2}{c}{
\begin{tabular}{c}
\hyperlink{cam}{Containerized Application Management} \\ \hline
\end{tabular}
} \\ \bottomrule
\end{longtable}

     \newpage
\begin{longtable}{p{3.7cm}p{3.7cm}p{3.7cm}p{3.7cm}}\toprule
\multicolumn{2}{l}{\large \textbf{ \hypertarget{psql}{PostgreSQL} } }
& \multicolumn{2}{l}{(product in: COTS SW
)}
\\ \hline
\textbf{\footnotesize Manager} & \textbf{\footnotesize Owner} &
\textbf{\footnotesize WBS} & \textbf{\footnotesize Team} \\ \hline
\parbox{3.5cm}{

\vspace{2mm}%
} &
\begin{tabular}{@{}l@{}}
\parbox{3.5cm}{

\vspace{2mm}%
} \\
\end{tabular} &
\begin{tabular}{@{}l@{}}
 \\
\end{tabular} & \begin{tabular}{@{}l@{}}
 \\
\end{tabular} \\ \hline
\multicolumn{4}{c}{
{\footnotesize Short name: \textbf{ PostgreSQL
 } - Product key: \textbf{ PSQL } }
}\\ \hline
\end{longtable}

Open Source Relational Database


\begin{longtable}{p{3.7cm}p{3.7cm}p{3.7cm}p{3.7cm}}\hline
\textbf{References} &
\multicolumn{3}{l}{\href{https://www.postgresql.org/}{https://www.postgresql.org/} }
\\ \hline \\ \hline
\textbf{\footnotesize Uses:}  & & \textbf{\footnotesize Used in:} & \\ \hline
\multicolumn{2}{c}{
} &
\multicolumn{2}{c}{
} \\ \bottomrule
\end{longtable}

     \newpage
\begin{longtable}{p{3.7cm}p{3.7cm}p{3.7cm}p{3.7cm}}\toprule
\multicolumn{2}{l}{\large \textbf{ \hypertarget{pth}{Python} } }
& \multicolumn{2}{l}{(product in: COTS SW
)}
\\ \hline
\textbf{\footnotesize Manager} & \textbf{\footnotesize Owner} &
\textbf{\footnotesize WBS} & \textbf{\footnotesize Team} \\ \hline
\parbox{3.5cm}{

\vspace{2mm}%
} &
\begin{tabular}{@{}l@{}}
\parbox{3.5cm}{

\vspace{2mm}%
} \\
\end{tabular} &
\begin{tabular}{@{}l@{}}
 \\
\end{tabular} & \begin{tabular}{@{}l@{}}
 \\
\end{tabular} \\ \hline
\multicolumn{4}{c}{
{\footnotesize Short name: \textbf{ Python
 } - Product key: \textbf{ PTH } }
}\\ \hline
\end{longtable}

Python is an interpreted, high-level, general-purpose programming
language.


\begin{longtable}{p{3.7cm}p{3.7cm}p{3.7cm}p{3.7cm}}\hline
\textbf{\footnotesize Uses:}  & & \textbf{\footnotesize Used in:} & \\ \hline
\multicolumn{2}{c}{
} &
\multicolumn{2}{c}{
} \\ \bottomrule
\end{longtable}

     \newpage
\begin{longtable}{p{3.7cm}p{3.7cm}p{3.7cm}p{3.7cm}}\toprule
\multicolumn{2}{l}{\large \textbf{ \hypertarget{puppet}{Puppet} } }
& \multicolumn{2}{l}{(product in: COTS SW
)}
\\ \hline
\textbf{\footnotesize Manager} & \textbf{\footnotesize Owner} &
\textbf{\footnotesize WBS} & \textbf{\footnotesize Team} \\ \hline
\parbox{3.5cm}{

\vspace{2mm}%
} &
\begin{tabular}{@{}l@{}}
\parbox{3.5cm}{

\vspace{2mm}%
} \\
\end{tabular} &
\begin{tabular}{@{}l@{}}
 \\
\end{tabular} & \begin{tabular}{@{}l@{}}
 \\
\end{tabular} \\ \hline
\multicolumn{4}{c}{
{\footnotesize Short name: \textbf{ Puppet
 } - Product key: \textbf{ PUPPET } }
}\\ \hline
\end{longtable}

Puppet is an open-source software configuration management tool.


\begin{longtable}{p{3.7cm}p{3.7cm}p{3.7cm}p{3.7cm}}\hline
\textbf{References} &
\multicolumn{3}{l}{\href{https://en.wikipedia.org/wiki/Puppet_(software)}{https://en.wikipedia.org/wiki/Puppet\_(software)} }
\\ \cline{2-4}
& \multicolumn{3}{l}{\href{https://puppet.com/}{https://puppet.com/} }
\\ \hline \\ \hline
\textbf{\footnotesize Uses:}  & & \textbf{\footnotesize Used in:} & \\ \hline
\multicolumn{2}{c}{
} &
\multicolumn{2}{c}{
\begin{tabular}{c}
\hyperlink{}{ICT Provisioning and Management} \\ \hline
\end{tabular}
} \\ \bottomrule
\end{longtable}

     \newpage
\begin{longtable}{p{3.7cm}p{3.7cm}p{3.7cm}p{3.7cm}}\toprule
\multicolumn{2}{l}{\large \textbf{ \hypertarget{rucio}{Rucio} } }
& \multicolumn{2}{l}{(product in: COTS SW
)}
\\ \hline
\textbf{\footnotesize Manager} & \textbf{\footnotesize Owner} &
\textbf{\footnotesize WBS} & \textbf{\footnotesize Team} \\ \hline
\parbox{3.5cm}{

\vspace{2mm}%
} &
\begin{tabular}{@{}l@{}}
\parbox{3.5cm}{

\vspace{2mm}%
} \\
\end{tabular} &
\begin{tabular}{@{}l@{}}
 \\
\end{tabular} & \begin{tabular}{@{}l@{}}
 \\
\end{tabular} \\ \hline
\multicolumn{4}{c}{
{\footnotesize Short name: \textbf{ Rucio
 } - Product key: \textbf{ RUCIO } }
}\\ \hline
\end{longtable}

Rucio is a project that provides services and associated libraries for
allowing scientific collaborations to manage large volumes of data
spread across facilities at multiple institutions and organisations.


\begin{longtable}{p{3.7cm}p{3.7cm}p{3.7cm}p{3.7cm}}\hline
\textbf{References} &
\multicolumn{3}{l}{\href{http://rucio.cern.ch/}{http://rucio.cern.ch/} }
\\ \cline{2-4}
& \multicolumn{3}{l}{\href{http://rucio.readthedocs.io/}{http://rucio.readthedocs.io/} }
\\ \hline \\ \hline
\textbf{\footnotesize Uses:}  & & \textbf{\footnotesize Used in:} & \\ \hline
\multicolumn{2}{c}{
} &
\multicolumn{2}{c}{
\begin{tabular}{c}
\hyperlink{bulkdsrv}{Bulk Distribution} \\ \hline
\end{tabular}
} \\ \bottomrule
\end{longtable}

     \newpage
\begin{longtable}{p{3.7cm}p{3.7cm}p{3.7cm}p{3.7cm}}\toprule
\multicolumn{2}{l}{\large \textbf{ \hypertarget{security}{IT Security SW} } }
& \multicolumn{2}{l}{(product in: COTS SW
)}
\\ \hline
\textbf{\footnotesize Manager} & \textbf{\footnotesize Owner} &
\textbf{\footnotesize WBS} & \textbf{\footnotesize Team} \\ \hline
\parbox{3.5cm}{

\vspace{2mm}%
} &
\begin{tabular}{@{}l@{}}
\parbox{3.5cm}{

\vspace{2mm}%
} \\
\end{tabular} &
\begin{tabular}{@{}l@{}}
 \\
\end{tabular} & \begin{tabular}{@{}l@{}}
 \\
\end{tabular} \\ \hline
\multicolumn{4}{c}{
{\footnotesize Short name: \textbf{ IT Security
 } - Product key: \textbf{ SECURITY } }
}\\ \hline
\end{longtable}




\begin{longtable}{p{3.7cm}p{3.7cm}p{3.7cm}p{3.7cm}}\hline
\textbf{\footnotesize Uses:}  & & \textbf{\footnotesize Used in:} & \\ \hline
\multicolumn{2}{c}{
} &
\multicolumn{2}{c}{
\begin{tabular}{c}
\hyperlink{itsec}{IT Security} \\ \hline
\end{tabular}
} \\ \bottomrule
\end{longtable}

     \newpage
\begin{longtable}{p{3.7cm}p{3.7cm}p{3.7cm}p{3.7cm}}\toprule
\multicolumn{2}{l}{\large \textbf{ \hypertarget{vsphere}{vSphere} } }
& \multicolumn{2}{l}{(product in: COTS SW
)}
\\ \hline
\textbf{\footnotesize Manager} & \textbf{\footnotesize Owner} &
\textbf{\footnotesize WBS} & \textbf{\footnotesize Team} \\ \hline
\parbox{3.5cm}{

\vspace{2mm}%
} &
\begin{tabular}{@{}l@{}}
\parbox{3.5cm}{

\vspace{2mm}%
} \\
\end{tabular} &
\begin{tabular}{@{}l@{}}
 \\
\end{tabular} & \begin{tabular}{@{}l@{}}
 \\
\end{tabular} \\ \hline
\multicolumn{4}{c}{
{\footnotesize Short name: \textbf{ vSphere
 } - Product key: \textbf{ VSPHERE } }
}\\ \hline
\end{longtable}

Third party software product for virtualization.


\begin{longtable}{p{3.7cm}p{3.7cm}p{3.7cm}p{3.7cm}}\hline
\textbf{References} &
\multicolumn{3}{l}{\href{https://www.vmware.com/products/vsphere.html}{https://www.vmware.com/products/vsphere.html} }
\\ \hline \\ \hline
\textbf{\footnotesize Uses:}  & & \textbf{\footnotesize Used in:} & \\ \hline
\multicolumn{2}{c}{
} &
\multicolumn{2}{c}{
\begin{tabular}{c}
\hyperlink{}{ICT Provisioning and Management} \\ \hline
\end{tabular}
} \\ \bottomrule
\end{longtable}

    
   \newpage
\subsubsection{Third Party Libraries}\label{thpl}
\begin{longtable}{p{3.7cm}p{3.7cm}p{3.7cm}p{3.7cm}}\hline
\textbf{Manager} & \textbf{Owner} & \textbf{WBS} & \textbf{Team} \\ \hline
\parbox{3.5cm}{

\vspace{2mm}%
} &
\begin{tabular}{@{}l@{}}
\parbox{3.5cm}{

\vspace{2mm}%
} \\
\end{tabular}
 &
\begin{tabular}{@{}l@{}}
 \\
\end{tabular} &
\begin{tabular}{@{}l@{}}
 \\
\end{tabular} \\ \hline
\multicolumn{4}{c}{
{\footnotesize Short name: \textbf{ Third Party Libs
 } - Product key: \textbf{ THPL } }
}\\ \hline
\end{longtable}

  External libraries required by the DM SW products in order to compile
and to run.


{\footnotesize
\textbf{Products included in this section}:
\begin{itemize}
\item Boost - \hyperlink{boost}{BOOST}
\end{itemize}
}

   \newpage
\begin{longtable}{p{3.7cm}p{3.7cm}p{3.7cm}p{3.7cm}}\toprule
\multicolumn{2}{l}{\large \textbf{ \hypertarget{boost}{Boost} } }
& \multicolumn{2}{l}{(product in: Third Party Libs
)}
\\ \hline
\textbf{\footnotesize Manager} & \textbf{\footnotesize Owner} &
\textbf{\footnotesize WBS} & \textbf{\footnotesize Team} \\ \hline
\parbox{3.5cm}{

\vspace{2mm}%
} &
\begin{tabular}{@{}l@{}}
\parbox{3.5cm}{

\vspace{2mm}%
} \\
\end{tabular} &
\begin{tabular}{@{}l@{}}
 \\
\end{tabular} & \begin{tabular}{@{}l@{}}
 \\
\end{tabular} \\ \hline
\multicolumn{4}{c}{
{\footnotesize Short name: \textbf{ Boost
 } - Product key: \textbf{ BOOST } }
}\\ \hline
\end{longtable}




\begin{longtable}{p{3.7cm}p{3.7cm}p{3.7cm}p{3.7cm}}\hline
\textbf{\footnotesize Uses:}  & & \textbf{\footnotesize Used in:} & \\ \hline
\multicolumn{2}{c}{
} &
\multicolumn{2}{c}{
} \\ \bottomrule
\end{longtable}

    
     \newpage
\subsection{Reference Data Products}\label{refd}
\begin{longtable}{p{3.7cm}p{3.7cm}p{3.7cm}p{3.7cm}}\hline
\textbf{Manager} & \textbf{Owner} & \textbf{WBS} & \textbf{Team} \\ \hline
\parbox{3.5cm}{

\vspace{2mm}%
} &
\begin{tabular}{@{}l@{}}
\parbox{3.5cm}{

\vspace{2mm}%
} \\
\end{tabular}
 &
\begin{tabular}{@{}l@{}}
 \\
\end{tabular} &
\begin{tabular}{@{}l@{}}
 \\
\end{tabular} \\ \hline
\multicolumn{4}{c}{
{\footnotesize Short name: \textbf{ Ref Data
 } - Product key: \textbf{ REFD } }
}\\ \hline
\end{longtable}

  (from lsst-dev01:/datasets/refcats/htm/README.txt)\\
~\\
This directory contains LSST style gen2 butler reference catalogs in
the\\
"indexed HTM" format, designed for use by\\
`lsst.meas.algorithms.LoadIndexedReferenceObjectsTask`.\\
~\\
For more about how the gen2 symlinked refcats work, see:\\
https://developer.lsst.io/services/datasets.html\#reference-catalogs\\
~\\
The v0 directory has refcats with fluxes implicitly in Janksy units.\\
The v1 directory has refcats with fluxes explicitly in nanojansky
units.\\
Support for the version 0 refcats will be removed from the stack in the
future.\\
For more information about the change from v0-\textgreater{}v1, see this
Community post:\\
https://community.lsst.org/t/photocalib-has-replaced-calib-welcoming-our-nanojansky-overlords/3648\\
~\\
The sub-directories containing each refcat may contain 100,000 or more
files\\
(each file representing one HTM pixel): running `ls`, `find`, or using\\
tab-completion in such a directory may take a very long time. To help
with this,\\
any files of interest beyond the index files, the `config.py` generated
by the\\
ingester, and each catalog's `README.txt` are listed in each section
below.


\newpage
\subsubsection{Gaia Data}\label{gaiad}
\begin{longtable}{p{3.7cm}p{3.7cm}p{3.7cm}p{3.7cm}}\hline
\textbf{Manager} & \textbf{Owner} & \textbf{WBS} & \textbf{Team} \\ \hline
\parbox{3.5cm}{

\vspace{2mm}%
} &
\begin{tabular}{@{}l@{}}
\parbox{3.5cm}{

\vspace{2mm}%
} \\
\end{tabular}
 &
\begin{tabular}{@{}l@{}}
 \\
\end{tabular} &
\begin{tabular}{@{}l@{}}
 \\
\end{tabular} \\ \hline
\multicolumn{4}{c}{
{\footnotesize Short name: \textbf{ Gaia Catalogs
 } - Product key: \textbf{ GAIAD } }
}\\ \hline
\end{longtable}

  Gaia Data release catalogs.


{\footnotesize
\textbf{Products included in this section}:
\begin{itemize}
\item Gaia\_DR1\_v1 - \hyperlink{gdr1v1}{GDR1v1}
\item Gaia\_DR2\_20190808 - \hyperlink{gdr2198}{GDR2198}
\item Gaia\_DR2\_20200414 - \hyperlink{gdr2204}{GDR2204}
\end{itemize}
}

   \newpage
\begin{longtable}{p{3.7cm}p{3.7cm}p{3.7cm}p{3.7cm}}\toprule
\multicolumn{2}{l}{\large \textbf{ \hypertarget{gdr1v1}{Gaia\_DR1\_v1} } }
& \multicolumn{2}{l}{(product in: Gaia Catalogs
)}
\\ \hline
\textbf{\footnotesize Manager} & \textbf{\footnotesize Owner} &
\textbf{\footnotesize WBS} & \textbf{\footnotesize Team} \\ \hline
\parbox{3.5cm}{

\vspace{2mm}%
} &
\begin{tabular}{@{}l@{}}
\parbox{3.5cm}{
Colin Slater
\vspace{2mm}%
} \\
\end{tabular} &
\begin{tabular}{@{}l@{}}
 \\
\end{tabular} & \begin{tabular}{@{}l@{}}
 \\
\end{tabular} \\ \hline
\multicolumn{4}{c}{
{\footnotesize Short name: \textbf{ Gaia\_DR1\_v1
 } - Product key: \textbf{ GDR1v1 } }
}\\ \hline
\end{longtable}

This catalog is available in both v0 and v1 formats.


\begin{longtable}{p{3.7cm}p{3.7cm}p{3.7cm}p{3.7cm}}\hline
\textbf{\footnotesize Uses:}  & & \textbf{\footnotesize Used in:} & \\ \hline
\multicolumn{2}{c}{
} &
\multicolumn{2}{c}{
} \\ \bottomrule
\end{longtable}

     \newpage
\begin{longtable}{p{3.7cm}p{3.7cm}p{3.7cm}p{3.7cm}}\toprule
\multicolumn{2}{l}{\large \textbf{ \hypertarget{gdr2198}{Gaia\_DR2\_20190808} } }
& \multicolumn{2}{l}{(product in: Gaia Catalogs
)}
\\ \hline
\textbf{\footnotesize Manager} & \textbf{\footnotesize Owner} &
\textbf{\footnotesize WBS} & \textbf{\footnotesize Team} \\ \hline
\parbox{3.5cm}{

\vspace{2mm}%
} &
\begin{tabular}{@{}l@{}}
\parbox{3.5cm}{

\vspace{2mm}%
} \\
\end{tabular} &
\begin{tabular}{@{}l@{}}
 \\
\end{tabular} & \begin{tabular}{@{}l@{}}
 \\
\end{tabular} \\ \hline
\multicolumn{4}{c}{
{\footnotesize Short name: \textbf{ Gaia\_DR2\_20190808
 } - Product key: \textbf{ GDR2198 } }
}\\ \hline
\end{longtable}

The full Gaia DR2 catalog in indexed HTM format.\\
See the README.txt in the refcat directory for details.\\
~\\
This catalog is only available in v1 format.\\
~\\
The configuration that was used to ingest the data is included as\\
`gaia\_dr2\_20190808/IngestIndexedReferenceTask.py`.\\
~\\
WARNING: The coordinate errors in this catalog are incorrect.\\
WARNING: This version of the Gaia dr2 catalog contains incorrect time
epochs.\\
WARNING: Please use the gaia\_dr2\_20200414 version of the catalog
instead.


\begin{longtable}{p{3.7cm}p{3.7cm}p{3.7cm}p{3.7cm}}\hline
\textbf{\footnotesize Uses:}  & & \textbf{\footnotesize Used in:} & \\ \hline
\multicolumn{2}{c}{
} &
\multicolumn{2}{c}{
} \\ \bottomrule
\end{longtable}

     \newpage
\begin{longtable}{p{3.7cm}p{3.7cm}p{3.7cm}p{3.7cm}}\toprule
\multicolumn{2}{l}{\large \textbf{ \hypertarget{gdr2204}{Gaia\_DR2\_20200414} } }
& \multicolumn{2}{l}{(product in: Gaia Catalogs
)}
\\ \hline
\textbf{\footnotesize Manager} & \textbf{\footnotesize Owner} &
\textbf{\footnotesize WBS} & \textbf{\footnotesize Team} \\ \hline
\parbox{3.5cm}{

\vspace{2mm}%
} &
\begin{tabular}{@{}l@{}}
\parbox{3.5cm}{

\vspace{2mm}%
} \\
\end{tabular} &
\begin{tabular}{@{}l@{}}
 \\
\end{tabular} & \begin{tabular}{@{}l@{}}
 \\
\end{tabular} \\ \hline
\multicolumn{4}{c}{
{\footnotesize Short name: \textbf{ Gaia\_DR2\_20200414
 } - Product key: \textbf{ GDR2204 } }
}\\ \hline
\end{longtable}

The full Gaia DR2 catalog in indexed HTM format.\\
See the README.txt in the refcat directory for details.\\
~\\
This catalog is only available in v1 format.\\
~\\
The configuration that was used to ingest the data is included as\\
`gaia\_dr2\_20200414/IngestIndexedReferenceTask.py`.


\begin{longtable}{p{3.7cm}p{3.7cm}p{3.7cm}p{3.7cm}}\hline
\textbf{\footnotesize Uses:}  & & \textbf{\footnotesize Used in:} & \\ \hline
\multicolumn{2}{c}{
} &
\multicolumn{2}{c}{
} \\ \bottomrule
\end{longtable}

    
   \newpage
\subsubsection{Other Catalogs}\label{odata}
\begin{longtable}{p{3.7cm}p{3.7cm}p{3.7cm}p{3.7cm}}\hline
\textbf{Manager} & \textbf{Owner} & \textbf{WBS} & \textbf{Team} \\ \hline
\parbox{3.5cm}{

\vspace{2mm}%
} &
\begin{tabular}{@{}l@{}}
\parbox{3.5cm}{

\vspace{2mm}%
} \\
\end{tabular}
 &
\begin{tabular}{@{}l@{}}
 \\
\end{tabular} &
\begin{tabular}{@{}l@{}}
 \\
\end{tabular} \\ \hline
\multicolumn{4}{c}{
{\footnotesize Short name: \textbf{ Other Catalogs
 } - Product key: \textbf{ ODATA } }
}\\ \hline
\end{longtable}

  


{\footnotesize
\textbf{Products included in this section}:
\begin{itemize}
\item Astrometry.net Data - \hyperlink{and}{AND}
\item sdss-dr9-fink-v5b - \hyperlink{sdssdr9v5b}{SDSSDR9v5b}
\item ps1\_pv3\_3pi\_20170110 - \hyperlink{ps1pv3}{PS1PV3}
\end{itemize}
}

   \newpage
\begin{longtable}{p{3.7cm}p{3.7cm}p{3.7cm}p{3.7cm}}\toprule
\multicolumn{2}{l}{\large \textbf{ \hypertarget{and}{Astrometry.net Data} } }
& \multicolumn{2}{l}{(product in: Other Catalogs
)}
\\ \hline
\textbf{\footnotesize Manager} & \textbf{\footnotesize Owner} &
\textbf{\footnotesize WBS} & \textbf{\footnotesize Team} \\ \hline
\parbox{3.5cm}{

\vspace{2mm}%
} &
\begin{tabular}{@{}l@{}}
\parbox{3.5cm}{

\vspace{2mm}%
} \\
\end{tabular} &
\begin{tabular}{@{}l@{}}
 \\
\end{tabular} & \begin{tabular}{@{}l@{}}
 \\
\end{tabular} \\ \hline
\multicolumn{4}{c}{
{\footnotesize Short name: \textbf{ Astronomy.net Data
 } - Product key: \textbf{ AND } }
}\\ \hline
\end{longtable}




\begin{longtable}{p{3.7cm}p{3.7cm}p{3.7cm}p{3.7cm}}\hline
\multicolumn{2}{r}{\textbf{GitHub Packages:}} &
\multicolumn{2}{l}{\href{https://github.com/lsst/astrometry_net_data}{astrometry\_net\_data} }\ref{astrometry_net_data}
\\ \hline \\ \hline
\textbf{\footnotesize Uses:}  & & \textbf{\footnotesize Used in:} & \\ \hline
\multicolumn{2}{c}{
} &
\multicolumn{2}{c}{
\begin{tabular}{c}
\hyperlink{spdist}{Science Pipelines Distribution} \\ \hline
\end{tabular}
} \\ \bottomrule
\end{longtable}

     \newpage
\begin{longtable}{p{3.7cm}p{3.7cm}p{3.7cm}p{3.7cm}}\toprule
\multicolumn{2}{l}{\large \textbf{ \hypertarget{sdssdr9v5b}{sdss-dr9-fink-v5b} } }
& \multicolumn{2}{l}{(product in: Other Catalogs
)}
\\ \hline
\textbf{\footnotesize Manager} & \textbf{\footnotesize Owner} &
\textbf{\footnotesize WBS} & \textbf{\footnotesize Team} \\ \hline
\parbox{3.5cm}{

\vspace{2mm}%
} &
\begin{tabular}{@{}l@{}}
\parbox{3.5cm}{

\vspace{2mm}%
} \\
\end{tabular} &
\begin{tabular}{@{}l@{}}
 \\
\end{tabular} & \begin{tabular}{@{}l@{}}
 \\
\end{tabular} \\ \hline
\multicolumn{4}{c}{
{\footnotesize Short name: \textbf{ sdss-dr9-fink-v5b
 } - Product key: \textbf{ SDSSDR9v5b } }
}\\ \hline
\end{longtable}

This catalog is available in both v0 and v1 formats.\\
\hspace*{0.333em}\hspace*{0.333em}\hspace*{0.333em}\hspace*{0.333em}\\
Converted from the astrometry.net style SDSS catalog in\\
`/datasets/refcats/astrometry\_net\_data/sdss-dr9-fink-v5b`.


\begin{longtable}{p{3.7cm}p{3.7cm}p{3.7cm}p{3.7cm}}\hline
\textbf{\footnotesize Uses:}  & & \textbf{\footnotesize Used in:} & \\ \hline
\multicolumn{2}{c}{
} &
\multicolumn{2}{c}{
} \\ \bottomrule
\end{longtable}

     \newpage
\begin{longtable}{p{3.7cm}p{3.7cm}p{3.7cm}p{3.7cm}}\toprule
\multicolumn{2}{l}{\large \textbf{ \hypertarget{ps1pv3}{ps1\_pv3\_3pi\_20170110} } }
& \multicolumn{2}{l}{(product in: Other Catalogs
)}
\\ \hline
\textbf{\footnotesize Manager} & \textbf{\footnotesize Owner} &
\textbf{\footnotesize WBS} & \textbf{\footnotesize Team} \\ \hline
\parbox{3.5cm}{

\vspace{2mm}%
} &
\begin{tabular}{@{}l@{}}
\parbox{3.5cm}{

\vspace{2mm}%
} \\
\end{tabular} &
\begin{tabular}{@{}l@{}}
 \\
\end{tabular} & \begin{tabular}{@{}l@{}}
 \\
\end{tabular} \\ \hline
\multicolumn{4}{c}{
{\footnotesize Short name: \textbf{ ps1\_pv3\_3pi\_20170110
 } - Product key: \textbf{ PS1PV3 } }
}\\ \hline
\end{longtable}

(from REQDME.txt)\\
~\\
This reference catalog, intended for use with the LSST Science
Pipelines\\
(https://pipelines.lsst.io) was constructed from the "3pi.pv3.20160422"
DVO\\
catalog of Processing Version 3 of the Pan-STARRS1 3pi survey, released
to\\
the Pan-STARRS1 Science Consortium. Following the public release of this
data\\
in December 2016 (http://panstarrs.stsci.edu), you may distribute this
catalog\\
freely.


\begin{longtable}{p{3.7cm}p{3.7cm}p{3.7cm}p{3.7cm}}\hline
\textbf{\footnotesize Uses:}  & & \textbf{\footnotesize Used in:} & \\ \hline
\multicolumn{2}{c}{
} &
\multicolumn{2}{c}{
} \\ \bottomrule
\end{longtable}

    
         



\newpage
\section{DM Jira Components}\label{sec:jiracomponents}

This section will list the components used in the DM Jira project.
Some of them are crealy related to a product included in the product tree or a low level git package.
Other Jira components are not mapped in the above sections, and will be described here.

The information should be extracted from Jira automatically and checks are executed in order to mach it with the information existing in MagicDraw and GitHub.
